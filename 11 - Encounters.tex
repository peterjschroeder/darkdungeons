\chapter[green]{Encounters}
\label{chap:Encounters}
\chapterimage[Encounters (Bordered)]
\thispagestyle{plain}

\begin{multicols*}{2}
As the characters explore, they will meet various denizens and need to deal with them. This will often result in combat, but could also result in diplomacy, trade, or the two parties simply ignoring each other.

\section{Surprise}\index[general]{Surprise}\label{sec:Surprise}
When two groups suddenly encounter each other, then the first thing that should happen is that the game should switch from general timekeeping to round-by-round timekeeping and each group should roll to see if they are surprised.

Assuming there are no special circumstances, each group rolls 1d6. If that group rolls a 1 or 2, the group is surprised and may not act this round.

If both groups are surprised or neither group is surprised, then round-by-round time simply starts as normal.

If one group is surprised and the other group is not, the group that is not surprised gets a single round in which to act before the other group can act.

In some circumstances, one side or the other might not need to roll for surprise. For example, if a party just spent three rounds trying to break a door down, the monsters at the other side of the door cannot be surprised. Similarly, if a rogue has scouted ahead and the party are aware of the presence of the monsters, the party cannot be surprised.

In some very unusual situations, it is possible that one particular member of a group may not be surprised while the rest of the group are. If that is the case, the unsurprised member will be able to act in the first round but other members of that group will not be able to.

\section{Reaction Roll}\index[general]{Reaction Roll}
When the monsters have their first action, the Game Master should determine what their reaction is to meeting the player characters.

If the monsters act before the player characters have acted because they surprised the players or won initiative, the Game Master will either know in advance how the monsters will react based on their personalities and the situation, or can consult \fullref{tab:Reaction Rolls}. The results of the table are explained in the following paragraphs.

\begin {table}[H]
  \caption{Reaction Rolls}\label{tab:Reaction Rolls}
  \begin{tabularx}{\columnwidth}{>{\bfseries}YY}
	\thead{2d6} & \thead{Reaction}\\
	2-3 & Hostile\\
	4-6 & Aggressive\\
	7-9 & Cautious\\
	10-11 & Neutral\\
	12 & Friendly
  \end {tabularx}
\end {table}

\textbf{Hostile:} The monsters will immediately attack, flee or surrender; depending on their numbers and strength compared to the apparent numbers and strength of the party.

\textbf{Aggressive:} The monsters will not immediately attack, but will threaten the party—either verbally or with growls and body language. If the reaction needs to be re-rolled because the party try to parley, the re-roll will take a -4 penalty.

\textbf{Cautious:} The monsters will not immediately attack, but will react with suspicion and may verbally challenge the party. They will ready themselves in case the party attack.

\textbf{Neutral:} The monsters will not attack, and will react in a neutral manner; ignoring the party or greeting them in a gruff or formal (but not overly friendly) manner. They will take precautions in case of attack by the party, but not in a threatening manner. If the reaction needs to be re-rolled because the party try to parley, the re-roll will have a +4 bonus.

\textbf{Friendly:} The monsters will greet the party in a friendly manner.

If the party respond to the monster’s reaction by attempting to parley, or the party act before the monsters and attempt to parley, when the monsters get a turn then the Game Master will either know how they will react based on the players actions, or can roll on the table again with whatever bonus or penalty came from the original roll and an additional bonus or penalty based on the \iref[sec:Charisma]{Charisma} modifier of the party leader or spokesperson.

The players have the option, if they are deliberately trying to insult or intimidate the monsters in an attempt to provoke them into a hostile reaction, of treating any \iref[sec:Charisma]{Charisma} bonus that the party leader has as if it were a penalty of equal magnitude.

If the result of these opening reactions (whether role played or rolled for) is that the party and the monsters end up talking, trading or otherwise acting in a non-hostile manner towards each other, then the game can switch back to general timekeeping.

If the result is that a fight or chase breaks out, then the game should stay in round-by-round timekeeping, and the combat should be resolved using the rules in \fullref{chap:Combat}.

\section{Dungeon Encounters}\index[general]{Dungeon Encounters}
One of the most common places adventurers will explore is a dungeon. A dungeon can be considered an actual dungeon beneath a stronghold, an ancient tomb, or a natural cave.

The adventuring party may not be the only people (or creatures) wandering around the dungeon. Some of the monsters that live in the dungeon will almost certainly move from place to place, and there may be other creatures or other adventurers that have also entered the dungeon for reasons of their own; whether looking for food, shelter or to loot the place.

These creatures that may be found wandering around the dungeon are referred to as “Wandering Monsters”.

The Game Master may decide that particular dungeons (or particular areas within a dungeon) are more or less likely places for adventuring parties to find wandering monsters; and may therefore alter the chance and frequency in those areas.

The default frequency for wandering monsters in a dungeon setting is for the Game Master to roll 1d6 every twenty minutes of game time (not real time). If the Game Master rolls a 1, then the party will encounter a group of wandering monsters.

The Game Master may have a pre-prepared list of what monsters (and how many) may be wandering around the dungeon. If not, roll on the wandering monster tables in this chapter (\autoref*{tab:Wandering Dungeon Monsters (Difficulty 1)} to \autoref*{tab:Wandering Dungeon Monsters (Difficulty 8-10)}).

The wandering monster tables are arranged by the estimated difficulty of the encounters, and the numbers of monsters encountered are tailored for this difficulty rather than necessarily matching the normal numbers that the monsters are found in.

The choice of which difficulty level to use should be based on the subjective difficulty of the dungeon itself.

A good guideline is to match the difficulty to the level of characters that the dungeon was designed for.

In any case, these tables are generic and although they can produce a wide variety of monsters, they can also produce wildly unrealistic results; indicating monsters that have no place in the current dungeon.

Game Masters are advised to use these tables only when they have not made a custom table for their dungeon, and to re-roll results that don’t “fit” the current dungeon.

\begin {table}[H]
  \caption{Wandering Dungeon Monsters (Difficulty 1)}\label{tab:Wandering Dungeon Monsters (Difficulty 1)}
  \begin{tabularx}{\columnwidth}{>{\bfseries}YYY}
	\thead{1d20} & \thead{Monster} & \thead{Number Encountered}\\
	1 & Beetle, Giant Fire & 1d6\\
	2 & Centipede, Giant & 1d6\\
	3 & Ghoul & 1d2\\
	4 & Goblin & 1d6\\
	5 & Human, Bandit & 1d6\\
	6-9 & Human, Commoner & 1d3\\
	10 & Kobold & 2d6\\
	11 & Lizard, Giant Gecko & 1d2\\
	12 & Locust, Giant & 1d6\\
	13 & NPC Party & 1 Party\\
	14 & Orc & 1d6\\
	15 & Skeleton & 1d10\\
	16 & Snake, Racer & 1d2\\
	17 & Spider, Giant Crab & 1d2\\
	18 & Stirge & 1d8\\
	19 & Troglodyte & 1d3\\
	20 & Zombie & 1d3
  \end {tabularx}
\end {table}

\begin {table}[H]
  \caption{Wandering Dungeon Monsters (Difficulty 2)}
  \begin{tabularx}{\columnwidth}{>{\bfseries}YYY}
  \thead{1d20} & \thead{Monster} & \thead{Number Encountered}\\
	1 & Beetle, Giant Bombard & 1d6\\
	2 & Ghoul & 1d4\\
	3 & Gnoll & 1d4\\
	4 & Goblin & 2d4\\
	5 & Gray Ooze & 1\\
	6 & Hobgoblin & 1d6\\
	7-9 & Human, Commoner & 1d3\\
	10 & Lizard, Giant Draco & 1\\
	11 & Lizardfolk & 1d6\\
	12 & Neanderthal & 2d4\\
	13 & NPC Party & 1 Party\\
	14 & Orc & 1d10\\
	15 & Skeleton & 2d6\\
	16 & Snake, Poisonous & 2d6\\
	17 & Spider, Giant Black Widow & 1\\
	18 & Troglodyte & 1d6\\
	19 & Worm, Cthonic & 1\\
	20 & Zombie & 1d6
  \end {tabularx}
\end {table}

\begin {table}[H]
  \caption{Wandering Dungeon Monsters (Difficulty 3)}
  \begin{tabularx}{\columnwidth}{>{\bfseries}YYY}
  \thead{1d20} & \thead{Monster} & \thead{Number Encountered}\\
	1 & Ape, Cave & 1d4\\
	2 & Beetle, Giant Tiger & 1d4\\
	3 & Bugbear & 1d6\\
	4 & Doppelganger & 1d2\\
	5 & Gargoyle & 1d3\\
	6 & Gelatinous Cube & 1\\
	7 & Ghast & 1d4\\
	8 & Harpy & 1d3\\
	9-10 & Human, Commoner & 1d3\\
	11 & Living Crystal Statue & 1d4\\
	12 & Lycanthrope, Wererat & 1d6\\
	13 & Medusa & 1\\
	14 & NPC Party & 1 Party\\
	15 & Ochre Jelly & 1\\
	16 & Ogre & 1d3\\
	17 & Shadow & 1d4\\
	18 & Spider, Giant Tarantella & 1\\
	19 & Wight & 1d3\\
	20 & Worm, Cthonic & 1d3
  \end {tabularx}
\end {table}

\begin {table}[H]
  \caption{Wandering Dungeon Monsters (Difficulty 4-5)}
  \begin{tabularx}{\columnwidth}{>{\bfseries}YYY}
  \thead{1d20} & \thead{Monster} & \thead{Number Encountered}\\
	1 & Blink Dog & 1d4\\
	2 & Bugbear & 1d6+4\\
	3 & Caecilian, Giant & 1\\
	4 & Cockatrice & 1d2\\
	5 & Coerl & 1d4+1\\
	6 & Gargoyle & 1\\
	7 & Giant, Hill & 1d4+1\\
	8 & Harpy & 1d4\\
	9 & Hellhound (1d3+2 HD) & 1\\
	10 & Hydra (5-headed) & 1d4\\
	11 & Lycanthrope, Werewolf & 1d2\\
	12 & Medusa & 1\\
	13 & Mummy & 1d3\\
	14 & NPC Party & 1 Party\\
	15 & Ochre Jelly & 1\\
	16 & Rhagodessa, Giant & 1d3\\
	17 & Rust Monster & 1d2\\
	18 & Scorpion, Giant & 1d3\\
	19 & Troll & 1d2\\
	20 & Wraith & 1d2
  \end {tabularx}
\end {table}

\begin {table}[H]
  \caption{Wandering Dungeon Monsters (Difficulty 6-7)}
  \begin{tabularx}{\columnwidth}{>{\bfseries}YYY}
  \thead{1d20} & \thead{Monster} & \thead{Number Encountered}\\
	1 & Basilisk & 1d3\\
	2 & Caecilian, Giant & 1d4\\
	3 & Cockatrice & 1d3\\
	4 & Giant, Hill & 1d2\\
	5 & Giant, Stone & 1d2\\
	6 & Hellhound (5-7 HD) & 1d4\\
	7 & Hydra (6-8 headed) & 1\\
	8 & Lycanthrope* & 1d3\\
	9 & Manticore & 1\\
	10 & Minotaur & 1d4\\
	11 & Mummy & 1d4\\
	12 & NPC Party & 1 Party\\
	13 & Ochre Jelly & 1\\
	14 & Ogre & 2d4\\
	15 & Rust Monster & 1d3+1\\
	16 & Spectre & 1d3\\
	17 & Spider, Giant Tarantella & 1d3\\
	18 & Salamander, Flame & 1d2\\
	19 & Troll & 1d4+1\\
	20 & Vampire & 1
  \end {tabularx}
	*Either Werebear or Weretiger
\end {table}

\begin {table}[H]
  \caption{Wandering Dungeon Monsters (Difficulty 8-10)}\label{tab:Wandering Dungeon Monsters (Difficulty 8-10)}
  \begin{tabularx}{\columnwidth}{>{\bfseries}YYY}
  \thead{1d20} & \thead{Monster} & \thead{Number Encountered}\\
	1 & Basilisk & 1d6\\
	2 & Black Pudding & 1\\
	3 & Chimera & 1\\
	4 & Construct* & 1\\
	5 & Dragon* & 1d2\\
	6-7 & Giant* & 1d6\\
	8 & Golem* & 1d4+1\\
	9 & Hydra (7-12 headed) & 1\\
	10-11 & Lycanthrope, Werebear & 1d6+1\\
	12 & NPC Party & 1 Party\\
	13 & Phantom, Apparition & 1\\
	14 & Rust Monster & 1d4+1\\
	15 & Salamander* & 1d4\\
	16 & Snake, Poisonous & 1d4+1\\
	17 & Spectre & 1d3\\
	18 & Spider* & 1d4+1\\
	19 & Vampire & 1d2\\
	20 & Worm, Purple & 1\\
	\end {tabularx}
	*Any one type; modify the number encountered for the level of monster.
\end {table}

\section{Wilderness Encounters}\index[general]{Wilderness Encounters}\label{sec:Wilderness Encounters}
Unlike dungeon situations, where there tend to be fixed structures with fixed creatures living in them, adventuring in the wilderness is a lot more random.

While there may be particular fixed locations that the Game Master has marked on their map as being the lairs of monsters or the territories of particular races; most of the time it is not feasible to work this out in advance for every square mile of the country or even planet that the players might want to explore.

In the same way that characters may encounter wandering monsters in dungeons, they may also encounter wandering monsters in the wilderness. The Game Master should check twice per 24-hour period; once during the day and once during the night. The chance of an encounter occurring is based on the type of terrain that the party is traveling through, and can be found on \fullref{tab:Wilderness Encounter Chances}.

If the party is traveling through terrain that fits more than one category (e.g. wooded hills), or is traveling through more than one type of terrain during the day, then the Game Master should pick whichever type is most suitable.

Once the type of encounter has been determined, the exact encounter can either be determined by the Game Master’s wishes or rolled randomly using 1d12 on the relevant table.

The number of creatures encountered is not given on the encounter tables. Instead, it is found in the monster descriptions in \fullref{chap:Monsters}. In the monster descriptions in that chapter, two numbers are given for each monster—a lair group and a wandering group. The Game Master is free to select whether the party have come across a wandering group of the monsters or whether they have come across the monsters’ lair. When selecting this, the Game Master should take into account both the party’s current activity (exploring, traveling along a well-worn road, or stationary) and what type of lair the monsters are likely to have.

If the Game Master wishes, they can replace these tables with tables specific to the areas of their own worlds—for example a particular mountain range might not contain kobolds, but might be known to contain lots of orcs. A replacement table could be made for that mountain range with the “kobold” entries swapped for additional “orc” entries.

\subsection{Castles}
The “Settled” column of \fullref{tab:Wilderness Encounters} has an entry labeled “Castle”. Unlike the other entries on that table, this entry does not link to another table.

If the Game Master already has a detailed map of the area, and there is no such castle, then this entry should be re-rolled. Otherwise, it means that the party has arrived at a castle or other stronghold.

To generate a random castle, refer to \fullref{tab:Castle Owner} to see who the owner of the castle is.

\begin {table}[H]
  \caption{Castle Owner}\label{tab:Castle Owner}
  \begin{tabularx}{\columnwidth}{>{\bfseries}YY}
	\thead{1d20} & \thead{Owner}\\
	1-3 & Cleric\\
	4 & Dwarf\\
	5 & Elf\\
	6-13 & Fighter\\
	14 & Gnome\\
	15 & Halfling\\
	16-17 & Rogue\\
	18-20 & Wizard
  \end {tabularx}
\end {table}

This owner will be a level 1d20+8 character of that class.

The Game Master should also refer to \fullref{tab:Castle Owner Allegiance}, to determine randomly what allegiance the castle’s owner has to the rulers of the country.

\begin {table}[H]
  \caption{Castle Owner Allegiance}\label{tab:Castle Owner Allegiance}
  \begin{tabularx}{\columnwidth}{>{\bfseries}YY}
	\thead{1d6} & \thead{Loyalty}\\
	1-2 & Fanatically loyal\\
	3-5 & Reasonably loyal\\
	6 & Disloyal
  \end {tabularx}
\end {table}

Obviously, this allegiance will not usually be openly displayed to a passing adventuring party.

\subsection{Encounter Balance}
The encounters listed on the following pages vary tremendously in strength, ranging from simple kobolds to mighty dragon queens.

Some encounters may be very easy for the party to overcome, and others may well be nigh impossible to overcome in any way other than the party simply hiding or fleeing from the creature(s) that they have encountered.

This variation is an essential part of the game—it is dangerous out in the wilderness and low level parties venture away from settled areas at their own risk—and therefore the Game Master shouldn’t feel that they have to re-roll encounters that are unsuitable for the party’s level.

The players should not get the feeling that the world is “leveling up” as they do, and that the Game Master is simply selecting monsters of an appropriate difficulty.

On the other hand, it is important for the Game Master to be fair to the players. There’s no fun in a low level party leaving town and getting eaten by a dragon on the first night.

The Game Master should therefore ensure that overwhelming fights can be avoided, whether that is through the party spotting the encounter before it spots them and hiding or avoiding it, or whether it is through the encounter not necessarily being hostile.

Obviously, if the party act in a belligerent or hostile manner to creatures that are far more powerful than they are, then they may well be killed. But it is unfair (and not fun for the players) to put them straight into a combat situation that they can’t win just because the dice rolled a particularly hard encounter, without giving them any chance to avoid a fight by fleeing, parleying or hiding.

\begin {table}[H]
  \caption{Wilderness Encounter Chances}\label{tab:Wilderness Encounter Chances}
  \begin{tabularx}{\columnwidth}{>{\bfseries}YYY}
	\thead{} & \multicolumn{2}{c}{\thead{Chance of Encounter (1d12)}}\\
	\thead{Terrain} & \thead{Day} & \thead{Night}\\
	Arctic & 1-2 & 1\\
	Barren Lands & 1-4 & 1-2\\
	City & 1-4 & 1-2\\
	Clear & 1-2 & 1\\
	Desert & 1-4 & 1-2\\
	Flying (any terrain) & 1-4 & 1-2\\
	Forest & 1-4 & 1-2\\
	Hills & 1-4 & 1-2\\
	Jungle & 1-6 & 1-3\\
	Mountains & 1-6 & 1-3\\
	Ocean & 1-4 & 1-2\\
	River & 1-4 & 1-2\\
	Settled & 1-2 & 1\\
	Swamp & 1-6 & 1-3
  \end {tabularx}
\end {table}

\end{multicols*}
\begin {table}[H]
  \caption{Wilderness Encounters}\label{tab:Wilderness Encounters}
  \begin{tabularx}{\columnwidth}{>{\bfseries}YYYYY}
	\thead{1d8} & \thead{Arctic} & \thead{Barren, Mountains, Hills} & \thead{City} & \thead{Clear}\\
	1 & Animal & Animal & Human & Animal\\
	2 & Animal & Dragon & Human & Animal\\
	3 & Animal & Dragon & Human & Dragon\\
	4 & Animal & Flyer & Human & Flyer\\
	5 & Dragon (White) & Human & Human & Human\\
	6 & Humanoid & Humanoid & Human & Humanoid\\
	7 & Humanoid & Humanoid & Humanoid & Insect\\
	8 & Humanoid & Unusual & Undead & Unusual\\
	\thead{1d8} & \thead{Desert} & \thead{Jungle} & \thead{Ocean} & \thead{River}\\
	1 & Animal & Animal & Dragon & Animal\\
	2 & Animal & Animal & Flyer & Dragon\\
	3 & Dragon & Dragon & Human & Flyer\\
	4 & Flyer & Flyer & Swimmer & Human\\
	5 & Human & Human & Swimmer & Humanoid\\
	6 & Human & Humanoid & Swimmer & Insect\\
	7 & Humanoid & Insect & Swimmer & Swimmer\\
	8 & Undead & Insect & Swimmer & Swimmer\\
	\thead{1d8} & \thead{Settled} & \thead{Swamp} & \thead{Woods}\\
	1 & Animal & Dragon & Animal\\ 
	2 & Animal & Flyer & Animal\\
	3 & Castle* & Human & Dragon\\
	4 & Dragon & Humanoid & Flyer\\ 
	5 & Flyer & Insect & Human\\
	6 & Human & Swimmer & Humanoid\\ 
	7 & Human & Undead & Insect\\
	8 & Humanoid & Undead & Unusual
  \end {tabularx}
	*See text for details of Castle encounters
\end {table}

\begin {table}[H]
  \caption{Animals}
  \begin{tabularx}{\columnwidth}{>{\bfseries}YYYYY}
	\thead{1d12} & \thead{Arctic} & \thead{Barren Lands, Mountains, Hills} & \thead{Clear} & \thead{Desert}\\
	1 & Ape, Snow & Ape, Cave & Ape, Rock Baboon & Camel\\
	2 & Ape, Snow & Ape, Rock Baboon & Boar & Camel\\
	3 & Ape, Snow & Ape, Snow & Cat, Lion & Cat, Lion\\
	4 & Ape, Snow & Bear, Cave & Elephant & Cat, Lion\\
	5 & Ape, Snow & Bear, Grizzly & Ferret, Giant & Herd Animal\\
	6 & Ape, Snow & Cat, Mountain Lion & Herd Animal & Herd Animal\\
	7 & Bear, Polar & Herd Animal & Horse, Riding & Lizard, Giant Gecko\\
	8 & Bear, Polar & Mule & Lizard, Giant Draco & Lizard, Giant Tuatara\\
	9 & Bear, Polar & Snake, Poisonous & Mule & Snake, Poisonous\\
	10 & Bear, Polar & Snake, Poisonous & Snake, Poisonous & Snake, Poisonous\\
	11 & Bear, Polar & Wolf & Snake, Poisonous & Spider, Giant Black Widow\\
	12 & Bear, Polar & Wolf, Dire & Weasel, Giant & Spider, Giant Tarantella\\
	\thead{1d12} & \thead{Jungle} & \thead{River} & \thead{Settled} & \thead{Woods}\\
	1 & Boar & Boar & Boar & Boar\\
	2 & Cat, Panther & Cat, Panther & Cat, Tiger & Cat, Panther\\
	3 & Herd Animal & Cat, Tiger & Ferret, Giant & Cat, Tiger\\
	4 & Lizard, Giant Draco & Crab, Giant & Herd Animal & Herd Animal\\
	5 & Lizard, Giant Gecko & Crocodile & Herd Animal & Lizard, Giant Draco\\
	6 & Lizard, Giant Horned & Crocodile, Giant & Horse, Riding & Lizard, Giant Gecko\\
	7 & Rat, Giant & Fish, Giant Stone & Rat, Giant & Lizard, Giant Tuatara\\
	8 & Shrew, Giant & Herd Animal & Shrew, Giant & Snake, Poisonous\\
	9 & Snake, Constrictor & Leech, Giant & Snake, Poisonous & Snake, Poisonous\\
	10 & Snake, Poisonous & Rat, Giant & Snake, Racer & Unicorn\\
	11 & Snake, Poisonous & Shrew, Giant & Spider, Giant Tarantella & Wolf\\
	12 & Spider, Giant Crab & Toad, Giant & Wolf & Wolf, Dire
  \end {tabularx}
\end {table}

\begin {table}[H]
  \caption{Humans}
  \begin{tabularx}{\columnwidth}{>{\bfseries}YYYYYY}
	\thead{1d12} & \thead{Clear} & \thead{Desert} & \thead{Hill} & \thead{Jungle} & \thead{Ocean}\\
	1 & Adventurer & Adventurer & Adventurer & Adventurer & Adventurer\\
	2 & Bandit & Cleric & Bandit & Adventurer & Bandit (Buccaneer)\\
	3 & Bandit & Dervish & Bandit (Brigand) & Bandit & Bandit (Buccaneer)\\
	4 & Bandit (Brigand) & Dervish & Berserker & Bandit (Brigand) & Bandit (Pirate)\\
	5 & Berserker & Fighter & Berserker & Bandit (Brigand) & Bandit (Pirate)\\
	6 & Cleric & Merchant & Cleric & Bandit (Brigand) & Bandit (Pirate)\\
	7 & Fighter & Noble & Fighter & Berserker & Bandit (Pirate)\\
	8 & Merchant & Nomad & Merchant & Cleric & Merchant\\
	9 & Merchant & Nomad & Neanderthal & Fighter & Merchant\\
	10 & Noble & Nomad & Neanderthal & Merchant & Merchant\\
	11 & Nomad & Nomad & Neanderthal & Neanderthal & Merchant\\
	12 & Wizard & Wizard & Wizard & Wizard & Merchant\\
	\thead{1d12} & \thead{River} & \thead{Settled} & \thead{Swamp} & \thead{Woods}\\
	1 & Adventurer & Adventurer & Adventurer & Adventurer\\
	2 & Bandit & Bandit & Adventurer & Bandit\\
	3 & Bandit (Buccaneer) & Bandit & Bandit & Bandit\\
	4 & Bandit (Buccaneer) & Cleric & Bandit & Bandit (Brigand)\\
	5 & Bandit (Buccaneer) & Commoner & Bandit (Brigand) & Bandit (Brigand)\\
	6 & Cleric & Fighter & Berserker & Berserker\\
	7 & Cleric & Merchant & Cleric & Cleric\\
	8 & Fighter & Noble & Fighter & Druid\\
	9 & Merchant & NPC Party & Merchant & Fighter\\
	10 & Merchant & Trader & NPC Party & Merchant\\
	11 & NPC Party & Veteran & Trader & NPC Party\\
	12 & Wizard & Wizard & Wizard & Wizard
  \end {tabularx}
\end {table}

\begin {table}[H]
  \caption{Humanoids}
  \begin{tabularx}{\columnwidth}{>{\bfseries}YYYYYY}
	\thead{1d12} & \thead{Arctic} & \thead{Barren Lands, Hills, Mountains} & \thead{Clear} & \thead{City, Settled} & \thead{Desert}\\
	1 & Giant, Frost & Athach & Bugbear & Dwarf & Giant, Fire\\
	2 & Giant, Frost & Cyclops & Elf & Elf & Goblin\\
	3 & Giant, Frost & Dwarf & Giant, Hill & Giant, Hill & Goblin\\
	4 & Giant, Frost & Giant, Hill & Gnoll & Gnoll & Hobgoblin\\
	5 & Giant, Frost & Giant, Stone & Gnome & Gnome & Hobgoblin\\
	6 & Giant, Frost & Giant, Storm & Goblin & Goblin & Ogre\\
	7 & Sasquatch & Gnome & Halfling & Halfling & Ogre\\
	8 & Sasquatch & Goblin & Hobgoblin & Hobgoblin & Ogre\\
	9 & Sasquatch & Kobold & Ogre & Ogre & Orc\\
	10 & Sasquatch & Orc & Orc & Orc & Orc\\
	11 & Sasquatch & Troglodyte & Pixie & Pixie & Pixie\\
	12 & Sasquatch & Troll & Troll & Sprite & Sprite\\
	\thead{1d12} & \thead{Jungle} & \thead{River} & \thead{Swamp} & \thead{Woods}\\
	1 & Bugbear & Bugbear & Gnoll & Bugbear\\
	2 & Cyclops & Elf & Goblin & Cyclops\\
	3 & Elf & Gnoll & Hobgoblin & Dryad\\
	4 & Giant, Fire & Hobgoblin & Lizardfolk & Elf\\
	5 & Giant, Hill & Lizardfolk & Lizardfolk & Elf\\
	6 & Gnoll & Lizardfolk & Lizardfolk & Giant, Hill\\
	7 & Goblin & Lizardfolk & Nixie & Gnoll\\
	8 & Lizardfolk & Nixie & Ogre & Goblin\\
	9 & Ogre & Ogre & Orc & Hobgoblin\\
	10 & Orc & Orc & Troglodyte & Ogre\\
	11 & Troglodyte & Sprite & Troll & Orc\\
	12 & Troll & Troll & Troll & Troll
  \end {tabularx}
\end {table}

\begin {table}[H]
  \caption{Other Wilderness Encounters}
  \begin{tabularx}{\columnwidth}{>{\bfseries}YYYYYY}
	\thead{1d12} & \thead{Dragons} & \thead{Flyers (Mountain)} & \thead{Flyers (Desert)} & \thead{Flyers (Other)} & \thead{Insects}\\
	1 & Chimera & Bee, Giant & Bird of Prey, Giant & Bee, Giant & Ant, Giant\\
	2 & Dragon (Black) & Bird of Prey, Giant & Gargoyle & Cockatrice & Bee, Giant\\
	3 & Dragon (Blue) & Gargoyle & Gargoyle & Gargoyle & Beetle, Giant Bombard\\
	4 & Dragon (Gold) & Griffon & Griffon & Griffon & Beetle, Giant Fire\\
	5 & Dragon (Green) & Harpy & Harpy & Hippogriff & Beetle, Giant Tiger\\
	6 & Dragon (Red) & Hippogriff & Insect Swarm & Lizard, Giant Draco & Insect Swarm\\
	7 & Dragon (White) & Insect Swarm & Lizard, Giant Draco & Pegasus & Rhagodessa, Giant\\
	8 & Hydra & Manticore & Manticore & Pixie & Robber Fly, Giant\\
	9 & Hydra & Pegasus & Manticore & Roc & Scorpion, Giant\\
	10 & Salamander, Flame & Robber Fly, Giant & Manticore & Robber Fly, Giant & Spider, Giant Black Widow\\
	11 & Salamander, Frost & Roc & Roc & Sprite & Spider, Giant Crab\\
	12 & Wyvern & Roc, Gargantuan & Roc, Gargantuan & Stirge & Spider, Giant Tarantella\\
	\thead{1d12} & \thead{Swimmers (River/Lake)} & \thead{Swimmers (Ocean)} & \thead{Swimmers (Swamp)} & \thead{Undead} & \thead{Unusual}\\
	1 & Crab, Giant & Giant, Storm & Crab, Giant & Ghoul & Basilisk\\
	2 & Crocodile & Hydra, Sea & Crocodile & Ghoul & Blink Dog\\
	3 & Crocodile, Giant & Hydra, Sea & Crocodile & Ghoul & Centaur\\
	4 & Fish, Giant Bass & Hydra, Sea & Crocodile, Giant & Mummy & Coerl\\
	5 & Fish, Giant Sturgeon & Merfolk & Crocodile, Giant & Skeleton & Gorgon\\
	6 & Leech, Giant & Snake, Sea & Leech, Giant & Skeleton & Lycanthrope, Werebear\\
	7 & Leech, Giant & Snake, Sea & Leech, Giant & Spectre & Lycanthrope, Wereboar\\
	8 & Lizardfolk & Snake, Sea & Leech, Giant & Vampire & Lycanthrope, Wererat\\
	9 & Lizardfolk & Snake, Sea & Lizardfolk & Wight & Lycanthrope, Weretiger\\
	10 & Merfolk & Termite, Giant Water & Lizardfolk & Wraith & Lycanthrope, Werewolf\\
	11 & Nixie & Termite, Giant Water & Termite, Giant Water & Zombie & Medusa\\
	12 & Termite, Giant Water & Termite, Giant Water & Termite, Giant Water & Zombie & Treant
  \end {tabularx}
\end {table}

