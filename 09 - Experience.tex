\chapter[green]{Experience}
\label{chap:Experience}
\chapterimage[Experience (Bordered)]
\thispagestyle{plain}

\begin{multicols*}{2}
As characters adventure, they gain experience and get better at what they do. This is measured in experience points (often abbreviated to XP).

\section{Gaining Experience}\index[general]{Gaining Experience}\label{sec:Gaining Experience}
Characters are awarded experience points in various situations, and these get added to the character’s experience point total. With the exception of being caught by an \iref[sec:Energy Drain]{Energy Drain} attack, experience points are never deducted from a character’s total.

Experience earned by a character may be affected by their \iref[sec:Prime Requisite]{Prime Requisite} (see \fullref{sec:Prime Requisite}). This adjustment is applied to all experience points that the character gains (with the exception of experience gain or lost by spells or effects).

There are a variety of things that characters can be given experience points for:

\textbf{Treasure:} By far the most experience that a character normally gets is by gaining treasure. For each 1 gp value of treasure gained, 1 XP is gained.

Treasure from the following sources should always count as experience:

\begin{itemize}
	\item{Treasure found while adventuring.}
	\item{Money paid as “rewards” or other payment for missions or adventures.}
	\item{Money stolen from monsters or NPCs.}
	\item{Monthly income from a dominion.}
	\item{Income from the sale of gems, jewelery and other valuable items and goods (except magic items) found while adventuring.}
\end{itemize}

The following sources of income should never count experience:

\begin{itemize}
	\item{Money given to the character by (or stolen from) other party members.}
	\item{Income from plying a mundane trade.}
	\item{Income from the sale of magic items.}
\end{itemize}

In other cases, experience may or may not be awarded for money gained at the discretion of the Game Master.

Note that experience points gained for treasure are individual in nature. If an adventuring party finds treasure collectively, experience is gained by each party member separately based on how much they individually receive when the money is shared out (which may or may not be equal to the amount that other party members receive, depending on how the party decides to share out treasure and magic item(s).

\textbf{Monsters:} The party collectively gain experience for every monster that is defeated over the course of an adventure. This does not necessarily mean that the monsters have to be killed. Driving off a monster or forcing it to surrender still counts as defeating it.

Monsters can sometimes be evaded, or successfully dealt with diplomatically, avoiding a fight altogether; or sometimes they can be “defeated” in other ways such as by the party solving a riddle that the monster poses.

Only monsters that were a potential threat to the party should be worth experience for defeating. The party should not be given experience for “defeating” the high priestess of the local temple just because they persuaded her to heal a party member, for example.

See \fullref{chap:Monsters} for details of how many experience points each monster is worth.

Unless the adventure is some kind of special solo side quest for an individual character, experience gained for defeating monsters should be shared equally between all party members even if not all took an equal part in the fight (or even if not all party members were present and conscious for every fight).

It is usually most convenient to give experience for monsters in a single lump sum at the end of an encounter or series of encounters that take place within a single day of game time.

\textbf{Achieving Plot Goals:} If the party are taking part in an ongoing plot, they may be given bonus experience for achieving goals that move the plot towards conclusion, at the Game Master’s discretion.

\textbf{Other:} At the Game Master’s discretion, characters can be given other miscellaneous experience point awards for such things as good roleplaying, humor, or even being the only person to remember to bring dice! Whether or not this type of miscellaneous award is used (and how often) will depend heavily on the tone of the campaign.

\section{Gaining Levels}\index[general]{Gaining Levels}\label{sec:Gaining Levels}
The adventuring careers of player characters are split up into levels. Each character normally starts at level one, which means that they are inexperienced and have never adventured before. 

Every class has experience totals needed for each level of experience from 1 to 36 listed in their class table in \fullref{chap:Classes}. When a character gains enough experience points such that their experience total is equal to or higher than that needed for the next level, their level will increase.

An increase in level is accompanied by an increase in the character’s abilities and attributes, showing that the character is now more experienced and becoming more capable in their chosen adventuring profession.

The level increase will not happen immediately (such as in the middle of a fight, for example). It will happen the next time the character rests for the night and has chance to dwell on the experiences of the day.

The following morning, the character will have all the abilities of their newly acquired level; including extra hit points and possibly extra spells per day.

Weapon feats are not automatically gained, only the empty weapon feat slots are gained. In order to fill those available slots, characters must train separately.

Similarly, although all spellcasters gain extra spells per day, wizards and elves do not automatically learn new spells when increasing in level. Any new daily spell slots for spell levels in which the elf or wizard does not actually know any spells are useless until the character learns at least one spell of that level.
\end{multicols*}

