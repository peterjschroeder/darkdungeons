\chapter[red]{Skills}
\label{chap:Skills}
\chapterimage[Skills (Bordered)]
\thispagestyle{plain}

\begin{multicols*}{2}
Characters of different classes have different unique abilities, for example wizards can cast spells and rogues can pick pockets.

However, there are some things—such as trying to walk along a thin ledge or trying to remember which type of dragon breathes fire and which type breathes frost before venturing into a lair—that any character can attempt.

These situations are handled by skills.

\section{Using Skills}
All characters start with four skill points at \nth{1} level plus an extra skill point per point of \iref[sec:Intelligence]{Intelligence} bonus. If a character has an \iref[sec:Intelligence]{Intelligence} penalty, then this does not reduce the number of skill points the character has.

Characters also gain an extra skill point every four experience levels.

These skill points are spent immediately on skills listed in this chapter.

In most cases, each point spent on a particular skill will give your character a +1 bonus to their effective ability score when rolling for ability checks that correspond to the skill.

So a character with a \iref[sec:Strength]{Strength} of 12 would normally need to roll a 12 to make an ability check to use his \iref[sec:Strength]{Strength} to swim upstream in a river but if the character’s player has spent two points in the \iref[skill:Swimming]{Swimming} skill then that character’s \iref[sec:Strength]{Strength} is effectively 14 when making the check and their player will only need to roll a 14 or less.

Such skill modifiers stack with modifiers applied by the Game Master due to circumstances.

\example{Black Leaf is hot on the trail of the Eye of Harmaz, and comes across a narrow rope bridge spanning a chasm. Unfortunately the guide rope for the rope bridge is missing, so Black Leaf will have to try to balance without its aid if she wishes to cross. The Game Master tells Marcie that she will need to make a Balance check (an ability check against Dexterity, using the Balance skill) in order to cross the bridge—but that the strong winds blowing through the chasm will give her a -1 penalty to her effective Dexterity.

Black Leaf has a Dexterity of 16, and Marcie has spent two points on the Balance skill, so she can add these points to her effective Dexterity for the roll.

The two skill points and the wind penalty stack, so overall Marcie is going to have to roll against an effective Dexterity of (16+2-1=17). Marcie reckons that this is well worth trying, but tells the Game Master that Black Leaf is looping a rope around the bridge and around her waist so that if she falls off she won’t plummet to her almost certain death at the bottom of the chasm.

She also trails a rope behind her so that after she’s crossed she and the other companions can tie it off so that the less dexterous party members will have a guide rope and not need to cross the bridge the hard way.}

\section{Social Skills}
Special care needs to be used when using skills designed for social situations (\iref[skill:Bluff]{Bluff}, \iref[skill:Diplomacy]{Diplomacy}, \iref[skill:Intimidation]{Intimidation}, and \iref[skill:Sense Motive]{Sense Motive}).

Players and Game Masters should discuss the use of these skills before the game starts, since they have the potential to unbalance games.

Firstly, it is recommended that these skills are not used against players. If a player wishes to influence another player then this should be role played rather than rolled for using skills. Many players don’t like the loss of “free will” and the loss of control that they feel when their character is forced into particular behavior by dice rolls rather than because they wanted their character to behave in that way, especially if the forced behavior is the result of another player’s use of social skills against them. This can be very disruptive to your gaming group.

Secondly, the players and Game Master should agree what proportion of social interaction with NPCs should be governed by skill usage and what proportion should be governed by role play. Some people prefer more skill usage since it means that people can play silver-tongued characters even if they are not good talkers themselves. Others feel that simply rolling a \iref[skill:Diplomacy]{Diplomacy} check in order to find out whether the character can talk the king into pardoning their wrongly-imprisoned associates is something of a dramatic let-down and prefer to role play the issue to its conclusion. There is no single “correct” way to play using these skills, only the way that your group enjoys.

\section{Special Skills}
While most skills simply add bonuses to ability checks, some—notably language skills—work in other ways.

If a skill works in an unusual way, this will be detailed in the skill’s individual description.

\section{Adding New Skills}
The list of skills in this chapter is not meant to cover all possible situations. Game Masters may require (or players may ask for) ability checks for a variety of situations other than those given here. If a particular situation crops up repeatedly, a player may—with the Game Master’s permission—spend a skill point in order to buy it as a named skill in order to get a bonus to their effective ability score in that situation.

However, Game Masters and players should be careful that this does not overlap with class abilities. For example, sneaking up on someone without being heard is a specific \iref[class:Rogue]{Rogue} ability called \iref[sec:Move Silently]{Move Silently}. While in some specific circumstances it may be acceptable for players without this ability to make a \iref[sec:Dexterity]{Dexterity} check in order to have their character sneak up on someone, this should not become standard practice (and players certainly shouldn’t start spending points on a “Stealth” skill to help with it) since that will overlap with—and undermine—the \iref[sec:Move Silently]{Move Silently} ability.

\section{Alphabetical Skill List}
\begin {table}[H]
  \caption{Alphabetical Skill List}
  \begin{tabularx}{\columnwidth}{>{\bfseries}YY}
	\thead{Skill} & \thead{Ability Check}\\
	Arcane Lore & Intelligence\\
	Balance & Dexterity\\
	Bluff & Charisma\\
	Cooking & Wisdom\\
	Craft (Choose Medium) & Dexterity\\
	Diplomacy & Charisma\\
	Disguise & Charisma\\
	Engineering & Intelligence\\
	Escape Artist & Dexterity\\
	Etiquette (Choose Culture) & Charisma\\
	First Aid & Wisdom\\
	Gambling & Charisma\\
	Geography & Intelligence\\
	History & Intelligence\\
	Intimidation & Strength or Charisma\\
	Jumping & Dexterity\\
	Language (Choose) & -\\
	Laws (Choose Culture) & Intelligence\\
	Lip Reading & Wisdom\\
	Magical Engineering & Intelligence\\
	Monster Empathy & Wisdom\\
	Nature Lore & Intelligence\\
	Navigating & Wisdom\\
	Performance (Choose Medium) & Charisma\\
	Religious Lore & Intelligence\\
	Riding (Choose Animal) & Dexterity\\
	Sense Motive & Wisdom\\
	Snares & Intelligence\\
	Swimming & Strength\\
	Tracking & Wisdom
  \end {tabularx}
\end {table}

\subsection{Arcane Lore}\index[general]{Arcane Lore}\label{skill:Arcane Lore}
Each point spent on this skill gives a +1 bonus to \iref[sec:Intelligence]{Intelligence} checks made to recognize spells, magical effects, and magical creatures.

An “average” spell or effect (at the Game Master’s discretion) will give no bonus or penalty to the effective \iref[sec:Intelligence]{Intelligence} used for the roll, but a particularly common spell might give a bonus and a particularly rare spell might give a penalty. Relative obscurity is more important in this regard than level of power.

\example[Note]{Arcane Lore checks should not be used to allow the players to identify magic items without using the \iref[spell:Analyze]{Analyze} spell.}

\subsection{Balance}\index[general]{Balance}\label{skill:Balance}
Each point spent on this skill gives a +1 bonus to \iref[sec:Dexterity]{Dexterity} checks made to keep one’s footing on a small (or moving) surface or to cross narrow ledges, beams or even tightropes.

Modifiers to the effective \iref[sec:Dexterity]{Dexterity} score used for the roll can come from a wide variety of factors. Examples include:

\begin{itemize}
	\item{Strong wind}
	\item{Slippery surfaces}
	\item{Heavily encumbered character}
	\item{Trying to move quicker (or slower) than a normal walk}
	\item{Particularly narrow (or wide) ledges}
	\item{Trying to balance while dodging attacks}
	\item{Using a pole for balance}
\end{itemize}

\subsection{Bluff}\index[general]{Bluff}\label{skill:Bluff}
Each point spent on this skill gives a +1 bonus to \iref[sec:Charisma]{Charisma} checks made to convince NPCs of things without evidence. While the skill is most often used to convince NPCs of untruths, it can also be used to make a convincing emotional argument in favor of something you know is true but cannot prove to be true.

Bluff checks should get modifiers for both the plausibility of what is being claimed and the potential consequences of the bluff to the target of the skill. A guard who may get executed if they let a potential assassin into the royal palace will be harder to bluff your way past than a merchant who may make a bit less profit if you manage to convince him that you should get a discount because you’re a member of the city watch.

\example[Note]{See Social Skills for a warning about how some uses of this skill may disrupt the game.}

\subsection{Cooking}\index[general]{Cooking}\label{skill:Cooking}
Each point spent on this skill gives a +1 bonus to \iref[sec:Wisdom]{Wisdom} checks made to cook.

Modifiers to the effective \iref[sec:Wisdom]{Wisdom} used for this roll are likely to be only rarely needed, except in the case of not having adequate equipment or when using ingredients that the character has never cooked before.

In most cases, failing a cooking check won’t result in inedible food; merely food that is not as nice as that produced by a successful cooking check.

\subsection{Craft}\index[general]{Craft}\label{skill:Craft}
This skill is not a single skill. It is a group of related skills used when making things of different types. When spending a skill point on this skill, you must specify what sort of craft your character is skilled at.

An exhaustive list of possible craft skills is not possible, but examples include: carpentry, smithing, fletching, skinning, leatherworking, rope binding, tailoring, gem cutting, forgery, masonry, thatching, drawing/painting, sculpture, and machine building.

Each skill point spent on a specific craft skill gives a +1 bonus to \iref[sec:Dexterity]{Dexterity} checks used to make items with that craft.

Modifiers to the effective \iref[sec:Dexterity]{Dexterity} used in craft rolls can come from high or low quality materials and tools, as well as time constraints.

Depending on what is being made, success may be not be an all-or-nothing affair—a failed craft check is likely to still result in a finished item. However, the quality of the finished product should be subjectively determined by the amount that the craft check succeeded by or failed by.

Craft skills can also be used to assess the workmanship of items made using that skill. For example skill at carpentry can be used to help find the weak spot in a door, or skill at smithing can help judge how good a sword is. The exact details of this are left to the Game Master’s discretion. However, in no case should a craft skill be able to be used to determine magical properties of an item.

\subsection{Diplomacy}\index[general]{Diplomacy}\label{skill:Diplomacy}
Each point spent on this skill gives a +1 bonus to \iref[sec:Charisma]{Charisma} checks made to work out compromises and calm tensions.

Diplomacy checks should get modifiers for both the amount of hostility between the parties and the amount that either side has to lose if the diplomacy fails. For example, trying to persuade the leader of a marauding band of orcs not to attack a small and lightly armed party is going to be much more difficult than trying to persuade an offended landlady that you shouldn’t be thrown out of her inn and you should be allowed to spend your spare cash there instead.

\example[Note]{See Social Skills for a warning about how some uses of this skill may disrupt the game.}

\subsection{Disguise}\index[general]{Disguise}\label{skill:Disguise}
Each point spent on this skill gives a +1 bonus to \iref[sec:Charisma]{Charisma} checks made to disguise a character as someone else. 

These checks should be made when the disguise is first worn; with modifiers based on the answers to the following:

\begin{itemize}
	\item{Is the disguise meant to look like a specific person, or merely not look like the wearer?}
	\item{Is the disguise intended to make the character look like a different gender and/or race?}
	\item{Is the disguise meant to only be seen in from a distance or is it intended for close scrutiny?}
	\item{Is the disguise intended to fool close acquaintances of subject (or wearer) of the disguise?}
	\item{Does the person applying the disguise have access to make up and prosthetics?}
\end{itemize}

The exact value of these modifiers should be determined by the Game Master on a case by case basis, and the Game Master should roll the dice for the check so that the player doesn’t know whether their character’s disguise will successfully fool people.

A disguise that fails to achieve its target intentions may still work to a lesser extent.

\example{Black Leaf wishes to gain access to the back room of the cartographer’s guild in order to steal a map that she thinks may be there despite the guild members’ denials. While her allies “accidentally” encounter one of the guild members in an inn and get him drunk, Black Leaf disguises herself as the guild member. 

The Game Master decides that because Black Leaf is trying to impersonate a specific person of the opposite gender, she should get a -4 penalty to her effective Charisma when making the disguise check.

However, since she has deliberately chosen to impersonate the guild member who most closely resembles her in terms of build and facial features (gender notwithstanding) the Game Master reduces the penalty to a –3.

Unfortunately Black Leaf’s Charisma is only a rather average 11, and the penalty means it is effectively only an 8 for this roll. The Game Master rolls a 10.

Since she only failed by a small amount—and would have succeeded if not for the penalties— the Game Master decides that Black Leaf’s disguise is not going to fool people into thinking she is the guild member unless seen only from a distance, but is enough that people seeing her will not recognize her real identity.}

\subsection{Engineering}\index[general]{Engineering}\label{skill:Engineering}
Each point spent on this skill gives a +1 bonus to \iref[sec:Intelligence]{Intelligence} checks made to design machinery or identify the function of existing machinery such as siege weaponry or orreries.

Modifiers to these the effective \iref[sec:Intelligence]{Intelligence} used for checks to identify machinery should come from the complexity of the machinery being examined.

In terms of the design of machinery, the Game Master should also take into account the technological level of the campaign and should veto the “invention” of devices that rely on principles unknown in the game world. Similarly, the presence of magic in the game world may mean that the laws of nature work differently there than in the real world, so at the Game Master’s discretion machinery relying on certain physical principles (e.g. combustion powered guns or engines) may not work at all.

\example[Note]{Although this skill can be used to identify how a large mechanical trap works, it should not be used as a replacement for either the Find Traps or Remove Traps abilities. At best, knowing how the trap functions may give insight into how the trap may be bypassed by mundane means.}

\subsection{Escape Artist}\index[general]{Escape Artist}\label{skill:Escape Artist}
Each point spent on this skill gives a +1 bonus to \iref[sec:Dexterity]{Dexterity} checks made to escape from bonds or ties.

The effective \iref[sec:Dexterity]{Dexterity} used for this check should be modified by the extent to which the character is tied. Escaping from simply having ones hands tied with a scarf should be much easier than escaping from iron manacles.

\example[Note]{Although this skill allows the character to wriggle free from locked manacles or padlocked chains, it does not allow the actual picking of those locks.}

\subsection{Etiquette}\index[general]{Etiquette}\label{skill:Etiquette}
This skill is not a single skill. It is a group of related skills used in social situations in a variety of cultures.

Each point spent in the etiquette skill for a particular culture gives a +1 bonus to \iref[sec:Charisma]{Charisma} rolls used to behave properly in formal social situations in that culture.

Modifiers to the effective \iref[sec:Charisma]{Charisma} used for these checks should come from particularly common or particularly obscure social situations within that culture.

\subsection{First Aid}\index[general]{First Aid}\label{skill:First Aid}
Each point spent in this skill gives a +1 bonus on \iref[sec:Wisdom]{Wisdom} rolls used to treat injuries.

The most common use of this skill is to prevent people from dying once they have reached 0 hit points (see \fullref{sec:Dying and Death}). This use of the skill can be performed during combat, and can be used multiple times until it succeeds.

The other use of the skill is to patch people up after they have taken damage. After each time a character has been injured (a whole combat only counts as a single injury for this purpose) they can be healed 1d3 of the hit points lost in this particular injury by someone making a successful first aid check. Each character using the skill is only able to make one attempt per injury, and if this first attempt fails then further attempts will not succeed.

If someone succeeds with a first aid check against a particular injury then further first aid checks will be of no benefit.

There are normally no modifiers to the effective \iref[sec:Wisdom]{Wisdom} used in first aid checks.

\example{Black Leaf is unlucky enough to fall down a pit, and is injured, taking 5 points of damage out of her 9 hit points, leaving her on 4 hit points.

She uses the First Aid skill to treat this injury, and succeeds in her roll. Her player rolls 1d3 to see how many hit points are recovered and unfortunately only gets a 1. Black Leaf now recovers one of the hit points lost in the injury, taking her from 4 hit points to 5 hit points.

Since she has now had this injury treated, further first aid rolls won’t be able to recover any more of the lost hit points.

Shortly afterwords, the party encounters some goblins spoiling for a fight. After a brief combat with the goblins, Black Leaf has lost another 2 hit points, leaving her with 3. While the party are resting, she tries to use the First Aid skill on herself in order to recover some of these hit points. The skill check is not successful and Black Leaf is unable to heal this injury, and cannot try again.

Seeing that her friend is still injured, Elfstar tries to use the skill on her. Elfstar’s player succeeds her skill check and rolls 1d3 to see how much is healed, getting a 3. Although Black Leaf has lost a total of 6 hit points, only 2 were lost in the injury being treated by the skill roll, so the first aid can only result in the recovery of these 2 hit points.}

\subsection{Gambling}\index[general]{Gambling}\label{skill:Gambling}
Each point spent on this skill gives a +1 bonus to \iref[sec:Charisma]{Charisma} checks made to determine who wins in games of skill and chance.

Modifiers to the effective \iref[sec:Charisma]{Charisma} used for gambling checks should be rare, but may be used for games that the character is unfamiliar with.

The simple way to use this skill is to simulate a single game or round in a game, by each character making a gambling check, and the game or round is won by the character who succeeds their roll by the biggest margin (or fails by the least margin, if no character succeeds).

At the Game Master’s discretion, individual gambling games can be devised for their game world, which may operate on a more complex basis.

\example[Note]{This skill is designed for games where psychology, bluffing, and second-guessing opponents are an essential part of the game play. The skill does not represent a supernatural “luck” type of ability and should not be used to determine the outcome of games of pure chance.}

\subsection{Geography}\index[general]{Geography}\label{skill:Geography}
Each point spent on this skill gives a +1 bonus to \iref[sec:Intelligence]{Intelligence} checks made to recollect information about countries and regions of the game world.

The use of this skill should be modified by the obscurity of the facts that the player wishes their character to recollect. Knowing the name of a country and the majority race that lives there should be easier than knowing the name of the palace in which the king of that country lives or knowing the major trade imports and exports of that country.

\subsection{History}\index[general]{History}\label{skill:History}
Each point spent on this skill gives a +1 bonus to \iref[sec:Intelligence]{Intelligence} checks made to recollect information about the game world’s past.

The use of this skill should be modified by the obscurity of the facts that the player wishes their character to recollect. Knowing the name of countries involved in a war a couple of generations ago should be easier than knowing the name of the vizier of an empire that hasn’t existed for thousands of years.

\subsection{Intimidation}\index[general]{Intimidation}\label{skill:Intimidation}
Each point spent on this skill gives a +1 bonus to both \iref[sec:Charisma]{Charisma} checks and \iref[sec:Strength]{Strength} checks made to bully an NPC into co-operation through threats or shows of physical strength.

Intimidation checks should get modifiers for both the plausibility of the threats being made and the potential consequences to the target of the skill for co-operation.

Using a threat of immediate physical violence to get a bandit to flee is both a realistic threat and has only minor consequences to the bandit if they do flee; so it should be easier than using a threat of sending a dragon to kill the family of an ogre if they don’t betray their tribe’s location to you, which is both an unrealistic threat and has potentially fatal consequences to the ogre and their tribe.

\example[Note]{See Social Skills for a warning about how some uses of this skill may disrupt the game.}

\subsection{Jumping}\index[general]{Jumping}\label{skill:Jumping}
Each point spent on this skill gives a +1 bonus to \iref[sec:Strength]{Strength} checks made to jump long distances.

Rather than giving modifiers to the effective \iref[sec:Strength]{Strength} based on the intended length or height of the jump, the nature of jumping means that a jump will always be successfully made but the distance of the jump will depend on the result of the check.

With a running start, a character can jump 10’ horizontally or 4’ vertically (remember that if the character is jumping up to reach something then the character’s height should be added to the vertical distance jumped in order to determine how high they can reach).

The effects of the roll depend on whether the character is jumping for height or length.

For each point that the character succeeds the jump check by in the case of a long jump, they jump an extra foot horizontally; for each two points that the character fails the jump check by they jump a foot less horizontally.

In the case of a high jump, for each two points the character succeeds the jump roll by they jump a foot higher, and for each four points the character fails by they jump a foot less high.

In either case, jumping from a standing start means that only half of the total distance (after modification) can be jumped. 

\example{Black Leaf has a Strength of only 9, but has spent two skill points on the jump skill—so she has an effective Strength of 11 when making jump checks.

After having fallen into a 10-foot-deep pit, she needs to try to get out. Her player announces that she will try to jump up and grab the edge of the pit.

The pit is 10 feet deep, and Black Leaf is 5’5” tall; so in order to reach the edge of the pit she will have to jump just over four and a half feet.

Luckily for Black Leaf, the pit is 20 feet wide, so she has chance to get a run up at the jump. Marcie rolls a 6, which is five points better than the 11 that she needed to roll, so Black Leaf jumps two and a half feet more than the default 4 feet, for a total of six and a half feet— more than enough to grab the edge of the pit and pull herself out.

Had Black Leaf not been able to take a run up, the total distance jumped would have been halved from six and a half feet down to just over three feet—not high enough to get out.}

\subsection{Language}\index[general]{Language}\label{skill:Language}
Each skill point spent on this skill means that the character knows another language to an acceptable level that they can converse. However, even though the character speaks the language, their speech will be accented and they will be clearly noticeable as a foreigner.

If a character spends a second skill point on the same language, their skill increases to the point where they speak it like a native and no longer have a noticeable foreign accent.

The Game Master will determine which languages exist in their game world.

\subsection{Laws}\index[general]{Laws}\label{skill:Laws}
This skill is not a single skill. It is a group of related skills used in legal situations in a variety of cultures.

Each point spent in the law skill for a particular culture gives a +1 bonus to \iref[sec:Intelligence]{Intelligence} checks used to recall/interpret the laws and customs of that country.

Modifiers to the effective \iref[sec:Charisma]{Charisma} used for these checks should come from particularly common or particularly obscure customs within that culture. For example knowing that a country has the death penalty for banditry is much easier than knowing that the country requires anyone setting up a market stall to buy a trading license from the alderman of the town.

\subsection{Lip Reading}\index[general]{Lip Reading}\label{skill:Lip Reading}
Each point spent on this skill gives a +1 bonus to \iref[sec:Wisdom]{Wisdom} checks made to understand what someone is saying even when you can’t hear them by watching the movement of their mouth.

Modifiers to the effective \iref[sec:Wisdom]{Wisdom} used for this skill should be used based on how clearly the person’s mouth can be seen, whether they are deliberately enunciating clearly in order to make it easy for you to lip read, and whether you are a native speaker of the language they are speaking (i.e. it’s a starting language or you spent two skill points on it) or whether you are merely fluent in it (i.e. you spent one skill point on it).

\subsection{Magical Engineering}\index[general]{Magical Engineering}\label{skill:Magical Engineering}
Each point spent on this skill gives a +1 bonus to \iref[sec:Intelligence]{Intelligence} checks made to design large scale magical effects or identify the function of existing effects such as wormholes, gates, and some large scale magical traps.

Modifiers to these the effective \iref[sec:Intelligence]{Intelligence} used for checks to identify such magic should come from the obscurity of the effect being examined. Something that uses a standard spell effect should be easier than something that uses a unique effect.

When designing such effects, this skill is used for the design itself, but the building of large-scale magical constructions will involve spellcasting and crafting skills.

\example[Note]{Although this skill can be used to identify how a large magical trap works, it should not be used as a replacement for either the Find Traps or Remove Traps class abilities. At best, knowing how the trap functions may give insight into how the trap may be bypassed by mundane means.}

\subsection{Monster Empathy}\index[general]{Monster Empathy}\label{skill:Monster Empathy}
This skill is not a single skill. It is a group of related skills used when empathizing with different types of monsters. When spending a skill point on this skill, you must specify what sort of monster you empathize with. Only non-intelligence monsters may be chosen.

Each skill point spent on a specific type of monster gives a +1 bonus to \iref[sec:Wisdom]{Wisdom} checks used to sense and communicate basic feelings with them when they are within 100 feet. 

If the monster's hit dice is higher than the character's level, than the difference is applied to the skill check as a penalty.

If the monster is the character's mount this skill applies a +2 bonus to the character's \iref[skill:Riding]{Riding} skill check.

\subsection{Nature Lore}\index[general]{Nature Lore}\label{skill:Nature Lore}
Each point spent on this skill gives a +1 bonus to \iref[sec:Intelligence]{Intelligence} checks made to recognize plants, natural terrain features, and natural creatures.

Most natural creatures or terrain features will give no bonus or penalty to the effective \iref[sec:Intelligence]{Intelligence} used for the roll, but a particularly common plant or animal might give a bonus and a particularly rare one might give a penalty.

\subsection{Navigating}\index[general]{Navigating}\label{skill:Navigating}
Each point spent in this skill gives a +1 bonus on \iref[sec:Wisdom]{Wisdom} checks used to avoid getting lost by using landmarks and the position of the sun and stars.

The effective \iref[sec:Wisdom]{Wisdom} used for navigating checks may be modified by factors such as weather conditions, visibility, access (or lack of it) to equipment such as maps and/or sextants, and familiarity with the area being traveled.

\subsection{Performance}\index[general]{Performance}\label{skill:Performance}
This skill is not a single skill. It is a group of related skills used when putting on different types of artistic performance. When spending a skill point on this skill, you must specify what sort of artistic performance your character is skilled at. An exhaustive list of possible artistic media is not possible, but examples include: singing, musical instrument, dancing, acting, juggling, jesting, storytelling, and poetry.

Each skill point spent on a specific craft skill gives a +1 bonus to both \iref[sec:Dexterity]{Dexterity} checks and \iref[sec:Charisma]{Charisma} checks used to make these artistic performances.

Modifiers to the effective \iref[sec:Dexterity]{Dexterity} or \iref[sec:Charisma]{Charisma} used in performance rolls can come from such things as particularly high or low quality costume and how much rehearsal time the performance has had.

Success in a performance may not be an all-or-nothing affair—a failed performance check is likely to still result in a performance of some kind. However, the quality of the finished work should be subjectively determined by the amount that the performance check succeeded by or failed by.

\subsection{Religious Lore}\index[general]{Religious Lore}\label{skill:Religious Lore}
Each point spent on this skill gives a +1 bonus to \iref[sec:Intelligence]{Intelligence} checks made to recall knowledge about religions, their targets of worship, and the creatures associated with them.

The effective \iref[sec:Intelligence]{Intelligence} score used for the religious lore check should be modified by the obscurity of the knowledge being sought.

Identifying a statue of a well known god should be much easier than identifying the purpose of a particular religious rite from a foreign religion.

\example[Note]{Characters should be assumed to know all about the rites and details their own religion without needing to make religious lore checks.}

\subsection{Riding}\index[general]{Riding}\label{skill:Riding}
This skill is not a single skill. It is a group of related skills used when making rolls to control or stay on various riding animals in unusual circumstances. When spending a skill point on this skill, you must specify what sort of animal your character is skilled at riding. An exhaustive list of animals is not possible, but examples include: Horse (includes mules and donkeys), \iref[monster:Camel]{Camel}, \iref[monster:Elephant]{Elephant}, \iref[monster:Dire Wolf]{Dire Wolf}, \iref[monster:Griffon]{Griffon}, \iref[monster:Pegasus]{Pegasus}, \iref[monster:Hippogriff]{Hippogriff}, and \iref[monster:Giant/Gargantuan Monster]{Gargantuan} \iref[monster:Bird of Prey]{Bird of Prey}.

Each skill point spent on a specific riding skill gives a +1 bonus to \iref[sec:Dexterity]{Dexterity} checks used to ride that type of animal.

Generally, simply riding a calm animal as it walks shouldn’t require a riding check. However, staying on an animal when it bolts or stumbles, or when it is damaged by an attack, should require a check—with the effective \iref[sec:Dexterity]{Dexterity} modified by the exact circumstances provoking the check.

\subsection{Sense Motive}\index[general]{Sense Motive}\label{skill:Sense Motive}
Each point spent on this skill gives a +1 bonus to \iref[sec:Wisdom]{Wisdom} checks made to read the expressions and body language of NPCs in order to tell if they have an ulterior motive for their speech or behavior. The skill is most often used to detect when an NPC is lying, although it can also in other situations, for example to try to tell if an NPC who is ignoring a character who is hiding is deliberately ignoring that character while trying to keep an eye on them or whether they simply haven’t seen them.

\example[Note]{See Social Skills for a warning about how some uses of this skill may disrupt the game.}

\subsection{Snares}\index[general]{Snares}\label{skill:Snares}
Each point spent on this skill gives a +1 bonus to \iref[sec:Intelligence]{Intelligence} checks made to building traps designed to snare animals, monsters, and intruders. A successful check indicates that the trap functions correctly. The Game Master may assign modifiers to the check depending on the availability of time, materials, etc.

\subsection{Swimming}\index[general]{Swimming}\label{skill:Swimming}
Each point spent on this skill gives a +1 bonus to \iref[sec:Strength]{Strength} checks made to swim while weighed down or in fast flowing or turbulent waters. Characters are assumed to be able to swim in calm water (providing they are not weighed down) without needing to make a swimming check, but heavily encumbered characters or characters trying to swim in difficult conditions will need to make such checks.

The effective \iref[sec:Strength]{Strength} used for swimming checks should be given modifiers based on the specific conditions that the character is trying to swim in, such as those listed above.

\subsection{Tracking}\index[general]{Tracking}\label{skill:Tracking}
Each point spent on this skill gives a +1 bonus to \iref[sec:Wisdom]{Wisdom} checks made to follow the tracks left by one of more creatures. Following (at a slow walk) a single human-sized creature who walked through light woodland or farmland less than a day ago would require a roll with no modifiers.

The effective \iref[sec:Wisdom]{Wisdom} used for the tracking check should be modified by many factors such as:

\begin{itemize}
	\item{Number of creatures being tracked}
	\item{Size of creatures being tracked}
	\item{Method of locomotion of creatures being tracked}
	\item{Age of the tracks}
	\item{Terrain being tracked through}
	\item{Weather conditions}
	\item{Tracking faster than a slow walk}
\end{itemize}
\end{multicols*}

