\chapter[red]{Equipment}
\label{chap:Equipment}
\chapterimage[Equipment (Bordered)]
\thispagestyle{plain}

\begin{multicols*}{2}
They say that money makes the world go around, and that applies to the Dark Dungeons world just as much as it applies to the real world. Trade and commerce are the backbone of civilization, and even brave adventurers need to buy things—from food to armor to ships or even siege artillery.

\section{Money}\index[general]{Money}
The actual type of money that people use, in terms of the names and denominations of coins, will vary from campaign setting to campaign setting and probably from country to country within an individual campaign setting.

However, for simplicity’s sake and consistency’s sake, the Dark Dungeons rules assume a simple gold standard. The standard coin that most prices are measured in is the gold piece (usually abbreviated to simply gp). The value of a gold piece is—quite literally—its weight in gold, rather than any artificial economic value. Complex economic factors such as exchange rates are assumed not to exist.

A gold piece is a large (by real-world standards) gold coin approximately 3cm in diameter and 1/3cm in thickness. This amount of gold weighs 1.6 ounces, or 1/10 of a pound. Therefore, a pound of gold is worth 10 gp, and 10 gp weighs a pound in weight.

Other coins are made from a similar weight of different metals—namely copper, silver, electrum (a natural alloy of gold and silver) and platinum. The relative rarity of these metals gives the different coins their different values relative to gold as indicated on \fullref{tab:Money}.

\begin {table}[H]
	\caption{Money}\label{tab:Money}
	\begin{tabularx}{\columnwidth}{>{\bfseries}YY}
		\thead{Coin} & \thead{Gold Value} \\
		1 cp (copper piece) & 1/100 gp \\
		1 sp (silver piece) & 1/10 gp \\
		1 ep (electrum piece) & 1/2 gp \\
		1 pp (platinum piece) & 5 gp \
  \end {tabularx}
\end {table}

\section{Encumbrance and Weight}\index[general]{Encumbrance and Weight}\label{sec:Encumbrance and Weight}
Just as the gold coin is the standard for prices, the weight of a gold coin—1/10 of a pound—is used as the standard for weights. The weights of most items are measured in coins (usually abbreviated to cn). A character can normally carry only 400 cn without being slowed by the weight. Anything heavier reduces the character’s movement as shown on \fullref{tab:Encumbrance and Weight}.

\begin {table}[H]
	\caption{Encumbrance and Weight}\label{tab:Encumbrance and Weight}
	\begin{tabularx}{\columnwidth}{>{\bfseries}YY}
		\thead{Coins} & \thead{Movement}\\
		Up to 400 cn & Full\\
		401 cn to 800 cn & 3/4\\
		801 cn to 1,200 cn & 1/2\\
		1,201 cn to 1,600 cn & 1/4\\
		1,601 cn to 2,400 cn & 1/8\\
		2,401 cn and over & Cannot move\
  \end {tabularx}
\end {table}
Armor is counted when calculating the encumbrance of a character, but normal clothing is not counted if it is being worn, only if it is being carried.

\section{Mundane Items}
\begin {table}[H]
  \caption{Mundane Items}
	\begin{tabularx}{\columnwidth}{>{\bfseries}cYY}
		\thead{Item} & \thead{Cost} & \thead{Weight}\\
		Backpack (holds 400 cn) & 5 gp & 20 cn\\
		Belt & 2sp & 5 cn\\
		Boots, Fancy & 5 gp & 15 cn\\
		Boots, Plain & 2 gp & 10 cn\\
		Cloak, Long & 1 gp & 15 cn\\
		Cloak, Short & 5sp & 10 cn\\
		Clothes, Merchant & 5 gp & 20 cn\\
		Clothes, Noble & 20 gp & 20 cn\\
		Clothes, Peasant & 5sp & 20 cn\\
		Clothes, Royal & 50+gp & 30 cn\\
		Garlic & 5 gp & 1 cn\\
		Grappling Hook & 25 gp & 80 cn\\
		Hammer, Small & 2 gp & 10 cn\\
		Hat & 2sp & 3 cn\\
		Holy Symbol & 25 gp & 1 cn\\
		Holy Water (small vial) & 25 gp & 1 cn\\
		Iron Spike & 1sp & 5 cn\\
		Lantern & 1 gp & 30 cn\\
		Mirror, Steel & 5 gp & 5 cn\\
		Oil (flask) & 2 gp & 10 cn\\
		Pole (10 ft.) & 1 gp & 100 cn\\
		Purse (holds 50 cn) & 5sp & 2 cn\\
		Quiver & 1 gp & 5 cn\\
		Rations, Dried (1 week) & 15 gp & 70 cn\\
		Rations, Fresh (1 week) & 5 gp & 200 cn\\
		Red Powder (flask) & 5 gp & 10 cn\\
		Rope (50 ft.) & 1 gp & 50 cn\\
		Sack (holds 200 cn) & 1 gp & 1 cn\\
		Sack (holds 600 cn) & 2 gp & 5 cn\\
		Shoes & 5 sp & 8 cn\\
		Spell Book & 25 gp & 25 cn\\
		Stakes (3) and Mallet & 3 gp & 10 cn\\
		Tent & 20 gp & 200 cn\\
		Rogues’ Tools & 25 gp & 10 cn\\
		Tinder Box & 3 gp & 5 cn\\
		Torch & 2sp & 20 cn\\
		Waterskin, Empty & 1 gp & 5 cn\\
		Waterskin, Full (1 day) & 1 gp & 30 cn\\
		Wine & 1 gp & 30 cn\\
		Wolfsbane & 10 gp & 1 cn\
  \end {tabularx}
\end {table}

\paragraph{Backpack (holds 400 cn)}\index[equipment]{Backpack}\textbf{:} A leather or canvas backpack with shoulder straps for carrying things while leaving the hands free.

\paragraph{Belt}\index[equipment]{Belt}\textbf{:} A sturdy leather belt.

\paragraph{Boots, Plain}\index[equipment]{Plain Boots}\textbf{:} Simple yet sturdy hard leather boots for walking or riding in.

\paragraph{Boots, Fancy}\index[equipment]{Fancy Boots}\textbf{:} Ornate boots, possibly with fold-down tops or designs and patterns on the leather.

\paragraph{Cloak, Short}\index[equipment]{Short Cloak}\textbf{:} A waist length weatherproof outer garment.

\paragraph{Cloak, Long}\index[equipment]{Long Cloak}\textbf{:} A knee or ankle length weatherproof outer garment.

\paragraph{Clothes, Peasant}\index[equipment]{Peasant Clothes}\textbf{:} Simple clothes that a peasant or manual laborer (or even slave) would wear.

\paragraph{Clothes, Merchant}\index[equipment]{Merchant Clothes}\textbf{:} High quality clothes that a middle-class artisan or merchant would wear.

\paragraph{Clothes, Noble}\index[equipment]{Noble Clothes}\textbf{:} Fancy clothing that a minor noble or other rich person would wear.

\paragraph{Clothes, Royal}\index[equipment]{Royal Clothes}\textbf{:} Extravagant and ostentatious clothing fit for a king or even an emperor.

\paragraph{Garlic}\index[equipment]{Garlic}\textbf{:} A pungent flavored herb used for cooking or repelling vampires.

\paragraph{Grappling Hook}\index[equipment]{Grappling Hook}\textbf{:} A large iron hook that is tied to the end of a rope and then swung over a target such as a wall or tree branch. The hooks catch on the target and support the weight of someone climbing the rope.

\paragraph{Hammer, Small}\index[equipment]{Small Hammer}\textbf{:} A working hammer, for hammering nails, spikes or tent pegs.

\paragraph{Hat}\index[equipment]{Hat}\textbf{:} A weatherproof hat made out of waxed linen.

\paragraph{Holy Symbol}\index[equipment]{Holy Symbol}\textbf{:} This is a small symbol, usually metal, that represents an icon of the character’s religion. Most clerics carry holy symbols, although they can operate without one.

\paragraph{Holy Water (small vial)}\index[equipment]{Holy Water (small vial)}\textbf{:} Water that has been especially prepared by a cleric or other priest. It is used in some religious ceremonies and can damage some undead monsters.

\paragraph{Iron Spike}\index[equipment]{Iron Spike}\textbf{:} A 9-inch-long iron nail. These find a multitude of uses, from wedging doors open (or shut) or to use as tent pegs. They are most useful if accompanied by a hammer for driving them into hard surfaces.

\paragraph{Lantern}\index[equipment]{Lantern}\textbf{:} An oil fired lamp that gives of light within a 30-foot radius. A single flask of oil will last 4 hours.

\paragraph{Mirror, Steel}\index[equipment]{Steel Mirror}\textbf{:} Useful for personal grooming, and also for seeing around corners or fighting creatures without looking at them directly. Fighting in such a manner gives a -2 penalty to attack rolls.

\paragraph{Oil (flask)}\index[equipment]{Oil}\textbf{:} Fine grade lantern oil. If a burning wick is inserted, it can also make an emergency missile weapon, doing 1d8 damage to all within a 5-foot radius, with a chance to also set them alight (see \fullref{sec:Environmental Damage} for the dangers of being set alight).

\paragraph{Pole (10 ft.)}\index[equipment]{Pole}\textbf{:} A stout wooden pole often used to probe ahead or to prod suspicious looking piles of refuse.

\paragraph{Purse (holds 50 cn)}\index[equipment]{Purse}\textbf{:} A small leather or cloth purse or pouch that ties to a belt.

\paragraph{Quiver}\index[equipment]{Quiver}\textbf{:} A leather container that holds 20 arrows or crossbow bolts.

\paragraph{Rations, Dried (1 week)}\index[equipment]{Dried Rations}\textbf{:} Enough dried, smoked, or otherwise preserved food to feed a person for a week. The preservation means that this food will last for two months before going bad.

\paragraph{Rations, Fresh (1 week)}\index[equipment]{Fresh Rations}\textbf{:} Enough fresh food to feed a person for a week. The food will go bad after a week.

\paragraph{Red Powder (flask)}\index[equipment]{Red Powder}\label{eq:Red Powder}\textbf{:} Red powder is a naturally occurring magical substance with unusual properties. In large quantities such as in a barrel it is perfectly safe, but in small quantities it becomes dangerously flammable or even explosive. Although it is normally found in large underground deposits with the consistency of talc or other soft rock, it is normally ground up (in a water bath to stop it igniting) and then sold in powdered form. Water temporarily suppresses the explosive properties of red powder, but does not spoil it—it is usable again once it has dried sufficiently.

Red powder is normally used to power guns; a single flask is enough to power 100 gun shots or a single cannon shot. However, in an emergency a full flask can also have a wick or fuse stuck in it and lit, and then be hurled as a grenade like weapon. In this case it will explode for 2d6 damage to all within 10 feet. Targets that can make a saving throw vs. breath weapon take only half damage.

Because of red powder’s unusual self-stabilizing property, multiple flasks cannot be combined in order to make bigger explosions.

\paragraph{Rope (50 ft.)}\index[equipment]{Rope}\textbf{:} A strong hemp or silk rope that can support up to 7,500 cn (approximately three people and their equipment).

\paragraph{Sack (holds 200 cn)}\index[equipment]{Sack (holds 200 cn)}\textbf{:} A canvas sack for either carrying in one hand or loading onto a horse or other beast of burden. Small enough to be tied around a belt.

\paragraph{Sack (holds 600 cn)}\index[equipment]{Sack (holds 600 cn)}\textbf{:} A canvas sack for either carrying in two hands or loading onto a horse or other beast of burden.

\paragraph{Spell Book}\index[equipment]{Spell Book}\textbf{:} A large book for an elf or wizard to write their spells into. A spell book contains 100 pages, and can hold up to 100 levels worth of spells.

\paragraph{Shoes}\index[equipment]{Shoes}\textbf{:} These are normal leather shoes. It is recommended that characters wear some sort of footwear, as the Game Master may apply damage to barefoot characters who walk through rough terrain.

\paragraph{Stakes (3) and Mallet}\index[equipment]{Stakes and Mallet}\textbf{:} Three 18 inch wooden stakes and a wooden mallet.

\paragraph{Tent}\index[equipment]{Tent}\textbf{:} A waxed canvas tent with wooden poles. Although heavy, such tents are invaluable when traveling through inclement weather and cold nights.

\paragraph{Rogues’ Tools}\index[equipment]{Rogues’ Tools}\textbf{:} A variety of lockpicks, needles, wire, pliers, etc. A rogue cannot pick locks or disarm traps without a set of these tools.

\paragraph{Tinder Box}\index[equipment]{Tinder Box}\textbf{:} A set of fire making equipment in a water resistant box. Starting a fire with a tinder box takes 1d3 rounds.

\paragraph{Torch}\index[equipment]{Torch}\textbf{:} A 1-to-2-foot length of wood dipped in pitch or tallow. A torch gives off light in a 30-foot radius and burns for one hour. If used in combat, a torch has the same statistics (and uses the same weapons feats) as a club.

\paragraph{Waterskin}\index[equipment]{Waterskin}\textbf{:} A waterskin is a flexible bladder that can hold up to two pints of water or other liquid, enough for a person for one day.

\paragraph{Wine}\index[equipment]{Wine}\textbf{:} One quart of common wine without a container.

\paragraph{Wolfsbane}\index[equipment]{Wolfsbane}\label{eq:Wolfsbane}\textbf{:} A hood-shaped wildflower used to repel werewolves.

\section{Weapons}
Refer to \fullref{chap:Weapon Feats} for the weapon’s damage, range and abilities.

\begin {table}[H]
  \caption{Weapons}
  \begin{tabularx}{\columnwidth}{>{\bfseries}cYYY}
		\thead{Item} & \thead{Cost} & \thead{Weight} & \thead{Size}\\
		Axe, Battle & 7 gp & 60 cn & Medium\\
		Axe, Hand & 4 gp & 30 cn & Small\\
		Blackjack & 5 gp & 5 cn & Small\\
		Blowgun, Large & 3 gp & 15 cn & Medium\\
		Blowgun, Small & 6 gp & 6 cn & Small\\
		Bolas & 5 gp & 5 cn & Medium\\
		Boomerang & 10 gp & 50 cn & Medium\\
		Bow, Long & 40 gp & 30 cn & Large\\
		Bow, Short & 25 gp & 20 cn & Medium\\
		Cestus & 5 gp & 10 cn & Small\\
		Chakram & 1 gp & 5 cn & Small\\
		Claw, Bagh nakh & 45 gp & 10 cn & Small\\
		Club & 3 gp & 50 cn & Medium\\
		Crossbow, Heavy & 50 gp & 80 cn & Large\\
		Crossbow, Light & 30 gp & 50 cn & Medium\\
		Dagger & 3 gp & 10 cn & Small\\
		Dagger, Silver & 30 gp & 10 cn & Small\\
		Dagger, Stiletto & 3 gp & 5 cn & Small\\
		Gun, Pistol & 250 gp & 20 cn & Small\\
		Gun, Smoothbore & 150 gp & 75 cn & Medium\\
		Halberd & 7 gp & 150 cn & Large\\
		Hammer, Throwing & 4 gp & 25 cn & Medium\\
		Hammer, War & 5 gp & 50 cn & Medium\\
		Javelin & 1 gp & 20 cn & Medium\\
		Lance & 10 gp & 180 cn & Large\\
		Lasso & 5 sp & 30 cn & Medium\\
		Mace & 5 gp & 30 cn & Medium\\
		Net & 4 gp & 40 cn & Medium\\
		Pike & 3 gp & 80 cn & Large\\
		Poleaxe & 5 gp & 120 cn & Large\\
		Shield, Horned & 15 gp & 20 cn & Small\\
		Shield, Knife & 65 gp & 70 cn & Small\\
		Shield, Sword & 200 gp & 185 cn & Medium\\
		Shield, Tusked & 200 gp & 275 cn & Large\\
		Sling & 2 gp & 20 cn & Small\\
		Spear & 3 gp & 30 cn & Large\\
		Staff & 5 gp & 40 cn & Medium\\
		Sword, Bastard & 15 gp & 80 cn & Large\\
		Sword, Normal & 10 gp & 60 cn & Medium\\
		Sword, Rapier & 15 gp & 40 cn & Medium\\
		Sword, Short & 7 gp & 30 cn & Small\\
		Sword, Two-Handed & 15 gp & 100 cn & Large\\
		Swordstick & 10 gp & 20 cn & Small\\
		Thrown Object & Varies & 10 cn & Small\\
		Trident & 5 gp & 25 cn & Medium\\
		Whip & 10 gp & 100 cn & Medium\
  \end {tabularx}
\end {table}

\paragraph{Axe, Battle}\index[equipment]{Battle Axe}\textbf{:} A battle axe is a two handed axe 2-4 feet in length, usually with a double blade or a spike on the reverse of the blade.

\paragraph{Axe, Hand}\index[equipment]{Hand Axe}\textbf{:} A hand axe is a one handed axe 1-2 feet in length, usually with a single blade. A hand axe is often thrown while its owner rushes towards melee range.

\paragraph{Blackjack}\index[equipment]{Blackjack}\textbf{:} A blackjack, also known as a cosh, is a small leather club usually filled with sand. Blackjacks are too soft to do significant damage, but are very useful for temporarily knocking people unconscious without doing permanent damage.

\paragraph{Blowgun, Small}\index[equipment]{Small Blowgun}\textbf{:} Small blowguns are tubes 1-2 feet in length through which tiny darts are blown. Blowgun darts are too small to cause more than a scratch, but are an effective means by which poison can be delivered. Small blowguns can be used in one hand. Blowguns require the use of darts as ammunition.

\paragraph{Blowgun, Large}\index[equipment]{Large Blowgun}\textbf{:} Large blowguns are tubes 2-4 feet in length through which tiny darts are blown. Blowgun darts are too small to cause more than a scratch, but are an effective means by which poison can be delivered. Large blowguns require two hands to fire them. Blowguns require the use of darts as ammunition.

\paragraph{Bolas}\index[equipment]{Bolas}\textbf{:} A bolas is a set of weights (usually three) on the ends of connected ropes 3-4 feet in length. The bolas is whirled around and then thrown at the opponent in order to entangle or even strangle them.

\paragraph{Boomerang}\index[equipment]{Boomerang}\textbf{:} A boomerang is a 14-to-18-inch curved wooden device. When thrown the boomerang returns to the owner if it doesn’t hit anything.

\paragraph{Bow, Long}\index[equipment]{Long Bow}\textbf{:} A long bow is a 4-to-6-foot bow, either made from a single piece of wood or a composite of different woods. A long bow requires both hands to fire. Bows require the use of arrows as ammunition.

\paragraph{Bow, Short}\index[equipment]{Short Bow}\textbf{:} A short bow is a 3-to-4-foot bow, either made from a single piece of wood or a composite of different woods. A short bow requires both hands to fire. Bows require the use of arrows as ammunition.

\paragraph{Cestus}\index[equipment]{Cestus}\textbf{:} A cestus is a spiked or bladed metal band that may either be worn around the hand or built into a gauntlet. Although only a small weapon, it can be used without incurring off-hand penalties.

\paragraph{Chakram}\index[equipment]{Chakram}\textbf{:} A chakram is small steel ring, 5-12 inches in diameter. The outer edge of the ring is very sharp. A chakram can be either thrown or used in melee.

\paragraph{Claw, Bagh nakh}\index[equipment]{Bagh nakh}\textbf{:} A bagh nakh is a claw-like weapon designed to fit over the knuckles. The claws are 2-3 inches in length and are designed to cut through flesh.

\paragraph{Club}\index[equipment]{Club}\textbf{:} A club is a crude blunt weapon—little more than a roughly shaped piece of wood—that can be used in one hand.

\paragraph{Crossbow, Heavy}\index[equipment]{Heavy Crossbow}\textbf{:} A heavy crossbow is a large two-handed missile weapon. It has powerful metal arms and a string that is pulled back using a crank. Crossbows require the use of bolts as ammunition.

\paragraph{Crossbow, Light}\index[equipment]{Light Crossbow}\textbf{:} A light crossbow is a medium-sized missile weapon, although it still needs two hands to wield. It has powerful metal arms and a string that is pulled back using a lever. Crossbows require the use of bolts as ammunition.

\paragraph{Dagger}\index[equipment]{Dagger}\textbf{:} A dagger is a short light blade which is 18 inches long or less. Daggers are popular weapons because their small size makes them easy to conceal and they can be either thrown or used in melee.

\paragraph{Dagger, Stiletto}\index[equipment]{Stiletto}\textbf{:} A stiletto is a dagger with a 8-to-9-inch slender blade, topped with a needle-like point. Like normal daggers, they can also be thrown or used in melee.

\paragraph{Gun, Pistol}\index[equipment]{Pistol}\textbf{:} A pistol is a short gun, 9-12 inches long. It can be fired using only one hand, but needs both hands to reload. Pistols require both bullets as ammunition and \iref[eq:Red Powder]{Red Powder} to fire them.

\paragraph{Gun, Smoothbore}\index[equipment]{Smoothbore}\textbf{:} A smoothbore is a long gun, 3-5 feet long. It requires two hands to both fire and reload. Smoothbores require both bullets as ammunition and \iref[eq:Red Powder]{Red Powder} to fire them.

\paragraph{Halberd}\index[equipment]{Halberd}\textbf{:} A halberd is a large pole-arm 6-8 feet in length, the head of which has a long spike with an axe blade on one side and a hook on the reverse side.

\paragraph{Hammer, Throwing}\index[equipment]{Throwing Hammer}\textbf{:} A throwing hammer is a one handed hammer from 18 inches to 2 feet in length that is weighted for throwing. Despite the name, it can also be used in melee.

\paragraph{Hammer, War}\index[equipment]{War Hammer}\textbf{:} A war hammer is a large one handed hammer, usually 2-3 feet in length. The head of a war hammer is often symmetrical to aid in balance.

\paragraph{Javelin}\index[equipment]{Javelin}\textbf{:} A javelin is a light one-handed throwing spear. Although primarily used as a thrown weapon, it can also be used in melee.

\paragraph{Lance}\index[equipment]{Lance}\textbf{:} A lance is an extremely long spear, 9-12 feet long. It is too unwieldy to use on foot, and can only be used when mounted—in which case it can be used in one hand despite its length.

\paragraph{Lasso}\index[equipment]{Lasso}\textbf{:} A lasso is a coil rope, up to 40 feet long. It has a noose on one end that can be used to entangle targets.

\paragraph{Mace}\index[equipment]{Mace}\textbf{:} A mace is a one handed melee weapon consisting of a 2-to-3-foot shaft with a heavy metal head. The head can be smooth (round or pear shaped), can contain flanges or studs, or can even be sculpted into the shape of a fist or skull or other roundish object.

\paragraph{Net}\index[equipment]{Net}\textbf{:} Nets designed for use in combat are generally 6-9 feet in diameter, and usually have small weights around their edge to hold them open when they are flung. This arrangement lets them be thrown one-handed if held in their center. Nets do no damage to opponents in combat, but are excellent defensive weapons.

\paragraph{Pike}\index[equipment]{Pike}\textbf{:} A pike is an extremely long spear, 7-15 feet long. It can only be wielded using two hands.

\paragraph{Poleaxe}\index[equipment]{Poleaxe}\textbf{:} A poleaxe consists of an axe blade (usually single sided, but occasionally double-sided) with a 6-to-8-foot shaft. It is wielded in both hands like a large two-handed axe.

\paragraph{Shield, Horned}\index[equipment]{Horned Shield}\textbf{:} A horned shield is a small (1 foot) buckler (strap-on shield) with a horn, spike or blade protruding from the center at right angles to the shield. It is primarily used for attacking, and does not provide a normal shield bonus; only an AC bonus based on proficiency level.

\paragraph{Shield, Knife}\index[equipment]{Knife Shield}\textbf{:} A knife shield is a small (1 foot) buckler (strap-on shield) with a knife blade protruding from either side, at right angles to the arm. It is primarily used for attacking, and does not provide a normal shield bonus; only an AC bonus based on proficiency level.

\paragraph{Shield, Sword}\index[equipment]{Sword Shield}\textbf{:} A sword shield is a medium-sized (1 foot) buckler (strap-on shield) with a sword or spear blade protruding from the end, parallel to the wielder’s arm so that it sticks out from behind their hand. It is primarily used for attacking, and does not provide a normal shield bonus; only an AC bonus based on proficiency level.

\paragraph{Shield, Tusked}\index[equipment]{Tusked Shield}\textbf{:} A tusked shield is a large hold-out shield, with spikes all around the edge and a spike protruding from the center. This heavy shield must be used in two hands, and it does not provide a normal shield bonus; only an AC bonus based on proficiency level.

\paragraph{Sling}\index[equipment]{Sling}\textbf{:} A sling is a long (4-6 feet) leather cord with a pouch half way along. The wielder straps one end of the cord around their wrist and holds the other end in the same hand. They then place a lead pellet in the pouch and swing the cord over their head. At the mid-point of the swing, they let go of the loose end, which releases the pellet. Slings normally require pellets as ammunition, although they can be used with stones taken from the ground, albeit with a -1 penalty on to-hit and damage rolls.

\paragraph{Spear}\index[equipment]{Spear}\textbf{:} A spear is a 5-to-7-foot shaft with a stabbing blade on the end. Despite its length, a spear is a well balanced weapon and can either be used in melee one-handed or thrown.

\paragraph{Staff}\index[equipment]{Staff}\textbf{:} A staff is one of the most simple weapons. It consists of just a length of wood 5-7 feet in length, which is wielded in two hands.

\paragraph{Sword, Bastard}\index[equipment]{Bastard Sword}\textbf{:} A bastard sword, also known as a longsword, is a sword with a long (3-4 feet) narrow blade that is used either one-handed or two-handed. The same weapon proficiency covers both forms of usage.

\paragraph{Sword, Normal}\index[equipment]{Normal Sword}\textbf{:} A normal sword is a one-handed sword with a straight or curved blade 2-3 feet long. This category of sword encompasses a variety of different styles of sword, ranging from scimitars to broadswords.

\paragraph{Sword, Rapier}\index[equipment]{Rapier}\textbf{:} A rapier is a one-handed sword with a thin blade 3-4 feet long. It is primarily used as a thrusting weapon, but may also be used as a slashing weapon.

\paragraph{Sword, Short}\index[equipment]{Short Sword}\textbf{:} A short sword is a sword with a straight 18-inch-to-2-foot blade. Primarily used as a stabbing rather than slashing weapon, this small blade can be used in one hand.

\paragraph{Sword, Two-Handed}\index[equipment]{Two-Handed Sword}\textbf{:} A two-handed sword, sometimes known as a greatsword, is a large and heavy sword with a 4-to-5-foot blade. It is always used in two hands.

\paragraph{Swordstick}\index[equipment]{Swordstick}\textbf{:} A swordstick is a 2-to-4-foot cane containing a hidden blade. It is useful for keeping one’s self armed in places that restrict weapons.

Mechanically, it functions as a normal sword once the blade is exposed.

\paragraph{Trident}\index[equipment]{Trident}\textbf{:} A trident is a short (4-6 feet) spear which is split at the end into three tines, like a fork. Each tine is usually barbed. A trident is used in one hand either in melee or as a thrown weapon, and is often used as a fishing spear, since the barbs on the points can lift the fish out of the water when the trident is withdrawn.

\paragraph{Thrown Object}\index[equipment]{Thrown Object}\textbf{:} This can be any small hard object such as a rock. It causes 1d3 points of damage and has a range of 10/30/50. Proficiency is not required to throw an object.

\paragraph{Whip}\index[equipment]{Whip}\textbf{:} A whip is a woven leather cord from 5-15 feet in length that is used in one hand. Whips do little damage, and are more often used for corporal punishment than for serious combat.

\subsection{Miniature/Giant Weapons}\index[general]{Miniature/Giant Weapons}
The weapons listed in this section are assumed to be used by human-sized creatures. Some weapon-wielding monsters are smaller or larger than this. Weapons for these monsters require special care or additional materials resulting in double the listed cost. Miniature weapons deal one less die of damage (minimum of 1 hit point) and their weight is halved. Giant weapons deal one additional die (maximum d12, +2 damage thereafter) and their weight is doubled. Miniature projectile weapons halve their range, giant projectiles double it.

\section{Ammunition}\index[general]{Ammunition}
All missile weapons require ammunition, for example bows need arrows and guns need bullets. In addition, guns also require a charge of \iref[eq:Red Powder]{Red Powder} to work.

Any character can load a weapon with ammunition, even those who are not proficient in its use. Loading a weapon does not take an action, and is assumed to be part of the normal \iref[sec:Attack]{Attack} action (see \fullref{sec:Actions}).

If a character gets multiple attacks for any reason, they can reload between each attack. For example, a \nth{28} level fighter is able to make three attacks against opponents that they can hit on a 2+ (after modifiers). That fighter could make all three attacks with a bow, crossbow, sling or gun and reload between each attack all in a single \iref[sec:Attack]{Attack} action.

The only exception to this rule is with pistols. Although pistols require only one hand to fire, they require two hands to reload. Therefore, unless a character has a free hand, they can only fire a single shot from a pistol in one \iref[sec:Attack]{Attack} action even if they would normally get more than one attack.

Similarly, if a character has no free hand, they cannot make a second attack with a pistol in subsequent rounds until such time as they have a free hand in order to reload.

A character equipped with multiple pistols but no free hand can take advantage of having multiple attacks if they are prepared to drop each empty pistol after firing it and draw a new loaded one.

\begin {table}[H]
  \caption{Ammunition}
  \begin{tabularx}{\columnwidth}{>{\bfseries}YYY}
		\thead{Item} & \thead{Cost} & \thead{Weight}\\
		Arrows (20) & 5 gp & 10 cn\\
		Arrows, Silver (2) & 10 gp & 1 cn\\
		Bolts (30) & 10 gp & 10 cn\\
		Bolts, Silver (3) & 15 gp & 1 cn\\
		Bullets (20) & 2 gp & 4 cn\\
		Bullets, Silver (5) & 25 gp & 1 cn\\
		Darts (5) & 1 gp & 1 cn\\
		Pellets (30) & 1 gp & 6 cn\\
		Pellets, Silver (5) & 25 gp & 1 cn\
  \end {tabularx}
\end {table}

\paragraph{Arrows}\index[equipment]{Arrows}\textbf{:} Arrows are the ammunition used by bows. The same arrows are usable in either long or short bows. Arrows are often broken in use. At the end of a combat, a character will only be able to recover half (round down) of the arrows fired during that combat. The rest are either lost or unusable.

\paragraph{Bolts}\index[equipment]{Bolts}\textbf{:} Bolts are the ammunition used by crossbows. The same bolts are usable in either heavy or light crossbows. Bolts are often broken in use. At the end of a combat, a character will only be able to recover half (round down) of the bolts fired during that combat. The rest are either lost or unusable.

\paragraph{Bullets}\index[equipment]{Bullets}\textbf{:} Bullets are the ammunition used by guns. The same bullets are usable in either pistols or smoothbores. Bullets are not recoverable after they have been fired. Those that have not been lost will have been deformed beyond use.

\paragraph{Darts}\index[equipment]{Darts}\textbf{:} Darts are the ammunition used by blowguns. The same darts are usable in either small or large blowguns. Darts are usually dipped in poison of some kind before use, although such poison is not included in the basic price. Darts are often broken in use. At the end of a combat, a character will only be able to recover half (round down) of the darts fired during that combat. The rest are either lost or unusable.

\paragraph{Pellets}\index[equipment]{Pellets}\textbf{:} Pellets are the ammunition used by slings. Pellets are easy to lose in use. At the end of a combat, a character will only be able to recover half (round down) of the pellets fired during that combat. The rest are lost.

\section{Armor}
Armor is toughened clothing, made out of leather or metal, that protects its wearer. In game terms, this reduces the armor class of the character wearing it from the default value of 9 to a better (i.e. lower) value. This value may be further adjusted by \iref[sec:Dexterity]{Dexterity} or magical factors, or by use of a shield.

Monsters and demi-human characters with an above average natural armor class who wear armor get either their normal armor class or the armor class granted by the armor they are wearing, whichever is better.

Although armor is made from a variety of pieces—from vambraces and greaves through to helmets and gauntlets—armor in Dark Dungeons is assumed to come in sets. Two sets of chain mail are considered to be the same, even if one consists of a chain shirt, chain trews and a coif whereas the other consists of a chain hauberk with leather vambraces and a steel helmet. Only one set of armor may be worn at a time, and adding or removing a helmet does not change the armor class granted by armor.

Since most people making armor—and especially those making magical armor—will be doing so for customers of different races, most armor is made with straps and so forth that can adjust it to fit wearers of differing size. Therefore all humans and demi-humans can wear the same armor. Similarly, most humanoids of approximately the same human to halfling size range can also wear the same armor.

\begin {table}[H]
  \caption{Armor}
  \begin{tabularx}{\columnwidth}{>{\bfseries}YYYY}
		\thead{Item} & \thead{Cost} & \thead{Armor Class} & \thead{Weight}\\
		Hide Armor & 10 gp & 8 & 100 cn\\
		Leather Armor & 20 gp & 7 & 200 cn\\
		Scale Mail & 30 gp & 6 & 300 cn\\
		Chain Mail & 40 gp & 5 & 400 cn\\
		Banded Mail & 50 gp & 4 & 450 cn\\
		Plate Mail & 60 gp & 3 & 500 cn\\
		Suit Armor & 250 gp & 0 & 750 cn\
  \end {tabularx}
\end {table}

\paragraph{Banded Mail}\index[equipment]{Banded Mail}\textbf{:} This is a suit primarily composed of chain mail with horizontal metal strips fastened into the mail. Banded mail doesn’t quite have the protection of plate mail, but is cheaper and lighter.

\paragraph{Chain Mail}\index[equipment]{Chain Mail}\textbf{:} This is a suit primarily composed of small metal rings that are linked together to form a flexible protective material.

\paragraph{Hide Armor}\index[equipment]{Hide Armor}\textbf{:} This is a suit made of the hide of a creature. Because it is made from natural materials it can be worn by druids. It is also light and quiet enough to be worn by rogues.

\paragraph{Leather Armor}\index[equipment]{Leather Armor}\textbf{:} This is a suit primarily composed of leather plates and strips. The leather is often boiled in wax to harden it. Because it is made from natural materials it can be worn by druids. It is also light and quiet enough to be worn by rogues.

\paragraph{Plate Mail}\index[equipment]{Plate Mail}\textbf{:} This is a suit primarily composed of large metal plates—the largest being the breastplate—linked together with chain mail.

It is the best armor that can be bought second hand or looted, since suit armor must be custom-made.

\paragraph{Scale Mail}\index[equipment]{Scale Mail}\textbf{:} This is a suit primarily composed of leather plates that have metal scales or studs sewn onto them for added protection.

\paragraph{Suit Armor}\index[equipment]{Suit Armor}\textbf{:} This is a suit primarily composed of overlapping metal plates, with chain mail underneath.

Also, known as jousting plate, suit armor must be specially made to fit its owner—and is often worn by nobility as a status symbol as much as for protection. However, it does provide more protection than any other (non-magical) armor. Suit armor reduces all damage done to the character by area effect attacks (such as \iref[spell:Fireball]{Fireball} spells and dragon’s breath) by one point per die of damage. If the suit armor is magical, it reduces such damage by an additional point per two points of enhancement bonus.

Unfortunately, since suit armor is designed for tourney use and parades where the wearer spends their time on horseback, it is very bulky and noisy when worn on foot. When on foot, someone wearing suit armor can he heard moving up to 120 feet away, negating any chance of surprise; and only has a 1 in 6 chance per round of being able to stand unaided if knocked \iref[sec:Prone]{Prone}.

\subsection{Miniature/Giant Armor}\index[general]{Miniature/Giant Armor}
The armor listed in this book is assumed to be used by human-sized creatures. Some armor-using monsters are smaller or larger than this. Armor for these monsters require special care or additional materials resulting in triple the listed cost. Miniature armor weighs half the listed weight, giant armor weighs double.

\section{Shields}
A shield is a large solid piece of wood or metal that is either held in one hand or strapped to one arm. It is used to parry melee attacks and provide cover from missile attacks. A shield is normally used in conjunction with armor and provides an extra bonus to the character’s armor class. If used without other armor, a shield provides a bonus to the character’s natural armor class. These bonuses do not apply if the wearer is attacked from behind. Wooden shields may be used by druids, since they are made of natural materials.

\begin {table}[H]
  \caption{Shields}
  \begin{tabularx}{\columnwidth}{>{\bfseries}YYYY}
		\thead{Item} & \thead{Cost} & \thead{Armor Class} & \thead{Weight}\\
		Shield, Small & 5 gp & -1 to AC & 50 cn\\
		Shield, Normal & 10 gp & -2 to AC & 100 cn\\
		Shield, Kite & 15 gp & -3 to AC & 200 cn\\
		Shield, Tower & 20 gp & -4 to AC & 300 cn\
  \end {tabularx}
\end {table}

\paragraph{Shield, Small}\index[equipment]{Small Shield}\textbf{:} Small shields are around 12 to 18 inches in diameter and are held in the fist.

\paragraph{Shield, Normal}\index[equipment]{Normal Shield}\textbf{:} A normal shield is a medium-sized shield around 2-3 feet in diameter that is held in the fist. They come in a variety of shapes.

\paragraph{Shield, Kite}\index[equipment]{Kite Shield}\textbf{:} A kite shield is a large, almond-shaped shield around 18 inches wide and up to 4 feet tall. Too big to be held in the fist, it is worn strapped to the arm.

\paragraph{Shield, Tower}\index[equipment]{Tower Shield}\textbf{:} A tower shield is a large, oblong or rectangular shield around 2 feet wide and up to 5 feet tall. Too big to be held in the fist, it is worn strapped to the arm. These shields are too heavy and bulky to be used while on a mount.

\subsection{Miniature/Giant Shields}\index[general]{Miniature/Giant Shields}
The shields listed in this book are assumed to be used by human-sized creatures. Some shield-using monsters are smaller or larger than this. Shields for these monsters require special care or additional materials resulting in double the listed cost. Miniature shields weighs half the listed weight, giant shields weighs double.

\section{Land Transport}
Whether trekking across a desert, riding from city to city, or driving a caravan of wagons filled with a dragon’s hoard, adventurers often need land transport other than walking.

Pretty much all land transport is powered by animals, and the vast majority of these animals are horses and mules, ranging from the smallest pony to the largest draft horse.

The speeds listed in \fullref{tab:Pack and Riding Animals} are for the animals when carrying a load up to their listed carrying capacity. Animals can carry twice the listed load, but can only move at half the listed speed while doing so.

Animals pulling vehicles use the carrying capacity of the cart or wagon rather than their own capacity, and cannot be loaded whilst hitched to a vehicle in this manner. When pulling a vehicle, an animal moves at its normal speed when the vehicle is carrying up to its capacity and at half speed when the vehicle is carrying up to twice its capacity.

\begin {table}[H]
  \caption{Pack and Riding Animals}\label{tab:Pack and Riding Animals}
  \begin{tabularx}{\columnwidth}{>{\bfseries}YYYY}
		\thead{Item} & \thead{Cost} & \thead{Carrying Capacity} & \thead{Speed}\\
		Camel & 100 gp & 3,000 cn & 50 ft./rnd\\
		Donkey & 20 gp & 2,000 cn & 30 ft./rnd\\
		Horse, Draft & 40 gp & 4,500 cn & 30 ft./rnd\\
		Horse, Riding & 75 gp & 3,000 cn & 80 ft./rnd\\
		Horse, War & 250 gp & 4,000 cn & 40 ft./rnd\\
		Mule & 30 gp & 3,000 cn & 40 ft.rnd\\
		Pony & 35 gp & 2,000 cn & 70 ft./rnd\
  \end {tabularx}
\end {table}

\begin {table}[H]
  \caption{Land Transport Equipment}
	\begin{tabularx}{\columnwidth}{>{\bfseries}cYYY}
		\thead{Item} & \thead{Cost} & \thead{Carrying Capacity} & \thead{Weight}\\
		Saddle and Tack & 25 gp & 200 cn & 300 cn\\
		Saddle Bags & 5 gp & 800 cn & 100 cn\\
		Trap (1 donkey/mule/pony) & 50 gp & 2,000 cn & -\\
		Cart (1 horse or 2 mules/ponies) & 100 gp & 4,000 cn & -\\
		Cart (2 horses or 4 mules/ponies) & 100 gp & 8,000 cn & -\\
		Chariot (2 horse) & 200 gp & 4,000 cn & -\\
		Chariot (4 horses) & 200 gp & 6,000 cn & -\\
		Wagon (2 horses) & 200 gp & 15,000 cn & -\\
		Wagon (4 horses) & 200 gp & 25,000 cn & -\
  \end {tabularx}
\end {table}

\begin {table}[H]
  \caption{Barding}
  \begin{tabularx}{\columnwidth}{>{\bfseries}YYYY}
		\thead{Item} & \thead{Cost} & \thead{Armor Class} & \thead{Weight}\\
		Leather Barding & 40 gp & 7 & 250 cn\\
		Scale Barding & 75 gp & 6 & 400 cn\\
		Chain Barding & 150 gp & 5 & 600 cn\\
		Banded Barding & 400 gp & 4 & 1,500 cn\\
		Plate Barding & 500 gp & 3 & 3,000 cn\\
		Field Barding & 600 gp & 2 & 4,000 cn\\
		Joust Barding & 700 gp & 0 & 5,000 cn\
  \end {tabularx}
\end {table}

\paragraph{Banded Barding}\index[equipment]{Banded Barding}\textbf{:} This is a suit primarily composed of chain mail with horizontal metal strips fastened into the mail.

Banded barding gives an animal an armor class of 4, unless the animal’s armor class is already better than 4.

\paragraph{Camel}\index[equipment]{Camel}\textbf{:} Camels are normally only found in desert or semi-desert environments where horses don’t fare well.

\paragraph{Cart}\index[equipment]{Cart}\textbf{:} A medium-sized two-wheeled vehicle pulled by one or more horses. A cart is designed for carrying cargo, and may optionally have a seat for a driver. If there is no seat, the horse must be led.

\paragraph{Chariot}\index[equipment]{Chariot}\textbf{:} Chariots are armored medium-sized two-wheeled vehicles made of wood, chitin, and hardened leather. They are pulled by 2 or more horses and are designed for riding and combat. Untrained riders attacking from a moving chariot suffer a -2 penalty to attack and the damage is reduced to half.

\paragraph{Chain Barding}\index[equipment]{Chain Barding}\textbf{:} This is a suit primarily composed of small metal rings that are linked together to form a flexible protective material. Chain barding gives an animal an armor class of 5, unless the animal’s armor class is already better than 5.

\paragraph{Donkey}\index[equipment]{Donkey}\textbf{:} Donkeys are rarely ridden, but they make a cheap—if somewhat stubborn and willful—pack animal.

\paragraph{Field Barding}\index[equipment]{Field Barding}\textbf{:} This is a suit primarily composed of large metal plates linked together with chain mail. The plates are heavier and more numerous than normal plate barding. Field barding gives an animal an armor class of 2, unless the animal’s armor class is already better than 2.

\paragraph{Horse, Draft}\index[equipment]{Draft Horse}\textbf{:} A large powerful horse, that sacrifices speed for strength and can carry heavy loads over long distances.

\paragraph{Horse, Riding}\index[equipment]{Riding Horse}\textbf{:} A typical horse, fast and light but easily spooked and not suitable for combat situations.

\paragraph{Horse, War}\index[equipment]{War Horse}\textbf{:} A large horse specially trained to not panic in combat situations.

\paragraph{Joust Barding}\index[equipment]{Joust Barding}\textbf{:} This is a suit primarily composed of overlapping metal plates that completely protect the front of the animal, to protect it from injury in jousting competitions. The plates are even heavier than field barding. Joust barding gives an animal an armor class of 0, unless the animal’s armor class is already better than 0.

\paragraph{Leather Barding}\index[equipment]{Leather Barding}\textbf{:} This is a suit primarily composed of leather plates and strips. The leather is often boiled in wax to harden it. Leather armor gives an animal an armor class of 7, unless the animal’s armor class is already better than 7.

\paragraph{Mule}\index[equipment]{Mule}\textbf{:} A cross between a donkey and a horse, combining the best features of both. It is larger and stronger than a donkey, but smarter than a horse.

\paragraph{Plate Barding}\index[equipment]{Plate Barding}\textbf{:} This is a suit primarily composed of large metal plates linked together with chain mail. Plate barding gives an animal an armor class of 3, unless the animal’s armor class is already better than 3.

\paragraph{Pony}\index[equipment]{Pony}\textbf{:} A small light horse that is cheaper to feed and easier to care for than other breeds of horse, but lacks their strength.

\paragraph{Saddle and Tack}\index[equipment]{Saddle and Tack}\textbf{:} A saddle, blanket, bridle and reins—everything needed to ride a horse safely. It is possible to ride a horse bareback without these items, but \iref[sec:Dexterity]{Dexterity} checks made to control the horse will be made at a -3 to effective \iref[sec:Dexterity]{Dexterity}.

The 200 cn carrying capacity of a saddle and tack does not refer to the weight of the rider, but to the weight that can be carried in the bags and pouches that come with it.

\paragraph{Saddle Bags}\index[equipment]{Saddle Bags}\textbf{:} Saddle bags are long pairs of sacks sewn together at the top with a length of material. They are slung over a horse’s saddle so that one bag hangs down either side of the horse distributing the weight evenly.

\paragraph{Scale Barding}\index[equipment]{Scale Barding}\textbf{:} This is a suit primarily composed of leather plates that have metal scales or studs sewn onto them for added protection. Scale barding gives an animal an armor class of 6, unless the animal’s armor class is already better than 6.

\paragraph{Trap}\index[equipment]{Trap}\textbf{:} A small two-wheeled vehicle with two seats, that is pulled by a single pony, mule or donkey. Although a trap can be used for transporting cargo, it is primarily designed as a means of personal transport.

\paragraph{Wagon}\index[equipment]{Wagon}\textbf{:} A large four-wheeled vehicle, pulled by a team of horses. Wagons are mostly used for cargo transport, although some traveling people live in them as an alternative to tents.

\section{Sea Transport}
For long distance travel, it is much more efficient to travel by boat than by land—and depending on the destination, land travel may not be possible.

If characters are traveling along an established route, they can book passage on an existing ship. If not, they may need to buy a ship and hire crew to sail it.

All ships and boats need skilled crew to sail them, and some also need unskilled rowers. If more than 5 crew are needed one of them must be a captain, and if more than 15 crew are needed one of them must be a captain and one of them must be a first mate.

Crew (and troops, if mentioned in the description) do not count towards a ship’s carrying capacity.

\end{multicols*}
\begin {table}[H]
  \caption{Ships and Boats}
  \begin{tabularx}{\columnwidth}{>{\bfseries}YYYYYY}
		\thead{Item} & \thead{Cost} & \thead{Minimum Crew} & \thead{Capacity} & \thead{Move per Day} & \thead{Weight†}\\
		Barque & 20,000 gp & 20 Crew & 20,000 cn & 90 miles & 400 tons\\
		Canoe, River & 50 gp & 1 Crew & 6,000 cn & 18 miles & 1,000 cn\\
		Canoe, Sea & 100 gp & 1 Crew & 6,000 cn & 18 miles & 3,000 cn\\
		Galley* & 10,000 gp & 10 Crew, 60 Row & 10,000 cn & 90 miles & 50 tons\\
		Longship* & 15,000 gp & 75 Crew & 15,000 cn & 90 miles & 30 tons\\
		Passage, Average & 1sp/5 miles & - & 6,000 cn & - & -\\
		Passage, Basic & 1sp/20 miles & - & 1,500 cn & - & -\\
		Passage, Luxury & 1sp/mile & - & 15,000 cn & - & -\\
		Passage, Skysailing & 1 gp/mile & - & 15,000 cn & - & -\\
		Quinquirime & 60,000 gp & 30 Crew, 300 Row & 60,000 cn & 72 miles & 120 tons\\
		Raft, Professional & 100 gp & - & 10,000 cn & 12 miles & 5,000 cn\\
		Raft, Scavenged & - & - & 5,000 cn & 12 miles & 5,000 cn\\
		River Barge* & 4,000 gp & 2 Crew, 8 Row & 40,000 cn & 36 miles & 10 tons\\
		Rowing Boat* & 1,000 gp & - & 1,000 cn & 18 miles & 5,000 cn\\
		Sail of Skysailing & 200,000 gp & 1 Spellcaster & 100 tons & Varies & 500 cn\\
		Skiff* & 3,000 gp & 1 Crew & 20,000 cn & 72 miles & 5 tons\\
		Sloop* & 5,000 gp & 10 Crew & 5,000 cn & 72 miles & 70 tons\\
		Trireme* & 30,000 gp & 20 Crew, 180 Row & 30,000 cn & 72 miles & 80 tons\\
		Troopship & 30,000 gp & 20 Crew & 30,000 cn & 54 miles & 400 tons\
  \end {tabularx}
	*These ships may be equipped with a Sail of Skysailing
	†1 ton = 20,000 cn
\end {table}
\begin{multicols*}{2}

\paragraph{Barque}\index[equipment]{Barque}\textbf{:} A barque is a three masted ocean going ship 100-150 feet long and 25-30 feet wide, with a draft of 10-12 feet. There are raised decks at the fore and aft, and the ship can be fitted with up to two light catapults or cannons.

In addition to its crew, a barque may house 50 troops.

\paragraph{Canoe, River}\index[equipment]{River Canoe}\textbf{:} A canoe is a small boat 15 feet long and 3 feet wide with a 1-foot draft. A canoe is normally made of waxed canvas or hides stretched over a wooden frame, and has one or two seats. Canoes are designed for use in rivers and swamps, and can easily be carried across land. A canoe weighs 1,000 cn if carried—but two people can share the weight.

\paragraph{Canoe, Sea}\index[equipment]{Sea Canoe}\textbf{:} A sea canoe is a small boat 15 feet long and 3 feet wide with a 1-foot draft, with one or two floats held out either side for stability. A sea canoe is normally made of waxed canvas or hides stretched over a wooden frame, and has one or two seats. Sea canoes are designed for use in coastal waters, and can be carried across land.

A canoe weighs 3,000 cn if carried—but two to four people can share the weight.

\paragraph{Galley}\index[equipment]{Galley}\textbf{:} A galley is an ocean-going ship 60-100 feet long and 10-15 feet wide with a 2-to-3-foot draft. Because of its small draft, it can travel along rivers as long as they are wide enough for it.

The listed speed is for the galley under sail.

If becalmed, it can be rowed at 18mi/day. A galley has a single line of rowers.

A galley can be fitted with a light ship’s ram (at a cost of 3,000 gp) and up to two light catapults or cannons.

In addition to its crew, and rowers it will normally house 20 troops.

\paragraph{Longship}\index[equipment]{Longship}\textbf{:} A longship is a single masted boat 60-80 feet long and 10-15 feet wide, with a 2-to-3-foot draft. It is designed for troop transport along rivers and coasts. The 75 crew normally act as both rowers and troops when necessary.

The listed speed is for the longship under sail. If becalmed, it can be rowed at 18mi/day.

\paragraph{Passage, Average}\index[equipment]{Average Passage}\textbf{:} Average passage includes a small shared cabin that can hold up to 1,000 cn plus the character, basic meals, and an additional 5,000 cn of hold space.

\paragraph{Passage, Basic}\index[equipment]{Basic Passage}\textbf{:} Basic passage includes a bunk in a shared hold with space for 500 cn plus the character, no meals unless the character brings their own food or pays for ship’s food separately, and 1,000 cn of hold space.

\paragraph{Passage, Luxury}\index[equipment]{Luxury Passage}\textbf{:} Luxury passage includes a spacious private cabin that can store up to 5,000 cn plus the character, excellent meals, and an additional 10,000 cn of hold space.

\paragraph{Passage, Skysailing}\index[equipment]{Skysailing Passage}\textbf{:} Skysailing passage is passage on a ship equipped with a \iref[eq:Sail of Skysailing]{Sail of Skysailing} which enables it to fly at great speed. This price is for atmospheric travel only. Voidspeed travel is charged at 100 gp/day in addition to the mileage to the edge of the atmosphere.

\paragraph{Quinquirime}\index[equipment]{Quinquirime}\textbf{:} A quinquirime is an ocean-going ship 120-150 feet long and 20-30 feet wide with a 4-to-6-foot draft. Because of its width, it cannot usually travel along rivers.

The listed speed is for the quinquirime under sail. If becalmed, it can be rowed at 12mi/day by five tiers of rowers.

A quinquirime includes a heavy ship’s ram and fore and aft towers. It can be fitted with up to three light catapults or cannons. In addition to its crew, it will normally house 75 troops.

\paragraph{Raft}\index[equipment]{Raft}\textbf{:} A raft is a flat platform 10 feet long and 10 feet wide, with a 3-to-6-inch draft. A raft is the most basic of vessels, and is normally limited to swamps, lakes and slow moving rivers. Up to 12 rafts can be lashed together to make a single larger raft with proportionally higher carrying capacity.

\paragraph{River Barge}\index[equipment]{River Barge}\textbf{:} A river barge is a flat-bottomed sail-less boat 20-30 feet long and 10 feet wide with a 2-to-3-foot draft. It is normally used for carrying cargo up and down slow moving rivers.

\paragraph{Rowing Boat}\index[equipment]{Rowing Boat}\textbf{:} A rowing boat is a small sail-less boat 20 feet long and 4-5 feet wide, with a 1-to-2-foot draft. Rowing boats are often stored on larger ships and used to transport people to and from shore or, in an emergency, used as lifeboats. Each rowing boat stored in this manner takes up 5,000 cn of its parent ship’s capacity.

\paragraph{Sail of Skysailing}\index[equipment]{Sail of Skysailing}\label{eq:Sail of Skysailing}\textbf{:} A sail of skysailing is a magical sail made chiefly from the silk of phase spiders. It can be attached to most boats and ships that weigh 100 tons or less, although some very small boats such as rafts and canoes are unsuitable as they have nowhere to fasten it.

When fitted to a new ship for the first time, the sail must be left in place for a week in order to attune to that ship. During that time, the sail changes shape to fit the rigging of the ship.

Once attunement has taken place, the sails can be hoist (on that individual ship) or stowed repeatedly without losing the attunement. However, a sail of skysailing can only be attuned to one ship at a time, and cannot be used on a different ship—not even one of the same design—without being re-attuned.

See \fullref{sec:Skysailing} for detailed rules on using a Sail of Skysailing to fly a ship.

\paragraph{Skiff}\index[equipment]{Skiff}\textbf{:} A skiff is a single masted boat 15-45 feet long and 5-15 feet wide with a 3-to-8-foot draft. It is designed for lakes and coastal waters, and can sometimes be too deep keeled for river use.

Skiffs are commonly used as fishing boats.

\paragraph{Sloop}\index[equipment]{Sloop}\textbf{:} A sloop is a one or two masted ocean going ship 60-80 feet long and 20-30 feet wide, with a draft of 5-8 feet. There may be a raised deck at the aft.

In addition to its crew, a sloop may house 25 troops.

\paragraph{Trireme}\index[equipment]{Trireme}\textbf{:} A trireme is an ocean-going ship 120-150 feet long and 15-20 feet wide with a 3-foot draft. Because of its small draft, it can travel along rivers as long as they are wide enough for it.

The listed speed is for the trireme under sail. If becalmed, it can be rowed at 18mi/day by three tiers of rowers.

A trireme can be fitted with a heavy ship’s ram (at a cost of 10,000 gp) and up to two light catapults or cannons. In addition to its crew, it will normally house 50 troops.

\paragraph{Troopship}\index[equipment]{Troopship}\textbf{:} A troopship is a three masted ocean going ship 100-150 feet long and 25-30 feet wide, with a draft of 10-12 feet. The hull is the same as that of a large sailing ship, but the ship has been converted to carry the maximum number of troops. A troopship often has large fold-down doors on its sides so that troops—often including cavalry—can rapidly exit.

In addition to its crew, a troopship will normally house 100 troops.

\section{Services}
Whether looking for hired help to take out a goblin’s lair, or looking for crew to accompany you on a sea voyage, or looking for people to manage your estates, or even looking for a cleric to raise your friend; adventurers need the services of others. Employees of the characters come in three types—\iref[sec:Hirelings]{Hirelings}, \iref[sec:Mercenaries]{Mercenaries} and \iref[sec:Specialists]{Specialists}.

\subsection{Hirelings}\index[general]{Hirelings}\label{sec:Hirelings}
Hirelings are adventurers (or would-be adventurers) willing to temporarily join a party for a mission.

Although hirelings travel with—and adventure with—a party, they are not equal party members. They view the party as employers rather than companions, and may desert or even rebel against their employers if maltreated or exposed to excessive danger (see \fullref{sec:Morale}).

Since parties are prone to argument and internal disagreements, hirelings are normally employed by—and follow the orders of—a single designated “party leader”.

This will normally be the party member with the highest \iref[sec:Charisma]{Charisma}. See \fullref{sec:Charisma} for details on how \iref[sec:Charisma]{Charisma} affects the employment of hirelings. Hirelings will not normally obey suicidal orders or allow themselves to be used as “trap detectors”. They are there to help out in fights and to help in transporting supplies and loot, not to act as monster fodder.

Getting the right hirelings can be a tricky business. On the one hand a party will want to hire competent help who are likely to survive the adventure and prove useful, but on the other hand there is both a cost consideration and also the consideration that particularly competent adventurers are likely to be interested in adventuring on their own rather than seeking employment with an existing group, since that is much more lucrative.

Although exact details may vary depending on the campaign world, a useful rule of thumb is that hirelings that are available will range from commoners to adventurers of half the level of the party leader (rounded down, but with a minimum of first level and a maximum of fifth level).

They can be hired for a single adventure for a cost in gold of a tenth of the experience that a character of their class would need for the level above theirs (or 50 gp for a commoner), or for an extended expedition into the wilderness for that cost per month. The party are expected to provide whatever equipment or mounts are needed by the hirelings, and they will expect half payment in advance.

\example{A party whose leader is fifth level wishes to employ some hirelings to accompany them on a mission to drive out an orc encampment that has been raiding in the local area.

Since the party leader is fifth level, hirelings of up to \nth{2} level will be available.

A \nth{2} level rogue needs a total of 2,400 XP to reach \nth{3} level, that \nth{2} level rogue would demand payment of a tenth of that in gold—i.e. 240 gp. A \nth{2} level wizard, on the other hand, would need 5,000 XP to reach \nth{3} level, that \nth{2} level wizard would demand payment of a tenth of that in gold—i.e. 500 gp.

Hirelings do not normally expect a share of treasure or magic items over and above their pay, although if given (or promised) such a share then their morale will be higher. Hirelings who are given magic items to test will expect to be allowed to keep those magic items as payment for the dangers involved in testing them.}

\subsection{Mercenaries}\index[general]{Mercenaries}\label{sec:Mercenaries}
If characters need an entire army, rather than just a few helpers, they can hire mercenaries. Mercenaries are trained troops that will work and fight for pay. The cost for mercenaries of different types are listed in \fullref{tab:Pack and Riding Animals}. The listed costs are for peacetime guarding and patrolling duties. For active war-time duties, double all costs.

When hiring mercenaries, they are assumed to come with captains and other leaders as part of the cost.

Mercenaries provide their own equipment when first hired, but if garrisoned for an extended period their employer is expected to provide armorers and blacksmiths to repair and maintain their equipment.

\begin {table}[H]
  \caption{Mercenaries}
	\begin{tabularx}{\columnwidth}{>{\bfseries}cM{.35in}YYM{.38in}Y}
		\thead{} & \multicolumn{5}{c}{\thead{Cost per Month}}\setcounter{rownum}{0}\\
		\thead{Mercenary Type} & \thead{Human} & \thead{Dwarf} & \thead{Elf} & \thead{Goblin} & \thead{Orc}\\
		Archer & 5 gp & - & 10 gp & 2 gp & 3 gp\\
		Cavalry, Heavy & 20 gp & - & - & - & -\\
		Cavalry, Light & 10 gp & - & 20 gp & - & -\\
		Cavalry, Medium & 15 gp & - & - & - & -\\
		Crossbowman & 4 gp & 6 gp & - & - & 2 gp\\
		Crossbowman, Pony & - & 15 gp & - & - & -\\
		Footman, Heavy & 3 gp & 5 gp & 6 gp & - & 15sp\\
		Footman, Light & 2 gp & - & 4 gp & 5sp & 1 gp\\
		Horse Archer & 15 gp & - & 30 gp & - & -\\
		Longbowman & 10 gp & - & 20 gp & - & -\\
		Militia & 1 gp & - & - & - & -\\
		Wolf Rider & - & - & - & 5 gp & -\
  \end {tabularx}
\end {table}

\textbf{Archer:} First level fighters (or racial equivalent) armed with short bows and swords and wearing leather armor.

\textbf{Cavalry, Heavy:} First level fighters armed with swords and lances and wearing plate armor, riding war horses in plate barding.

\textbf{Cavalry, Light:} First level fighters (or racial equivalent) armed with lances and wearing leather armor, riding war horses in leather barding.

\textbf{Cavalry, Medium:} First level fighters (or racial equivalent) armed with lances and wearing chain mail armor, riding war horses in chain barding.

\textbf{Crossbowman:} First level fighters (or racial equivalent) armed with heavy crossbows and wearing chain mail armor.

\textbf{Crossbowman, Pony:} First level dwarves armed with crossbows and riding ponies.

\textbf{Footman, Heavy:} First level fighters (or racial equivalent) armed with swords and shields and wearing chain mail armor.

\textbf{Footman, Light:} First level fighters (or racial equivalent) armed with swords and shields and wearing leather armor.

\textbf{Horse Archer:} First level fighters (or racial equivalent) armed with short bows, riding normal horses.

\textbf{Longbowman:} First level fighters (or racial equivalent) armed with longbows and swords and wearing chain mail armor.

\textbf{Militia:} Commoners armed with spears.

\textbf{Wolf Rider:} Goblins armed with spears and wearing leather armor, riding dire wolves.

\subsection{Specialists}\index[general]{Specialists}\label{sec:Specialists}
Sometimes a character will need the help of a different kind of specialist. Maybe they need an engineer to oversee the building of a large castle, maybe they need a scribe to write their memoirs, or maybe they just need a cleric to raise a dead party member.

All of these situations require the character to employ a skilled specialist of some kind or another.

Specialists will not expose themselves to danger, and will not accompany characters on adventures.

The costs for different types of specialists are listed in \fullref{tab:Specialists}. Any specialist not listed in this table (for example a leatherworker or a scribe) should be assumed to have a cost of 5 gp/month if skilled or 2 gp/month if unskilled.

\begin {table}[H]
  \caption{Specialists}\label{tab:Specialists}
  \begin{tabularx}{\columnwidth}{>{\bfseries}YY}
		\thead{Specialist Type} & \thead{Cost per Month}\\
		Animal Trainer & 500 gp\\
		Armorer & 100 gp\\
		Artillerist & 750 gp\\
		Bailiff & 5 gp\\
		Blacksmith & 25 gp\\
		Castellan & 2,000 gp\\
		Chamberlain & 5 gp\\
		Chaplain & 500 gp\\
		Chemist & 1,000 gp\\
		Chief Magistrate & 2,000 gp\\
		Engineer & 750 gp\\
		Equerry & 5 gp\\
		Guard Captain & 4,000 gp\\
		Herald & 400 gp\\
		Magist & 3,000 gp+\\
		Marshal & 5 gp\\
		Provost & 5 gp\\
		Reeve & 500 gp\\
		Rower & 2 gp\\
		Sage & 2,000 gp\\
		Sailor & 10 gp\\
		Seneschal & 4,000 gp\\
		Sheriff & 5 gp\\
		Ship’s Captain & 250 gp\\
		Ship’s Navigator & 150 gp\\
		Spellcaster for Single Spell & Special\\
		Steward & 1,000 gp\\
		Warden & 5 gp\
  \end {tabularx}
\end {table}

\textbf{Animal Trainer:} An animal trainer domesticates and trains unusual animals.

An animal trainer is not required for horses, mules, donkeys or dogs, but other animals can only be taught “tricks” by an animal trainer, who can handle up to six creatures of the same species at a time. The length of time needed to train an animal will depend on the animal type, but a month is average.

\textbf{Armorer:} An armorer makes and repairs armor. One armorer is needed per 50 troops, whether the troops are conscripted or mercenaries.

\textbf{Artillerist:} An artillerist is usually a fighter (or racial equivalent class) of \nth{3} to \nth{5} level, who is in charge of the placement, maintenance and operation of siege weapons.

\textbf{Bailiff:} A bailiff is an official who looks after part or all of a castle, and makes sure that that part of the stronghold is in good repair.

\textbf{Blacksmith:} A blacksmith extracts pure iron from iron ore and makes steel.

Sometimes a blacksmith will also make simple metal goods; other times the blacksmith will simply make ingots of metal that other craftsmen will use.

\textbf{Castellan:} A castellan is usually a fighter (or racial equivalent class) of \nth{5} to \nth{9} level, and is in overall charge of the military aspects of a stronghold.

\textbf{Chamberlain:} A chamberlain is in charge of cleaning and cooking staff in a stronghold.

\textbf{Chaplain:} A chaplain is a cleric who works full-time at a stronghold looking after the chapel and performing religious services. Unlike clerics who work in independent temples around a dominion, the chaplain is salaried rather than living on donations and tithes.

\textbf{Chemist:} A chemist is a non-spell-caster who studies alchemy, and may make potions like a wizard, but takes twice the time and cost to do so.

\textbf{Chief Magistrate:} A chief magistrate is in charge of justice within the dominion of a stronghold, and oversees the common magistrates and sheriffs.

\textbf{Engineer:} An engineer oversees the design and construction of buildings, roads, bridges, and other large scale structures. One engineer is needed per 100,000 gp cost of a building project.

\textbf{Equerry:} An equerry is in charge of the stables, and is a specialized form of bailiff.

\textbf{Guard Captain:} A guard captain is usually a fighter (or racial equivalent) of \nth{8} level or higher, and is in charge of both the rulers personal guard and the guarding of the stronghold.

\textbf{Herald:} A herald is in charge of making announcements, and also in charge of maintaining up to date news about (and coats of arms of) the rulers of nearby dominions. A herald also acts as an adviser on matters of etiquette.

\textbf{Magist:} A magist is a wizard (or racial equivalent class) of \nth{9} level or higher who acts as an adviser on magical affairs.

\textbf{Marshal:} A marshal is a fighter (or racial equivalent class) who is in charge of recruiting and training troops.

\textbf{Provost:} A provost collects taxes.

\textbf{Reeve:} A reeve is in charge of book-keeping and accounts within a stronghold.

\textbf{Rower:} A rower is an unskilled seaman who rows in a galley.

\textbf{Sage:} A sage is an adviser who specializes in history and lore.

\textbf{Sailor:} A sailor is skilled at operating and maintaining ships and boats.

\textbf{Seneschal:} A seneschal, sometimes called a vizier, is a ruler by proxy who speaks for the actual ruler of the dominion when the ruler is absent or when the ruler does not wish to bother with a particular matter.

\textbf{Sheriff:} A sheriff is responsible for law enforcement in an area of dominion.

\textbf{Ship’s Captain:} A ship’s captain is in charge of the well-being of a ship and its crew.

\textbf{Ship’s Navigator:} A ship’s navigator is in charge of ensuring a ship follows the correct course and getting it back on course if there have been problems.

\textbf{Spellcaster for Single Spell:} Sometimes a character doesn’t need to employ a spellcaster over a long term. They simply need a single spell cast, whether it is to identify a magic item or to raise a dead companion back to life.

The issues involved in this vary depending on the type of caster. \iref[class:Cleric]{Clerics} vary from religion to religion. While some secretive cults will simply refuse to cast spells for outsiders (or will only do so if paid in a similar manner to wizards), the clerics of most mainstream religions are dedicated to good works and spreading the popularity of their religion. As such, they will generally cast spells for free, especially if it is for something they consider to be a good cause.

However, the problem is availability rather than cost—particularly for healing (and raising) clerical spells or others that would compete for the same spell slots. There may be a waiting list for such spells, as locals also need them. A few clerics may accept a bribe to push adventurers to the front of the queue, but most will not since their own lay members and followers come first.

When trying to get a spell cast by a mainstream cleric, consult \fullref{tab:Spellcaster for Single Spell}.

\begin {table}[H]
  \caption{Spellcaster for Single Spell}\label{tab:Spellcaster for Single Spell}
  \begin{tabularx}{\columnwidth}{>{\bfseries}cY}
		\thead{1d8} & \thead{Availability}\\
		0 or less & There is no cleric who can (or is willing to) cast the spell at this time.\\
		1-2 & There is a long queue of people wanting the spell. It will be available after 3d6 days.\\
		3-4 & There is a queue of people wanting the spell. It will be available after 1d6 days.\\
		5-6 & Spell is used up or not learned, but will be available the following day.\\
		7 or more & Spell is available immediately.\
  \end {tabularx}
\end {table}

Most of the time, the size of the temple or chapel will not affect the roll, since larger temples have more (and higher level) clerics but also serve greater numbers of lay worshipers.

However, if there is a particular reason for a high level cleric to be in a small temple, the roll should get a +1 bonus; and if there is a particular reason for a large temple to have a shortage of high level clerics then the roll should get a -1 penalty. Similarly, if the desired spell is \nth{2} level or lower the roll should get a +1 bonus and if the desired spell is \nth{5} level or higher the roll should get a -1 penalty with an additional -1 penalty if the spell is \nth{7} level.

\iref[class:Wizard]{Wizards} and \iref[class:Elf]{elves}, on the other hand, tend to set artificially high prices in order to not be constantly disturbed with petty requests. Generally, this will be a cost of 10 gp/caster level for each consultation (whether that results in a spell being cast or not). \iref[class:Druid]{Druids} set similar prices, but will generally be interested in an equivalent value of goods, since they have little use for money.

\textbf{Steward:} A steward is in charge of household affairs at a stronghold, including housekeeping, maintaining food supplies, and so forth.

\textbf{Warden:} A warden is a military adviser subordinate to the castellan and is responsible for the defenses of a particular area within the dominion or stronghold.

\section{Siege Weaponry}
Siege weaponry is, naturally enough, used when besieging a stronghold. However, it can also be used defensively firing from the stronghold itself or even used in naval warfare. Full rules for siege combat are found in \fullref{sec:Mass Combat}.

\end{multicols*}
\begin {table}[H]
  \caption{Siege Weaponry}
	\begin{tabularx}{\columnwidth}{>{\bfseries}cccYcYcccM{.6in}c}
		\thead{Item} & \thead{Cost} & \thead{AC} & \scriptsize\thead{Hit Points} & \scriptsize\thead{Artillerists} & \scriptsize\thead{Other Crew} & \thead{Range} & \thead{Damage} & \thead{Fire Rate} & \thead{Ammo Cost} & \thead{Weight}\\
		Ballista & 75 gp & 4 & 9 & - & 4 & 100/200/300 & 1d10+6 & 1/2 rnds & 2,000 gp/wk & 6,000 cn\\
		Battering Ram & 100 gp & -4 & 50 & - & 10 & - & 1d6+8 & 1/2 rnds & - & 3,000 cn\\
		Belfry & 750 gp & 0 & 75 & - & - & - & - & - & - & 250,000 cn\\
		Bore & 150 gp & -4 & 50 & - & 10 & - & 1d6+14 & 1/2 rnds & - & 3,000 cn\\
		Cannon & 1,000 gp & -4 & 75 & 2 & 3 & 250/350/450 & 1d10+10 & 1/3 rnds & 12,000 gp/wk & 10,000 cn\\
		Catapult, Heavy & 250 gp & 0 & 27 & 1 & 7 & 250/325/400* & 1d10+10 & 1/6 rnds & 6,000 gp/wk & 18,000 cn\\
		Catapult, Light & 150 gp & 4 & 18 & 1 & 5 & 200/250/300* & 1d8+8 & 1/5 rnds & 4,000 gp/wk & 12,000 cn\\
		Gallery Shed & 300 gp & 4 & 40 & - & 8 & - & - & - & - & 8,000 cn\\
		Hoist & 150 gp & 4 & 15 & - & 6 & - & - & - & - & 12,000 cn\\
		Ladder & 3 gp & 4 & 3 & - & 2 & - & - & - & - & 900 cn\\
		Mantlet & 16 gp & 0 & 16 & - & - & - & - & - & - & 4,800 cn\\
		Ship’s Ram, Heavy & 10,000 gp & - & - & - & - & - & 6d6† & - & - & -\\
		Ship’s Ram, Light & 3,000 gp & - & - & - & - & - & 3d8† & - & - & -\\
		Timber Fort & 32 gp & 0 & 32 & - & - & - & - & - & - & 7,200 cn\\
		Trebuchet & 400 gp & 0 & 50 & 1 & 11 & 250/400/500* & 1d12+13 & 1/6 rnds & 8,000 gp/wk & 24,000 cn\
  \end {tabularx}
*Catapults and trebuchets have a minimum range of 150 feet
†Ship’s Rams do full damage against other ships.
\end {table}
\begin{multicols*}{2}

\paragraph{Ballista}\index[equipment]{Ballista}\textbf{:} A ballista is a large crossbow mounted on a sturdy platform. It fires bolts that are the size of spears. It is most commonly used to shoot into formations of troops, as the spears are of little use against fortifications. A ballista on wheels can be pulled by a single horse, mule or pony.

\paragraph{Battering Ram}\index[equipment]{Battering Ram}\textbf{:} A large heavy post or log that is usually used to break down wooden walls or doors.

If it is mounted inside a belfry or gallery sled, it only needs half the normal crew.

\paragraph{Belfry}\index[equipment]{Belfry}\textbf{:} A portable tower 30 feet tall with a drawbridge at the top and a door at the bottom, connected by an internal staircase. The belfry is pushed up to the walls of a fortification and then the drawbridge is lowered so that it rests on top of those walls. Troops climb the stairs and cross the drawbridge onto the fortification walls. A belfry needs to be pulled by a team of four horses or pushed by 20 humans. A belfry provides troops inside it with a -8 bonus to armor class.

\paragraph{Bore}\index[equipment]{Bore}\textbf{:} A bore is a large drill hung from chains like a battering ram. It is pushed against the side of a fortification and used to drill into the side of it, weakening it.

\paragraph{Cannon}\index[equipment]{Cannon}\textbf{:} A cannon is basically an over-sized gun on wheels that shoots large iron balls rather than bullets. Although lighter than catapults and trebuchets and requiring fewer untrained crew; cannons are expensive to make, require more trained crew, and require a steady supply of \iref[eq:Red Powder]{Red Powder} to work.

\paragraph{Catapult}\index[equipment]{Catapult}\textbf{:} A catapult consists of a frame containing a wooden pole with a basket or bowl on the end that is pulled back under tension and then quickly released so that it springs forward within the frame hurling whatever was put in the bowl towards the enemy.

A light catapult can be towed by a single horse, and a heavy catapult can be towed by a pair of horses.

\paragraph{Gallery Shed}\index[equipment]{Gallery Shed}\textbf{:} A gallery shed is a wooden frame with side walls and a roof but no end walls. It is used to either give cover to troops, allowing them to get close to a fortification, or to house a battering ram or bore. A gallery shed provides troops inside it with a -12 bonus to armor class.

\paragraph{Hoist}\index[equipment]{Hoist}\textbf{:} A hoist is a small platform fastened to a 30-foot-tall pole which rests on a mobile base. It is pushed towards the wall of a fortification, and then elite troops stand on the platform while the crew pull it to the top of the pole by means of a rope and pulley system. Although it does not offer the protection that a belfry does, it is both cheaper and far more portable.

\paragraph{Ladder}\index[equipment]{Ladder}\textbf{:} This is simply a 30-foot-long normal ladder that is put up against the wall of a fortification and climbed. Since it is very vulnerable to being dislodged or broken, it is normally only used for either stealth attacks or secondary attacks while most wall defenders are busy trying to fight off belfries and hoists.

\paragraph{Mantlet}\index[equipment]{Mantlet}\textbf{:} A mantlet is an 8-foot-long-by-4-foot-high wooden palisade on wheels that up to five troops can use to shield themselves from missile fire as they advance towards a fortification.

A mantlet provides troops behind it with a -4 bonus to armor class.

\paragraph{Ship’s Ram}\index[equipment]{Ship’s Ram}\textbf{:} A ship’s ram is like a battering ram, except fastened to the front of a ship below the water line, offset to either side. The ship attacks by ramming its target with a glancing blow, so that the ship’s itself doesn’t get significantly damaged by impacting the enemy vessel with force, but the ram scrapes along the enemy’s hull as the two ships pass. Ship’s rams do full damage to other ships.

\paragraph{Timber Fort}\index[equipment]{Timber Fort}\textbf{:} A timber fort is a number of 8-foot-long-by-5-foot-high sections of wooden palisade that can be carried on a wagon and quickly deployed in the field on order to provide cover from missiles. Once deployed, it cannot be moved with any great speed.

A timber fort provides troops inside or behind it with a -8 bonus to armor class.

\paragraph{Trebuchet}\index[equipment]{Trebuchet}\textbf{:} A trebuchet is a long pole with a sling on one end and a heavyweight on the other. The whole thing is mounted on a wheeled frame. The pole is pulled down so that the sling is down to the ground and can be loaded, then it is released—at which time the weight forces the pole to swing and the sling to fire the ammunition in the direction of the enemy.

\end{multicols*}

