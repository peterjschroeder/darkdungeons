\begingroup
\titleformat{\chapter}{\filcenter\librebaskerville\fontsize{20}{24}\bfseries\color{\pgcolor}}{\chaptertitlename~\thechapter:} {0.5em} {}
\titlespacing*{\chapter}{0pt}{-53pt}{0pt}
\chapter[green]{Strongholds and Dominions}
\endgroup
\label{chap:Strongholds and Dominions}
\chapterimage[Strongholds and Dominions (Bordered)]
\thispagestyle{plain}

\begin{multicols*}{2}
When characters have come to the attention of the rulers of their land, usually by performing the sort of deeds that only heroes can manage, they may be granted titles of nobility and land grants.

This will vary from campaign to campaign depending on the preferences of the players and the Game Master. As a rough guideline, it should happen when the party are somewhere between \nth{9} and \nth{15} level—although some groups or some individual players may wish their characters to continue the life of a traveling adventurer rather than taking on the responsibility of ruling.

The area of land ruled by a noble (whether a player character or otherwise) is called a dominion. This applies whether or not the noble is given their title by a ruler or the noble strikes out on their own and simply claims land and assumes a title. A single dominion consists of a stronghold and all the surrounding land that is ruled from and protected by the stronghold. If a ruler had more than one stronghold (except for when one is simply a garrison) then each one and its land is considered a separate dominion.

\section{Titles of Nobility}\index[general]{Titles of Nobility}
Although there may be local and cultural variations specific to parts of the campaign setting, this chapter assumes that the following titles of nobility are in use (in ascending order of rank).

\textbf{Knight:} A knight is the lowest title of nobility. A grant of knighthood does not come with a dominion, and a knight does not normally rule such land. The children of nobles of all non-royal blood are normally knighted as a matter of course when they come of age, receiving a greater title when (or if) they gain their inheritance.

The title “Knight” applies to both sexes, although it is not used in the name of the noble. Instead, male knights are referred to as “Sir (Name)” and female knights are referred to as “Dame (Name)”. Male knights are addressed as “Sir” and female knights are addressed as “Ma’am”.

\textbf{Baron:} A grant of baroncy comes with a single dominion. It is the lowest form of landed nobility, and may be granted by a count or higher. A baron is granted a single domain, which is called a barony.

The title “Baron (Name)” is used for male barons, and “Baroness (Name)” is used for female barons. Male barons are addressed as “Your Lordship”, and female barons are addressed as “Your Ladyship”.

\textbf{Viscount:} If a baron is granted a second dominion, or conquers a second dominion belonging to their ruler’s enemy and adds it to their ruler’s country, they will likely be granted the title of viscount. This title and all higher titles may only be granted by a duke or higher. The viscount may rule directly from one of the strongholds and allow an allied baron to rule the other, or may remain at court and allow allied barons to rule both their dominions.

A viscount who is granted more dominions by their ruler or who gains them by expanding into unclaimed land remains a viscount. Viscount is therefore the highest possible title for (non-royal) nobility who are yet to prove themselves as war leaders.

The title “Lord (Name)” is used for male viscounts, and “Lady (Name)” is used for female viscounts. Male viscounts are addressed as “Your Lordship” and female viscounts are addressed as “Your Ladyship”.

\textbf{Count:} A viscount who conquers a third dominion belonging to one of their ruler’s enemies and adds it to their ruler’s country will be granted the title of count. A count will therefore rule at least three dominions, at least one of which was won by military conquest. Note that the “enemy” may not necessarily be a human country—but a distinction is made between the military capture of a domain formerly ruled by an organized goblin nation, and the annexing by an adventuring party of terrain formerly populated by a few sparse orc tribes, for example. The dominions ruled by a count are collectively referred to as a county.

Like a viscount, a count may delegate the rule of some or all of their dominions to barons, and may remain at court. A count may grant the title of baron, although etiquette demands that this is normally not done without at least checking with the king or queen first.

The title “Count (Name) of (County)” is used for male counts, and the title “Countess (Name) of (County)” is used for female counts. Male counts are addressed as “Your Lordship” and female counts are referred to as “Your Ladyship”.

\textbf{Marquis:} A count who continues to add dominions to their ruler’s lands (either by conquest or expansion into unclaimed land) may be given the title of Marquis. Although a marquis has greater station and influence at court than a “mere” count, there is little practical difference between the two.

The title “The Marquis of (County)” is used for male marquises and the title “The Marquise of (County)” is used for female marquises. Male marquises are addressed as “Your Lordship” and female marquises are addressed as “Your Ladyship”.

\textbf{Duke:} The highest rank of non-royal nobility is the Duke. The title of duke is granted to marquises who have served their ruler well and continued to add dominions to their rule. The collected dominions of a duke are referred to as a “duchy”. It is rare (but not unheard-of) for a duke to actually reside in one of the strongholds in their duchy. Dukes normally stay at court except in times of war or emergency. A duke may grant any lesser title (assuming the candidate has achieved the necessary status), although etiquette demands that the king or queen be informed before such grants happen.

The title “The Duke of (Duchy)” is used for male dukes, and “The Duchess of (Duchy)” is used for female dukes. Both male and female dukes are addressed as “Your Grace”.

\textbf{Archduke:} The title of Archduke is given to members of the royal family who are also dukes.

Some archdukes may be minor royalty who were originally granted baronies and have genuinely worked to gain their dominions, but others may be princes or other high ranking royals who have been granted archdukedoms for reasons of nepotism rather than ability.

While archdukes appointed via royal fiat do technically have the status and responsibility that their position entails, they are often not taken seriously and are sidelined when it comes to important councils of war and so forth. However, it can be dangerous to underestimate them completely. After all, they are close relatives of the king or queen, and do have their ears; as well as an array of lesser nobles (and their armies) at their disposal.

It is possible (although rare) for someone to first become a duke and then be adopted into the royal family to become an archduke. On very rare occasions, this will happen to an enemy duke who “defects” from their former ruler bringing their lands with them. Such defections cause much political turbulence, and can often be the start of major wars.

The title “The Duke of (Duchy)” is used for male archdukes, and “The Duchess of (Duchy)” is used for female archdukes. Both male and female archdukes are addressed as “Your Grace”.

\textbf{Prince:} A prince is the child of a king or queen (or emperor), or the child of a crown prince or imperial prince. This may be by birth, or it may be by adoption—such as the adoption that turns a duke into an archduke.

A prince will always be at least a baron, although in many cases their title is a technicality and no actual domains are ruled. However, many rulers encourage their offspring to go out and actively rule—partly to keep them occupied and reduce internecine squabbling, and partly to “toughen them up” and get them used to responsibility.

A prince may not appoint lesser nobles unless entitled to do so by their own rank of nobility. The collective dominion held by a prince is called a principality.

The title “Prince (Name) is used for male princes, and “Princess (Name)” is used for female princes. Both male and female princes are addressed as “Your Highness”.

\textbf{Crown Prince:} A crown prince is a prince who is the heir to the throne. A crown prince is almost always a prince by birth. Only if a king or queen is childless would it be possible to adopt someone as crown prince without causing outright rebellion amongst nobility.

Although it would seem that crown princes would be the most pampered of princes, the opposite is true. Because of their future responsibilities they tend to have the most preparation—and are therefore the most likely to work their way up from baroncy rather than just have higher titles awarded to them.

Being a crown prince is a risky business, since there are often many other potential heirs keen to see one fail or die, as well as foreign powers keen to cause internal strife.

The title “Crown Prince (Name) is used for male princes, and “Crown Princess (Name)” is used for female princes, although occasionally (depending on the rules of succession in the country in question) only princes of one sex will be eligible to be crown princes. Both male and female crown princes are addressed as “Your Royal Highness”.

\textbf{Imperial Prince:} An imperial prince is a prince who is the heir to an empire. An imperial prince is almost always a prince by birth. Only if an emperor or empress is childless would it be possible to adopt someone as an imperial prince without causing outright rebellion among the nobility.

Like crown princes, imperial princes tend to be active nobles. Because of their future responsibilities they tend to have the most preparation—and are therefore the most likely to work their way up from baroncy rather than just have higher titles awarded to them.

Being an imperial prince is even more dangerous than being a crown prince, since there are often many more interested parties keen to see one fail or die.

The title “Imperial Prince (Name) is used for male princes, and “Imperial Princess (Name)” is used for female princes, although occasionally (depending on the rules of succession in the empire in question) only princes of one sex will be eligible to be imperial princes. Both male and female imperial princes are addressed as “Your Imperial Highness”.

\textbf{King:} A king is the ruler of an entire country. The title is passed down in an hereditary manner, and therefore the only way to become a king (if one is not already heir to a throne) is to declare yourself the king of an area and get away with it by having enough military and political support.

In some countries, being the spouse of a king or queen makes one a king or queen yourself. This depends upon the rules of succession for the country in question.

In theory, a king has absolute power over their country. In practice, however, the king is reliant on the income and military might provided by their nobles; and it therefore is a foolish king indeed that does not take advice from those nobles.

The title “King (Name)” is used for male kings, and the tile “Queen (Name)” is used for female kings. Both male and female kings are addressed as “Your Majesty”.

\textbf{Emperor:} An emperor is the king of a country that has taken over (and had surrender to it) one or more other countries; but rather than simply expand to cover the whole area, the countries that have been taken over are allowed to remain autonomous (although sometimes their kings are deposed and replaced with more friendly kings). These countries become client countries as part of a larger empire, and the king of the country that is doing the taking over becomes the emperor of the entire empire.

The heirs of an emperor will take on the title of emperor themselves without necessarily taking over more countries than the founder of the empire did.

The kings of the client countries mostly run their states as before, although they may have policies dictated by the emperor—particularly foreign policies, and they may have to pay a yearly tribute to the emperor.

Depending on the size of the tribute and the policies imposed, the empire may be seen as a useful and benevolent thing to have (especially if it brings peace between previously antagonistic countries that have now become client states), or it may be seen as an oppressive force that the client countries would overthrow if they could.

In many cases, client countries lose control of their armies and find them replaced by a single “Imperial Army” which is paid by—and loyal to—the empire.

Empires have a tendency to be too big to be stable, and rarely last more than three or four generations before disintegrating; although the disintegration often causes utter chaos and anarchy and the former kingdoms that made up the empire rarely survive the break up.

The title “Emperor (Name)” is used for male emperors, and “Empress (Name)” is used for female emperors. Both male and female emperors are addressed as “Your Imperial Majesty”.

\section{Rogue States}\index[general]{Rogue States}
Of course, it is entirely possible for player character (or non-player character) to ignore this whole hierarchy and simply claim some area of wilderness and proclaim themselves the ruler of it using whatever title they feel like. Depending on the location they choose and the title they adopt, this may be met with anything from indifference to derision to downright hostility by other local rulers.

While it may be attractive to not have a ruler to pay \iref[Salt Tax]{Salt Tax} to, the independent dominion runs the constant risk of invasion—not only by monsters (since it has no allies to back it up) but also by neighboring countries who may wish to add the land to their own.

While some lucky independents—usually those in the most isolated areas away from other states—are able to grow from a single stronghold to a whole country, the vast majority soon become part of a neighboring country; either by being taken over militarily or by the political expedient of the independent ruler accepting a title of nobility from the royalty of a nearby country and swearing allegiance to that country in order to avoid a war they cannot win.

And, of course, some simply disappear; struck by plague or famine or worse.

However, claiming wilderness and declaring oneself to be an independent ruler is always an option for a particularly desperate or adventurous character who wishes to own a dominion without having to impress someone else enough to grant them one.

\section{Building a Stronghold}\index[general]{Building a Stronghold}
No dominion can survive without a stronghold of some sort. The stronghold provides not only an administrative center for the dominion, but also a secure place to store the dominion’s wealth and to retreat to in times of war.

The area of land covered by a dominion is measured in fiefs. A single fief is an area of about 12 miles radius.

Usually, a dominion will consist of a single fief, with the stronghold roughly in the center so that no point is too far away for easy access.

A large stronghold with several external troop garrisons can increase the effective dominion to anything up to seven fiefs (one containing the stronghold and another six surrounding it). However, the increased travel time needed for either troops to get from the stronghold to an outlying village or for the villagers from that village trying to seek refuge in the stronghold limits the maximum size of the dominion to no larger than this.

If someone wishes to clear out more land and enlarge their dominion beyond this size than they must build another stronghold to protect the newly cleared land—and this then becomes the center of a second dominion.

Before a stronghold can be built, the surrounding area must first be cleared of monsters that would threaten the builders. This job is ideally suited to adventuring parties.

Once the area is clear, the stronghold itself can be designed and built.

\subsection{Costs and Time}
\fullref{tab:Stronghold Elements} lists the costs of various stronghold elements. These prices all include the cost of the unskilled and semi-skilled labor that does the building, but do not include the cost of skilled architects and engineers.

Strongholds take one day per 500 gp (or part of 500 gp) total cost, and need one engineer on site to oversee the building process per 100,000 gp (or part of 100,000 gp) total cost.

The listed costs assume that the stronghold is being built as an outpost in a remote but not inaccessible region. If a stronghold is being built in an inaccessible region, double the costs—and if a stronghold is being built in a heavily settled region halve the costs.

Often a stronghold will require features of custom size, for example larger doors than normal. The prices of these features should be based on the standard prices for similar features and increased or decreased proportionally.

\begin {table}[H]
  \caption{Stronghold Elements}\label{tab:Stronghold Elements}
  \begin{tabularx}{\columnwidth}{>{\bfseries}YY}
	\thead{Item} & \thead{Cost}\\
	Arrow Slit & 10 gp\\
	Barbican & 37,000 gp\\
	Battlement (100 ft.) & 500 gp\\
	Building, Stone & 3,000 gp\\
	Building, Wood & 1,500 gp\\
	Door, Exterior Iron/Stone & 100 gp\\
	Door, Interior Iron/Stone & 50 gp\\
	Door, Interior Reinforced & 20 gp\\
	Door, Interior Wood & 10 gp\\
	Door, Secret & Cost x5\\
	Drawbridge & 250 gp\\
	Dungeon Corridor & 500 gp\\
	Floor, Flagstone & 100 gp\\
	Floor, Tile & 100 gp\\
	Floor, Wood & 40 gp\\
	Gate, Wooden & 1,000 gp\\
	Gatehouse & 6,500 gp\\
	Keep, Square & 75,000 gp\\
	Moat, Filled & 800 gp\\
	Moat, Unfilled & 400 gp\\
	Shifting Wall & 1,000 gp\\
	Shutters, Window & 5 gp\\
	Staircase, Stone & 60 gp\\
	Staircase, Wood & 20 gp\\
	Tower, Bastion & 9,000 gp\\
	Tower, Round Large & 30,000 gp\\
	Tower, Round Small & 15,000 gp\\
	Trap Door & Cost x2\\
	Wall, Castle & 5,000 gp\\
	Wall, Wood & 1,000 gp\\
	Window, Barred & 20 gp\\
	Window, Open & 10 gp
  \end {tabularx}
\end {table}

\textbf{Arrow Slit:} A narrow window designed to let defenders shoot out whilst not exposing them to returning fire.

\textbf{Barbican:} A pair of 30 feet tall x 20 feet wide towers flanking a 20-square-foot gatehouse built as a single unit. Price includes iron portcullis.

\textbf{Battlement (100 ft.):} 100 feet of crenelated wall with a parapet behind it. The price only includes the crenelations and parapet, not the wall that the battlement is on.

\textbf{Building, Stone:} A two story stone building, such as a large stone house, stables, or an inn.

\textbf{Building, Wood:} A two story wooden building, such as a large wooden house, stables, or an inn.

\textbf{Door, Secret:} A door that is disguised and hidden so that it will not be noticed unless searched for.

\textbf{Door, Exterior Iron/Stone:} A heavy exterior double-door, 7 feet tall by 6 feet wide.

\textbf{Door, Interior Iron/Stone:} A heavy internal door, 7 feet tall and 3 feet wide.

\textbf{Door, Interior Reinforced:} A wooden internal door reinforced with iron bands, 7 feet tall and 3 feet wide.

\textbf{Door, Interior Wood:} A standard wooden internal door, 7 feet tall and 3 feet wide.

\textbf{Drawbridge:} A 10-foot-wide, 20-foot-long reinforced wooden bridge that can be raised or lowered.

\textbf{Dungeon Corridor:} A 10-by-10-by-10-foot section dug out from rock. The cost is multiplied by the depth of the dungeon (in multiples of 50 feet), for example digging a 10-by-10-by-10-foot section at a depth of 150 feet will cost triple the listed price: 1,500 gp.

\textbf{Floor, Flagstone:} A 10-by-10-foot section of floor covered in flagstones.

\textbf{Floor, Tile:} A 10-by-10-foot section of floor covered in tiles.

\textbf{Floor, Wood:} A 10-by-10-foot floor covered in polished fitted wood.

\textbf{Gate, Wooden:} A 20-foot-tall-by-10-foot-wide wooden gate, reinforced and barred, suitable for putting in a stockade wall.

\textbf{Gatehouse:} A 30-foot-high building 20 by 20 feet in area. Price includes iron portcullis.

\textbf{Keep, Square:} A heavily reinforced stone building 80 feet tall and 60 by 60 feet in area.

\textbf{Moat, Filled:} 100-foot length of 10-foot deep, 20 feet wide canal.

\textbf{Moat, Unfilled:} 100-foot length of 10-foot deep, 20 feet wide ditch.

\textbf{Shifting Wall:} A 10-by-10-foot wall which has a counterbalance and mechanism for moving it.

\textbf{Shutters, Window:} Window shutters that provide little military defense, but protect against bad weather.

\textbf{Staircase, Stone:} A stone staircase 3 feet wide with a 10-foot ascent.

\textbf{Staircase, Wood:} A wooden staircase 3 feet wide with a 10-foot ascent.

\textbf{Tower, Bastion:} A half-circle tower, 30 feet tall and 30 feet diameter.

\textbf{Tower, Round Large:} A 30-foot tall, 30-foot diameter round tower.

\textbf{Tower, Round Small:} A 30-foot tall, 20-foot diameter round tower.

\textbf{Trap Door:} A 5-by-5-foot section of false floor with an opening mechanism that allows it to drop anyone standing on it through a hole in the floor.

\textbf{Wall, Castle:} 100-foot length of 20-foot tall and 5-foot thick reinforced stone wall, with a walkway and battlements on the top.

\textbf{Wall, Wood:} 100-foot length of 20-foot tall and 5-foot thick reinforced wooden wall, with a walkway on the top.

\textbf{Window, Barred:} A 3-by-1-foot window with bars to prevent ingress and egress.

\textbf{Window, Open:} A 3-by-1-foot open window.

\section{Terrain and Resources}
In order to determine the resources available to a dominion, the terrain of each fief must be determined.

Each fief is classified as either Civilized, Borderlands or Wilderness, according to \fullref{tab:Dominion Fief Classification}, depending on the terrain type of the fief and how close it is to a major city or to other civilized fiefs. Note that the other civilized fiefs don’t necessarily need to belong to the same dominion or even the same country, as long as there are trade links between them and the fief in question (which will usually be the case, barring embargoes).

\begin {table}[H]
  \caption{Dominion Fief Classification}\label{tab:Dominion Fief Classification}
	\begin{tabularx}{\columnwidth}{>{\bfseries}cYM{1in}Y}
	\thead{Terrain Type} & \thead{Within 144 miles of a city} & \thead{More than 144 miles from a city but within 72 miles of a Civilized fief} & \thead{Not near a city or Civilized fief}\\
	Arctic &  Borderlands & Wilderness & Wilderness\\
	Barren Lands & Borderlands & Wilderness & Wilderness\\
	Clear* & Civilized & Borderlands & Wilderness\\
	Desert & Borderlands§ & Wilderness & Wilderness\\
	Forest† & Civilized & Borderlands & Wilderness\\
	Hills* & Civilized & Borderlands & Wilderness\\
	Jungle† & Borderlands & Wilderness & Wilderness\\
	Mountains‡ & Borderlands & Wilderness & Wilderness\\
	Ocean & Wilderness & Wilderness & Wilderness\\
	Settled & Civilized & Civilized & Borderlands\\
	Swamp & Borderlands & Wilderness & Wilderness\\
	Woods* & Civilized & Borderlands & Wilderness
  \end {tabularx}
	*Fiefs of this type can become Settled if populated by anyone\\
	†Fiefs of this type can become Settled if populated by elves\\
	‡Fiefs of this type can become Settled if populated by dwarves\\
	§Fiefs containing oases are considered to be Civilized*
\end {table}

This civilization level of the fief determines both the number of families that will be attracted to settle the area when the stronghold is built and also the maximum number of families that the fief can support. See \fullref{tab:Civilization Levels} for details.

\begin {table}[H]
  \caption{Civilization Levels}\label{tab:Civilization Levels}
  \begin{tabularx}{\columnwidth}{>{\bfseries}YYY}
	\thead{Level} & \thead{Settling Families} & \thead{Max Families}\\
	Wilderness & 1d10x10 & 1,500\\
	Borderlands & 2d6x100 & 3,000\\
	Civilized & 1d10x500 & 6,000
  \end {tabularx}
\end {table}

\subsection{Settled Terrain}
Any Clear, Forest, Hills or Woods fief that is has over 1,000 families living in it is considered to be of terrain type Settled rather than its basic terrain type.

Mountains fiefs with over 1,000 families become Settled only if populated by dwarves, and Forest and Jungle fiefs with over 1,000 families become Settled only if populated by elves.

This has two effects. Firstly, settled terrain uses different columns on wilderness encounter tables than other terrain types (see \fullref{sec:Wilderness Encounters}). Secondly, the change to settled terrain may change the civilization level of the fief, with a corresponding increase in the maximum number of families that the fief can contain.

The change in civilization level of the fief may have a knock-on effect on other nearby fiefs, since they may now be within 72 miles of a civilized fief.

\example{One of Lady Gretchen’s dominions consists of a castle and its fief. The entire area is Mountains, and is well away from other civilized lands. When the castle is first built, the fief is therefore at the Wilderness level of civilization.

After a few years, the population of her fief grows to 1,033 families. Because Lady Gretchen’s people are primarily dwarves, that fief is now considered to be Settled rather than Mountains, and therefore becomes Borderlands and can support a higher population.

Another of Lady Gretchen’s dominions is in the hills closer to the rest of the kingdom. It is also not within 144 miles of a city or within 72 miles of a Civilized fief, but two of the fiefs adjacent to it are within 72 miles of a Civilized fief.

Those two fiefs (which belong to allied barons) are therefore considered to be Borderlands Hills and Lady Gretchen’s fief is considered to be Wilderness Hills.

After a few years, one of the adjacent Borderlands fiefs reaches 1,014 families. It is now considered to be Settled terrain rather than Hills. This changes the civilization level of the fief to Civilized.

This change in civilization level means that Lady Gretchen’s fief is now within 72 miles of a Civilized fief, and its civilization level is now upgraded from Wilderness to Borderlands accordingly.}

Any fief that loses enough population that it no longer has 1,000 families also loses its Settled type, and reverts back to its normal terrain type. Again, this may have a knock-on effect on other fiefs; which may no longer be within 72 miles of a Civilized fief, and therefore may drop in civilization level themselves.

Should this cause the maximum population of a fief to drop below its current population, the population of that fief will reduce by 20\% per month until it is no longer unsupportable.

\section{Material Resources}\index[general]{Material Resources}
Each fief of the dominion will produce between one and four resources that must be exploited to generate income for the dominion, determined randomly by consulting \fullref{tab:Material Resources Quantity}.

\begin {table}[H]
  \caption{Material Resources Quantity}\label{tab:Material Resources Quantity}
  \begin{tabularx}{\columnwidth}{>{\bfseries}YY}
	\thead{1d10} & \thead{Quanity}\\
	1 & 1 resource\\
	2-7 & 2 resources\\
	8-9 & 3 resources\\
	10 & 4 resources
  \end {tabularx}
\end {table}

Although the actual resources available can be very varied, for game purposes they are simply split into three categories: animal, vegetable and mineral.

For each resource found, the type is determined randomly as indicated on \fullref{tab:Material Resources Type}.

\begin {table}[H]
  \caption{Material Resources Type}\label{tab:Material Resources Type}
  \begin{tabularx}{\columnwidth}{>{\bfseries}YY}
  \thead{1d10} & \thead{Type}\\
	1d10 & Resource\\
	1-3 & Animal\\
	4-8 & Vegetable\\
	9-10 & Mineral
  \end {tabularx}
\end {table}

The Game Master or players may wish to go into further detail about exactly what types of resources these are; for example a mineral resource could be a gold seam or a source of strong stone for building or a source of fine clay or any one of dozens of other types of mineral. This detail may enhance role playing, particularly if the players like doing trade negotiations, but it does not affect the dominion rules. In the example above, although gold is far more expensive than building stone, there will also be far less of it and the relative income for a gold mine or a quarry in a fief will be similar.

\example{When Lady Gretchen was granted her land and built her castle, the Game Master rolled for resources for the fief. He rolled that the fief had three resources: two mineral and a vegetable. Jim discussed what those three resources could be with the Game Master, and between them they decided that there was a silver seam that could be mined, a source of granite that could be quarried, and—because the mountain fief is in a warm region and on the edge of the mountain range—olive groves in the valleys and foothills.}

\section{Ruling a Dominion}\index[general]{Ruling a Dominion}
In Dark Dungeons, ruling of a dominion takes place in the timescale of months and years, dropping down to a day-to-day basis only during unusual situations.

To be specific, the population change and the economy (the income and expenditure for the dominion) are handled on a monthly basis, and the level of satisfaction—or unrest—of the populace is usually handled on a yearly basis but may need to be checked in exceptional circumstances.

\subsection{Population Change}\index[general]{Population Change}
Each month, the number of families in each fief of the dominion will change due to a variety of factors. Rather than try to account for each individual factor, Dark Dungeons abstracts the whole population change for the month into a single check.

For each fief, the basic population change is based on the existing population of the fief as indicated on \fullref{tab:Population Change}.

\begin {table}[H]
  \caption{Population Change}\label{tab:Population Change}
  \begin{tabularx}{\columnwidth}{>{\bfseries}YY}
	\thead{Families} & \thead{Population Change}\\
	1-100 & +25\%\\
	101-200 & +20\%\\
	201-300 & +15\%\\
	301-400 & +10\%\\
	401-500 & +5\%\\
	501-750 & +3\%\\
	750-1,000 & +2\%\\
	1,001+ & +1\%
  \end {tabularx}
\end {table}

In addition to this percentage increase, each fief with fewer than 250 families may randomly lose or gain families as indicated on \fullref{tab:Family Change}.

\begin {table}[H]
  \caption{Family Change}\label{tab:Family Change}
  \begin{tabularx}{\columnwidth}{>{\bfseries}YY}
	\thead{1d6} & \thead{Population Change}\\
	1-3 & Lose 1d10 families\\
	4-6 & Gain 1d10 families\\
  \end {tabularx}
\end {table}

In the case of more populous fiefs, these small changes are simply assumed to be irrelevant compared to the normal population growth.

\example{When Lady Gretchen builds her castle, the fief is wilderness. Therefore, it attracts 1d10x10 families as settlers. Jim rolls a 7, so 70 families settle the fief.

After a month, Jim checks the fief for population growth. There are less than 100 families, so there is a 25\% increase, making 94 families. Additionally, because there are less than 250 families in the fief, Jim rolls a d6 to see what the random fluctuation is. He rolls a 6, which is good news because it means that there is a further increase in population of 1d10 families, but is disappointed when he then only rolls a 2 on the 1d10. Two extra families arrive, making a total of 96 families at the start of month two.}

\subsection{Monthly Economy Check}
Each game month, the ruler of the dominion, along with the Game Master, needs to check the economy and tally up the income and expenditure for the month.

\subsection{Income}\index[general]{Income}
Monthly income comes from four sources:

\textbf{Resources:} Each fief of the dominion will have between 1 and 4 types of resource in it. These resources provide income for the dominion ruler based on their resource type as indicated on \fullref{tab:Income}.

\begin {table}[H]
  \caption{Income}\label{tab:Income}
  \begin{tabularx}{\columnwidth}{>{\bfseries}YY}
	\thead{Resource} & \thead{Income}\\
	Animal & 2 gp/family\\
	Vegetable & 1 gp/family\\
	Mineral & 3 gp/family
  \end {tabularx}
\end {table}

Each family within the fief may work on a single resource within the fief.

The ruler of the dominion may simply let the populace split themselves evenly between the available resources, or may direct the populace to concentrate on exploiting a particular resource.

However, doing so is subject to a few limitations.

Firstly, given the infrastructure needed to exploit a particular resource (animals need breeding, crops need sowing, mines need digging), the ruler of a dominion can only change the emphasis once per year. The ruler must decide what their priorities will be at the beginning of each year, and the actual change to those new priorities will happen at the beginning of the following year. When doing so, it is convenient to assign priorities in terms of percentages of families rather than in absolute numbers of families, since the total number of families in the fief will change from month to month.

Secondly, the populace must work all the resources in the fief for the local economy to thrive and for the populace to be content. In particular, forcing too much of the population to work in dangerous and unhealthy mines makes the ruler very unpopular.

In game terms, each resource must be worked by at least 20\% of the families in the fief. For each 1\% below that threshold in a year, there is a cumulative -1 penalty to the dominion’s Confidence Rating. Similarly, no more than 50\% of the families in the fief should be made to exploit mineral wealth. For each 1\% above that threshold in a year, there is a cumulative -1 penalty to the dominion’s Confidence Rating.

Thirdly, any fief that brings in a monthly revenue of 15,000 gp or more will attract corruption, black markets and bandits. Unless that fief contains the stronghold from which the dominion is ruled, 1d10x10\% of the potential resource income will be lost to such forces.

\textbf{Service:} Each family in the dominion brings in the equivalent of income worth 10 gp per month in service, such as building works, growing food, tending animals, and so forth.

Unlike other sources of income, this is not actually received by the ruler of the dominion as money. However, it can be used to offset expenses such as holidays, tithes, \iref[Salt Tax]{Salt Tax}, and the paying of armies (mercenary or otherwise). Any service income that is not used is wasted and cannot be stored.

\textbf{Poll Tax:} Each family in the dominion normally pays 1 gp per month in poll tax. This is actual money-in-the-coffers tax paid in coinage.

The ruler of the dominion can set the tax rate higher or lower if they desire. For each extra 5sp that is paid per family, there is a -10 penalty to the dominion’s Confidence Rating per year. For each 5sp less that is paid per family, there is a +5 bonus to the dominion’s Confidence Rating per year.

Additionally, when the ruler increases the tax rate, this gives an instant -25 penalty to the dominion’s Confidence Rating and forces an immediate confidence check. Similarly, decreasing the tax rate gives an instant +10 bonus to the dominion’s Confidence Rating.

\textbf{Salt Tax:}\label{sec:Salt Tax} If the ruler of the dominion has other nobles who have sworn fealty to them, they are given 20\% of the total income of each lesser noble’s dominion.

This income is normally paid in the form of services, and therefore doesn’t actually arrive as coinage. However, like other service income it can be used to offset expenditure. Like service income, this income cannot be stored, and must be used or wasted.

\example{In the fief containing her castle, Lady Gretchen has assigned 25\% of the families to work in the silver mine, 25\% of the families to work in the granite quarry, and 50\% of the families to work in the olive groves. Since she has at least 20\% of the population working on each resource and she does not have more than 50\% of the population working on mineral resources there is no effect on her dominion’s Confidence Level.

After a few years of growth, there are 447 families living in the fief. Splitting these families into the different resources (with some rounding) gives:

447x25\% = 112 families mining silver

447x25\% = 112 families quarrying granite

447x50\% = 223 families farming olives

Therefore, the resource income for the fief in the first month of that year is: (112x3)+(112x3)+(223x1) = 895 gp

The service income of the fief is simply ten times the population, which is: (10x447) = 4,470 gp

Lady Gretchen has not set taxes higher or lower than the 1 gp/family, so in Poll Tax she receives: (1x447) = 447 gp

Therefore, for this fief, Lady Gretchen receives a total of 1,342 gp in cash and 4,470 gp in services that can offset expenditure.

The Game Master then instructs Jim to add 920 gp of extra service income for the Salt Tax paid to Lady Gretchen by the baron who looks after her second dominion.}

\subsection{Expenditure}\index[general]{Expenditure}
\textbf{Castle Staff and Maintenance:} With the exception of armies, which must be accounted for, the cost of castle staff and routine maintenance is assumed to already be covered by the service income of the dominion.

However, extraordinary expenses such as rebuilding works in the wake of a siege or a monster attack must be paid for out of the ruler’s pocket. Service income may be used to pay for these expenses.

\textbf{Troops:} Whether a full time standing army, a “special forces” unit of adventurers, or a group of mercenaries; troops must be paid for.

Armies and mercenaries can be paid for with service income, based on their costs in \fullref{sec:Mercenaries} but adventurers usually only work for cold hard cash.

In times of dire need, a peasant militia can be formed from the local populace.

Up to 10\% of the families in an area can provide “poor” quality peasant militia (providing an average of 2.5 troops per family). A further 10\% of the families in an area can provide “untrained” quality peasant militia (providing an average of 2.5 troops per family).

If either are called up, the families providing militia will not produce income of any type during the months in which the militia is active.

\textbf{Tithes:} One tenth of all gross income (income before any expenditure has been taken out) must be given in tithes to the various churches and temples that are worshiped throughout the dominion.

Tithes may be paid with either service income or money, or a combination of the two.

Failure to provide the full amount of tithes results in the churches (and possibly \iref[chap:Immortals]{Immortals} associated with them) being angered, and they make their anger known to the populace.

The net result of this is that any year in which tithes are not paid in full gives a -50 penalty to the dominion’s Confidence Rating.

If tithes are short-changed for more than one year in a row, there is a 25\% chance each year that an extra “Disaster” event will happen that year as the \iref[chap:Immortals]{Immortals} show their displeasure. If such an event is going to happen, it will be preceded by omens and prophetic dreams.

\textbf{Salt Tax:} In just the same way that the dominion may receive salt tax from subservient dominions, it must also pay twenty percent of its gross income (income before any expenditure has been taken out) to the noble or royal that the ruler of the dominion has sworn fealty to.

Salt tax may be paid with either service income or money, or a combination of the two.

\textbf{Festivals and Holidays:} Some days during the year are declared as festivals or holidays. These may have been declared by the ruler of the country, or by one of the major religions of the country, or the ruler of the dominion may declare their own.

The overall cost of a holiday is 5 gp per family. This represents both the expenditure for celebrations and also the lost income because people are not working. This cost may be paid with either service income or money, or a combination of the two.

If the holiday was one declared by the churches, its cost can be recouped from the tithes paid to the church. Similarly, if the holiday was one declared by the ruler of the country, its cost can be recouped from the salt tax paid to that ruler.

However, if the cost of the holiday is too great to be covered by the tithes or salt tax (or if the holiday was declared by the dominion ruler rather than by a higher power) the dominion ruler must pay the remaining cost themselves.

Each time a regular holiday or festival that the populace are expecting is canceled, a -5 penalty is applied to the dominion’s Confidence Rating, and an immediate confidence check must be made.

Each time an extraordinary holiday or festival day is announced, a +2 bonus is applied to the dominion’s Confidence Rating.

\textbf{Entertaining Visitors:} Etiquette demands that visiting nobles and royalty are entertained according to their station.

The costs on \fullref{tab:Entertaining Visitors} apply whenever a noble (and their retinue) are visiting.

\begin {table}[H]
  \caption{Entertaining Visitors}\label{tab:Entertaining Visitors}
  \begin{tabularx}{\columnwidth}{>{\bfseries}YYYYY}
	\thead{Visitor} & \thead{Cost}\\
	Knight & No extra cost\\
	Baron & 100 gp/day\\
	Viscount & 150 gp/day\\
	Count & 300 gp/day\\
	Marquis & 400 gp/day\\
	Duke & 600 gp/day\\
	Archduke & 700 gp/day\\
	Prince & As nobility + 100 gp/day\\
	King & 1,000 gp/day\\
	Emperor & 1,500 gp/day
  \end {tabularx}
\end {table}

\example{Lady Gretchen has a total income from her dominion for the month (including salt tax from her second dominion) of 1,342 gp in cash, and 5,390 gp worth of services.

Firstly, she takes 30\% of that out (20\% in salt tax to the queen and 10\% in tithes to the church). She ends up paying 1,346 gp to the queen and 673 gp to the church for a total of 2,018 gp.

The 2,018 gp is all paid out of service income, leaving her with: 5,390-2,018 = 3,372 gp left.

There was a religious festival for one day during the month. Since the population of her dominion is 447 families, this costs her: 447x5 = 2,235 gp

Since this was a religious festival, she can use tithes to help fund it. She should be paying 673 gp in tithes, so that leaves: 2,235-673 = 1,562 gp to pay.

She pays the 1,562 gp out of her remaining service income, leaving her with: 3,372-1,562 = 1,810 gp left

Out of this 1,810 gp, she pays for her standing army consisting of 300 heavy dwarven infantry (costing 5 gp each per month) and 100 dwarven crossbowmen (costing 6 gp each per month).

These troops cost her: (300x5)+(100x6) = 2,100 gp

She can pay 1,810 gp of this wage bill using services, leaving her: 2,100-1,810 = 290 gp to pay

She pays the 290 gp out of her 1,342 gp cash income, leaving her: 1,342-290 = 1,052 gp

Having balanced her finances for the month, Lady Gretchen discovers that she has managed to pay most of her expenses out of service income, but she considers cutting back on troop numbers, since she’s having to dip into real cash in order to pay them.

She keeps the 1,052 gp of cash and puts it in her coffers.}

\subsection{Experience for Income}\index[general]{Experience for Income}
When calculating the amount of experience the ruler of a dominion gets from their monthly income, there are two rules that must be applied.

Firstly, only cash income (i.e. Material Resources and Poll Tax) provide experience points. Service income and income from \iref[Salt Tax]{Salt Tax} of lesser nobles does not provide experience points.

Secondly, experience points are derived from the gross income of cash (income before any expenditure has been taken out). Even if all the income is spent due to heavy expenditure and the ruler ends up making a net loss, they will still receive full experience.

\example{Lady Gretchen received 1,342 gp of cash and 5,390 gp of services this month. She gets experience for all of the cash even though she had to spend some of it, but does not get money for the services. Lady Gretchen therefore gains 1,342 XP this month.}

\subsection{Confidence Level}\index[general]{Confidence Level}
Each dominion has a Confidence Rating. This is a number that represents the general state of content (or discontent!) of the populace.

There is a single confidence rating for the whole dominion—different fiefs do not have separate ratings.

When a dominion is first established, the initial confidence rating is set to the sum of the ability scores of the ruler plus 150 plus an additional d100 roll.

In addition to the confidence rating, a dominion also has a confidence level. The confidence level is based on the rating, and periodically a “confidence check” is made. Whenever a confidence check needs to be made, look up the current confidence rating on \fullref{tab:Confidence Levels} and this will indicate the new confidence level.

It is important to remember that although the confidence rating may change frequently, the confidence level only changes when a confidence check is made—even if the rating moves into a different range between checks.

\subsubsection{Yearly Confidence Check}\index[general]{Yearly Confidence Check}
At the beginning of each year, the Game Master checks the current confidence rating on \fullref{tab:Confidence Levels} in order to determine the confidence level of the dominion.

This confidence check may also be required as a result of certain actions by the dominion ruler (e.g. when an expected holiday is canceled.) or as a result of a disaster striking the dominion.

\begin {table}[H]
  \caption{Confidence Levels}\label{tab:Confidence Levels}
  \begin{tabularx}{\columnwidth}{>{\bfseries}YY}
	\thead{Confidence Rating} & \thead{Confidence Level}\\
	49 or less & Turbulent\\
	50 to 99 & Belligerent\\
	100 to 149 & Rebellious\\
	150 to 199 & Defiant\\
	200 to 229 & Unsteady\\
	230 to 269 & Average\\
	270 to 299 & Steady\\
	300 to 349 & Healthy\\
	350 to 399 & Prosperous\\
	400 to 449 & Thriving\\
	450 or higher & Ideal
  \end {tabularx}
\end {table}

Descriptions of the various confidence levels and their effects on the dominion are given below:

\textbf{Average:} The dominion is running smoothly. There are no special conditions or effects.

\textbf{Belligerent:} In each fief that has fewer troops than one half of the number of families, half the families will form a peasant militia (providing an average of 2.5 troops per family).

No Poll Tax can be collected.

A quarter of normal service income can be collected in areas without a peasant militia, but none can be collected in areas with a peasant militia.

A quarter of normal resource income can be collected in areas without a peasant militia, but none can be collected in areas with a peasant militia.

A -10 penalty is applied to the confidence rating.

All trade caravans and traveling officials will be attacked by bandits.

Any of the dominion ruler’s troops that move or deploy within the dominion will be attacked by peasant militia, deserters, bandits or enemy agents.

There is a 50\% chance that an enemy state will provide the peasant militia with military support.

\textbf{Defiant:} In each fief that has fewer troops than one third of the number of families, half the families will form a peasant militia (providing an average of 2.5 troops per family). However, these militia will not attack unless provoked.

No Poll Tax can be collected.

A half of normal service income can be collected in areas without a peasant militia, but only a third can be collected in areas with a peasant militia.

A half of normal resource income can be collected in areas without a peasant militia, but only a third can be collected in areas with a peasant militia.

\textbf{Healthy:} All income is 10\% greater than normal.

There is a 25\% chance per agent that enemy agents working in the dominion will be exposed.

\textbf{Ideal:} All income is 10\% greater than normal.

There is a 75\% chance per agent that enemy agents working in the dominion will be exposed.

If a random check indicates that a disaster will occur during the coming year, there is a 25\% chance that it will not happen.

A +25 bonus is applied to the confidence rating.

The confidence rating cannot drop below 400 before the next confidence check.

\textbf{Prosperous:} All income is 10\% greater than normal.

There is a 25\% chance per agent that enemy agents working in the dominion will be exposed.

If a random check indicates that a disaster will occur during the coming year, there is a 25\% chance that it will not happen.

\textbf{Rebellious:} In each fief that has fewer troops than one third of the number of families, half the families will form a peasant militia (providing an average of 2.5 troops per family). However, these militia will not attack unless provoked.

No Poll Tax can be collected.

A third of normal service income can be collected in areas without a peasant militia, but only a quarter can be collected in areas with a peasant militia.

A third of normal resource income can be collected in areas without a peasant militia, but only a quarter can be collected in areas with a peasant militia.

A -10 penalty is applied to the Confidence Rating.

\textbf{Steady:} There is a 25\% chance per agent that enemy agents working in the dominion will be exposed.

\textbf{Thriving:} All income is 10\% greater than normal.

There is a 50\% chance per agent that enemy agents working in the dominion will be exposed.

If a random check indicates that a disaster will occur during the coming year, there is a 25\% chance that it will not happen.

\textbf{Turbulent:} 95\% of families will form a peasant militia (providing an average of 2.5 troops per family).

No income of any kind may be collected, except by force.

A -10 penalty is applied to the Confidence Rating.

The Confidence Rating cannot rise above 100 until the ruler of the dominion is removed.

All trade caravans and traveling officials will be attacked by bandits.

Any of the dominion ruler’s troops that move or deploy within the dominion will be attacked by peasant militia, deserters, bandits or enemy agents.

One or more enemy states will provide the peasant militia with military support.

\textbf{Unsteady:} There is a 20\% chance that a -10 penalty will apply to the Confidence Rating.

\section{Events}\index[general]{Events}
Each year, 1d4 random events will happen in the dominion.

Due to the huge variety of events that can occur, it is not possible to list them here. However, they can be roughly classified into types of event.

For each event that occurs, roll on \fullref{tab:Dominion Events} to determine the type of event.

Although this table is random, the Game Master should be fair to the players and should not let players’ dominions be wiped out by a few bad rolls which indicate disaster after disaster. If the dice seem to be against the players, then the Game Master should introduce plot elements or potential adventures into the game that can mitigate the worst situations. Similarly, if the dice are favoring the players and they are getting bored just raking in the money every month without challenge, the Game Master should introduce plot elements or adventures that can cause additional problems.

However, in either case the Game Master should be careful not to railroad the players and make them feel that the status quo is being forcibly maintained. The Game Master should make sure that the players’ decisions have a real impact on the way their dominions prosper or struggle.

\begin {table}[H]
  \caption{Dominion Events}\label{tab:Dominion Events}
  \begin{tabularx}{\columnwidth}{>{\bfseries}YYYYY}
	\thead{d100} & \thead{Event Type}\\
	01-05 & Major Positive Event\\
	06-20 & Minor Positive Event\\
	25-40 & Neutral Event\\
	41-75 & Minor Negative Event\\
	76-95 & Major Negative Event\\
	96-00 & Disaster
  \end {tabularx}
\end {table}

Types of event and their effects are listed below:

\textbf{Major Positive Event:} A major positive event will benefit the dominion greatly. It may result in a bonus to the Confidence Rating of up to +25, up to a doubling of income for a month, a population increase of up to +25\%, or some combination of the above. Depending on the nature of the event, the ruler may need to get involved personally in order to get the best results—but there should be some positive results even if the ruler does nothing.

\example[Examples]{New resource type found, ancient treasure found, An Immortal decides to become the patron of the dominion, A clan of demi-human refugees joins the dominion.}

\textbf{Minor Positive Event:} A minor positive event will benefit the dominion, or at the very least not harm it. It may result in a bonus to the Confidence Rating of up to +15, up to 50\% extra income for a month, a population increase of up to +15\%, or some combination of the above. The ruler may need to get involved personally in order to get the benefits—but there should be no negative results even if the ruler does nothing.

\example[Examples]{A new trade route is proposed, a hostile tribe of humanoids moves away from the dominion, Passing adventurers clear out local bandits without needing to be hired to do so, A druid moves into the area.}

\textbf{Neutral Event:} A neutral event may benefit the dominion or harm it, depending on how it is dealt with. It may result in a change to the Confidence Rating of up to +/-10, up to 25\% extra or less income for a month, a population change of up to +/-10\%, or some combination of the above. Whether the event works out positively or negatively should depend on how the ruler handles it.

\example[Examples]{A VIP visitor arrives unexpectedly, Comets or other omens are seen in the sky, heresy is discovered in a local church, a local tribe of humanoids is displaced by a different tribe.}

\textbf{Minor Negative Event:} A minor positive event will harm the dominion, or at the very least not benefit it. It may result in a penalty to the Confidence Rating of up to +15, up to 50\% less income for a month, a population decrease of up to -15\%, or some combination of the above. The ruler may need to get involved personally in order to avoid the harm—but there should be no significant positive results no matter how well the ruler handles the situation.

\example[Examples]{Bandits start raiding, an official is assassinated, low level wandering monsters arrive in the area, a disease breaks out.}

\textbf{Major Negative Event:} A major negative event will harm the dominion greatly. It may result in a penalty to the Confidence Rating of up to -25, up to 75\% less income for a month, a population decrease of up to -25\%, or some combination of the above. Depending on the nature of the event, the ruler may need to get involved personally in order to get the least bad results—but there should be some negative results no matter how well the ruler handles the situation.

\example[Examples]{One of the fief’s resources runs out, an epidemic strikes, a high level wandering monster enters the dominion, agitants foment rebellion against the ruler, a major fire breaks out.}

\textbf{Disaster:} A disaster event will harm the dominion greatly in a similar way to a major negative event. It may result in a penalty to the Confidence Rating of up to -25, up to 75\% less income for a month, a population decrease of up to -25\%, or some combination of the above. It will also result in an immediate confidence check. Depending on the nature of the event, the ruler may need to get involved personally in order to get the least bad results—but there should be seriously negative results no matter how well the ruler handles the situation.

\example[Examples]{Examples: An extremely high level monster attacks the dominion, plague strikes, a hurricane, tornado or avalanche sweeps the dominion, an earthquake strikes, an Immortal smites the dominion.}

\end{multicols*}

