\chapter[red]{Character Creation}
\index[general]{Character Creation}\label{chap:Character Creation}
\chapterimage[Character Creation (Bordered)]
\thispagestyle{plain}

\begin{multicols*}{2}
There are two types of characters, player characters (abbreviated to “PCs”) which are adventurers created and controlled by the players and non-player characters (abbreviated to “NPCs”) which are created and controlled by the Game Master (abbreviated to “Game Master”).

This chapter covers the process of creating a character. As each step is gone through, the results should be written down on a sheet of paper.

\section{Name and Concept}
The first thing to do is to decide what sort of character you want to play. Do you want to be a brave warrior or a stealthy rogue? Do you want to be male or female? Do you want to be dour and sullen or happy-go-lucky? Are you a paragon of virtue or a conniving schemer?

It’s important to decide what sort of character you want to play before picking up any dice, but also important to be flexible in such a concept and to talk to the Game Master first. If the Game Master is starting a game with first level characters (which is the usual way to start a game) then deciding you want to play an experienced swashbuckling pirate captain who is an expert fencer isn’t going to work. You would be better deciding you want to play someone who aspires to be a swashbuckling pirate captain but is just starting out on their adventuring career. Similarly, if your Game Master has decided that the campaign will take place in a world where humans are the only race, it’s no good deciding that you want to play an elf.

\example{Marcie decides that she’d like to play a carefree young tearaway who is bored with traveling on merchant caravans with her parents and wants to set off on a life of adventure instead. She sees the character—who she names Black Leaf—as possibly being magical (maybe an elf) but definitely being someone who is tall and slender, light on their feet, and highly mobile rather than being weighed down with armor.}

\section{Roll Ability Scores}\index[general]{Rolling Ability Scores}
Write down each of the six abilities in order (\iref[sec:Strength]{Strength}, \iref[sec:Dexterity]{Dexterity}, \iref[sec:Constitution]{Constitution}, \iref[sec:Intelligence]{Intelligence}, \iref[sec:Wisdom]{Wisdom}, \iref[sec:Charisma]{Charisma}) on a piece of scratch paper, and roll 3d6 for each one in order, noting down the result. These are your basic scores in each of the abilities.

If none of your rolls are above 9, or if two or more of your rolls are 6 or less, then re-roll all six rolls.

\example{Marcie rolls 3d6 six times in order and gets: Str: 10, Int: 8, Wis: 12, Dex: 14, Con: 9, Cha: 11.

She is pleased with the high Dexterity, which fits well with her concept, but realizes that with an Intelligence of only 8 she’s going to have to abandon thoughts of Black Leaf using magic.}

\section{Choose a Class}
Once you have your basic ability scores, you will be able to see your character’s basic strengths and weaknesses. Now you must choose which class your character will have. You may have already decided this as part of your character’s concept, or you may have changed your mind after seeing that your basic ability scores are particularly suitable (or unsuitable) for particular classes.

\example{Marcie has given up on the idea of Black Leaf using magic, and has decided she should be a slippery and lithe character adept at getting out of (and into) trouble. In terms of classes, either rogue, mystic or halfling would fit. She dismisses the idea of being a halfling since she imagines Black Leaf as being tall and willowy; and decides that she just wouldn’t be the disciplined type so mystic is out too. She makes Black Leaf a rogue.}

\section{Adjust Ability Scores}\index[general]{Adjusting Ability Scores}
When you choose your class, you are able to raise your \iref[sec:Prime Requisite]{Prime Requisites} by lowering your \iref[sec:Strength]{Strength}, \iref[sec:Intelligence]{Intelligence}, or \iref[sec:Wisdom]{Wisdom} scores as long as they are not also a \iref[sec:Prime Requisite]{Prime Requisite}.

For every two points of ability scores sacrificed, you may raise a \iref[sec:Prime Requisite]{Prime Requisite} by one. Abilities can not be lowered below 9 or raised above 18 in this way.

Please note that some classes have minimum required scores in some abilities. These requirements must be met after adjustments have been made.

\example{Marcie sees that by choosing the rogue class she can raise her Dexterity by sacrificing Strength or Wisdom. She can’t sacrifice Intelligence because it is already below 9, so she decides to sacrifice as much Strength and Wisdom as she can. By lowering her Strength by 1 point and her Wisdom by 3 points (the maximum she can, since that puts both scores on 9) she has sacrificed a total of 4 points and can therefore raise her Dexterity by two points—to 16.

Black Leaf now has the following ability scores: Str: 9, Int: 8, Wis: 9, Dex: 16, Con: 9, Cha: 11.}

\section{Choose Starting Age}\index[general]{Aging}
The player is free to choose their starting age as long as it meets the Game Master's approval. Rather than choosing their starting age, the player may use \fullref{tab:Aging} to determine it randomly. This is done by using the adult column of the table. 

The max column on the table is a modifier added to the old age column to determine what the characters maximum age is. When a character reaches old age, the Game Master will secretly roll this modifier and make note of it. When the character reaches this age, they will die of natural causes.

\begin {table}[H]
  \caption{Aging}\label{tab:Aging}
	\scriptsize
	\begin{tabularx}{\columnwidth}{>{\bfseries}ccccccc}
		\thead{Class} & \thead{Child} & \thead{Juvenile} & \thead{Adult} & \thead{Middle Age} & \thead{Old} & \thead{Max}\\
		Dwarf & 5 & 32 & 40+5d6 & 187 & 250 & 375+1d100\\
		Elf & 13 & 80 & 100+5d6 & 375 & 500 & 750+2d100\\
		Elf, Aquatic & 13 & 80 & 100+3d6 & 300 & 400 & 600+1d100\\
		Elf, Dark & 11 & 64 & 80+4d4 & 262 & 350 & 525+1d100\\
		Elf, Half- & 2 & 12 & 15+1d6 & 62 & 83 & 125+3d20\\
		Gnome & 6 & 48 & 60+3d12 & 100 & 133 & 200+3d100\\
		Halfling & 3 & 16 & 20+3d4 & 95 & 126 & 190+2d20\\
		Human & 2 & 12 & 15+1d4 & 47 & 63 & 95+2d12\
  \end {tabularx}
\begin{flushleft}
	Characters will have modifiers to their ability scores depending on their age category. All of these modifiers are cumulative.
	\statblock{\textbf{Baby:} Strength -4, Dexterity -1, Constitution -2, Wisdom -2
	\textbf{Child:} Strength +2, Dexterity +1, Wisdom +1
	\textbf{Teenager:} Strength +1, Constitution +1
	\textbf{Adult:} Strength +1, Constitution +1, Wisdom +1
	\textbf{Middle Age:} Dexterity -1, Constitution -1, Intelligence +1
	\textbf{Old:} Strength -2, Constitution -1, Wisdom +1}
\end{flushleft}
\end {table}

\section{Choose Height and Weight}\index[general]{Height and Weight}
The player is free to choose their starting height and weight as long as it meets the Game Master's approval. Rather than choosing their starting height and weight, the player may use \fullref{tab:Height and Weight} to determine them randomly. Weight is measured in coin for use with the Encumbrance rules (see Encumbrance and Weight).

\begin {table}[H]
	\caption{Height and Weight}\label{tab:Height and Weight}
  \begin{tabularx}{\columnwidth}{>{\bfseries}YYYYY}
		\thead{} & \multicolumn{2}{c}{\thead{Height in Inches}} & \multicolumn{2}{c}{\thead{Weight in Coin}}\\
		\thead{Class} & \thead{Base*} & \thead{Modifier} & \thead{Base*} & \thead{Modifier}\\
		Dwarf & 43/41 & 1d10 & 1,300/1,050 & 4d10x10\\
		Elf & 55/50 & 1d10 & 900/700 & 3d10x10\\
		Elf, Aquatic & 50/50 & 1d8 & 850/750 & 2d12x10\\
		Elf, Dark & 50/55 & 1d10 & 800/950 & 3d10x10\\
		Elf, Half- & 60/58 & 2d6 & 1,100/850 & 3d12x10\\
		Gnome & 38/36 & 1d6 & 720/680 & 5d4x10\\
		Halfling & 32/30 & 2d8 & 520/480 & 5d4x10\\
		Human & 60/59 & 2d10 & 1,400/1,000 & 6d10x10\
  \end {tabularx}
\begin{flushleft}
	*The second value is for females, as they tend to be lighter and shorter than males.
\end{flushleft}
\end {table}

\section{Level and Experience Points}
Unless otherwise specified by the Game Master, all characters start at first level and have no experience points (see \fullref{chap:Experience}).

\section{Roll Hit Dice}
The character’s starting hit points are determined by their chosen class. Refer to the description of that class in \fullref{chap:Classes} to determine which dice to roll.

\example{Marcie, being a \nth{1} level rogue, rolls a 1d4 for hit dice. She rolls 1d4 and gets a 3, which becomes her starting hit points. Marcie’s Constitution score is 9, which gives no bonus to the hit points rolled.}

\section{Choose Alignment}
Each character must select an alignment (see \fullref{chap:Alignment}). This represents the philosophical outlook of the character. There are three alignments to chose from: \iref[sec:Order]{Order}, \iref[sec:Chaos]{Chaos}, or \iref[sec:Neutral]{Neutral}.

\example{Marcie’s rogue steals for fun and the sense of adventure, not for greed or survival, which would not align her with Chaos. Her profession is frowned upon by the eyes of the law, which would not align her with Order. Marcie selects Neutral as her alignment, as the best fits her character’s persona.}

\section{Choose Skills}
Skills (see \fullref{chap:Skills}) are similar to class abilities, but are available to all classes. All characters will start with at least four skills that must be spent immediately.

\example{Marcie does not have a high Intelligence, so she only starts with four skill points. When working a heist, Marcie would like to be able quickly grab the most expensive items. So she decided to spend two skill points in Arcane Lore and two skill points in History.}

\section{Choose Weapon Feats}
Weapon feats (see \fullref{chap:Weapon Feats}) indicate how well a character can use a particular weapon. All characters will start with at least two weapon feats that must be spent immediately.

\example{Marcie, being a rogue, starts with two weapon feats. She spends one weapon feat on dagger, which is the weapon she plans on carrying while adventuring. She spends the other weapon feat on crossbow, which she doesn’t plan on owning right away, but may purchase one at a later time.}

\section{Starting Money and Equipment}\index[general]{Starting Money and Equipment}
Newly created first level characters start with a set of peasant clothes, and with 3d6x10 gp to spend on other items (see \fullref{chap:Equipment}).

A newly created \iref[class:Elf]{Elf} or \iref[class:Wizard]{Wizard} character will also start with a spell book containing \iref[spell:Read Magic]{Read Magic} and one other first level spell.

If a character is introduced in the middle of a campaign (to replace a dead character) then the Game Master should give the character money and equipment—including magic items—in line with the amount that the rest of the party have.

\example{Marcie rolls 3d6 and gets a 13 total. She than times this total by 10, which gives her 310 gp to start with. She purchases leather armor (20 gp), a lantern (1 gp), dried rations (15 gp), rogues’ tools (25 gp) so she can pick locks, a dagger (3 gp), and a silver dagger (30 gp) just in case she encounters vampires. This leaves Marcie with 36 gp to spend at a later time.}

\end{multicols*}

