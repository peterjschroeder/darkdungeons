\chapter[red]{Spells and Spellcasting}
\label{chap:Spells and Spellcasting}
\chapterimage[Spells and Spellcasting (Bordered)]
\thispagestyle{plain}

\begin{multicols*}{2}
Many characters in Dark Dungeons can use spells. Humans are the most versatile spellcasters. They can become clerics, druids, or wizards and learn to cast many spells per day. Experienced rogues can learn to use wizard scrolls to cast spells with a good chance of success. And even fighters can—if they take knightly vows—start casting low level clerical spells.

Many non-human races use spells too. Some creatures have natural (or magical) abilities that work like spells, but many of the intelligent races (humanoid or otherwise) can have shamans or sorcerers who cast spells like clerics and wizards, although doing so through innate ability rather than formal training they lack the versatility of human spellcasters.

The demi-human races are somewhat unusual, in that they have no clerical or shaman spellcasters, and with the exception of elves—who can cast spells just like human wizards—have no wizards or sorcerers either.

All these major types of spellcaster (clerics, druids, shamans, sorcerers, wizards) share a common basis for spellcasting, as detailed in this chapter.

\section{Spell Levels}\index[general]{Spell Levels}
All spells have a level. This is a measure of how powerful the spell is and how difficult it is to prepare. Spells of the same level are roughly similar in terms of power level, and are interchangeable when it comes to spell preparation, providing the caster has access to each spell.

\iref[class:Cleric]{Cleric} and druid spells range in level from 1-7. \iref[class:Elf]{Elf} and wizard spells range in level from 1-9. Shamans and sorcerers can only cast spells from levels 1-6.

Each spellcasting character is limited in the number of spells they can prepare each day. When first able to cast spells they will probably be limited to only preparing a single \nth{1} level spell per day, but as they get more experienced they will be able to cast more spells and also more powerful spells. See \fullref{chap:Classes} for exactly how many spells of each level a character gets based on their class.

Shamans get the same number of spells as clerics of the same level, and sorcerers get the same number of spells as wizards of the same level.

\section{Acquiring Spells}\index[general]{Acquiring Spells}
Spellcasting characters acquire spells in different ways.

\textbf{Clerics:} \iref[class:Cleric]{Clerics} (and fighters who have taken knightly vows) always have access to all clerical spells of levels that they can cast. As part of the new cleric’s rite of investiture (the details of which will vary from religion to religion and may be full of pomp and ceremony or be a quiet personal affair) their patron \iref[chap:Immortals]{Immortal} magically alters the cleric’s life force so that it will accept spells from that \iref[chap:Immortals]{Immortal}. This is a permanent alteration, and costs the \iref[chap:Immortals]{Immortal} some of its own essence; so it is not done lightly. The cleric now has access to all their clerical spells, providing nothing untoward happens to the \iref[chap:Immortals]{Immortal} who grants them the spells. This sort of thing rarely happens, but it is not unheard-of.

Since the change done to the cleric has a permanent effect, the cleric will continue to gain spells even if they split from their church into heresy or even give up their religion altogether. However, depending on the attitude of the church towards this, it may result in the cleric being denounced and hunted down and killed.

In extreme cases it is possible for the \iref[chap:Immortals]{Immortal} to personally visit the offending cleric and remove their clerical powers in the same way that they granted them. However, since this also costs the \iref[chap:Immortals]{Immortal} some of its essence it is not done by any but the most forgiving \iref[chap:Immortals]{Immortals}. Less forgiving \iref[chap:Immortals]{Immortals} will either ignore the offending cleric and leave them for their worshipers to deal with or simply kill them outright.

Forsaking one’s church can be a dangerous business.

\textbf{Dervishes, Druids, and Fey:} Dervishes, druids, and fey get their spells from nature itself. Like clerics, they have access to all spells they can cast, but unlike clerics this access cannot be blocked—since nature itself is never in a situation where it is unable to grant spells.

\textbf{Elves, Wise Women, and Wizards:} Elves, wise women, and wizards do not automatically acquire spells. Each spell they wish to learn must be formally studied and recorded in a spell book. Each spell book can contain up to 100 levels of spells, so a powerful magic user may need up to 6 spell books to hold all their spells.

A starting elf or wizard begins with a free spell book, which contains the \iref[spell:Read Magic]{Read Magic} spell and one other first level spell of their choice.

A spell book is a very personal item, and the spells written in it are scribed using an arcane symbolism unique to the individual spellcaster. Each caster needs their own spell book in order to prepare spells, and no caster can prepare spells from anyone else’s spell book.

Although each elf or wizard is dependent on their spell book to be able to prepare spells, their knowledge of spells does not rely on their book. If they lose their book for any reason or it gets destroyed, they can—given time and effort—make a replacement of it by writing down all the spells that they know.

However, because of the personal connection to the book, each elf or wizard can only have one set of spell books in existence at a time. Writing a spell into a new one makes the old copy obsolete and useless; so it is not possible to keep a spare set around just in case.

In order to acquire mastery of a new spell in order to write it into their spell book the elf or wizard needs two things. Firstly they need time to practice the spell. Secondly they need a source containing the spell. This source can be another elf or wizard willing to teach the spell to them, or it can be another wizard’s spell book or even a Spells Scroll.

Depending on the campaign world, there may be organized wizard’s guilds or magical universities that teach magic, or characters may have to seek out individual teachers. The Game Master should be careful not to make it too hard for wizards and elves to find suitable teachers; since the game assumes that the cost in terms of money and time is enough on its own to balance the utility of having more spells.

The teacher must know, and be able to cast, the spell that the student is trying to learn; and the student must be high enough level to cast the spell.

Once the character has found a suitable teacher, they must start their study. The length of time that the study must take is listed on \fullref{tab:Learning New Spells}, along with the weekly cost of the study if the trainer is an NPC.

\autoref*{tab:Learning New Spells} also shows the chance of the study being successful. This check is made halfway through the study, at which point it will be apparent to the teacher whether the student is progressing or not. Unscrupulous teachers may keep quiet and keep taking money from the student for the rest of the duration anyway, but most teachers will warn the student that the study isn’t working out and give them the choice of continuing the study until the end or abandoning it at that halfway mark (and therefore saving the money they would have paid for the second half of the study).

If the studying was not successful, the character gets a +10\% cumulative bonus on all future study (or research) for the same spell.

\end{multicols*}
\begin {table}[H]
	\caption{Learning New Spells}\label{tab:Learning New Spells}
	\begin{tabularx}{\columnwidth}{>{\bfseries}YYYYYYYY}
	\thead{} & \thead{} & \multicolumn{6}{c}{\thead{Level of Teacher}}\\
	\thead{} & \thead{} & \thead{0} & \thead{1-4} & \thead{5-8} & \thead{9-14} & \thead{15-20} & \thead{21+}\\
	\thead{Spell Level} & \thead{Time Needed} & \thead{(50 gp/week)} & \thead{(100 gp/week)} & \thead{(250 gp/week)} & \thead{(500 gp/week)} & \thead{(750 gp/week)} & \thead{(1,000 gp/week)}\\
	1-2 & 1 week & 1\%* & 60\% & 80\% & 95\% & 99\% & 99\%\\
	3-4 & 2 weeks & - & 1\%* & 50\% & 70\% & 90\% & 95\%\\
	5-6 & 4 weeks & - & - & 1\%* & 40\% & 60\% & 80\%\\
	7-8 & 8 weeks & - & - & - & 1\%* & 30\% & 50\%\\
	9 & 12 weeks & - & - & - & - & 1\%* & 20\%\\
  \end {tabularx}
	* A teacher of this level cannot teach the spell. This is the chance for individual research done without a teacher
\end {table}
\begin{multicols*}{2}

Unlike weapon training, it is possible for an elf or wizard to do their own research in order to learn a spell if there is no teacher available.

In order to do such research, the elf or wizard must have access to either a scroll of the spell or someone else’s spell book that contains the spell. Although an elf or wizard cannot use another caster’s spell book to prepare spells, casting a \iref[spell:Read Magic]{Read Magic} spell on it will allow them to read its contents and use it as a research tool in order to learn the spells themselves.

In the case of individual research, the chance of success is based on the amount of money that the researcher is prepared to spend per week on research materials and laboratory usage.

However, the maximum that a researcher can spend is the equivalent of a teacher of one level group lower than the researcher’s actual level.

Unlike study with a teacher, characters doing research do not roll for their success until after the whole research period has finished, however if their research fails they still get the +10\% cumulative bonus on future research or study of the same spell.

\example{Aloysius is a \nth{16} level wizard who does not know the \nth{3} level spell Create Air. Unfortunately, the small town that he is in does not have any wizards other than himself, and he can’t leave the town for any extended period since he has sworn an oath to stay there and protect it for the rest of the season—so there is no-one to learn the spell from. He does own a spell book containing the spell though. One that he once looted from a lich’s tomb and kept.

Since he has no teacher, he must try to research the spell on his own.

The spell is \nth{3} level, so Aloysius must spend at least 100 gp/week on his research in order to have a chance of learning it, and can spend up to a maximum of 500 gp/week on research, since that is the equivalent of a level 9-14 teacher—one level group lower than his own \nth{16} level.

Since he knows that he’s going to be in town for a while, he decides that there’s no point throwing excess money at the issue, so he decides to spend the more modest 250 gp/week on his research.

Researching a \nth{3} level spell at 250 gp per week is exactly the same as being taught the \nth{3} level spell by a teacher of level 5-8—it will take two weeks, cost 250 gp per week, and have a 50\% chance of success. However, this chance of success is rolled at the end of the two-week period, rather than in the middle of it.

Aloysius spends the two weeks and the 500 gp, and at the end of that period his player rolls a 17. Aloysius now knows the Create Air spell and can add it to his spell book.}

\textbf{Medicine Men:} Medicine men are granted their spells by their spirit totem. They follow exactly the same rules as clerics.

\textbf{Shamans:} Shamans are granted their spells either by an \iref[chap:Immortals]{Immortal} or by nature, depending on the individual shaman and their race and religion. Therefore, they follow exactly the same rules as either clerics or druids.

\textbf{Sorcerers:} Sorcerers are able to use magic via a natural talent. Although they prepare spells in the same way as wizards—and all sorcerer spells are also wizard spells—they are able to do so without the aid of spell books or any kind of formal training.

A sorcerer’s spellcasting is simply intuitive. Every so often, as they get more experienced, they will simply gain an insight and “know” how to prepare a new spell that they weren’t previously able to prepare.

\section{Preparing Spells}\index[general]{Preparing Spells}
Spells are complicated affairs involving powerful magical energies, and can not simply be cast on a whim.

Before a spell can be cast, it must be prepared by the caster. When preparing a spell, a caster gathers the magical energy and shapes it into the spell that they wish to cast, leaving it primed. Then at the time of casting the spellcaster says the final words and makes the final gestures, which set the parameters for the spell effect and release the energy. This final release of the spell’s energy is known as casting the spell.

On a typical day, and adventuring spellcaster will prepare all the spells they can first thing in the morning after waking up, and then cast them at various times during the day—often in the middle of combat.

Since each extra spell that is prepared adds more magical energy to the caster’s control, and higher level spells require more precise control than lower level ones, the number of spells a caster can have prepared at one time is limited by their experience level, as listed in the various tables in \fullref{chap:Classes}.

Once a spell has been prepared, it will not fade away. Prepared spells can be kept primed for days or even weeks at a time before being cast; although a caster can dissipate the magical energy of a spell at any time in order to free up the spell slot that it was taking up.

Preparing spells requires a clear head, and must be done directly after a good night’s sleep, and before any strenuous or stressful activity has happened.

The character must then spend an hour performing meditation and rites (involving their spell book if they use one) in relatively quiet surroundings. Preparing spells is possible in the bustle of a campsite or shared room, and the caster can wave away interruptions without disturbing their preparation, but preparation is not possible in a crowded marketplace or in the middle of a fight.

At the end of the hour, any spare spell slots the character has that don’t already contain prepared spells can be filled with the character’s choice of spells from those that they know (of the appropriate level).

If a spellcaster expects to cast a spell twice during a day, they must prepare it twice, taking up two slots of the appropriate level.

\section{Reversible Spells}\index[general]{Reversible Spells}
Some spells are marked in the spell lists with an asterisk (*). This indicates that the spell is reversible. Reversible spells can be cast in one of two ways, usually with the opposite effect. For example, a \iref[spell:Raise Dead]{Raise Dead} spell is normally cast to bring someone back from the dead, but it can be reversed and cast as a Finger of Death in order to kill someone.

\iref[class:Cleric]{Clerics}, dervishes, druids, fey, medicine men and shamans always prepare their spells in the normal form, and can choose to either use them in that form or reverse them at the time of casting. Although any cleric can cast any clerical spell, some religions may discourage or even prohibit the reversal of some spells.

Elves, wise women, wizards and sorcerers must decide at the time of preparation whether they wish to prepare the normal or reversed form of the spell. Although they can, of course, prepare both forms of the same spell if they have two or more spell slots of an appropriate level.

\section{Environmental Effects}\index[general]{Environmental Effects}
Some spells are marked in the spell lists with a dagger (†). This indicates that the spell functions differently or not at all when cast in certain environments. For example, a \iref[spell:Fly]{Fly} spell normally allows a target to fly, but when cast underwater it instead allows a target to swim faster.

\section{Casting Spells}\index[general]{Casting Spells}
Unless the spell description explicitly says otherwise, any spell can be cast in a character’s turn in combat as a normal action.

However, casting a spell requires precise gestures and speaking. Therefore, a spellcaster must have at least one hand free and must be able to speak in order to be able to cast a spell.

If you wish to cast a spell, you must announce what spell you are casting at the beginning of the round before initiative is rolled. If you then take any damage before your initiative, your spellcasting will be disrupted and you will lose the prepared spell but it will dissipate without effect.

When casting a spell that targets creatures or objects other than yourself, you must be able to see those targets. However, you can hit unseen or invisible targets with area effect spells even if you can’t see them (and even if you don’t know they are there).

Some spells, such as \ilink{spell:Cause Light Wounds}{Cause Light Wounds} require the caster to make a successful melee attack in order to touch the target in order for the spell to work.

This attack does not require a separate action in combat—it can be performed as part of the Cast Spell action.

The attack must be done with bare hands and does not use the Unarmed Strike weapon feat. Touching someone with a spell is very different from striking someone in melee.

However, if the attack misses the spell is not wasted. The spell that the caster was attacking with will stay “primed” for ten minutes or until they cast another spell. Therefore, the caster can try repeatedly to touch the same (or a different) target. These additional attempts to touch a target each require an Attack action.

\section{Saving Throws}\index[general]{Saving Throws}
Most spells that have an effect on others allow their targets to roll a saving throw in order to lessen or negate the effect. Unless the specific spell description mentions otherwise, all such saving throws are vs. spells.

If a spell does not mention giving a saving throw, then none should be given.

\section{Conflicting Spells}\index[general]{Conflicting Spells}
When different spells are cast on the same target, each of the spells has its normal effect for its duration. However, multiple castings of the same spell (or spells that are the same except in terms of area of effect—such as Protection from Evil and \iref[spell:Protection from Evil 10-foot radius]{Protection from Evil 10-foot radius}) do not stack. The target gets the effect only once.

However, in cases where the multiple castings of the same spell could apply to the target without conflict (such as a target having two active \iref[spell:Quest]{Quest} spells at the same time, or the target being simultaneously charmed by two different casters and therefore believing both of them to be friends) each of the spells stays active.

\section{Anti-Magic}\index[general]{Anti-Magic}\label{sec:Anti-Magic}
Some creatures have an anti-magic ability, either as a personal ability or occasionally as an area effect ability.

Anti-magic is usually listed with a percentage (e.g. “50\% Anti-Magic”). In cases where it is not, assume that it is 100\%.

In an anti-magic area, all existing magic has a chance of not working. Each magical effect that is brought into the area has a percentage chance equal to the anti-magic percentage of being temporarily suppressed while in the area and ceasing to work. Existing magic with a duration still has its duration counted while in the area, even if it is not working. Existing magic brought into an anti-magic area will resume functionality (assuming its duration has not run out) once it leaves the area.

\example{Oeric, using a +2 Sword of Flaming and under the influence of a Potion of Flying, is attacking a gazer. The gazer turns its anti-magic ray in his direction.

Since the anti-magic ray of a gazer is not given a percentage, it is assumed to be 100\% effective. Oeric’s +3 Sword of Flaming goes out, and is now a normal sword. Similarly, Oeric falls to the ground as his Potion of Flying is suppressed.

Two rounds later, the gazer looks away. Oeric’s sword immediately bursts into flame again and regains its +2 bonus, and Oeric can now fly again. The two rounds that he spent unable to fly still count towards the duration of his potion.}

Additionally, while in an anti-magic area, any spell cast or any magical ability used has a percentage chance of failing to work. In this case, the spell or ability will not suddenly kick in after the area is left, and the spell or ability does count as having been used for purposes of charges or uses-per-day.

If the anti-magic is a personal ability rather than an area effect, it is simpler. The creature with the ability simply has that percentage chance of being unaffected by the magic. The magic will still function normally other than not affecting the creature.

\example{Aloysius casts a Fireball spell from an artifact that he owns at a marilith and the trolls that are accompanying her. Because the fireball is from an artifact, it gets past the marilith’s immunity to mortal level magic. However, her anti-magic still applies.

The marilith has a 25\% anti-magic against immortal level magic, so the Game Master rolls for the Fireball and gets a 12.

The fireball goes off as normal, hurting all the trolls. However, the marilith stands in the middle of it totally unaffected by the magic and totally undamaged.}

\end{multicols*}
\section{Spell Lists}\label{sec:Spell Lists}
\begin {table}[H]
  \caption{Cleric Spells}
  \begin{tabularx}{\columnwidth}{YYYY}
	\thead{\nth{1} Level} & \thead{\nth{2} Level} & \thead{\nth{3} Level} & \thead{\nth{4} Level}\\
	Cure Light Wounds* & Bless* & Continual Light* & Animate Dead\\
	Detect Evil & Find Traps & Cure Blindness & Create Water\\
	Detect Magic & Hold Person* & Cure Disease* & Cure Serious Wounds*\\
	Light* & Know Alignment* & Growth of Animal & Dispel Magic\\
	Protection from Evil & Resist Fire & Locate Object & Neutralize Poison*\\
	Purify Food and Water & Silence 15-foot radius & Remove Curse* & Protection from Evil 10-foot radius\\
	Remove Fear* & Snake Charm & Speak with Dead & Speak with Plants\\
	Resist Cold & Speak with Animal & Striking & Sticks to Snakes†\\
	\thead{\nth{5} Level} & \thead{\nth{6} Level} & \thead{\nth{7} Level}\\
	Commune & Aerial Servant† & Earthquake†\\
	Create Food† & Animate Objects & Holy Word\\
	Cure Critical Wounds* & Barrier* & Raise Dead Fully*\\
	Dispel Evil & Create Normal Animals & Restore*\\
	Insect Plague† & Find the Path & Survival\\
	Quest* & Heal & Travel\\
	Raise Dead* & Speak with Monsters* & Wish\\
	Truesight & Word of Recall & Wizardry\
	\end {tabularx}
	*Reversible spell\\
	†Effected by environment
\end {table}

\begin {table}[H]
  \caption{Dervish Spells}
  \begin{tabularx}{\columnwidth}{YYYY}
	\thead{\nth{1} Level} & \thead{\nth{2} Level} & \thead{\nth{3} Level} & \thead{\nth{4} Level}\\
	Detect Magic & Hold Person* & Call Lightning† & Charm Animal\\
	Detect Water & Obscure & Growth of Plants & Create Water\\
	Faerie Fire & Snake Charm & Hold Animal & Neutralize Poison*\\
	Locate & Speak with Animal & Know Destiny & Speak with Plants\\
	Predict Weather & Truth or Else & Shift Sand & Summon Animals\\
	\thead{\nth{5} Level} & \thead{\nth{6} Level} & \thead{\nth{7} Level}\\
	Commune & Anti-Animal Shell & Creeping Doom†\\
	Conjure Elemental† & Find the Path & Earthquake†\\
	Control Winds† & Speak with Monsters* & Holy Word\\
	Passwall & Summon Weather & Survival\\
	Quest* & Word of Recall & Weather Control\
  \end {tabularx}
	*Reversible spell\\
	†Effected by environment
\end {table}

\begin {table}[H]
  \caption{Druid Spells}
  \begin{tabularx}{\columnwidth}{YYYY}
	\thead{\nth{1} Level} & \thead{\nth{2} Level} & \thead{\nth{3} Level} & \thead{\nth{4} Level}\\
	Cure Light Wounds* & Bless* & Call Lightning† & Animate Dead\\
	Detect Danger & Find Traps & Continual Light* & Control Temperature 10-foot radius\\
	Detect Evil & Heat Metal & Cure Blindness & Create Water\\
	Detect Magic & Hold Person* & Cure Disease* & Cure Serious Wounds*\\
	Faerie Fire & Know Alignment* & Growth of Animal & Dispel Magic\\
	Light* & Obscure & Hold Animal & Neutralize Poison*\\
	Locate & Produce Fire† & Locate Object & Plant Door\\
	Predict Weather & Resist Fire & Protection from Poison & Protection from Evil 10-foot radius\\
	Protection from Evil & Silence 15-foot radius & Remove Curse* & Protection from Lightning\\
	Purify Food and Water & Snake Charm & Speak with Dead & Speak with Plants\\
	Remove Fear* & Speak with Animal & Striking & Sticks to Snakes†\\
	Resist Cold & Warp Wood & Water Breathing* & Summon Animals\\
	\thead{\nth{5} Level} & \thead{\nth{6} Level} & \thead{\nth{7} Level}\\
	Anti-Plant Shell & Aerial Servant† & Creeping Doom†\\
	Commune & Animate Objects & Earthquake†\\
	Control Winds† & Anti-Animal Shell & Holy Word\\
	Create Food† & Barrier* & Metal to Wood\\
	Cure Critical Wounds* & Create Normal Animals & Raise Dead Fully*\\
	Dispel Evil & Find the Path & Restore*\\
	Dissolve* & Heal & Summon Elemental†\\
	Insect Plague† & Speak with Monsters* & Survival\\
	Pass Plant & Summon Weather & Travel\\
	Quest* & Transport Through Plants & Weather Control\\
	Raise Dead* & Turn Wood & Wish\\
	Truesight & Word of Recall & Wizardry\
  \end {tabularx}
	*Reversible spell\\
	†Effected by environment
\end {table}

\begin {table}[H]
  \caption{Elf Spell}
  \begin{tabularx}{\columnwidth}{YYYY}
	\thead{\nth{1} Level} & \thead{\nth{2} Level} & \thead{\nth{3} Level} & \thead{\nth{4} Level}\\
	Analyze & Detect Danger & Call Lightning† & Charm Monster\\
	Command Word & Detect Evil & Clairvoyance & Confusion\\
	Charm Person & Detect Invisible & Cure Disease & Dimension Door\\
	Detect Magic & Entangle & Cure Light Wounds & Enchanted Weapon\\
	Faerie Fire & ESP* & Dispel Magic & Fear\\
	Faerie Lights & Invisibility & Fly† & Growth of Animal\\
	Fellowship & Know Alignment* & Haste* & Growth of Plants*\\
	Light* & Levitate & Heat Metal & Hallucinatory Terrain\\
	Locate & Locate Object & Hold Animal* & Massmorph†\\
	Longstride & Mirror Image & Hold Person* & Polymorph Other\\
	Magic Missile & Phantasmal Force & Invisibility 10-foot radius & Polymorph Self\\
	Precipitation & Predict Weather & Obscure & Remove Curse*\\
	Protection from Evil & Produce Fire† & Protection from Evil 10-foot radius & Summon Animals\\
	Read Languages & Purify Food and Water & Protection from Normal Missiles & Wall of Fire†\\
	Read Magic & Resist Fire & Protection from Poison & Wizard Eye\\
	Resist Cold & Silence & Speak with Animal &\\
	Ventriloquism & Warp Wood & Water Breathing* &\\
	Watcher & Web &  &\\
	\thead{\nth{5} Level} & \thead{\nth{6} Level} & \thead{\nth{7} Level} & \thead{\nth{8} Level}\\
	Conjure Elemental† & Anti-Animal Shell & Charm Plant & Charm, Mass*\\
	Contact Outer Plane & Anti-Magic Shell & Create Normal Monsters & Creeping Doom†\\
	Control Temperature 10-foot radius & Cure Serious Wounds & Dispel Evil & Dance\\
	Control Winds† & Geas* & Invisibility, Mass & Force Field\\
	Dissolve* & Lower Water† & Lore & Metal to Wood\\
	Feeblemind & Move Earth & Magic Door* & Mind Barrier*\\
	Insect Plague† & Pass Plant & Statue & Permanence\\
	Magic Jar & Projected Image & Teleport Any Object & Polymorph Any Object\\
	Neutralize Poison & Reincarnation & Transport Through Plants & Symbol\\
	Plant Door & Stone to Flesh & Truesight & Travel\\
	Protection from Lightning & Weather Control & Turn Wood &\\
	Telekinesis & & &\\
	Teleport & & &\\
	\multicolumn{4}{c}{\thead{\nth{9} Level}}\\
	Contingency & Immunity & Summon Object & Wish\\
	Gate* & Maze & Sword &\\
	Heal & Shapechange & Timestop &\
	\end {tabularx}
	*Reversible spell\\
	†Effected by environment
\end {table}

\begin {table}[H]
  \caption{Fey Spells}
  \begin{tabularx}{\columnwidth}{YYYY}
	\thead{\nth{1} Level} & \thead{\nth{2} Level} & \thead{\nth{3} Level} & \thead{\nth{4} Level}\\
	Chill & Continual Light* & Anti-Plant Shell & Anti-Animal Shell\\
	Charm Person & Detect Danger & Charm Monster & Call Lightning†\\
	Courier & Entangle & Clairvoyance & Charm Plant\\
	Detect Evil & ESP* & Create Water & Confusion\\
	Detect Invisible & Faerie Slumber & Dispel Magic & Create Food†\\
	Detect Magic & Heat Metal & Fear & Cure Blindness or Disease\\
	Faerie Fire & Hold Animal* & Fly† & Dimension Door\\
	Hold Portal & Horse-rush & Gaseous Form & Enchanted Weapon\\
	Light* & Infravision & Growth of Plants* & Find the Path\\
	Locate & Knock & Hallucinatory Terrain & Growth of Animal\\
	Magic Missile & Know Alignment* & Haste* & Massmorph†\\
	Precipitation & Levitate & Hold Person* & Pass Plant\\
	Predict Weather & Locate Object & Invisibility & Plant Door\\
	Protection from Evil & Mirror Image & Polymorph Self & Polymorph Other\\
	Read Languages & Obscure & Protection from Evil 10-foot radius & Second Sight\\
	Read Magic & Phantasmal Force & Protection from Normal Missiles & Summon Weather\\
	Shield & Produce Fire† & Protection from Poison & Telekinesis\\
	Sleep & Purify Food and Water & Remove Curse* & Wizard Eye\\
	Ventriloquism & Silence & Silence 15-foot radius &\\
	Watcher & Warp Wood & Summon Animals &\\
	& Web & Water Breathing* &\\
	& Wizard Lock &  &\\
	\thead{\nth{5} Level} & \thead{\nth{6} Level} & \thead{\nth{7} Level}\\
	Animate Objects & Conjure Elemental† & Contingency\\
	Anti-Magic Shell & Create Normal Monsters & Create Magical Monsters\\
	Charm, Mass* & Dance & Creeping Doom†\\
	Control Temperature 10-foot radius & Geas* & Disintegrate\\
	Control Winds† & Lower Water† & Dispel Evil\\
	Create Normal Animals & Metal to Wood & Maze\\
	Dissolve* & Move Earth & Mind Barrier*\\
	Feeblemind & Polymorph Any Object & Permanence\\
	Hold Monster* & Projected Image & Polymorph Any Object\\
	Insect Plague† & Shapechange & Power Word Blind\\
	Invisibility, Mass* & Speak with Dead & Reincarnation\\
	Lore & Speak with Monsters* & Summon Object\\
	Magic Jar & Stone to Flesh* & Teleport Any Object\\
	Neutralize Poison* & Survival & Timestop\\
	Protection from Lightning & Sword &\\
	Statue & Truesight &\\
	Transport Through Plants &  &\\
	Weather Control &  &\
	\end {tabularx}
	*Reversible spell\\
	†Effected by environment
\end {table}

\begin {table}[H]
  \caption{Medicine Man Spells}
  \begin{tabularx}{\columnwidth}{YYYY}
	\thead{\nth{1} Level} & \thead{\nth{2} Level} & \thead{\nth{3} Level} & \thead{\nth{4} Level}\\
	Command Word & Animal Charm & Call Lightning† & Bind Totem\\
	Detect Evil & Bless & Call Totem & Control Temperature 10-foot radius\\
	Detect Magic & Cure Light Wounds* & Continual Light* & Cure Serious Wounds\\
	Detect Poison & Find Traps & Cure Blindness & Dispel Magic\\
	Detect Totem & Hold Person* & Cure Disease & Fate\\
	Fellowship & Hunting Paint & Detect Curse & Fire Gate\\
	Light* & Know Alignment & Firebow & Growth of Plants\\
	Locate & Locate Totem & Hold Animal & Neutralize Poison\\
	Minor Blessing & Produce Fire† & Hold Spirit & Protection from Lightning\\
	Pass Without Trace & Resist Fire & Invisibility to Spirits & Sanctify\\
	Predict Weather & Shimmer & Locate Object & Speak with Plants\\
	Protection from Evil & Silence 15-foot radius & Protection from Poison & Sticks to Snakes†\\
	Purify Food and Water & Snake Charm & Remove Curse* & Summon Animals\\
	Remove Fear* & Speak with Animal & Silent Move & Summon Herd\\
	Trance & Spirit Sending & Thunder Drum & Summon Lesser Animal Spirit\\
	&  & War Paint &\\
	\thead{\nth{5} Level} & \thead{\nth{6} Level} & \thead{\nth{7} Level}\\
	Commune & Animate Objects & Creeping Doom†\\
	Confound & Barrier* & Earthquake†\\
	Control Winds† & Find the Path & Holy Word\\
	Cure Critical Wounds & Heal & Metal to Wood\\
	Dispel Evil & Infusion & Spell Turning\\
	Dissolve* & Lore & Summon Greater Animal Spirit\\
	Eye of the Eagle & Madness & Survival\\
	Polymorph Self & Spirit Storm & Travel\\
	Quest & Speak with Monsters* & Weather Control\\
	Spirit Walk & Summon Weather & Wish\\
	Strength of Mind & Symbol &\\
	Sword of Fire & Turn Wood &\\
	Totem Mastery & Wrath of Amerind &\\
	Truesight & Word of Recall &\
	\end {tabularx}
	*Reversible spell\\
	†Effected by environment
\end {table}

\begin {table}[H]
  \caption{Shaman Spells}
  \begin{tabularx}{\columnwidth}{YYY}
	\thead{\nth{1} Level} & \thead{\nth{2} Level} & \thead{\nth{3} Level}\\
	Cure Light Wounds* & Bless* & Continual Light*\\
	Detect Magic & Hold Person* & Cure Blindness\\
	Light* & Snake Charm & Cure Disease*\\
	Protection from Evil & Speak with Animal & Remove Curse*\\
	\thead{\nth{4} Level} & \thead{\nth{5} Level} & \thead{\nth{6} Level}\\
	Cure Serious Wounds* & Create Food† & Find the Path\\
	Dispel Magic & Cure Critical Wounds* & Heal\\
	Neutralize Poison* & Dispel Evil & Speak with Monsters*\\
	Speak with Plants & Insect Plague† & Word of Recall\
	\end {tabularx}
	*Reversible spell\\
	†Effected by environment
\end {table}

\begin {table}[H]
  \caption{Sorcerer Spells}
  \begin{tabularx}{\columnwidth}{YYY}
	\thead{\nth{1} Level} & \thead{\nth{2} Level} & \thead{\nth{3} Level}\\
	Detect Magic & Continual Light* & Clairvoyance\\
	Light* & Detect Evil & Dispel Magic\\
	Protection from Evil & Detect Invisible & Fireball†\\
	Read Languages & Invisibility & Fly†\\
	Read Magic & Levitate & Lightning Bolt†\\
	Sleep & Web & Water Breathing*\\
	\thead{\nth{4} Level} & \thead{\nth{5} Level} & \thead{\nth{6} Level}\\
	Charm Monster & Animate Dead & Death Spell\\
	Growth of Plants* & Cloudkill† & Move Earth\\
	Ice Storm/Wall of Ice† & Dissolve* & Projected Image\\
	Massmorph† & Hold Monster* & Reincarnation\\
	Remove Curse* & Passwall & Stone to Flesh*\\
	Wall of Fire† & Wall of Stone & Wall of Iron\
	\end {tabularx}
	*Reversible spell\\
	†Effected by environment
\end {table}

\begin {table}[H]
  \caption{Wizard Spells}
  \begin{tabularx}{\columnwidth}{XYYY}
	\thead{\nth{1} Level} & \thead{\nth{2} Level} & \thead{\nth{3} Level} & \thead{\nth{4} Level}\\
	Analyze & Continual Light* & Clairvoyance & Charm Monster\\
	Charm Person & Detect Evil & Create Air & Clothform\\
	Detect Magic & Detect Invisible & Dispel Magic & Confusion\\
	Floating Disc & Entangle & Fireball† & Dimension Door\\
	Hold Portal & ESP* & Fly† & Growth of Plants*\\
	Light* & Invisibility & Haste* & Hallucinatory Terrain\\
	Magic Missile & Knock & Hold Person* & Ice Storm/Wall of Ice†\\
	Protection from Evil & Levitate & Infravision & Massmorph†\\
	Read Languages & Locate Object & Invisibility 10-foot radius & Polymorph Other\\
	Read Magic & Mirror Image & Lightning Bolt† & Polymorph Self\\
	Shield & Phantasmal Force & Protection from Evil 10-foot radius & Remove Curse*\\
	Sleep & Uncontrollable Hideous Laughter & Protection from Normal Missiles & Wall of Fire†\\
	Ventriloquism & Web & Water Breathing* & Wizard Eye\\
	& Wizard Lock & &\\
	\thead{\nth{5} Level} & \thead{\nth{6} Level} & \thead{\nth{7} Level} & \thead{\nth{8} Level}\\
	Animate Dead & Anti-Magic Shell & Charm Plant & Charm, Mass*\\
	Cloudkill† & Death Spell & Create Normal Monsters & Clone\\
	Conjure Elemental† & Disintegrate & Delayed Blast Fireball & Create Magical Monsters\\
	Contact Outer Plane & Geas* & Invisibility, Mass* & Dance\\
	Dissolve* & Invisible Stalker† & Ironform & Explosive Cloud\\
	Earth to Air* & Lower Water† & Lore & Force Field\\
	Feeblemind & Move Earth & Magic Door* & Mind Barrier*\\
	Hold Monster* & Projected Image & Power Word Stun & Permanence\\
	Ice to Water* & Reincarnation & Reverse Gravity & Polymorph Any Object\\
	Magic Jar & Stone to Flesh* & Statue & Power Word Blind\\
	Passwall & Stoneform & Summon Object & Steelform\\
	Telekinesis & Wall of Iron & Sword & Symbol\\
	Teleport & Weather Control & Teleport Any Object & Travel\\
	Wall of Stone &  &  &\\
	Woodform &  &  &\\
	\multicolumn{4}{c}{\thead{\nth{9} Level}}\\
	Contingency & Immunity & Prismatic Wall & Wish\\
	Create Any Monster & Maze & Shapechange &\\
	Gate* & Meteor Swarm & Survival &\\
	Heal & Power Word Kill & Timestop &\
	\end {tabularx}
	*Reversible spell\\
	†Effected by environment
\end {table}
\begin{multicols*}{2}

\section{Spell Format}
Each spell description contains a stat block containing the following entries.

\textbf{Target:} This shows the possible targets and how many of them that the caster may affect with this spell.

\textbf{Range:} This shows how far or how much area a caster can target with this spell.

\textbf{Duration:} This lists how long the spell holds it’s effect.

\section{Spell Descriptions}\label{sec:Spell Descriptions}
\subsection{Aerial Servant}\index[spells]{Aerial Servant}\label{spell:Aerial Servant}

\statblock{\textit{Cleric 6, Druid 6}

\textbf{Target:} None

\textbf{Range:} 60 ft.

\textbf{Duration:} 1 day/level}

This spell summons an aerial servant (see \fullref{monster:Aerial Servant}) and gives it a specific task. When the aerial servant arrives, the caster must describe a creature or object (and can optionally describe its location to make the aerial servant’s task easier and more likely to succeed).

The aerial servant will then search for the object or creature and do its best (fighting to the death if necessary) to return it to the caster. It will not be distracted from this task except to defend itself.

However, if the duration of the spell runs out before the aerial servant has returned with the item then the aerial servant will drop the item (if it already has it) and return immediately to the caster and attempt to slay them unless the caster was an \iref[chap:Immortals]{Immortal}.

\textbf{Environmental Effect:} If this spell is cast underwater, an Undine is summoned rather than an Aerial Servant.

\subsection{Analyze}\index[spells]{Analyze}\label{spell:Analyze}
\statblock{\textit{Elf 1, Wizard 1}

\textbf{Target:} One magic item

\textbf{Range:} Special (see below)

\textbf{Duration:} Instant}

To use an analyze spell, the caster must imitate using the item. This includes wearing armor (which won’t disrupt this spell), wielding weapons, wetting one’s lips with a potion, and so on.

Any curses or other ill effects of the item may affect the caster.

The caster then has a chance equal to 15\% plus 5\% per level to identify one magical property of the item and how to activate it (or that the item has no more unrevealed magical properties). If an item has more than one magical property, then the one that is revealed will be one that the caster does not already know about.

The caster does find out command words or other instructions needed to activate the power(s) that are discovered on the item, but does not discover how many charges each power has.

\subsection{Animal Charm}\index[spells]{Animal Charm}\label{spell:Animal Charm}
\statblock{\textit{Medicine Man 2}

\textbf{Target:} 1 non-magical creature

\textbf{Range:} 60 ft.

\textbf{Duration:} 1 hour}


\subsection{Animate Dead}\index[spells]{Animate Dead}\label{spell:Animate Dead}
\statblock{\textit{Cleric 4, Druid 4, Sorcerer 5, Wizard 5}

\textbf{Target:} One or more corpses

\textbf{Range:} 60 ft.

\textbf{Duration:} Permanent}

When this spell is cast, a number of dead bodies or skeletons within range will be animated and will become zombies or skeletons respectively.

A created skeleton will have the same number of hit dice as the race of the original creature had (not including extra hit dice gained from class levels). A created zombie will have one more hit dice than the original creature had (not including extra hit dice gained from class levels).

Therefore, a human or demi-human skeleton will always have 1 hit die, and a human or demi-human zombie will always have 2 hit dice.

Each casting of the spell will create a total number of hit dice of undead equal to the caster’s level, starting with those nearest the caster.

See \fullref{sec:Monster Descriptions} for more details about skeletons and zombies.

The animated undead will mindlessly obey the commands of the caster, and there is no limit to the total number of undead that the caster can create and control using multiple castings of this spell.

The zombies and skeletons created by this spell can be turned or destroyed normally. Unless the caster of this spell is an \iref[chap:Immortals]{Immortal}, they are also vulnerable the \iref[spell:Dispel Magic]{Dispel Magic} spell.

\subsection{Animate Objects}\index[spells]{Animate Objects}\label{spell:Animate Objects}
\statblock{\textit{Cleric 6, Druid 6, Fey 5}

\textbf{Target:} One or more objects

\textbf{Range:} 60 ft.

\textbf{Duration:} 1 hour}

This spell will animate a number of non-magical objects, giving them the capability of movement and causing them to mindlessly obey the caster’s commands, including being made to attack opponents. Each casting of the spell will animate one large object (such as a bed or tree), two medium objects (such as a table or chest), four small objects (such as chairs), or eight tiny objects (such as candlesticks). Two objects of a size smaller can be substitutes for one object of any given size—for example a single casting can animate a medium table, two small chairs and four tiny candlesticks.

This spell can be used to animate normally static plants, even if they are alive; but cannot be used to animate or control the movement of normally mobile creatures unless they are dead.

A dead body animated by this spell is not undead and uses the statistics of an animated object of the appropriate size and not those of a zombie.

See \fullref{sec:Animated Object} for details and abilities of animated objects of different sizes.

\subsection{Anti-Animal Shell}\index[spells]{Anti-Animal Shell}\label{spell:Anti-Animal Shell}
\statblock{\textit{Dervish 6, Druid 6, Elf 6, Fey 4}

\textbf{Target:} Caster

\textbf{Range:} Personal

\textbf{Duration:} 10 mins/level}

This spell prevents any animals from coming within an inch of the caster’s body, and therefore prevents them from physically attacking the caster. This protection also prevents the caster from physically attacking animals for the duration.

Since the protection is only against the animals themselves, it does not prevent such things as animals throwing things at the caster or spitting at the caster. Similarly, it does not prevent the caster from using missile weapons or spells against animals—unless those spells have a touch range.

\subsection{Anti-Magic Shell}\index[spells]{Anti-Magic Shell}\label{spell:Anti-Magic Shell}
\statblock{\textit{Elf 6, Fey 5, Wizard 6}

\textbf{Target:} Caster

\textbf{Range:} Personal

\textbf{Duration:} 2 hours}

When this spell is cast, it creates an invisible shell around the caster that prevents the passage of magic in either direction. For the duration of the spell, the caster can not be affected by any spell or other magical effect except those that they cast themselves; but also cannot affect anything other than themselves with their magic; since magic cannot pass through the shell in either direction.

The shell cannot be bypassed by any other magic except a \iref[spell:Wish]{Wish} spell, which can be used to destroy it.

The caster can cancel the shell at any time before the duration has expired, but if they do so then they can’t put it back without casting this spell again.

\subsection{Anti-Plant Shell}\index[spells]{Anti-Plant Shell}\label{spell:Anti-Plant Shell}
\statblock{\textit{Druid 5, Fey 3}

\textbf{Target:} Caster

\textbf{Range:} Personal

\textbf{Duration:} 1 rnd/level}

This spell prevents any plants from coming within an inch of the caster’s body, and therefore prevents plant-like monsters from physically attacking the caster. This protection also prevents the caster from physically attacking plants or plant-like monsters for the duration.

Since the protection is only against the plants themselves, it does not prevent such things as plants shooting spores at the caster. Similarly, it does not prevent the caster from using missile weapons or spells against plants—unless those spells have a touch range.

While using this spell, the caster can push their way through thorny and tangled undergrowth without danger, leaving a trail that others can follow.

\subsection{Barrier}\index[spells]{Barrier}\label{spell:Barrier}
\statblock{\textit{Cleric 6, Druid 6}

\textbf{Target:} 15-foot radius ring

\textbf{Range:} 60 ft.

\textbf{Duration:} 2 hours}

When this spell is cast, a ring shaped wall of spinning and whirling hammers appears up to 15 feet in radius and up to 30 feet tall.

The barrier must be created where there is room for it—it cannot be created on top of creatures or objects. Once it is in place, any creature passing through it takes 7d10 damage from the hammers.

\textbf{Reverse:} \hypertarget{spell:Remove Barrier}{Remove Barrier}\index[spells]{Remove Barrier} will destroy the effect created by any one of the following spells: \iref[spell:Barrier]{Barrier}, \iref[spell:Clothform]{Clothform}, \ilink{spell:Wall of Ice}{Wall of Ice}, \iref[spell:Wall of Fire]{Wall of Fire}, \iref[spell:Wall of Stone]{Wall of Stone}, \iref[spell:Woodform]{Woodform}.

\subsection{Bind Totem}\index[spells]{Bind Totem}\label{spell:Bind Totem}
\statblock{\textit{Medicine Man 4}

\textbf{Target:} One animal

\textbf{Range:} 60 ft.

\textbf{Duration:} Permanent}



\subsection{Bless}\index[spells]{Bless}\label{spell:Bless}
\statblock{\textit{Cleric 2, Druid 2, Medicine Man 2, Shaman 2}

\textbf{Target:} All allies in a 20-foot radius

\textbf{Range:} 60 ft.

\textbf{Duration:} 1 hour}

When this spell is cast, all friendly creatures in the area of effect gain a +1 bonus to attack, damage, and morale rolls for the duration, providing they are not already engaged in combat.

\textbf{Reverse:} \hypertarget{spell:Blight}{Blight}\index[spells]{Blight} gives each enemy in the area a -1 penalty to attack, damage and morale rolls for the duration providing they are not already in combat. Each of these enemies may make a saving throw vs. spells to avoid the effect.

\subsection{Call Lightning}\index[spells]{Call Lightning}\label{spell:Call Lightning}
\statblock{\textit{Dervish 3, Druid 3, Elf 3, Fey 4, Medicine Man 3}

\textbf{Target:} One or more 10-foot radius bolts

\textbf{Range:} 360 ft.

\textbf{Duration:} 10 mins/level}

This spell can only be used outside when there is stormy weather. Once per 10 minutes (60 rounds) the caster may direct the storm to strike the area of their choosing (within range) with lightning. All creatures within 10 feet of the strike take 8d6 points of electrical damage, but can make a saving throw vs. spells to take half damage.

\textbf{Environmental Effect:} This spell has no effect if cast underwater.

\subsection{Call Totem}\index[spells]{Call Totem}\label{spell:Call Totem}
\statblock{\textit{Medicine Man }

\textbf{Target:} One totem animal

\textbf{Range:} Special

\textbf{Duration:} Special}



\subsection{Charm Animal}\index[spells]{Charm Animal}\label{spell:Charm Animal}
\statblock{\textit{Dervish 4, Elf 4, Sorcerer 4, Wizard 4}

\textbf{Target:} One or more animals

\textbf{Range:} 120 ft.

\textbf{Duration:} Special}

When this spell is cast, one or more animals will imagine the caster to be their best friend, and treat them accordingly. It does not make them fanatically loyal and will not make them attack their other friends.

If cast on animals with 3 hit dice or fewer, it will affect 3d6 targets, otherwise it will affect only a single target. In either case, all targets get a saving throw vs. spells to avoid the effect.

If a target fails its saving throw, then it gets another one periodically to throw off the charm effect. The frequency of the saving throw is based on the animal’s \iref[sec:Intelligence]{Intelligence} as indicated on \fullref{tab:Charm}.

If the caster behaves in an overtly hostile manner to the charmed target, such as attacking it or ordering others to attack it, then the charm is broken.

If an \iref[chap:Immortals]{Immortal} is charmed using this spell, they may make a save every 10 minutes even if they have less than 21 \iref[sec:Intelligence]{Intelligence}.

\subsection{Charm Monster}\index[spells]{Charm Monster}\label{spell:Charm Monster}
\statblock{\textit{Elf 4, Fey 3, Sorcerer 4, Wizard 4}

\textbf{Target:} One or more creatures

\textbf{Range:} 120 ft.

\textbf{Duration:} Special}

When this spell is cast, one or more creatures will imagine the caster to be their best friend, and treat them accordingly. It does not make them fanatically loyal and will not make them attack their other friends.

This spell works on any living creature, but not on undead or non-living creatures such as golems.

If cast on creatures with 3 hit dice or fewer, it will affect 3d6 targets, otherwise it will affect only a single target. In either case, all targets get a saving throw vs. spells to avoid the effect.

If a target fails its saving throw, then it gets another one periodically to throw off the charm effect. The frequency of the saving throw is based on the creature’s \iref[sec:Intelligence]{Intelligence} as indicated on \fullref{tab:Charm}.

If the caster behaves in an overtly hostile manner to the charmed target, such as attacking it or ordering others to attack it, then the charm is broken.

If an \iref[chap:Immortals]{Immortal} is charmed using this spell, they may make a save every 10 minutes even if they have less than 21 \iref[sec:Intelligence]{Intelligence}.

\subsection{Charm Person}\index[spells]{Charm Person}\label{spell:Charm Person}
\statblock{\textit{Elf 1, Fey 1, Wizard 1}

\textbf{Target:} One or more humanoids

\textbf{Range:} 120 ft.

\textbf{Duration:} Special}

When this spell is cast, one humanoid creature will imagine the caster to be their best friend, and treat them accordingly. It does not make them fanatically loyal and will not make them attack their other friends.

This spell works on any living humanoid, but not on undead or non-living creatures such as golems and not on non-humanoid creatures. The target gets a saving throw vs. spells to avoid the effect.

If the caster behaves in an overtly hostile manner to the charmed target, such as attacking it or ordering others to attack it, then the charm is broken.

If the target fails its saving throw, then they get another one periodically to throw off the charm effect. The frequency of the saving throw is based on the target’s \iref[sec:Intelligence]{Intelligence} as indicated on \fullref{tab:Charm}.

\begin {table}[H]
\caption{Charm}\label{tab:Charm}
  \begin{tabularx}{\columnwidth}{>{\bfseries}YY}
	\thead{Intelligence} & \thead{Frequency}\\
	0 & Save every 120 days\\
	1 & Save every 90 days\\
	2 & Save every 60 days\\
	3 & Save every 45 days\\
	4-5 & Save every 30 days\\
	6-8 & Save every 15 days\\
	9-12 & Save every 7 days\\
	13-15 & Save every 3 days\\
	16-17 & Save every 24 hours\\
	18 & Save every 8 hours\\
	19 & Save every 3 hours\\
	20 & Save every hour\\
	21-29 & Save every 10 minutes\\
	30+ & Save every round
  \end {tabularx}
\end {table}

If an \iref[chap:Immortals]{Immortal} is charmed using this spell, they may make a save every 10 minutes even if they have less than 21 \iref[sec:Intelligence]{Intelligence}.

\subsection{Charm Plant}\index[spells]{Charm Plant}\label{spell:Charm Plant}
\statblock{\textit{Elf 7, Fey 4, Wizard 7}

\textbf{Target:} One or more plants

\textbf{Range:} 120 ft.

\textbf{Duration:} Special}

When this spell is cast, one plant or plant-like creature will imagine the caster to be their best friend, and treat them accordingly. In the case of plant-like creatures, it does not make them fanatically loyal and will not make them attack their other friends.

This spell will affect one tree or plant-like creature, six medium-sized bushes, or 24 smaller plants. If the target is a plant-like creature such as a treant, it gets a saving throw vs. spells to avoid the effect. Normal plants get no save.

Normal plants affected by this spell will understand the commands of the caster and mindlessly carry them out, such as entangling or attacking creatures that come near them.

On normal plants, the effect lasts for 6 months.

If an intelligent plant-like target fails its saving throw, then it gets another one periodically to throw off the charm effect. The frequency of the saving throw is based on the target’s \iref[sec:Intelligence]{Intelligence} as indicated on \fullref{tab:Charm}.

If the caster behaves in an overtly hostile manner to a charmed plant-like creature, such as attacking it or ordering others to attack it, then the charm is broken.

\subsection{Charm, Mass}\index[spells]{Charm, Mass}\label{spell:Charm, Mass}
\statblock{\textit{Elf 8, Fey 5, Wizard 8}

\textbf{Target:} One or more creatures

\textbf{Range:} 120 ft.

\textbf{Duration:} Special}

When this spell is cast, one or more creatures will imagine the caster to be their best friend, and treat them accordingly. It does not make them fanatically loyal and will not make them attack their other friends.

If a target fails its saving throw, then it gets another one periodically to throw off the charm effect. The frequency of the saving throw is based on the creature’s \iref[sec:Intelligence]{Intelligence} as indicated on \fullref{tab:Charm}.

If the caster does something to break the charm against one target, such as attacking it, then other targets who see the incident get an immediate saving throw to break their charm too.

This spell works on any living creature, but not on undead or non-living creatures such as golems. The spell will affect a total of 30 hit dice or levels of creatures, and all targets get a saving throw vs. spells at a penalty of -2 to avoid the effect. Creatures who make the saving throw still count towards the 30 hit dice limit.

\textbf{Reverse:} \hypertarget{spell:Remove Charm}{Remove Charm}\index[spells]{Remove Charm} removes all charm effects in a 10-foot radius with no chance of failure, and also prevents any object or creature within that radius from creating charm effects for 10 minutes.

\subsection{Chill}\index[spells]{Chill}\label{spell:Chill}
\statblock{\textit{Fey 1}

\textbf{Target:} One creature

\textbf{Range:} 30 ft.

\textbf{Duration:} Concentration}

The spell reduces the body temperature of a creature, which inflict 1 point of damage per round for as long as the caster concentrates. The first round of damage is automatic, with subsequent rounds allowing a saving throw vs. spells. The spell end when either the caster's concentration is broken or the target makes a successful save.

\subsection{Clairvoyance}\index[spells]{Clairvoyance}\label{spell:Clairvoyance}
\statblock{\textit{Elf 3, Fey 3, Sorcerer 3, Wizard 3}
\textbf{Target:} One creature at a time

\textbf{Range:} 60 ft.

\textbf{Duration:} 2 hours}

This spell allows the caster to see through the eyes of any one creature within range. Once the spell is cast, it takes 10 minutes to start seeing through the eyes of a creature, but then once the connection is established, the caster can change to a different creature (within range of the caster and within sight of the current creature) instantly. The caster can keep changing creatures until the spell duration runs out.

The creatures affected by this spell do not get a saving throw and are not aware that they are affected by it. The sight is temporarily blocked if the creature goes out of range or there is more than two feet or rock (or a thin coating of lead) between the creature and the caster.

\subsection{Clone}\index[spells]{Clone}\label{spell:Clone}
\statblock{\textit{Wizard 8}

\textbf{Target:} One creature

\textbf{Range:} 10 ft.

\textbf{Duration:} Permanent}

This spell is cast on a piece of flesh that has been taken from a single living or dead creature (but not a non-living or undead one).

The caster must place the flesh in a vat of alchemical reagents and let it slowly grow into a copy of the original creature.

The difficulty (and effect) of making a clone depends on the type of creature cloned.

\textbf{Human or Demi-Human:} The process takes a week per level of the target, and each such week costs 5,000 gp in reagents and components. The piece of flesh used must be at least one pound in weight.

When the clone is fully grown, it wakes up and has the memories, personality and abilities (including level) that the original had at the time the flesh was taken. Note that if the target is alive then they will have gained more memories—and possibly more experience—since that time. The clone will not have these.

If the original is still alive when the clone wakes up (or if the original is raised from the dead after this time) and is on the same plane (and within the same Celestial Sphere, if on the prime material plane) as the clone, a mind link is immediately established between the two of them. They are both aware of each other’s existence and emotions. Further, any damage taken by one of them is also taken by the other, although the other can make a saving throw vs. spells to only take half damage.

The clone will also immediately become obsessed with destroying their original, even at the cost of their own life. This mind link remains even if the original and clone are separated onto different planes or Celestial Spheres, but if either one dies then it will be canceled. The clone cannot be raised or reincarnated, but if the original was the one that died then raising or resurrecting them will cause the mind-link to re-establish itself.

After the pair have been mind-linked for a number of days equal to the level of the caster of the clone spell, and both are still alive, the clone will become completely insane. The original will permanently lose a point of \iref[sec:Intelligence]{Intelligence} and a point of \iref[sec:Wisdom]{Wisdom} when this happens, and has a 5\% chance (not cumulative) per day of also going insane. Once both are insane, they will both die a week later. The clone cannot be raised or reincarnated, and the original can only be raised by a \iref[spell:Wish]{Wish} spell.

A human or demi-human can only have one clone at a time. Any attempt to make a second clone will fail. However, if the original is dead, the clone is effectively an independent being, and can be cloned (or raised) itself. Should the original be raised, all clones will be mind-linked as above.

If a clone is made of the preserved flesh of a person who has become an \iref[chap:Immortals]{Immortal}, it will retain the personality that the \iref[chap:Immortals]{Immortal} had during their mortal life, but not their abilities or memories. Instead, it will be a \nth{1} level character of the immortal’s old class with no memories (and no knowledge that they are connected with the \iref[chap:Immortals]{Immortal} in any way).

\textbf{Other Creatures:} If the clone is made from another living creature other than a human or demi-human, the process takes a week per hit die of the target, and each such week costs 500 gp in reagents and components. The piece of flesh used must be at least one percent of the weight of the original creature.

Once the clone wakes up, it will unfailingly obey the commands of its creator, and the caster can mentally command the clone when within 10 feet of it.

The clone will begin with only 50\% of the physical abilities (hit points, damage caused by attacks, strength, size) of the original creature, and will have a 50\% chance to possess each special ability (except spell like abilities) that the original creature possessed.

Each week after waking, the clone continues to grow. The physical abilities of the clone increase by a further 5\%, and it can re-roll for any special abilities (including spell like abilities) that it has not yet acquired, until 8 weeks have passed (at which time it will have 90\% of the physical attributes of the original and will have had 8 chances to re-roll for special abilities). At this point it stops growing.

The clone does not have any special connection to the original creature.

\subsection{Clothform}\index[spells]{Clothform}\label{spell:Clothform}
\statblock{\textit{Wizard 4}

\textbf{Target:} None

\textbf{Range:} Touch

\textbf{Duration:} Instant}

When this spell is cast, it creates a single piece of un-worked and un-dyed linen up to 30 by 30 feet in size.

If the caster makes a \iref[sec:Dexterity]{Dexterity} check (with an appropriate craft skill) then the cloth can be created in a more finished form, such as with seams or twisted into 60 feet of rope. The Game Master must decide on any penalties to the caster’s effective \iref[sec:Dexterity]{Dexterity} based on the complexity of what is desired.

The cloth is non-magical once created, and cannot be dispelled.

If the caster chooses to leave one or more ends of the cloth unfinished, with loose thread hanging from it, then a second casting of this spell can add to the existing cloth at that edge rather than creating a separate piece.

The cloth comes out of the caster’s hands when created and falls in a heap on the floor. The caster cannot use this spell to create cloth over targets’ heads or create cloth attached to (or tying) anything.

\subsection{Cloudkill}\index[spells]{Cloudkill}\label{spell:Cloudkill}
\statblock{\textit{Sorcerer 5, Wizard 5}

\textbf{Target:} 15-foot radius

\textbf{Range:} 1 ft.

\textbf{Duration:} 1 hour}

This spell creates a 15-foot radius and 20-foot-tall cloud of poisonous gas, the closest edge of which must be within 1 foot of the caster. The gas is dense enough to be visible, but does not block sight. If cast within an enclosed space, the cloud may be smaller than the dimensions above. It will not expand to fill the same volume.

The cloud moves in a straight line away from the caster at a rate of 20 feet per round, although it is also affected by winds. It is heavier than air, so will sink through holes or into pits, and will go around rather than over obstacles. If the cloud moves into thick vegetation it will dissipate.

The poisonous nature of the cloud means that all living (but not undead or non-living) creatures within it take 1 point of damage per round. Living creatures with fewer than 5 hit dice must also make a saving throw vs. poison each round or be killed.

\textbf{Environmental Effect:} If this spell is cast underwater, the cloud is instead a mass of dark green bubbles that rise 60 feet per round.

\subsection{Command Word}\index[spells]{Command Word}\label{spell:Command Word}
\statblock{\textit{Elf 1, Medicine Man 1}

\textbf{Target:} One living creature

\textbf{Range:} 10 ft.

\textbf{Duration:} 1 round}

The caster utters a single word that the target living creature must abide by. The word must be in a language the creature understands. A command of "Die" will result in the creature fainting for a round. Creatures with an Intelligence of 13 or higher or with 6 or more HD are allowed a saving throw vs. spells to ignore the command.

\subsection{Commune}\index[spells]{Commune}\label{spell:Commune}
\statblock{\textit{Cleric 5, Dervish 5, Druid 5, Medicine Man 5}

\textbf{Target:} Caster

\textbf{Range:} Personal

\textbf{Duration:} 30 minutes}

This spell can be cast only once per week. The caster uses it to ask three questions of their patron \iref[chap:Immortals]{Immortal} that can be answered with a “yes” or a “no”. The \iref[chap:Immortals]{Immortal} will answer these questions to the best of their ability, which will almost always be sufficient to answer correctly. The \iref[chap:Immortals]{Immortal} may lie if they have a reason to mislead the caster, and on very rare occasions may be forced to answer “unknown”.

If the \iref[chap:Immortals]{Immortal} is more than 10 planar boundaries away from the caster, or is in a different Celestial Sphere, this spell will fail to contact them.

If this spell is cast on one particular day of the year (which will vary from religion to religion), the caster will get to ask 6 questions instead of the usual 3.

When cast by an \iref[chap:Immortals]{Immortal}, this spell allows a full telepathic conversation with another \iref[chap:Immortals]{Immortal} of their choice (providing that \iref[chap:Immortals]{Immortal} is no more than 10 planes away and is in the same Celestial Sphere) rather than just a limited number of yes/no questions

\subsection{Confound}\index[spells]{Confound}\label{spell:Confound}
\statblock{\textit{Medicine Man }

\textbf{Target:}

\textbf{Range:}

\textbf{Duration:} }



\subsection{Confusion}\index[spells]{Confusion}\label{spell:Confusion}
\statblock{\textit{Elf 4, Fey 4, Wizard 4}

\textbf{Target:} 3d6 creatures in a 30-foot radius

\textbf{Range:} 120 ft.

\textbf{Duration:} 12 rounds}

When this spell is cast, 3d6 creatures within 30 feet of the target point of the spell, starting with the closest, will be both confused and enraged, wishing to lash out at enemies but being unable to determine who those enemies are. Creatures with 2 hit dice or fewer will be automatically affected, but creatures with more than 2 hit dice may make a saving throw vs. spells each round to shake off the effect. Creatures who leave the area automatically shake off the effect, and new creatures who enter the area are unaffected.

Confused creatures act randomly each round as indicated on \fullref{tab:Confusion}.

\begin {table}[H]
  \caption{Confusion}\label{tab:Confusion}
  \begin{tabularx}{\columnwidth}{>{\bfseries}YY}
	\thead{2d6} & \thead{Effect}\\
	2-5 & Attack the caster’s party\\
	6-8 & Do nothing but shout and scream\\
	9-12 & Attack the creature’s own party
  \end {tabularx}
\end {table}

\subsection{Conjure Elemental}\index[spells]{Conjure Elemental}\label{spell:Conjure Elemental}
\statblock{\textit{Dervish 5, Elf 5, Fey 6, Wizard 5}

\textbf{Target:} None

\textbf{Range:} 240 ft.

\textbf{Duration:} Concentration}

When this spell is cast, a 16 hit dice Elemental (see \fullref{sec:Elemental}) will appear within 240 feet of the caster. If this spell is cast more than once during the same day, a different type of elemental must be conjured each time.

While the caster controls the elemental, they can make it do anything it is capable of doing, including fighting to the death on the caster’s behalf. The caster can also send the controlled elemental home.

The caster must concentrate to keep controlling the elemental, and cannot fight or cast other spells or move at more than half normal speed. If the caster’s concentration is broken, either because they did one of those things or they take damage, then the control is broken.

Once the control is broken, it can not be re-established. The elemental will try to kill the caster who conjured it, but will not commit suicide doing so. If it looks to be a hopeless fight the elemental will flee instead.

A conjured elemental is blocked by a Protection from Evil, and can be sent home by a Dispel Magic or a Dispel Evil.

\textbf{Environmental Effect:} If this spell is cast underwater, only an Earth Elemental or Water Elemental may be conjured. Earth Elemental must stay in contact with the ground or they will be sent home in 1d4 rounds. Each type of elemental may be summoned up to four times a day.

\subsection{Contact Outer Plane}\index[spells]{Contact Outer Plane}\label{spell:Contact Outer Plane}
\statblock{\textit{Elf 5, Wizard 5}

\textbf{Target:} Caster

\textbf{Range:} Personal

\textbf{Duration:} Special}

This spell contacts an \iref[chap:Immortals]{Immortal} or other powerful entity on another plane and asks it questions. It can only be cast once per month. It is a risky procedure, since the \iref[chap:Immortals]{Immortal} is under no obligation to the caster and may not appreciate being disturbed. The caster may not get accurate answers and may be driven insane by the mental contact.

The caster must choose which plane to direct their questions to—further planes allow more questions but also increase the chance of insanity, as shown on \fullref{tab:Contact Outer Plane}. The chance of insanity must be checked first before any questions are asked.

For each level of the caster above level 20, the chance of insanity is reduced by 5\%. If the caster is driven insane, no questions are answered and in will take a number of weeks equal to the number of questions asked for them to recover.

If the caster is not driven insane, they may ask a number of questions equal to the distance to the plane contacted plus two, with the listed chance of each question being answered correctly. Incorrect answers may be because the \iref[chap:Immortals]{Immortal} does not know the answer, or may be simply because the \iref[chap:Immortals]{Immortal} is unhappy about being contacted and is lying to the caster.

When cast by an \iref[chap:Immortals]{Immortal}, this spell allows a full telepathic conversation with another \iref[chap:Immortals]{Immortal} of their choice (providing that \iref[chap:Immortals]{Immortal} is no more than 10 planes away and is in the same Celestial Sphere) rather than just a limited number of questions, and the caster has no chance of going insane.

\begin {table}[H]
  \caption{Contact Outer Plane}\label{tab:Contact Outer Plane}
  \begin{tabularx}{\columnwidth}{>{\bfseries}cYYY}
	\thead{} & \multicolumn{3}{c}{\thead{Chance Of...}}\\
	\thead{Distance /Number of Questions} & \thead{Correct} & \thead{False} & \thead{Insanity}\\
	1/3 & 25\% & 75\% & 5\%\\
	2/4 & 30\% & 70\% & 10\%\\
	3/5 & 35\% & 65\% & 15\%\\
	4/6 & 40\% & 60\% & 20\%\\
	5/7 & 50\% & 50\% & 25\%\\
	6/8 & 60\% & 40\% & 30\%\\
	7/9 & 70\% & 30\% & 35\%\\
	8/10 & 80\% & 20\% & 40\%\\
	9/11 & 90\% & 10\% & 45\%\\
	10/12 & 95\% & 5\% & 50\%
  \end {tabularx}
\end {table}

\subsection{Contingency}\index[spells]{Contingency}\label{spell:Contingency}
\statblock{\textit{Elf 9, Fey 7, Wizard 9}

\textbf{Target:} One creature, object or place

\textbf{Range:} Touch

\textbf{Duration:} Special}

When this spell is cast, the caster also casts a second spell at the same time. The second spell must be \nth{4} level or lower, and must not be a spell that causes damage.

The second spell does not go off immediately. Instead, the caster describes a situation upon which the second spell will activate, and the spell remains dormant until that time.

If the contingent spell has parameters that need to be decided at the time of casting, they must be decided at the time the contingency is set.

When the situation comes about, the spell triggers automatically and immediately. This may interrupt the action that caused the contingency to apply.

\example{Aloysius casts a Contingency on himself with a Dimension Door spell setting the following condition: “If I am about to be dealt a blow that would knock me unconscious, Dimension Door me one hundred feet to the left of my position, or to the nearest open space to that point.”

Some weeks later he is bitten by a large dragon. The Game Master rolls the damage for the attack, and it would knock him unconscious. The contingency kicks in and he is teleported away before taking the damage.}

A contingency will last until discharged, and each creature or object can only have a single contingency active on it at any time. Casting a second contingency dissipates the first. A waiting contingency cannot be dispelled, but if the contingent spell has a duration it can be dispelled normally once it has activated.

\subsection{Continual Light}\index[spells]{Continual Light}\label{spell:Continual Light}
\statblock{\textit{Cleric 3, Druid 3, Fey 2, Shaman 3, Sorcerer 2, Wizard 2}

\textbf{Target:} 30-foot radius

\textbf{Range:} 120 ft.

\textbf{Duration:} Permanent}

When this spell is cast, the area within 30 feet of the target point is lit with light as bright as daylight on an overcast day.

This area will continue to be lit until it is dispelled.

The caster can choose to either cast this spell in a location, in which case it will stay in that location, or cast it on an object—in which case it will move as the object moves.

This spell creates ambient light throughout the area, not a light source in the center of the area. There are no shadows in the area covered by this spell, and covering the object that the spell is centered on will not block out the light. However, any amount of lead or 6 inches of stone will block the area.

If this spell is cast on a creature’s eyes, that creature must make a saving throw vs. spells or be \iref[sec:Blinded]{Blinded} until the spell is canceled.

See \fullref{sec:Light vs. Darkness} for details about how different types of natural and magical light and darkness interact.

\textbf{Reverse:} \hypertarget{spell:Continual Darkness}{Continual Darkness}\index[spells]{Continual Darkness} causes the area within 30 feet of the target point to be absolutely dark, with not even the heat vision of some demi-humans or the dark vision of some monsters able to penetrate it.

This spell creates ambient darkness rather than darkness radiated from a center point, so covering the object that the spell was cast on will not block the darkness. However, any amount of lead or 6 inches of stone will block the area.

If this spell is cast on a creature’s eyes, that creature must make a saving throw vs. spells or be \iref[sec:Blinded]{Blinded} until the spell is canceled.

See \fullref{sec:Light vs. Darkness} for details about how different types of natural and magical light and darkness interact.

\subsection{Control Temperature 10-foot radius}\index[spells]{Control Temperature 10-foot radius}\label{spell:Control Temperature 10-foot radius}
\statblock{\textit{Druid 4, Elf 5, Fey 5}

\textbf{Target:} 10-foot radius

\textbf{Range:} Personal

\textbf{Duration:} 10 mins/level}

This spell allows the caster to alter the air temperature within 10 feet of themselves. The spell will protect the caster and all around them from hot or cold environments, but will not protect them from fire sources or hot or cold items.

The caster can change the ambient temperature by concentrating for a round, and can vary that temperature from -5 to 40 degrees celsius (23 to 104 fahrenheit).

\subsection{Control Winds}\index[spells]{Control Winds}\label{spell:Control Winds}
\statblock{\textit{Dervish 5, Druid 5, Elf 5, Fey 5}

\textbf{Target:} 10-foot radius/level

\textbf{Range:} Personal

\textbf{Duration:} 10 mins/level}

This spell lets the caster completely control the speed and direction of wind within the area, from dead calm to gale force.

Changing the wind is slow and it can take up to 10 minutes to change from one extreme to another.

If cast at an air elemental or other creature made of air, the creature can resist the spell by making a saving throw vs. spells. If this saving throw fails, the caster can completely control the creature for the duration of the spell—or even choose to kill it outright.

\textbf{Environmental Effect:} If this spell is cast in a desert, the spell creates a sandstorm causing all creatures in the area to become \iref[sec:Blinded]{Blinded}.

If this spell is cast underwater, it effects currents rather than wind and can target a Water Elemental rather than an Air Elemental.

\subsection{Courier}\index[spells]{Courier}\label{spell:Courier}
\statblock{\textit{Fey 1}

\textbf{Target:} None

\textbf{Range:} 10 ft.

\textbf{Duration:} 1 day/level}

The caster summons one small animal that will deliver a message. The message can be spoken to the animal in which the animal will understand and repeat the message for the recipient or the animal can carry a small piece of paper such as a scroll. The caster must be able to accurately describe the location of the recipient and the recipient must be within a mile of the location described for the message to be delivered. If the duration expires before the animal finds the recipient than the animal will forget it's task. If the duration allows, the recipient may use the animal to send a response. 

\subsection{Create Air}\index[spells]{Create Air}\label{spell:Create Air}
\statblock{\textit{Wizard 3}

\textbf{Target:} 8,000 cubic feet or one creature or object

\textbf{Range:} Touch

\textbf{Duration:} 1 hour/level}

This spell can be cast in a static area of up to 8,000 cubic feet (20 by 20 by 20 feet or the equivalent) in order to fill that area with breathable air for the duration.

It can also be cast on an enclosed object, from a small one such as a helmet to one as large as 20 by 20 by 20 feet such as the interior of a ship’s hold in order to fill that object with constantly refreshing air for the duration even if the object moves around. If the object is not airtight then the air will constantly leak out and be replaced.

Finally, it can be cast on a creature to surround the creature in a thin skin-like bubble of breathable air for the duration, even if the creature moves. In this case, a creature that flies using wings can use that air around it to fly even if there is no other air, so it can fly through an airless Void or even the Luminiferous Aether. This bubble will maintain its integrity underwater or in a such a Void.

In any of these cases, although the spell provides air where there may be none, it doesn’t stop poison from mixing with that air—so it provides no protection from \iref[spell:Cloudkill]{Cloudkill} spells or the poisonous breath of some monsters.

\subsection{Create Any Monster}\index[spells]{Create Any Monster}\label{spell:Create Any Monster}
\statblock{\textit{Wizard 9}

\textbf{Target:} None

\textbf{Range:} 90 ft.

\textbf{Duration:} 30 minutes}

This spell causes monsters to temporarily appear and obey the caster’s commands for the duration, before disappearing.

Any type of creature can be created except for humans and demi-humans, and the creatures will be typical for their species. Creatures of only one species can be created per casting.

The total number of hit dice of creatures that can be created at once is equal to the caster’s level, and if the creatures are humanoid then they appear with normal (non-magical) equipment that disappears when they do.

This spell can also be used to create a construct. See \fullref{chap:Monsters} for details of different types of construct and \fullref{sec:Making Constructs} for how to create them. When used to create a construct, the duration of the spell is permanent, and the construct cannot be destroyed by a \iref[spell:Dispel Magic]{Dispel Magic}, although it is still affected by \iref[spell:Protection from Evil]{Protection from Evil}, \iref[spell:Dispel Evil]{Dispel Evil}, and \iref[spell:Anti-Magic Shell]{Anti-Magic Shell}.

\subsection{Create Food}\index[spells]{Create Food}\label{spell:Create Food}
\statblock{\textit{Cleric 5, Druid 5, Fey 4, Shaman 5}

\textbf{Target:} None

\textbf{Range:} 10 ft.

\textbf{Duration:} Permanent}

Each time this spell is cast, it creates enough food to feed up to 36 humans or demi-humans. If mounts such as horses are fed using this spell, each one eats food equivalent to 2 humans.

For every caster level above 10, 36 extra people can be fed by this spell, although the caster can produce less food if desired.

The food created by this spell is similar to a bland and almost tasteless porridge, but is highly nutritious. However, it will spoil after 24 hours.

\textbf{Environmental Effect:} If this spell is cast underwater, the conjured food is polluted by the salt water. Anyone who eats it suffers from stomach cramps for 1d6 minutes unless they can make a saving throw vs. poison. Creatures effected by stomach cramps suffer a -4 penalty to hit, move a quarter of their normal speed, and suffer a -5 penalty to all saving throws made against spells that have an area effect.

\subsection{Create Magical Monsters}\index[spells]{Create Magical Monsters}\label{spell:Create Magical Monsters}
\statblock{\textit{Fey 7, Wizard 8}

\textbf{Target:} None

\textbf{Range:} 60 ft.

\textbf{Duration:} 20 minutes}

This spell causes monsters to temporarily appear and obey the caster’s commands for the duration, before disappearing.

Any type of creature that has up to two special abilities (up to two asterisks on its hit-dice) can be created except for humans and demi-humans, and the creatures will be typical for their species. Creatures of only one species can be created per casting.

The total number of hit dice of creatures that can be created at once is equal to the caster’s level, and if the creatures are humanoid then they appear with normal (non-magical) equipment that disappears when they do.

This spell can also be used to create a construct with up to two special abilities (up to two asterisks on its hit-dice). See \fullref{chap:Monsters} for details of different types of construct and \fullref{sec:Making Constructs} for how to create them. When used to create a construct, the duration of the spell is permanent, and the construct cannot be destroyed by a Dispel Magic, although it is still affected by Protection from Evil, Dispel Evil and Anti-Magic Shell.

If this spell is cast by an \iref[chap:Immortals]{Immortal}, the caster may choose to let the monsters remain in existence after the spell runs out, although if they do so the monsters will no longer be under their control.

\subsection{Create Normal Animals}\index[spells]{Create Normal Animals}\label{spell:Create Normal Animals}
\statblock{\textit{Cleric 6, Fey 5, Druid 6}

\textbf{Target:} None

\textbf{Range:} 30 ft.

\textbf{Duration:} 2 hours}

This spell causes animals to temporarily appear and obey the caster’s commands for the duration, before disappearing.

The spell will create one large animal (Camel, Dolphin, Elephant, etc.), three medium-sized animals (Black Bear, Mountain Lion, Pony, etc.), or six small animals (Bird of Prey, Rat, Spitting Cobra, etc.) and the creatures will be typical for their species. The caster can decide on the number and size of creature that they wish to create, but not on the actual species. Only one species of animal will appear per casting, and the animals will be typical for their species.

If this spell is cast by an \iref[chap:Immortals]{Immortal}, the caster may choose to let the animals remain in existence after the spell runs out, although if they do so the animals will no longer be under their control.

\subsection{Create Normal Monsters}\index[spells]{Create Normal Monsters}\label{spell:Create Normal Monsters}
\statblock{\textit{Elf 7, Fey 6, Wizard 7}

\textbf{Target:} None

\textbf{Range:} 30 ft.

\textbf{Duration:} 10 minutes}

This spell causes monsters to temporarily appear and obey the caster’s commands for the duration, before disappearing.

Any type of creature that has no special abilities (no asterisks on its hit-dice) can be created except for humans and demi-humans, and the creatures will be typical for their species. Creatures of only one species can be created per casting.

The total number of hit dice of creatures that can be created at once is equal to the caster’s level, and if the creatures are humanoid then they appear with normal (non-magical) equipment that disappears when they do.

This spell can also be used to create a construct with no special abilities (no asterisks on its hit-dice). See \fullref{chap:Monsters} for details of different types of construct and \fullref{sec:Making Constructs} for how to create them. When used to create a construct, the duration of the spell is permanent, and the construct cannot be destroyed by a \iref[spell:Dispel Magic]{Dispel Magic}, although it is still affected by \iref[spell:Protection from Evil]{Protection from Evil}, \iref[spell:Dispel Evil]{Dispel Evil}, and \iref[spell:Anti-Magic Shell]{Anti-Magic Shell}.

If this spell is cast by an \iref[chap:Immortals]{Immortal}, the caster may choose to let the monsters remain in existence after the spell runs out, although if they do so the monsters will no longer be under their control.

\subsection{Create Water}\index[spells]{Create Water}\label{spell:Create Water}
\statblock{\textit{Cleric 4, Dervish 4, Druid 4, Fey 3}

\textbf{Target:} None

\textbf{Range:} 10 ft.

\textbf{Duration:} 1 hour}

When this spell is cast, a magical spring will appear from the ground or a wall and flow for an hour.

The spring will provide enough water for 36 humans or demi-humans (50 gallons or enough water to fill a 10-by-10-by-2 foot pool) before drying up. If mounts such as horses are being watered, each one will take the same water as 2 humans. For each level of the caster above \nth{8}, enough water for an additional 36 humans will flow through.

The caster can stop the spring at any time before the duration has expired, although this will not make the existing water disappear.

\subsection{Creeping Doom}\index[spells]{Creeping Doom}\label{spell:Creeping Doom}
\statblock{\textit{Dervish 7, Druid 7, Elf 8, Fey 7, Medicine Man 8}

\textbf{Target:} None

\textbf{Range:} 120 ft.

\textbf{Duration:} 1 rnd/level}

This spell summons a swarm of hundreds of thousands of crawling insects and spiders. The swarm can vary from having a 10-foot radius to a 30-foot radius, and the caster can move the swarm up to 20 feet per round and also alter the radius on a round by round basis.

The swarm is initially capable of doing 1,000 points of damage per round, which must be split as evenly as possible between all creatures in the area of the swarm; although no individual creature can be dealt more than 100 damage, so if there are fewer than 10 creatures in the area then some of the potential damage will be wasted. The insects cannot damage creatures that can be hit only by magical weapons.

Normal attacks will slay many dozens of insects, with each point of damage reducing the damage potential of the swarm on a 1-for-1 basis, so if an attack deals 50 damage then the swarm will only be capable of doing 950 points of damage from then on. Area effect attacks, such as a \iref[spell:Fireball]{Fireball} spell, do double damage against the swarm.

A \iref[spell:Protection from Evil]{Protection from Evil} spell won’t keep the insects out, but a \iref[spell:Dispel Magic]{Dispel Magic} will work against the swarm with normal chances of success.

\textbf{Environmental Effect:} If this spell is cast underwater, shrimp are summoned rather than insects and spiders.

\subsection{Cure Blindness}\index[spells]{Cure Blindness}\label{spell:Cure Blindness}
\statblock{\textit{Cleric 3, Druid 3, Medicine Man 3, Shaman 3}

\textbf{Target:} One living creature

\textbf{Range:} Touch

\textbf{Duration:} Permanent}

This spell cures both mundane blindness and magical blindness caused by \iref[spell:Light]{Light} and \ilink{spell:Darkness}{Darkness} spells (and their continual versions). The only form of blindness it will not normally cure is that caused by a Curse. This spell can only cure blindness caused by a Curse when it is cast by an \iref[chap:Immortals]{Immortal}.

\subsection{Cure Blindness or Disease}\index[spells]{Cure Blindness or Disease}\label{spell:Cure Blindness or Disease}
\statblock{\textit{Fey 4}

\textbf{Target:} One living creature

\textbf{Range:} Touch

\textbf{Duration:} Permanent}

This spell duplicates the effects of a \iref[spell:Cure Blindness]{Cure Blindness} or \iref[spell:Cure Disease]{Cure Disease} spell. The caster chooses which effect will take place, unless the target suffers from both ailments in which case there is an equal chance of either being cured.

\subsection{Cure Critical Wounds}\index[spells]{Cure Critical Wounds}\label{spell:Cure Critical Wounds}
\statblock{\textit{Cleric 5, Druid 5, Medicine Man 5, Shaman 5}

\textbf{Target:} One living creature

\textbf{Range:} Touch

\textbf{Duration:} Permanent}

This spell cures one living (not undead or non-living) creature of 3d6+3 points of damage. The caster can cure themselves with this spell.

\textbf{Reverse:} \hypertarget{spell:Cause Critical Wounds}{Cause Critical Wounds}\index[spells]{Cause Critical Wounds} inflicts 3d6+3 damage to a touched living (not undead or non-living) target. The target gets no saving throw against the damage, but the caster must make a normal attack roll to touch an unwilling target.

\subsection{Cure Disease}\index[spells]{Cure Disease}\label{spell:Cure Disease}
\statblock{\textit{Cleric 3, Druid 3, Elf 3, Medicine Man 3, Shaman 3}

\textbf{Target:} One living creature

\textbf{Range:} 30 ft. (Elf: Touch)

\textbf{Duration:} Permanent}

This spell will cure any living (not undead or non-living) creature of a single disease. It will cure any mundane disease, and will even cure magical diseases such as Mummy Rot and kill disease-like monsters such as green slime.

This spell will only cure lycanthropy if cast by a caster of \nth{11} level or higher.

\textbf{Reverse:} \hypertarget{spell:Cause Disease}{Cause Disease}\index[spells]{Cause Disease} causes the target to contract a non-contagious disease that gives them a -2 to attack rolls, stops magical healing working on them, and causes natural healing to take twice as long. The target may make a saving throw vs. spells to avoid the effect.

This magical disease can only be cured by a Cure Disease spell, and if not cured it will prove fatal in 2d12 days.

\subsection{Cure Light Wounds}\index[spells]{Cure Light Wounds}\label{spell:Cure Light Wounds}
\statblock{\textit{Cleric 1, Druid 1, Elf 3, Medicine Man 2, Shaman 1}

\textbf{Target:} One living creature

\textbf{Range:} Touch

\textbf{Duration:} Permanent}

This spell cures one living (not undead or non-living) creature of 1d6+1 points of damage. The caster can cure themselves with this spell.

Alternately, if not cast by an elf, the spell can be used to cure paralysis (except that caused by a \iref[spell:Hold Person]{Hold Person} or \iref[spell:Hold Monster]{Hold Monster} spell), although if it does so then no damage will be cured at the same time.

\textbf{Reverse:} \hypertarget{spell:Cause Light Wounds}{Cause Light Wounds}\index[spells]{Cause Light Wounds} inflicts 1d6+1 damage to a touched living (not undead or non-living) target. The target gets no saving throw against the damage, but the caster must make a normal attack roll to touch an unwilling target.

\subsection{Cure Serious Wounds}\index[spells]{Cure Serious Wounds}\label{spell:Cure Serious Wounds}
\statblock{\textit{Cleric 4, Druid 4, Elf 6, Medicine Man 4, Shaman 4}

\textbf{Target:} One living creature

\textbf{Range:} Touch

\textbf{Duration:} Permanent}

This spell cures one living (not undead or non-living) creature of 2d6+2 points of damage. The caster can cure themselves with this spell.

\textbf{Reverse:} \hypertarget{spell:Cause Serious Wounds}{Cause Serious Wounds}\index[spells]{Cause Serious Wounds} inflicts 2d6+2 damage to a touched living (not undead or non-living) target. The target gets no saving throw against the damage, but the caster must make a normal attack roll to touch an unwilling target.

\subsection{Dance}\index[spells]{Dance}\label{spell:Dance}
\statblock{\textit{Elf 8, Fey 6, Wizard 8}

\textbf{Target:} One creature

\textbf{Range:} Touch

\textbf{Duration:} Special}

When this spell is cast, the caster must touch a single target by making a successful attack roll. The target gets no saving throw, and is forced to dance wildly for three or more rounds.

While dancing, the target is unable to attack, move quicker than a walk, or use spells or spell like abilities. The target also has a -4 penalty to all saving throws and a +4 penalty to armor class until they stop dancing.

The duration of the spell is based on the caster’s level as indicated on \fullref{tab:Dance}.

\begin {table}[H]
	\caption{Dance}\label{tab:Dance}
  \begin{tabularx}{\columnwidth}{>{\bfseries}YY}
	\thead{Level} & \thead{Duration}\\
	18-20 & 3 rounds\\
	21-24 & 4 rounds\\
	25-28 & 5 rounds\\
	29-32 & 6 rounds\\
	33-36 & 7 rounds
  \end {tabularx}
\end {table}

An \iref[chap:Immortals]{Immortal} target of this spell may make a saving throw each round to stop dancing.

\subsection{Death Spell}\index[spells]{Death Spell}\label{spell:Death Spell}
\statblock{\textit{Sorcerer 6, Wizard 6}

\textbf{Target:} One or more creatures in a 30-foot radius

\textbf{Range:} 240 ft.

\textbf{Duration:} Instant}

This spell sucks the life out of all creatures within a 30-foot radius of the target point of the spell.

Roll 4d8 to see how many hit dice worth of creatures are slain by the effect.

Go through all the living (not undead or non-living) creatures in the area with fewer than 8 hit dice or levels, starting with the weakest. If there are enough hit dice left from the roll, that creature is slain unless they can make a saving throw vs. death ray and their hit dice are taken from the running total (whether they make or fail the save). Once there are no more creatures left with fewer (or equal) hit dice to the number of hit dice left over, the spell stops.

\example{Aloysius casts a Death Spell at a mixed bunch of opponents. There are ten goblins with one hit die each, three wolves with 2 hit dice each, and a giant with 9 hit dice along with his 5 hit dice hellhound pet.

Aloysius’s player rolls 4d8 and gets a total of 19. The ten goblins are the weakest creatures, so they are all affected, taking 1 hit dice each from the total (even though two of them make their saving throws and survive) leaving 9 left. The three wolves take 2 hit dice from the total each, leaving 3 more left. This is not enough to kill the 5 hit dice hellhound, so it is left alive. The giant is unharmed since even if there were no other creatures, it has 8 or more hit dice and is too powerful for the spell to kill.}

Creatures with no hit points (vermin, insects, small plants, etc.) are instantly slain with no saving throw, and do not count towards this total.

\subsection{Delayed Blast Fireball}\index[spells]{Delayed Blast Fireball}\label{spell:Delayed Blast Fireball}
\statblock{\textit{Wizard 7}

\textbf{Target:} 20-foot radius

\textbf{Range:} 240 ft.

\textbf{Duration:} 0-60 rounds}

When this spell is cast, the caster chooses a length of delay, from 0 to 60 rounds. A small ruby-like gem then shoots out to the target location, waits for the specified number of rounds, and explodes into a ball of fire that does 1d6 points of damage per caster level (to a maximum of 20d6 unless the caster is an \iref[chap:Immortals]{Immortal}) to all within a 20-foot radius. Creatures within that radius can make a saving throw vs. spells to take half damage.

During the time between the casting of this spell and the explosion, the gem can be moved by normal means (carried, thrown, dropped, etc.) but it is immune to all magical attempts to move it (Teleport, Telekinesis, etc.)

Nothing can delay the gem’s explosion at the appointed time or make it explode before its time is up except a \iref[spell:Wish]{Wish} spell. However, a Dispel Magic has normal chances to destroy the gem and thus prevent the explosion.

\subsection{Detect Curse}\index[spells]{Detect Curse}\label{spell:Detect Curse}
\statblock{\textit{Medicine Man }

\textbf{Target:}

\textbf{Range:}

\textbf{Duration:} }


\subsection{Detect Danger}\index[spells]{Detect Danger}\label{spell:Detect Danger}
\statblock{\textit{Druid 1, Elf 2, Fey 2}

\textbf{Target:} 5 ft./level

\textbf{Range:} Personal

\textbf{Duration:} 1 hour or 1/2 hour}

This spell allows the caster to detect the presence of danger. If cast outdoors, it lasts for an hour, but if cast indoors then it only lasts for half an hour.

During that time, the caster can concentrate on a square foot of ground or wall, human-sized creature, or a chest sized object for a full round and know whether it is immediately dangerous, potentially dangerous, or benign (from the caster’s point of view).

Objects larger than those mentioned above can be examined, but will take correspondingly more time.

\subsection{Detect Evil}\index[spells]{Detect Evil}\label{spell:Detect Evil}
\statblock{\textit{Cleric 1, Druid 1, Elf 2, Fey 1, Medicine Man 1, Sorcerer 2,  Wizard 2}

\textbf{Target:} 120-foot radius

\textbf{Range:} Personal

\textbf{Duration:} 1 hour}

This spell allows the caster to see a glow around any creature or intelligent object within range that wishes them harm. The caster does not know exactly what harm the creature is intending or what they are capable of, merely that the intent is there.

\subsection{Detect Invisible}\index[spells]{Detect Invisible}\label{spell:Detect Invisible}
\statblock{\textit{Elf 2, Fey 1, Sorcerer 2, Wizard 2}

\textbf{Target:} 10-foot radius/level

\textbf{Range:} Personal

\textbf{Duration:} 1 hour}

This spell allows the caster to see all invisible creatures and objects within range. \iref[sec:Invisible]{Invisible} creatures do not get a saving throw against this effect.

\subsection{Detect Magic}\index[spells]{Detect Magic}\label{spell:Detect Magic}
\statblock{\textit{Cleric 1, Dervish 1, Druid 1, Elf 1, Fey 1, Medicine Man 1, Shaman 1, Sorcerer 1, Wizard 1}

\textbf{Target:} 60-foot radius

\textbf{Range:} Personal

\textbf{Duration:} 20 minutes}

This spell allows the caster to see a glow around any magical creature, object and place within range. Magical creatures do not get a saving throw against this effect.

The glow only extends a couple of inches around the magical object, so if it is in a container or behind another object then some or all of the glow may not be visible.

\subsection{Detect Poison}\index[spells]{Detect Poison}\label{spell:Detect Poison}
\statblock{\textit{Medicine Man }

\textbf{Target:}

\textbf{Range:}

\textbf{Duration:} }


\subsection{Detect Totem}\index[spells]{Detect Totem}\label{spell:Detect Totem}
\statblock{\textit{Medicine Man 1}

\textbf{Target:} One living creature

\textbf{Range:} None

\textbf{Duration:} Permanent}


\subsection{Detect Water}\index[spells]{Detect Water}\label{spell:Detect Water}
\statblock{\textit{Dervish 1}

\textbf{Target:} 300-foot radius

\textbf{Range:} Personal

\textbf{Duration:} 20 minutes}

This spell allows the caster to discern the location of any water within range. It will also allow the caster to determine the quantity of the water.

\subsection{Dimension Door}\index[spells]{Dimension Door}\label{spell:Dimension Door}
\statblock{\textit{Elf 4, Fey 4, Wizard 4}

\textbf{Target:} One creature

\textbf{Range:} 10 ft.

\textbf{Duration:} Instant}

This spell will teleport either the caster or a single creature within 10 feet a distance of up to 360 feet from its current location. If the caster cannot see the destination then it must be described in terms of direction and distance.

If the destination is occupied by solid matter, the spell fails and the target does not move.

An unwilling target may make a saving throw vs. spells to avoid being teleported by the spell.

\subsection{Disintegrate}\index[spells]{Disintegrate}\label{spell:Disintegrate}
\statblock{\textit{Fey 7, Wizard 6}

\textbf{Target:} One creature or object

\textbf{Range:} 60 ft.

\textbf{Duration:} Instant}

This spell destroys a single creature or a single non-magical object, leaving only a trace of fine silvery dust. If targeted on a creature, that creature may make a saving throw vs. death ray to avoid the effect.

\subsection{Dispel Evil}\index[spells]{Dispel Evil}\label{spell:Dispel Evil}
\statblock{\textit{Cleric 5, Druid 5, Elf 7, Fey 7, Medicine Man 5, Shaman 5}

\textbf{Target:} One or more creatures

\textbf{Range:} 30 ft.

\textbf{Duration:} 10 minutes}

When cast, the caster can choose to make this spell affect any animated, charmed, controlled, created, cursed, summoned, or undead creatures within range, or target it at a single such creature or object.

Each targeted creature must make a saving throw vs. spells (if a single creature is targeted it saves at -2) or take the following effect:

\textbf{Animated:} Creature is no longer animated.

\textbf{Charmed:} The charm is removed from the creature.

\textbf{Controlled:} The control is removed from the creature.

\textbf{Created:} The creature is destroyed.

\textbf{Cursed:} The curse is removed from the creature.

\textbf{Summoned:} Creature is banished to where it was summoned from.

\textbf{Undead:} The creature is destroyed.

Even if the creature makes its saving throw, it must still flee for the duration of the spell. If the caster moves before the duration is up, the spell is ended early.

In the case of objects, the spell will remove any curse from the object, but only if the object is specifically targeted by the spell.

If an elf cast this spell, only creatures that are elves are affected.

\subsection{Dispel Magic}\index[spells]{Dispel Magic}\label{spell:Dispel Magic}
\statblock{\textit{Cleric 4, Druid 4, Elf 3, Fey 3, Medicine Man 4, Shaman 4, Sorcerer 3, Wizard 3}

\textbf{Target:} All spells in 10-foot radius

\textbf{Range:} 120 ft.

\textbf{Duration:} Permanent}

When this spell is cast, it has a chance of canceling all ongoing spell effects that are wholly or partly within a 10-foot radius of the target point of the spell.

Spells cast by casters of equal or lower level to the caster of the dispel are automatically canceled. Spells cast by higher level casters have a 5\% chance per level of difference of resisting the dispel.

\example{Elfstar casts Dispel Magic on a Barrier spell cast by a \nth{15} level cleric. Elfstar is only \nth{12} level which is three levels difference, so there is a 15\% chance of the Barrier spell resisting the dispel.}

This spell will not destroy magic items unless cast by an \iref[chap:Immortals]{Immortal}, although it will cancel spell effects that were created by magic items. Even if cast by an \iref[chap:Immortals]{Immortal}, this spell will not destroy an artifact.

For purposes of this spell, \iref[chap:Immortals]{Immortals} are considered to be casters of twice their hit dice, rather than casters of their level. For example, a first level \iref[chap:Immortals]{Immortal} with 15 hit dice dispels other spells as if they were a \nth{30} level caster, and has their spells dispelled as if they were a \nth{30} level caster.

\iref[chap:Immortals]{Immortal} level spells can not be dispelled with this spell.

\subsection{Dissolve}\index[spells]{Dissolve}\label{spell:Dissolve}
\statblock{\textit{Druid 5, Elf 5, Fey 5, Sorcerer 5, Wizard 5}

\textbf{Target:} 3,000 square feet of ground

\textbf{Range:} 240 ft.

\textbf{Duration:} 3d6 days}

This spell turns an area of up to 3,000 square feet of soil or natural rock (not constructions or worked rock) into a slurry of mud. The area can be shaped how the caster desires, but all of it must be within the range of the spell.

The mud is too thin to walk on properly and too thick to swim through. Creatures attempting to wade through it can only move at 10\% of their normal speed.

The mud will dry out naturally in 3d6 days.

\textbf{Reverse:} \hypertarget{spell:Harden}{Harden}\index[spells]{Harden} will change 3,000 square feet of mud, up to 10 feet deep, into solid rock permanently.

Any creature standing in the mud must make a saving throw vs. spells to avoid being trapped by the solidifying mud.

\subsection{Earth to Air}\index[spells]{Earth to Air}\label{spell:Earth to Air}
\statblock{\textit{Wizard 5}

\textbf{Target:} 3,000 square feet of ground

\textbf{Range:} 240 ft.

\textbf{Duration:} 3d6 days}

This spell turns an area of up to 3,000 square feet of soil or natural rock (not constructions or worked rock) into air.

\textbf{Reverse:} \hypertarget{spell:Air to Earth}{Air to Earth}\index[spells]{Air to Earth} will change 3,000 square feet of air into solid rock permanently. The area can be shaped how the caster desires, but all of it must be within the range of the spell.

Any creature located within the area must make a saving throw vs. spells to avoid being trapped in the newly formed rock.

\subsection{Earthquake}\index[spells]{Earthquake}\label{spell:Earthquake}
\statblock{\textit{Cleric 7, Dervish 7, Druid 7}

\textbf{Target:} 60 ft.+5 ft./level diameter

\textbf{Range:} 360 ft.

\textbf{Duration:} 10 minutes}

This spell causes a powerful but localized earthquake. It will destroy small buildings and damage large ones, and may cause rockslides.

Each creature in the area has a 1 in 6 chance of being in danger of being engulfed in a crack in the ground. The creature must make a saving throw vs. death ray in order to stop themselves falling in and being crushed for 1d100+100 damage per round.

\textbf{Environmental Effect:} If this spell is cast on the sea floor, it creates shock waves that will stun all creatures within the area of effect unless they make a saving throw vs. death ray. If cast underwater, the area of effect is reduced by half. The stun effect last until 1d6 rounds after the creature has left the area of effect.

\subsection{Enchanted Weapon}\index[spells]{Enchanted Weapon}\label{spell:Enchanted Weapon}
\statblock{\textit{Elf 4, Fey 4}

\textbf{Target:} One weapon

\textbf{Range:} Touch

\textbf{Duration:} 5 rnd/level}

The target weapon of this spell becomes magical allowing it to be used against creatures that are invulnerable to normal weapons (e.g. gargoyles, lycanthrope, various undead).

\subsection{Entangle}\index[spells]{Entangle}\label{spell:Entangle}
\statblock{\textit{Elf 2, Fey 2, Wizard 2}

\textbf{Target:} One rope or vine

\textbf{Range:} 30 ft.

\textbf{Duration:} 1 rnd/level}

This spell animates a single rope or vine that can be up to 50 feet long plus 5 feet per level of the caster.

The rope cannot be ordered to attack, but it can be ordered to loop or tie around something, to knot or unknot itself, or to neatly coil up.

Using a combination of these commands, the rope can be used for climbing or for capturing enemies. The rope cannot stretch itself out, it can only grasp things within 1 foot of it—so it must be thrown by hand at the thing it is commanded to loop or tie around.

If the rope is thrown at a creature and commanded to tie it, the creature gets a saving throw vs. spells to avoid the rope.

\subsection{ESP}\index[spells]{ESP}\label{spell:ESP}
\statblock{\textit{Elf 2, Fey 2, Wizard 2}

\textbf{Target:} One creature at a time

\textbf{Range:} 60 ft.

\textbf{Duration:} 2 hours}

This spell allows the caster to hear and understand the thoughts of any living creature within range, regardless of language.

The caster must concentrate for six rounds. If there is more than one creature in the same direction, it takes the caster an additional six rounds to filter the thoughts of a single creature out of the cacophony.

The spell is blocked by two feet of stone or any thickness of lead, and each individual targeted can make a saving throw vs. spells to block out the ESP.

\textbf{Reverse:} \hypertarget{spell:Mindmask}{Mindmask}\index[spells]{Mindmask} makes the caster or a touched creature immune to all forms of mind reading for the duration.

\subsection{Explosive Cloud}\index[spells]{Explosive Cloud}\label{spell:Explosive Cloud}
\statblock{\textit{Wizard 8}

\textbf{Target:} 15-foot radius

\textbf{Range:} 1 ft.

\textbf{Duration:} 1 hour}

This spell creates a 15-foot radius and 20-foot-tall cloud of poisonous gas, the closest edge of which must be within 1 foot of the caster. The gas is dense enough to be visible, but does not block sight. If cast within an enclosed space, the cloud may be smaller than the dimensions above. It will not expand to fill the same volume.

From the outside, the cloud is indistinguishable from that created by a \iref[spell:Cloudkill]{Cloudkill} spell.

The cloud is poisonous, and each round all within it must make a saving throw vs. spells or be paralyzed for that round. Additionally, the cloud contains sparkling lights visible only to those within it. These lights are small explosions that do 1 point of damage per two caster levels to each creature within the cloud. This damage will affect any creature, even those immune to fire, gas and poison, and there is no saving throw against it.

\subsection{Eye of the Eagle}\index[spells]{Eye of the Eagle}\label{spell:Eye of the Eagle}
\statblock{\textit{Medicine Man }

\textbf{Target:} One living creature

\textbf{Range:} None

\textbf{Duration:} 1 day}



\subsection{Faerie Fire}\index[spells]{Faerie Fire}\label{spell:Faerie Fire}
\statblock{\textit{Dervish 1, Druid 1, Elf 1, Fey 1}

\textbf{Target:} One or more creatures

\textbf{Range:} 60 ft.

\textbf{Duration:} 1 rnd/level}

This spell causes one or more creatures or objects within the area to glow with flickering greenish flames as if on fire. The fire is bright enough to make the targets glow in the dark, but not bright enough to use as a light source.

The caster can outline one human-sized target or the equivalent per 5 levels.

All attacks against outlined creatures gain a +2 bonus to hit.

\subsection{Faerie Lights}\index[spells]{Faerie Lights}\label{spell:Faerie Lights}
\statblock{\textit{Elf 1}

\textbf{Target:} Area

\textbf{Range:} 40 ft. + 10 ft./level

\textbf{Duration:} 2 rounds/level}

This spell creates up to 4 hovering light sources such as lanterns, torches, etc. The light sources must be created where the caster can see but once created can me moved anywhere within the spell's range. The caster can alter the intensity of the light sources by concentrating for a round. Once the spell ends, the light sources disappear.

\subsection{Faerie Slumber}\index[spells]{Faerie Slumber}\label{spell:Faerie Slumber}
\statblock{\textit{Fey 2}

\textbf{Target:} 2-16 HD living creatures within a 40 ft. square

\textbf{Range:} 240 ft.

\textbf{Duration:} 4d4 turns}

This spell functions like the \iref[spell:Sleep]{Sleep} spell, but can effect creatures with greater than 4+1 hit dice and target's are allowed saving throws vs. spells. Also for each target, before a saving throw is made, the caster rolls 1d4 + their level and if the result is lower than the target's HD than the spell fails against that target.

\subsection{Fate}\index[spells]{Fate}\label{spell:Fate}
\statblock{\textit{Medicine Man }

\textbf{Target:}

\textbf{Range:}

\textbf{Duration:} }


\subsection{Fear}\index[spells]{Fear}\label{spell:Fear}
\statblock{\textit{Elf 4, Fey 3}

\textbf{Target:} One or more creatures

\textbf{Range:} 120 ft.

\textbf{Duration:} Special}

This spell creates a cone of fear, 60 feet long and 30 feed wide at the end. All creatures within the cone must make a saving throw vs. spells or flee in terror for 5 minutes. Affected creatures that are cornered will cower and fight only to defend themselves.

\subsection{Feeblemind}\index[spells]{Feeblemind}\label{spell:Feeblemind}
\statblock{\textit{Elf 5, Fey 5, Wizard 5}

\textbf{Target:} One spell-using creature

\textbf{Range:} 240 ft.

\textbf{Duration:} Permanent}

This spell blasts the mind of the target, who must be a spellcaster. The target must make a saving throw vs. spells (with a penalty of -4 to the roll unless the target is an \iref[chap:Immortals]{Immortal}) or be made \iref[sec:Helpless]{Helpless}, unable to think clearly and unable to cast spells. The victim’s \iref[sec:Intelligence]{Intelligence} is reduced to a score of 2.

The effect is permanent, although it can be removed by a Dispel Magic (with the normal chance of success) or by a Heal spell.

\subsection{Fellowship}\index[spells]{Fellowship}\label{spell:Fellowship}
\statblock{\textit{Elf 1}

\textbf{Target:} Caster

\textbf{Range:} Personal

\textbf{Duration:} 1 round/level}

This caster temporarily gains 2d4 points of \iref[sec:Charisma]{Charisma} in the eyes of anyone they meet who fails a saving throw vs. spells or a loss of 1d4 to those who pass.

Targets who fail the saving throw wish to become the caster's friend and will do their best to assist the caster anyway they can. Targets who pass the saving throw find the caster irritating and don't wish to be around them.

This spell has no effect on targets with animal intelligence or lower and does not effect \iref[sec:Charisma]{Charisma} skills.

\subsection{Find the Path}\index[spells]{Find the Path}\label{spell:Find the Path}
\statblock{\textit{Cleric 6, Dervish 6, Druid 6, Fey 4, Shaman 6}

\textbf{Target:} Caster

\textbf{Range:} Personal

\textbf{Duration:} 1 hour + 10 mins/level}

This spell mentally guides the caster to a specific place. The caster is subconsciously able to take the correct route, and even know the location of secret doors and know passwords. The caster is not conscious of any of this knowledge, however, and is therefore unable to remember it after the spell’s duration runs out or communicate it to others.

Once the spell runs out, the caster will remember the vague direction to the goal, but that is all.

The spell must be used to direct the caster to a fixed location that the caster has either visited before or had described in detail. It can not be used to simply find the current location of an object. It will only find a path that does not involve crossing planar boundaries. If there is no such path, then the caster will know this after casting the spell.

\subsection{Find Traps}\index[spells]{Find Traps}\label{spell:Find Traps}
\statblock{\textit{Cleric 2, Druid 2}

\textbf{Target:} 30-foot radius

\textbf{Range:} Personal

\textbf{Duration:} 20 minutes}

This spell causes the caster to see a glow around any mechanical or magical traps that are within 30 feet of them.

It does not give any indication about the type of trap or the triggering mechanism, and cannot find natural hazards or ambushes.

\subsection{Fire Gate}\index[spells]{Fire Gate}\label{spell:Fire Gate}
\statblock{\textit{Medicine Man }

\textbf{Target:}

\textbf{Range:}

\textbf{Duration:} }



\subsection{Fireball}\index[spells]{Fireball}\label{spell:Fireball}
\statblock{\textit{Sorcerer 3, Wizard 3}

\textbf{Target:} 20-foot radius

\textbf{Range:} 240 ft.

\textbf{Duration:} Instant}

This spell creates a small ball of flame that shoots out to the target point and then explodes into a 20-foot radius ball of fire.

The fire does 1d6 damage per caster level (to a maximum of 20d6, unless the caster is an \iref[chap:Immortals]{Immortal}) to each creature in the area. Creatures that make a saving throw vs. spells take only half damage.

\textbf{Environmental Effect:} If this spell is cast underwater, the ball is composed of lightning rather than fire.


\subsection{Firebow}\index[spells]{Firebow}\label{spell:Firebow}
\statblock{\textit{Medicine Man }

\textbf{Target:} One wooden bow

\textbf{Range:} None

\textbf{Duration:} Special}



\subsection{Floating Disc}\index[spells]{Floating Disc}\label{spell:Floating Disc}
\statblock{\textit{Wizard 1}

\textbf{Target:} None

\textbf{Range:} Personal

\textbf{Duration:} 1 hour}

This spell creates a small flying platform of force, about the size and shape of a round shield. The platform is invisible to all but the caster and hovers at the height of the caster’s waist. The platform follows the caster around, never getting more than 6 feet away from them.

The platform can support 500lbs of weight.

The platform has no edges and can not be used as a weapon in any way, since it has no physical existence other than to support weight.

\subsection{Fly}\index[spells]{Fly}\label{spell:Fly}
\statblock{\textit{Elf 3, Fey 3, Sorcerer 3, Wizard 3}

\textbf{Target:} One creature

\textbf{Range:} Touch

\textbf{Duration:} 10 mins/level + 1d6x10 mins}

This spell allows the target to fly at 120 feet per round by concentrating. If the target stops concentrating they will hover in place.

\textbf{Environmental Effect:} If this spell is cast underwater, instead of granting flight it allows the target to swim at three times their normal speed.

\subsection{Force Field}\index[spells]{Force Field}\label{spell:Force Field}
\statblock{\textit{Elf 8, Wizard 8}

\textbf{Target:} None

\textbf{Range:} 120 ft.

\textbf{Duration:} 1 hour}

This spell creates an impossibly hard field of force that cannot be dispelled and can only be broken by a \iref[spell:Disintegrate]{Disintegrate} or \iref[spell:Wish]{Wish} spell.

The force field must be a simple smooth shape—either a flat plane of up to 5,000 square feet, a sphere or hemisphere of up to 20-foot radius, or a rectangular box with a surface area of up to 5,000 square feet. Regardless of shape, it must always be created in an empty area. It can not be created inside any object or creature to cut them in half.

However, the edges of the force field will conform to the shape of surrounding material such as walls.

The force field does not need to be supported, its edges are not sharp, and it is completely immobile—only a \iref[spell:Wish]{Wish} spell can move a force field.

No physical or magical force except those noted above can destroy or pass through a force field, although a \iref[spell:Teleport]{Teleport} or \iref[spell:Dimension Door]{Dimension Door} spell will bypass it.

Creatures enclosed within a force field are magically preserved, and will not starve or suffocate.

This spell can be made permanent with a \iref[spell:Permanence]{Permanence} spell, and if this happens then the permanence can still be dispelled even though the force field cannot be. If the permanence is dispelled after the normal duration of the force field has expired then the force field will immediately disappear.

A force field cast by a mortal can be destroyed by an \iref[chap:Immortals]{Immortal}’s attacks (treat it as having an armor class of -10 and 50 hit points).

A force field cast by an \iref[chap:Immortals]{Immortal} cannot be broken by \iref[spell:Disintegrate]{Disintegrate} or \iref[spell:Wish]{Wish} spells cast by mortals, and cannot be penetrated by mortal \iref[spell:Teleport]{Teleport}, \iref[spell:Gate]{Gate}, or \iref[spell:Dimension Door]{Dimension Door} spells.

\subsection{Gaseous Form}\index[spells]{Gaseous Form}\label{spell:Gaseous Form}
\statblock{\textit{Fey 3}

\textbf{Target:} One creature

\textbf{Range:} Touch

\textbf{Duration:} 10 mins/level}

The target’s body transforms to a cloud of gas for up to 1 hour, causing all their equipment and carried items to fall to the floor. The target keeps control of their body, and can move through any non-airtight barrier.

While in gaseous form, the target cannot attack, but has an armor class of -2 and can only be hit by magical weapons.

\subsection{Gate}\index[spells]{Gate}\label{spell:Gate}
\statblock{\textit{Elf 9, Wizard 9}

\textbf{Target:} One extraplanar being

\textbf{Range:} 30 ft.

\textbf{Duration:} Special}

This spell opens a one-way rift between two planes, even those attached to two different Celestial Spheres.

Normally, the caster must specify the target plane and the name of a creature on that plane. If the creature is on the named plane, then the spell opens a one-way physical portal between the caster’s current location and the creature’s location allowing the creature to step through to the caster (but not vice versa). If the creature is not on the named plane then the spell fails. If the creature is an \iref[chap:Immortals]{Immortal}, it can refuse to allow the gate to open.

However, the spell can also be cast through the open end of an open one-way gate, targeting the other end of the gate rather than a specific creature. In this case, the caster does not need to specify (or even know) where the other end of the gate is located. If cast in this manner, the gate becomes two-way and creatures can pass through in either direction.

The open end of a one-way gate appears misty and is opaque. The closed end of a one-way gate is invisible, although detectable with a \iref[spell:Detect Magic]{Detect Magic} spell. A two-way gate can be seen through in either direction.

If the either end of the gate is an outer plane then the gate only stays open for 10 minutes. Otherwise, it stays open for 1d10 x 100 minutes.

Any creature can step through the open end of a gate, and there is a base 10\% chance of a random inhabitant of the far plane (modified by the location that the gate is opened to) noticing the open gate and investigating it.

A gate can be held open permanently with a \iref[spell:Wish]{Wish} spell, although if the gate is two-way then a separate Wish must be used for each direction.

It costs an \iref[chap:Immortals]{Immortal} 50 pp to step through a gate. This makes summoning one for frivolous reasons a risky proposition at best.

\textbf{Reverse:} \hypertarget{spell:Close Gate}{Close Gate}\index[spells]{Close Gate} closes a gate to another plane, whether one made by this spell or a natural planar rift.

\subsection{Geas}\index[spells]{Geas}\label{spell:Geas}
\statblock{\textit{Elf 6, Fey 6, Wizard 6}

\textbf{Target:} One creature

\textbf{Range:} 30 ft.

\textbf{Duration:} Special}

This spell forces the target to either perform a specific action or refrain from performing a specific action. The target may make a saving throw vs. spells to escape the effect.

The action must be something that is possible, and can’t be something suicidal—for example you can’t geas someone into jumping off a cliff or geas someone into never eating.

The target must perform (or refrain from performing) the action, but they are not mind controlled in any way, and they are fully aware that they may only be performing the action or refraining from it in order to avoid the consequences of this spell.

If the target goes against the geas, they receive a Curse, as if by the reversed form of the \iref[spell:Remove Curse]{Remove Curse} spell. Neither this curse nor the geas itself can be dispelled or removed via a \iref[spell:Remove Curse]{Remove Curse} spell, although a Dispel Evil will remove it. The curse will not lift until the geas is fulfilled (or, in the case of geases against performing actions, a week has passed without the action being performed).

This spell cannot affect an \iref[chap:Immortals]{Immortal}, even if cast by another \iref[chap:Immortals]{Immortal}.

\textbf{Reverse:} \hypertarget{spell:Remove Geas}{Remove Geas}\index[spells]{Remove Geas} will remove an unwanted geas, although for each level the caster of the geas is above the caster of the remove geas there is a 5\% chance of failure.

\subsection{Growth of Animal}\index[spells]{Growth of Animal}\label{spell:Growth of Animal}
\statblock{\textit{Cleric 3, Druid 3, Elf 4, Fey 4}

\textbf{Target:} One animal

\textbf{Range:} 120 ft.

\textbf{Duration:} 2 hours}

This spell causes one animal to grow to double its normal size, giving it twice its normal strength and making it do twice its normal damage in combat. The animal may also carry twice its normal load.

\subsection{Growth of Plants}\index[spells]{Growth of Plants}\label{spell:Growth of Plants}
\statblock{\textit{Dervish 3, Elf 4, Fey 3, Medicine Man 4, Sorcerer 4, Wizard 4}

\textbf{Target:} 3,000 square feet of undergrowth

\textbf{Range:} 120 ft.

\textbf{Duration:} Permanent}

This spell causes all undergrowth (grasses, brambles, vines etc.) within a 3,000 square foot area of whatever shape the caster desires to become thickly overgrown. This growth makes the area impassable to creatures smaller than giant sized.

The effect can be removed by a \iref[spell:Dispel Magic]{Dispel Magic} spell or by the reverse of this spell.

\textbf{Reverse:} \hypertarget{spell:Shrink Plants}{Shrink Plants}\index[spells]{Shrink Plants} reverses this spell or makes a similar area of naturally overgrown plants shrink and shrivel making the area easily passable.

\subsection{Hallucinatory Terrain}\index[spells]{Hallucinatory Terrain}\label{spell:Hallucinatory Terrain}
\statblock{\textit{Elf 4, Fey 3, Wizard 4}

\textbf{Target:} One terrain feature up to 480-foot-diameter

\textbf{Range:} Touch

\textbf{Duration:} Special}

This spell masks a terrain feature (either indoors or outdoors) and makes it look like a different feature. For example a pit could be made to look like solid floor or a small hill could be made to look like a patch of swamp.

The illusion last until it is touched by an intelligent creature.

\subsection{Haste}\index[spells]{Haste}\label{spell:Haste}
\statblock{\textit{Elf 3, Fey 3, Wizard 3}

\textbf{Target:} 24 creatures in a 30-foot radius

\textbf{Range:} 240 ft.

\textbf{Duration:} 30 minutes}

This spell speeds up creatures in a 30-foot radius of the target point. Up to 24 creatures can be affected.

Affected creatures move twice their speed and can make twice as many attacks. They also get a +2 bonus on their initiative rolls. Spellcasting (whether innate or through items) is not sped up.

Multiple haste spells from a single source do not stack, but different sources (e.g. a haste spell and a Potion of Speed) do stack, to a maximum of double effect.

This spell has no effect on \iref[chap:Immortals]{Immortals}.

\textbf{Reverse:} \hypertarget{spell:Slow}{Slow}\index[spells]{Slow} slows creatures in the area rather than speeding them up. Affected creatures move at half their speed and can only make half of their attacks. They also get a -2 penalty on their initiative rolls. Spellcasting (whether innate or through items) is not slowed down.

Creatures can only be affected by one slow spell at a time.

Creatures may find that they are making half an attack per round or one and a half attacks per round when effected by slow. In these cases, the creature’s “half” attack is made every other round.

\subsection{Heal}\index[spells]{Heal}\label{spell:Heal}
\statblock{\textit{Cleric 6, Druid 6, Elf 9, Medicine Man 6, Shaman 6, Wizard 9}

\textbf{Target:} One living creature

\textbf{Range:} Touch

\textbf{Duration:} Permanent}

This spell will cure nearly all damage from a living (not undead or non-living) target, leaving them with only 1d6 damage—although if the target is already healthier than that it won’t damage them.

Alternatively, the spell can be used as a \iref[spell:Remove Curse]{Remove Curse}, \iref[spell:Cure Disease]{Cure Disease} or \iref[spell:Cure Blindness]{Cure Blindness} spell, or it can be used to cure a \iref[spell:Feeblemind]{Feeblemind} spell. However, it will only cure one thing per casting.

If cast on someone who has recently been brought back to life by a \iref[spell:Raise Dead]{Raise Dead} spell, it will eliminate the rest period and bring the target back to full strength immediately.

\subsection{Heat Metal}\index[spells]{Heat Metal}\label{spell:Heat Metal}
\statblock{\textit{Druid 2, Elf 3, Fey 2}

\textbf{Target:} One metal object

\textbf{Range:} 30 ft.

\textbf{Duration:} 7 rounds}

This spell heats one metal object to red-hot over the course of 4 rounds. The object then takes a further three rounds to cool.

Any metal object of up to half a pound (5 cn) per caster level can be affected by this spell. The metal is not damaged by the heating and cooling, although if the metal forms part of an object that also has wood or leather then the non-metal parts of the object may be scorched unless the object is magical.

If the object is being held or worn by a creature, the creature will take 1 damage on the first round, 2 on the second, 4 on the third, 8 on the fourth, 4 on the fifth, 2 on the sixth, and 1 on the seventh. The holder of the item gets no saving throw, although creatures will usually drop the item before the duration is up. The holder cannot cast spells while holding a hot item.

\subsection{Hold Animal}\index[spells]{Hold Animal}\label{spell:Hold Animal}
\statblock{\textit{Dervish 3, Druid 3, Elf 3, Fey 2, Medicine Man 3}

\textbf{Target:} One or more animals

\textbf{Range:} 180 ft.

\textbf{Duration:} 10 mins/level}

This spell will paralyze several animals for the duration.

It will affect 1 hit dice of animal per level of the caster, although each animal gets a saving throw vs. spells to avoid the effect.

\subsection{Hold Monster}\index[spells]{Hold Monster}\label{spell:Hold Monster}
\statblock{\textit{Fey 5, Sorcerer 5, Wizard 5}

\textbf{Target:} One to four creatures

\textbf{Range:} 120 ft.

\textbf{Duration:} 1 hour + 10 min/level}

This spell will paralyze up to four living creatures (not undead or non-living creatures) for the duration.

It can be cast at a single target, in which case the target may save vs. spells at a -2 penalty to avoid the effect; or at a group, in which case it will affect 1d4 of them each of which may save vs. spells without penalty to avoid the effect.

\textbf{Reverse:} \hypertarget{spell:Free Monster}{Free Monster}\index[spells]{Free Monster} removes the paralysis caused by either this spell or a \iref[spell:Hold Person]{Hold Person} spell. It will also remove the paralysis effect of an \iref[chap:Immortals]{Immortal}’s aura, but only when cast by an \iref[chap:Immortals]{Immortal}.

\subsection{Hold Person}\index[spells]{Hold Person}\label{spell:Hold Person}
\statblock{\textit{Cleric 2, Dervish 2, Druid 2, Elf 3, Fey 3, Shaman 2, Wizard 3}

\textbf{Target:} One to four humanoids

\textbf{Range:} 180 ft.

\textbf{Duration:} 90 minutes}

This spell will paralyze up to four living humanoid creatures (not undead or non-living creatures) for the duration.

It can be cast at a single target, in which case the target may save vs. spells at a -2 penalty to avoid the effect; or at a group, in which case it will affect 4 of them each of which may save vs. spells without penalty to avoid the effect.

\textbf{Reverse:} \hypertarget{spell:Free Person}{Free Person}\index[spells]{Free Person} removes the paralysis caused by this spell. It will also remove the paralysis effect of an \iref[chap:Immortals]{Immortal}’s aura, but only when cast by an \iref[chap:Immortals]{Immortal}.

\subsection{Hold Portal}\index[spells]{Hold Portal}\label{spell:Hold Portal}
\statblock{\textit{Fey 1, Wizard 1}

\textbf{Target:} One door, chest or gate

\textbf{Range:} 10 ft.

\textbf{Duration:} 2d6x10 minutes}

This spell will hold a single portal or door closed by magical means.

The portal can only be forced open by creatures who have at least three hit dice more than the caster or opened with a \iref[spell:Knock]{Knock} spell, although in either case if the portal is allowed to close it will continue to be held for the duration of the spell.

\iref[chap:Immortals]{Immortals} can always open doors held by a mortal’s casting of this spell, even if they have fewer hit dice.

\subsection{Hold Spirit}\index[spells]{Hold Spirit}\label{spell:Hold Spirit}
\statblock{\textit{Medicine Man 3}

\textbf{Target:}

\textbf{Range:}

\textbf{Duration:} }



\subsection{Holy Word}\index[spells]{Holy Word}\label{spell:Holy Word}
\statblock{\textit{Cleric 7, Dervish 7, Druid 7}

\textbf{Target:} All creatures in range

\textbf{Range:} 40 ft.

\textbf{Duration:} Instant}

This spell affects all creatures within 40 feet of the caster. The exact effect varies depending on the creature’s level or hit dice as indicated on \fullref{tab:Holy Word}.

\begin {table}[H]
\caption{Holy Word}\label{tab:Holy Word}
  \begin{tabularx}{\columnwidth}{>{\bfseries}YY}
	\thead{Level or Hit Dice} & \thead{Effect}\\
	Up to 5 HD & Killed\\
	6-8 HD & Stunned for 2d10x10 minutes\\
	9-12 HD & Stunned for 1d10 rounds\\
	13+ HD & Deafened for 1d6x10 minutes
  \end {tabularx}
\end {table}

Only creatures of the same alignment as the caster (and any creature of 13 or more hit dice) may make saving throws vs. spells to avoid the effect.

\subsection{Horse-rush}\index[spells]{Horse-rush}\label{spell:Horse-rush}
\statblock{\textit{Fey 2}

\textbf{Target:} One living rush

\textbf{Range:} Touch

\textbf{Duration:} Up to 1 hour/level}

This spell enchants a rush allowing it to change into a Riding Horse and back. Once the spell takes effect, the caster can put the rush between their legs and say the words "gitty up" which the rush will than turn into a riding horse. Saying the words "simmer down" will revert it back to a rush. When the duration has expired, the rush will take it's natural form and stay that way unless the spell is recast on it.

\subsection{Hunting Paint}\index[spells]{Hunting Paint}\label{spell:Hunting Paint}
\statblock{\textit{Medicine Man 2}

\textbf{Target:} Body paints

\textbf{Range:} None

\textbf{Duration:} 1 day}


\subsection{Ice Storm/Wall of Ice}\index[spells]{Ice Storm/Wall of Ice}\label{spell:Ice Storm/Wall of Ice}\hypertarget{spell:Wall of Ice}
\statblock{\textit{Sorcerer 4, Wizard 4}

\textbf{Target:} Special

\textbf{Range:} 120 ft.

\textbf{Duration:} Instant or 2 hours}

This spell can be cast in one of two ways, and the caster can decide which way to use it at the time of casting.

The first way to use the spell is to create a 10-foot radius ice storm. This ice storm does 1d6 cold damage per caster level (maximum of 20d6) to every creature in the area. Each creature can make a saving throw vs. spells to take half damage.

Fire based creatures have a -4 penalty to their saving throws, but cold based creatures are immune to the spell.

The second way to use the spell is to create a vertical wall of ice of up to 1,200 square feet. The wall is not transparent, and must be created in unoccupied space on ground that will support it.

The wall will melt in two hours, but can be broken before that time by creatures of 5 hit dice or more. However, such creatures take 1d6 cold damage while doing so (fire based creatures take 2d6, cold based creatures take none).

\textbf{Environmental Effect:} If an ice storm is cast underwater, the damage is reduced to 1d4 damage per caster level due to losing some of its downward force.

If an ice wall is cast underwater, the wall will float to the surface at the rate of 60 feet per round unless it is braced.

\subsection{Ice to Water}\index[spells]{Ice to Water}\label{spell:Ice to Water}
\statblock{\textit{Wizard 5}

\textbf{Target:} 3,000 square feet of ground

\textbf{Range:} 240 ft.

\textbf{Duration:} 3d6 days}

This spell turns an area of up to 3,000 square feet of solid ice into slush. The area can be shaped how the caster desires, but all of it must be within the range of the spell.

The slush is too thin to walk on properly and too thick to swim through. Creatures attempting to wade through it can only move at 10\% of their normal speed.

The slush will naturally refreeze or melt (depending on the temperature of the area) in 3d6 days.

\textbf{Reverse:} \hypertarget{spell:Water to Ice}{Water to Ice}\index[spells]{Water to Ice} will change 3,000 square feet of slush, up to 10 feet deep, into solid ice permanently.

Any creature standing in the slush must make a saving throw vs. spells to avoid being trapped by the solidifying slush.

\subsection{Immunity}\index[spells]{Immunity}\label{spell:Immunity}
\statblock{\textit{Elf 9, Wizard 9}

\textbf{Target:} One creature

\textbf{Range:} Touch

\textbf{Duration:} 10 mins/level}

This spell gives the creature touched complete immunity to spells of \nth{3} level or less, and spells of \nth{5} level or less only have half normal effect (halve whatever quantifiable effects they have, such as damage, duration, penalties, etc.)

The spell also grants immunity to all missiles and all non-magical weapons.

The target can drop the immunity temporarily by concentrating, if they wish to be affected by a beneficial spell.

\subsection{Infravision}\index[spells]{Infravision}\label{spell:Infravision}
\statblock{\textit{Fey 2, Wizard 3}

\textbf{Target:} One living creature

\textbf{Range:} Touch

\textbf{Duration:} 1 day}

This spell gives a living (not undead or non-living) \iref[sec:Infravision]{Infravision} (see \fullref{sec:Infravision}).

\subsection{Infusion}\index[spells]{Infusion}\label{spell:Infusion}
\statblock{\textit{Medicine Man }

\textbf{Target:} Up to 1 creature/level

\textbf{Range:} None

\textbf{Duration:} Permanent}



\subsection{Insect Plague}\index[spells]{Insect Plague}\label{spell:Insect Plague}
\statblock{\textit{Cleric 5, Druid 5, Elf 5, Fey 5, Shaman 5}

\textbf{Target:} None

\textbf{Range:} 480 ft.

\textbf{Duration:} Concentration}

This spell summons a 30-foot radius swarm of insects which obscures vision in the area.

The insects do no damage, but will drive away any creature of less than 3 hit dice with their stinging and biting. There is no saving throw against this effect.

The insect swarm can be moved up to 20 feet per round by the caster, and lasts until the caster stops concentrating.

Elves can only cast this spell in wilderness environments.

\textbf{Environmental Effect:} If this spell is cast underwater, shrimp rather than insects are summoned.

\subsection{Invisibility}\index[spells]{Invisibility}\label{spell:Invisibility}
\statblock{\textit{Elf 2, Fey 3, Sorcerer 2, Wizard 2}

\textbf{Target:} One creature or object

\textbf{Range:} 240 ft.

\textbf{Duration:} Special}

This spell makes a single creature or object invisible.

If cast on a creature, the creature’s clothing and equipment also become invisible. Any item the creature drops will become visible, but items the creature picks up will not become invisible.

The creature will become visible again if it attacks another creature or if it casts a spell. Otherwise, the invisibility is permanent.

If the spell is cast on an object, the object remains invisible until touched by an intelligent creature.

\subsection{Invisibility 10-foot radius}\index[spells]{Invisibility 10-foot radius}\label{spell:Invisibility 10-foot radius}
\statblock{\textit{Elf 3, Wizard 3}

\textbf{Target:} One creature

\textbf{Range:} 120 ft.

\textbf{Duration:} Special}

This spell makes all creatures in a 10-foot radius around the target invisible.

The creatures’ clothing and equipment also become invisible. Any item any creature drops will become visible, but items that one of the creatures pick up will not become invisible.

Any creature that strays more than 10 feet from the target creature also becomes visible, and re-entering that radius does not make the creature become visible again.

Any creature will become visible again if it attacks another creature or if it casts a spell. Otherwise, the invisibility is permanent.

\subsection{Invisibility, Mass}\index[spells]{Invisibility, Mass}\label{spell:Invisibility, Mass}
\statblock{\textit{Elf 7, Fey 5, Wizard 7}

\textbf{Target:} One or more creatures

\textbf{Range:} 240 ft.

\textbf{Duration:} Special}

This spell makes all creatures in a 30-foot radius around the target point of the spell invisible. The creatures’ clothing and equipment also become invisible. Any item any creature drops will become visible, but items that one of the creatures pick up will not become invisible.

Any creature will become visible again if it attacks another creature or if it casts a spell. Otherwise, the invisibility is permanent.

\textbf{Reverse:} \hypertarget{spell:Appear}{Appear}\index[spells]{Appear} causes all invisible creatures and objects within a 10-foot radius of the target point of the spell to become visible and be unable to become invisible again for 10 minutes. Ethereal creatures are unaffected by this spell.

\subsection{Invisibility to Spirits}\index[spells]{Invisibility to Spirits}\label{spell:Invisibility to Spirits}
\statblock{\textit{Medicine Man }

\textbf{Target:}

\textbf{Range:}

\textbf{Duration:} }


\subsection{Invisible Stalker}\index[spells]{Invisible Stalker}\label{spell:Invisible Stalker}
\statblock{\textit{Wizard 6}

\textbf{Target:} None

\textbf{Range:} Personal

\textbf{Duration:} Special}

This spell summons an Invisible Stalker (see \fullref{sec:Invisible Stalker}).

The caster must specify one task for the stalker to perform, and the stalker will attempt to perform that task even at the cost of its own life until either the task becomes impossible or a year and a day have passed. In either case, the spell then ends.

\textbf{Environmental Effect:} If this spell is cast underwater, an Ice Spider is summoned rather than an Invisible Stalker. A \iref[spell:Dispel Evil]{Dispel Evil} spell will banish the Ice Spider back to their home plane.

\subsection{Ironform}\index[spells]{Ironform}\label{spell:Ironform}

\statblock{\textit{Wizard 7}

\textbf{Target:} None

\textbf{Range:} Touch

\textbf{Duration:} Permanent}

This spell creates a sheet of iron up to 2 inches thick and up to 500 square feet in area. The sheet can be created flat or can be created pre-shaped. The iron does not appear instantly, but takes time to form, ranging from a single round for a simple sheet of iron to two hours to create something with a precise specification.

Whatever the complexity of the shape, the iron must form a single piece with no moving parts. However, the caster can create the iron in a “rough” form, which can then have the ironform spell cast on it again in order to either add to the object or reshape it. If the caster does create the iron in rough form, then this spell is cast once again to “set” the iron in its final form so that other casters can’t cast this spell on it in order to modify it.

Note that whatever shape the sheet takes, it is always a maximum of 2 inches thick—so that if it is formed into a statue, for example, the statue would be hollow.

Once created, the iron is real and cannot be dispelled, and it will last until it is physically or magically destroyed.

\subsection{Knock}\index[spells]{Knock}\label{spell:Knock}
\statblock{\textit{Fey 2, Wizard 2}

\textbf{Target:} One lock

\textbf{Range:} 60 ft.

\textbf{Duration:} Special}

This spell will temporarily unlock any lock, or open a door that is closed by a \iref[spell:Hold Portal]{Hold Portal} or \iref[spell:Wizard Lock]{Wizard Lock} spell. If the door was locked by mundane means then it remains unlocked until physically locked again, but if it is held shut by magical means then it will re-fasten once closed.

This spell will even unbar a door that is barred on the other side, although if the door is both barred and locked then it will take two castings to remove both.

\subsection{Know Alignment}\index[spells]{Know Alignment}\label{spell:Know Alignment}
\statblock{\textit{Cleric 2, Druid 2, Elf 2, Fey 2, Medicine Man 2}

\textbf{Target:} One creature or item

\textbf{Range:} 10 ft.

\textbf{Duration:} Instant}

This spell allows the caster to discern the alignment of a single creature or item within 10 feet. Most items do not have an alignment, but some magical ones might.

\textbf{Reverse:} \hypertarget{spell:Confuse Alignment}{Confuse Alignment}\index[spells]{Confuse Alignment} lets the caster give a false alignment to a touched creature for 10 minutes per level of the caster. The creature does not actually take on the false alignment, but any magical effect that relies on alignment—such as the normal form of this spell, or an item that can only be operated by creatures of a particular alignment, or even a \iref[spell:Holy Word]{Holy Word} spell—will be fooled and will treat the target as if the false alignment were their true one.

\subsection{Know Destiny}\index[spells]{Know Destiny}\label{spell:Know Destiny}
\statblock{\textit{Dervish 3}

\textbf{Target:} Caster

\textbf{Range:} Personal

\textbf{Duration:} Instant}

This spell functions like the \iref[spell:Commune]{Commune} spell, but rather than a yes or no answer the caster receives a hint.

\subsection{Levitate}\index[spells]{Levitate}\label{spell:Levitate}
\statblock{\textit{Elf 2, Fey 2, Sorcerer 2, Wizard 2}

\textbf{Target:} Caster

\textbf{Range:} Personal

\textbf{Duration:} 1 hour + 10 mins/level}

This spell allows the caster to rise into the air supported by magic. The caster can rise or lower themselves at a rate of 20 feet per round, but the spell does not let the caster move horizontally.

The caster can move horizontally via other means while under the influence of this spell, such as by pulling themselves along a rope or crawling on the underside of a ceiling.

\textbf{Environmental Effect:} If this spell is cast underwater, instead of granting levitation it allows the caster to swim at twice their normal speed.

\subsection{Light}\index[spells]{Light}\label{spell:Light}
\statblock{\textit{Cleric 1, Druid 1, Elf 1, Fey 1, Shaman 1, Sorcerer 1, Wizard 1}

\textbf{Target:} 15-foot radius

\textbf{Range:} 120 ft.

\textbf{Duration:} 1 hour + 10 mins/level}

When this spell is cast, the area within 15 feet of the target point is lit with light as bright as torchlight.

The caster can choose to either cast this spell in a location, in which case it will stay in that location, or cast it on an object—in which case it will move as the object moves.

This spell creates a central light source that radiates light throughout the area, so there are shadows in the area covered by this spell, and covering the object that the spell is centered on will block out the light.

If this spell is cast on a creature’s eyes, that creature must make a saving throw vs. spells or be \iref[sec:Blinded]{Blinded} for the duration of the spell.

See \fullref{sec:Light vs. Darkness} or details about how different types of natural and magical light and darkness interact.

\textbf{Reverse:} \hypertarget{spell:Darkness}{Darkness}\index[spells]{Darkness} causes the area within 15 feet of the target point to be absolutely dark, although the heat vision of some demi-humans or the dark vision of some monsters is able to penetrate it.

This spell creates a central source that radiates darkness rather, so covering the object that the spell was cast on will block the darkness.

If this spell is cast on a creature’s eyes, that creature must make a saving throw vs. spells or be \iref[sec:Blinded]{Blinded} for the duration of the spell.

See \fullref{sec:Light vs. Darkness} for details about how different types of natural and magical light and darkness interact.

\subsection{Lightning Bolt}\index[spells]{Lightning Bolt}\label{spell:Lightning Bolt}
\statblock{\textit{Sorcerer 3, Wizard 3}

\textbf{Target:} 60 ft. x 5 ft. bolt

\textbf{Range:} 180 ft.

\textbf{Duration:} Instant}

This spell creates a bolt of lightning that strikes creatures in the area for 1d6 damage per caster level (to a maximum of 20d6 unless the caster is an \iref[chap:Immortals]{Immortal}). Each creature hit may make a saving throw vs. spells to take half damage.

The caster may start the lightning bolt up to 180 feet away from themselves, and the 60-foot area of effect is then measured directly away from the caster in the same direction. If the bolt hits a wall, door, or other solid obstacle before it reaches a length of 60 feet, it will double-back and return directly towards the caster. Creatures in a doubled portion of the bolt do not take double damage.

\textbf{Environmental Effect:} If this spell is cast underwater, the lightning becomes conducted by the water causing it to take the form of a 20-foot sphere. Anyone within the sphere takes normal damage.

\subsection{Locate}\index[spells]{Locate}\label{spell:Locate}
\statblock{\textit{Dervish 1, Druid 1, Elf 1, Fey 1}

\textbf{Target:} One animal or plant

\textbf{Range:} 120 ft.

\textbf{Duration:} 1 hour}

This spell lets the caster know the direction and distance to all instances of a particular animal or plant (chosen at time of casting) that fall within range. Animals detected by this spell do not get a saving throw. It will not locate animals or plants that are not on the same plane as the caster.

\subsection{Locate Object}\index[spells]{Locate Object}\label{spell:Locate Object}
\statblock{\textit{Cleric 3, Druid 3, Elf 2, Fey 2,  Wizard 2}

\textbf{Target:} One object

\textbf{Range:} 120 ft.

\textbf{Duration:} 1 hour}

This spell lets the caster know the direction (but not distance) to the closest instance of a particular type of object (chosen at time of casting) that falls within range. The description of the object can be as vague or detailed as the caster chooses, although the object must be described rather than named.

This spell will not detect living creatures, and the object must be on the same plane as the caster.

\subsection{Locate Totem}\index[spells]{Locate Totem}\label{spell:Locate Totem}
\statblock{\textit{Medicine Man 2}

\textbf{Target:} One living creature

\textbf{Range:} None

\textbf{Duration:} 1 hour}



\subsection{Longstride}\index[spells]{Longstride}\label{spell:Longstride}
\statblock{\textit{Elf 1}

\textbf{Target:} One bipedal human or demi-human

\textbf{Range:} Touch

\textbf{Duration:} 5-8 hours}

The target of this spell doubles their walking speed for 1d4+4 hours and can move up to that speed without tiring. After the spell ends, the target must rest and consume nurishment for the same amount of time they walked during this spell. If the target does not, they temporarily lose 2d4 points of \iref[sec:Constitution]{Constitution} which is recovered at 1d4 days of rest per point lost and they must still consume nurishment.

\subsection{Lore}\index[spells]{Lore}\label{spell:Lore}
\statblock{\textit{Elf 7, Fey 5, Medicine Man 6, Wizard 7}

\textbf{Target:} Caster

\textbf{Range:} Personal

\textbf{Duration:} Instant}

This spell allows the caster to meditate on an item, place or person and gain knowledge about them. The meditation can take a long time, so can be split between multiple castings of this spell.

If the spell is cast with respect to an item that the caster has to hand, it will take 1d4x10 minutes of meditation to receive the information. At the end of that time, the caster learns the name of the item if it has one, the details of one of the item’s magical powers and how to activate that power (and how many charges it has, if applicable).

The caster does not learn whether or not the item has any additional powers, and must cast this spell again to find out.

If the spell is cast with respect to a place, an object that is not present, or a person; then the spell takes 1d100 days of meditation for the caster to learn about the subject, and the exact information learned is up to the Game Master’s discretion.

\subsection{Lower Water}\index[spells]{Lower Water}\label{spell:Lower Water}
\statblock{\textit{Elf 6, Fey 6, Wizard 6}

\textbf{Target:} 10,000 square feet of water

\textbf{Range:} 240 ft.

\textbf{Duration:} 100 minutes}

This spell causes an area of water to be reduced to half its normal depth. If cast on part of a larger body of water, it will create a trench in the water’s surface and will hold back the sides for the duration.

Any boat or ship caught in such a trench will take 1d12+20 points of hull damage when the water rushes back at the end of the duration, and all items on deck (including people if they fail saving throws vs. spells) will be swept off the ship.

\textbf{Environmental Effect:} If this spell is cast underwater, the water will rise rather than be lowered. A boat or ship caught above the effected area will get lifted into the air and flung back into the water causing 2d20+20 points of hull damage to the vessel. Creatures in the effected area will also get flung out to sea, suffering 3d20 points of damage.

\subsection{Madness}\index[spells]{Madness}\label{spell:Madness}
\statblock{\textit{Medicine Man }

\textbf{Target:}

\textbf{Range:}

\textbf{Duration:} }


\subsection{Magic Door}\index[spells]{Magic Door}\label{spell:Magic Door}
\statblock{\textit{Elf 7, Wizard 7}

\textbf{Target:} One flat surface

\textbf{Range:} 10 ft.

\textbf{Duration:} 7 uses}

This spell creates a magical doorway in a solid non-living surface such as a wall or a floor. Behind the doorway is an invisible passage up to 10 feet long with a similar door at the other end. The caster can see both door and passage and can pass through the passage 7 times before both disappear.

The door and passage are completely undetectable by normal means, although they will show up on a \iref[spell:Detect Magic]{Detect Magic} spell. The only way to destroy the passage (without destroying the wall that it runs through) is to use a \iref[spell:Dispel Magic]{Dispel Magic} spell.

\textbf{Reverse:} \hypertarget{spell:Magic Lock}{Magic Lock}\index[spells]{Magic Lock} seals any one door, chest lid, gate, archway, or other portal up to 10 by 10 feet in size with an invisible barrier.

The barrier doesn’t prevent the door (if there is one) from being opened, but it prevents passage through the portal by any but the caster. Once the caster has passed through the portal 7 times, the spell ends and the barrier disappears.

The barrier can be removed by either a \iref[spell:Dispel Magic]{Dispel Magic} or \iref[spell:Disintegrate]{Disintegrate} spell.

\subsection{Magic Jar}\index[spells]{Magic Jar}\label{spell:Magic Jar}
\statblock{\textit{Elf 5, Fey 5, Wizard 5}

\textbf{Target:} One object

\textbf{Range:} 30 ft.

\textbf{Duration:} Special}

This spell takes the caster’s life-force and places it in one object within range (the object does not have to be an actual jar).

The caster’s body falls into a deep trance while their life-force is missing, and appears dead to all but a detailed examination. The caster’s body does not need air or sustenance while their life-force is in the jar, so the caster can stay in the jar indefinitely.

The caster may attempt to possess any creature within a range of 120 feet of the jar. The target must make a saving throw vs. spells to prevent this possession, and if the saving throw succeeds the caster may not try to possess that target again for 10 minutes—although they may try to possess a different target immediately.

If the caster successfully possesses the target, the caster’s life force moves into the target’s body, and the target’s life force is forced into the jar.

The caster can use the target’s body, and its natural physical abilities; but cannot use the target’s special or magical abilities, and can cast neither their own spells nor the target’s spells.

If the target’s body is killed while the caster is in it, the target dies and the caster’s life-force returns to the jar.

If the jar is destroyed while the caster’s life-force is in it, the caster is killed.

If the jar is destroyed while the target’s life-force is in it, the target is killed and the caster is trapped in the target’s body until death.

If the caster’s own body is killed while the caster is either in the jar or in a target’s body, there is no immediate effect but the caster can no longer return to their body and must either stay in the jar or possess other bodies.

A \iref[spell:Protection from Evil]{Protection from Evil} spell will stop the caster from possessing a potential target, and a \iref[spell:Dispel Evil]{Dispel Evil} spell will force the caster back into the jar.

\iref[chap:Immortals]{Immortal} bodies cannot be possessed by use of this spell, and if an \iref[chap:Immortals]{Immortal} tries to possess a mortal body, the body is destroyed as if by a \iref[spell:Disintegrate]{Disintegrate} spell.

\subsection{Magic Missile}\index[spells]{Magic Missile}\label{spell:Magic Missile}
\statblock{\textit{Elf 1, Fey 1, Wizard 1}

\textbf{Target:} One or more creatures

\textbf{Range:} 150 ft.

\textbf{Duration:} 1 hour}

This spell creates one or more glowing missiles in the form of arrows that appear in the air around the caster and follow the caster’s movements, hovering in position.

When the caster commands, each missile will launch itself at a single target that is visible to the caster (if the caster is firing more than one missile then they may be aimed at different targets) and automatically hit for 1d6+1 damage, with no saving throw allowed.

The arrows are intangible until used, and cannot be touched or destroyed except by a \iref[spell:Dispel Magic]{Dispel Magic} spell. Arrows that are not used within an hour of casting disappear.

For each 5 levels the caster has above \nth{1}, two more missiles are created (i.e. 3 missiles at \nth{6} level, 5 missiles at \nth{11} level, 7 missiles at \nth{16} level, 9 missiles at \nth{21} level, 11 missiles at \nth{26} level, 13 missiles at \nth{31} level, 15 missiles at \nth{36} level).

\subsection{Massmorph}\index[spells]{Massmorph}\label{spell:Massmorph}

\statblock{\textit{Elf 4, Fey 4, Sorcerer 4, Wizard 4}
\textbf{Target:} 120-foot radius

\textbf{Range:} 240 ft.

\textbf{Duration:} Special}

This spell creates an illusion that makes up to 100 human-sized creatures within a 120-foot radius of the target point appear to be trees. Creatures that are larger than human-sized may count as more than one towards this 100 creature total, for example horses count as two people each.

The illusion is effective against creatures both outside and passing through the area, and is not broken by movement of the illusion-covered creatures within the area. If creatures covered by the illusion leave the area or attack or cast spells then the illusion ends for those individual creatures (even if they return to the area), but continues to affect those that remain.

The spell lasts until none of the targeted creatures are still covered or until it is dispelled, although the caster can cancel it early if they desire.

\textbf{Environmental Effect:} If this spell is cast underwater, the targets appear as kelp.

\subsection{Maze}\index[spells]{Maze}\label{spell:Maze}
\statblock{\textit{Elf 9, Fey 7, Wizard 9}

\textbf{Target:} 1 creature

\textbf{Range:} 60 ft.

\textbf{Duration:} Varies}

This spell transports the target to an indestructible maze within a temporarily created outer planar space unless they can make a saving throw vs. spells. After the target has negotiated their way out of the maze, they re-appear in the exact location that they left (or as near to it as possible without appearing inside a solid object) and the outer planar space collapses into nothingness.

The length of time taken to escape the maze is based on the creature’s \iref[sec:Intelligence]{Intelligence} as indicated on \fullref{tab:Maze}.

\begin {table}[H]
	\caption{Maze}\label{tab:Maze}
  \begin{tabularx}{\columnwidth}{>{\bfseries}YY}
	\thead{Intelligence} & \thead{Time}\\
	Up to 8 & 1d6x10 mins\\
	9-12 & 2d20 rounds\\
	13-17 & 2d4 rounds\\
	18+ & 1d4 rounds
  \end {tabularx}
\end {table}

\subsection{Metal to Wood}\index[spells]{Metal to Wood}\label{spell:Metal to Wood}
\statblock{\textit{Druid 7, Elf 8, Fey 6}

\textbf{Target:} One metal item

\textbf{Range:} 120 ft.

\textbf{Duration:} Permanent}

This spell changes a single metal item weighing up to 5 pounds (50 cn) per level of the caster into wood.

If the metal is magical then this spell has only a 10\% chance of working, otherwise it automatically works.

The item becomes normal non-magical wood, and this spell cannot be dispelled. Armor affected by this spell becomes useless and metal weapons become non-magical clubs.

\subsection{Meteor Swarm}\index[spells]{Meteor Swarm}\label{spell:Meteor Swarm}
\statblock{\textit{Wizard 9}

\textbf{Target:} Special

\textbf{Range:} 240 ft.

\textbf{Duration:} Instant}

This spell creates a number of flaming meteors that streak out from the caster’s fingertips and strike enemies, before each explodes into a 20-foot radius fire ball.

The meteors automatically hit their targets (no attack roll or saving throw allowed) doing physical damage, and then explode for fire damage. All in the radius of the fire damage (including the target struck) can make saving throws vs. spells to take half damage from the explosion.

Each meteor must be aimed at a different target, although if the targets are close together then some or all of them may take damage from multiple explosions, which do stack with each other.

The caster can choose to shoot either four meteors that each do 8d6 physical damage (no save) to their targets plus 8d6 fire damage (save for half) in their explosions or eight meteors that each do 4d6 physical damage (no save) to their targets plus 4d6 fire damage (save for half) in their explosions.

\textbf{Environmental Effect:} If this spell is cast underwater, it creates balls of lightning rather than meteors.

\subsection{Mind Barrier}\index[spells]{Mind Barrier}\label{spell:Mind Barrier}
\statblock{\textit{Elf 8, Fey 7, Wizard 8}

\textbf{Target:} One creature

\textbf{Range:} 10 ft.

\textbf{Duration:} 1 hour/level}

This spell makes the target immune to the \iref[spell:ESP]{ESP} and \iref[spell:Clairvoyance]{Clairvoyance} spells, and any type of magical scrying or information gathering spell.

This includes protecting the target from being the subject of Lore and Locate spells and protects the target from having their location discovered via the use of a \iref[spell:Summon Object]{Summon Object} spell.

The target also gets a +8 bonus to all saving throws against mind-affecting attacks such as Charm and \iref[spell:Feeblemind]{Feeblemind} spells and abilities, illusions and phantasms that require saving throws, etc. However, it does not grant a saving throw against abilities that do not normally grant one.

\textbf{Reverse:} \hypertarget{spell:Open Mind}{Open Mind}\index[spells]{Open Mind} causes the victim to get a -8 penalty to all saving throws against mind-affecting attacks such as Charm and \iref[spell:Feeblemind]{Feeblemind} spells and abilities, illusions and phantasms that require saving throws, etc.

The target gets no saving throw against open mind, but the caster must make a melee attack to touch the target.

\subsection{Minor Blessing}\index[spells]{Minor Blessing}\label{spell:Minor Blessing}
\statblock{\textit{Medicine Man 1}

\textbf{Target:} One object, creature, or place

\textbf{Range:} None

\textbf{Duration:} Permanent}



\subsection{Mirror Image}\index[spells]{Mirror Image}\label{spell:Mirror Image}
\statblock{\textit{Elf 2, Fey 2, Wizard 2}

\textbf{Target:} Caster

\textbf{Range:} Personal

\textbf{Duration:} 1 hour}

This spell creates 1d4 duplicate images of the caster which follow the caster’s every move and shift into and through each other.

It is impossible to tell which is the real caster and which are the images.

Every time the caster is hit by an attack that requires an attack roll, the attack will instead strike an image, destroying it. The caster takes no damage or other effect from the attack. Attacks and other effects that cause damage automatically (without an attack roll), such as falling or the \iref[spell:Magic Missile]{Magic Missile} spell, affect the caster normally without destroying any images.

Any attack that affects everything in an area, such as a \iref[spell:Fireball]{Fireball} spell or a dragon’s breath, will destroy all remaining images and the caster will be affected normally by the attack.

\subsection{Move Earth}\index[spells]{Move Earth}\label{spell:Move Earth}
\statblock{\textit{Elf 6, Fey 6, Sorcerer 6, Wizard 6}

\textbf{Target:} Special

\textbf{Range:} 240 ft.

\textbf{Duration:} 1 hour}

This spell allows the caster to cause soil, clay or sand—but not rock—to move horizontally or vertically in order to build a rampart or hill or to dig a hole or trench.

The caster can move soil at a rate of 60 feet per turn, and will usually be able to dig to a depth of 240 feet before reaching solid rock.

Soil can only be dug and pushed around with this spell. It cannot be made into structures that won’t support themselves.

When the duration expires, the soil stays where it is, although wind and rain may make it settle over the course of time.

\subsection{Neutralize Poison}\index[spells]{Neutralize Poison}\label{spell:Neutralize Poison}
\statblock{\textit{Cleric 4, Dervish 4, Druid 4, Elf 5, Fey 5, Medicine Man 4, Shaman 4}

\textbf{Target:} One creature, object or container

\textbf{Range:} Touch

\textbf{Duration:} Instant}

This spell will make the poison in and on one creature, object or container harmless.

The spell affects all poisons present at the time of casting, but does not cure damage. However, if a creature has been killed by poison and this spell is cast within 10 rounds of the creature’s death then it will revive the creature.

If this spell is cast by an elf, it only effects targets who were poisoned by plants or animals.

\textbf{Reverse:} \hypertarget{spell:Create Poison}{Create Poison}\index[spells]{Create Poison} will either poison a creature touched by the caster (requiring a melee attack), killing it unless it can make a saving throw vs. poison, or poison the contents of a container so that anyone who drinks or eats those contents will be killed unless they can make a saving throw vs. poison.

\subsection{Obscure}\index[spells]{Obscure}\label{spell:Obscure}
\statblock{\textit{Dervish 2, Druid 2, Elf 3, Fey 2}

\textbf{Target:} Caster

\textbf{Range:} Personal

\textbf{Duration:} 10 mins/level}

This spell creates a cold, dense cloud of mist around the caster, 1 foot high per level of the caster and 10-foot radius per level of the caster.

The mist has no effect other than to completely obscure vision within it including \iref[sec:Infravision]{Infravision}. Only the caster and creatures able to see invisible things can see through the mist.

\textbf{Environmental Effect:} If this spell is cast in a desert, a dust devil is created rather than mist.

\subsection{Pass Plant}\index[spells]{Pass Plant}\label{spell:Pass Plant}
\statblock{\textit{Druid 5, Elf 6, Fey 4}

\textbf{Target:} Caster

\textbf{Range:} Special

\textbf{Duration:} Instant}

This spell allows the caster to step inside a tree that is large enough to enclose them and instantly step out of another tree of the same kind some distance away. The maximum distance that can be teleported using this spell depends on the type of tree as indicated on \fullref{tab:Pass Plant}.

\begin {table}[H]
\caption{Pass Plant}\label{tab:Pass Plant}
  \begin{tabularx}{\columnwidth}{>{\bfseries}YY}
	\thead{Type} & \thead{Distance}\\
	Oak & 1,800 ft.\\
	Ash, Elm, Linden & 1,080 ft.\\
	Other deciduous & 900 ft.\\
	Evergreen & 720 ft.
  \end {tabularx}
\end {table}

\subsection{Pass Without Trace}\index[spells]{Pass Without Trace}\label{spell:Pass Without Trace}
\statblock{\textit{Medicine Man }

\textbf{Target:}

\textbf{Range:}

\textbf{Duration:} }


\subsection{Passwall}\index[spells]{Passwall}\label{spell:Passwall}
\statblock{\textit{Dervish 5, Sorcerer 5, Wizard 5}

\textbf{Target:} One wall, ceiling or floor

\textbf{Range:} 30 ft.

\textbf{Duration:} 30 minutes}

This spell opens a tunnel through a stone wall, ceiling or floor by making the stone disappear. The tunnel is 5 feet wide and tall, and 10 feet long.

At the end of the duration, the stone re-appears and the tunnel closes. Anyone caught in it when that happens must make a saving throw vs. spells or be trapped in the re-appearing stone and killed.

Those who make their saving throws are ejected from the closest end of the tunnel.

\subsection{Permanence}\index[spells]{Permanence}\label{spell:Permanence}
\statblock{\textit{Elf 8, Fey 7, Wizard 8}

\textbf{Target:} Special

\textbf{Range:} 10 ft.

\textbf{Duration:} Permanent}

This spell causes another spell to become permanent in duration. The two spells must be cast together (the permanence spell cannot be cast on a spell that has already been cast and is currently still active).

The following spells can be made permanent using this spell: \iref[spell:Anti-Magic Shell]{Anti-Magic Shell}, \iref[spell:Cloudkill]{Cloudkill}, \iref[spell:Confusion]{Confusion}, \iref[spell:Create Air]{Create Air}, \iref[spell:Create Normal Monsters]{Create Normal Monsters}, \iref[spell:Detect Evil]{Detect Evil}, \iref[spell:Detect Invisible]{Detect Invisible}, \iref[spell:Detect Magic]{Detect Magic}, \iref[spell:ESP]{ESP}, \iref[spell:Floating Disc]{Floating Disc}, \iref[spell:Fly]{Fly}, \iref[spell:Force Field]{Force Field}, \iref[spell:Hold Person]{Hold Person}, \iref[spell:Hold Portal]{Hold Portal}, \iref[spell:Infravision]{Infravision}, \iref[spell:Levitate]{Levitate}, \iref[spell:Light]{Light}, \iref[spell:Lower Water]{Lower Water}, \iref[spell:Mirror Image]{Mirror Image}, \iref[spell:Move Earth]{Move Earth}, \iref[spell:Phantasmal Force]{Phantasmal Force}, \iref[spell:Polymorph Self]{Polymorph Self}, \iref[spell:Projected Image]{Projected Image}, \iref[spell:Protection from Normal Missiles]{Protection from Normal Missiles}, \iref[spell:Protection from Evil]{Protection from Evil}, \iref[spell:Protection from Evil 10-foot radius]{Protection from Evil 10-foot radius}, \iref[spell:Read Languages]{Read Languages}, \iref[spell:Read Magic]{Read Magic}, \iref[spell:Shield]{Shield}, \iref[spell:Statue]{Statue}, \iref[spell:Sword]{Sword}, \iref[spell:Telekinesis]{Telekinesis}, \iref[spell:Wall of Fire]{Wall of Fire}, \iref[spell:Ice Storm/Wall of Ice]{Wall of Ice}, \iref[spell:Water Breathing]{Water Breathing}, and \iref[spell:Web]{Web}.

If cast by an \iref[chap:Immortals]{Immortal}, this spell can also make the following spells permanent: \iref[spell:Explosive Cloud]{Explosive Cloud}, \iref[spell:Gate]{Gate}, \iref[spell:Polymorph Any Object]{Polymorph Any Object}, \iref[spell:Power Word Blind]{Power Word Blind}, \iref[spell:Prismatic Wall]{Prismatic Wall}, \iref[spell:Shapechange]{Shapechange}, \iref[spell:Survival]{Survival}, \iref[spell:Timestop]{Timestop}, and \iref[spell:Travel]{Travel}.

The permanence spell only makes the natural duration of the above spells permanent. Spells that can be partially or fully canceled. before their duration has expired by particular situations (e.g. Mirror Image ending because all the images have been struck, Phantasmal Force or Projected Image ending because the illusion has been touched, or Protection from Evil partially ending because the caster has attacked a target of the spell) will be still partially or fully canceled if that situation occurs.

This spell can be dispelled by a \iref[spell:Dispel Magic]{Dispel Magic} spell, and this will cause the spell that it is sustaining to immediately end—even if that spell can not normally be dispelled.

Any area or non-living object can only have one permanence spell active at a time. If a second one is cast on the same area or object then both immediately fail.

A living creature can have up to two permanence spells active at on it one time. If a third one is cast on the same creature than all three immediately fail.

\subsection{Phantasmal Force}\index[spells]{Phantasmal Force}\label{spell:Phantasmal Force}
\statblock{\textit{Elf 2, Fey 2, Wizard 2}

\textbf{Target:} 10-foot radius

\textbf{Range:} 240 ft.

\textbf{Duration:} Concentration}

This spell creates a visual illusion within the area of effect that disappears when touched or when the caster stops concentrating.

The illusion can alter the appearance of everything within the area, and create images where there is nothing. The images can be mobile or static.

Any illusionary creatures created by this spell are armor class 9, and disappear if they take any damage.

If an illusion is of something which attacks a target, the attack is made as if the caster was making it. If that illusionary attack would damage the target, the target may make a saving throw vs. spells. If the saving throw succeeds, the target realizes that the attack is illusionary and the attack has no effect.

If the target fails the saving throw, they take damage as normal from the attack (assuming it is something they would expect to damage them) but such damage is illusionary and fades away in 1d4x10 minutes. Such illusionary damage cannot kill the target. At the most it can knock them unconscious until it fades.

Since attacking a target with an illusion usually involves the illusion touching the target, this will normally end the illusion.

\subsection{Plant Door}\index[spells]{Plant Door}\label{spell:Plant Door}
\statblock{\textit{Druid 4, Elf 5, Fey 4}

\textbf{Target:} Caster

\textbf{Range:} Personal

\textbf{Duration:} 10 mins/level}

This spell causes the caster to be completely intangible to plants. The caster can walk through dense undergrowth and even step through living trees (or hide inside them).

The spell only works on living plants, not dead wood; and although the effect includes the caster’s equipment, it doesn’t include other creatures carried by the caster.

When the duration ends, flexible plants will be bent around the caster to give them room as they reappear, and solid plants such as trees will gently push the caster out.

\subsection{Polymorph Any Object}\index[spells]{Polymorph Any Object}\label{spell:Polymorph Any Object}
\statblock{\textit{Elf 8, Fey 6, Wizard 8}

\textbf{Target:} One object or creature

\textbf{Range:} 240 ft.

\textbf{Duration:} Special}

This spell will change any object or creature into another type of object or creature. If cast at a large object, it will only change a 10-by-10-by-10-foot section of the object.

If the spell is cast at a creature, the creature may make a saving throw vs. spells at a -4 penalty to avoid the effect.

The duration of the spell depends on the degree of change between the old and new forms as indicated on \fullref{tab:Polymorph Any Object Duration}.

\begin {table}[H]
  \caption{Polymorph Any Object Duration}\label{tab:Polymorph Any Object Duration}
  \begin{tabularx}{\columnwidth}{>{\bfseries}cYYY}
	\thead{} & \multicolumn{3}{c}{\thead{Polymorphed Object Type}}\\
	\thead{Original Object Type} & \thead{Animal} & \thead{Plant} & \thead{Non-Living}\\
	Animal & Permanent & 1 hour/level & 10 mins/level\\
	Plant & 1 hour/level & Permanent & 1 hour/level\\
	Non-Living & 10 mins/level & 1 hour/level & Permanent
  \end {tabularx}
\end {table}

Regardless of the duration, the change can be dispelled with a \iref[spell:Dispel Magic]{Dispel Magic} spell.

This spell can not affect a creature’s hit points, and only works on creatures with no more than 2 hit dice per level of the caster.

The target is given all special abilities of the new form, including thinking and behaving as the new form. It cannot create a duplicate of a specific individual, only a generic individual of the desired race or monster species; and if something is polymorphed into a human or demi-human then it will not have class levels.

If this spell is cast by an \iref[class:Elf]{Elf} or Fey, only natural objects (e.g. bone, flesh, stone, wood) can be changed.

\subsection{Polymorph Other}\index[spells]{Polymorph Other}\label{spell:Polymorph Other}
\statblock{\textit{Elf 4, Fey 4, Wizard 4}

\textbf{Target:} One living creature

\textbf{Range:} 60 ft.

\textbf{Duration:} Permanent}

This spell changes a living (not undead or non-living) creature into a different type of living creature.

The target creature type must have no more than twice the number of hit dice that the original creature has, and the creature’s hit points do not change.

The target of this spell can make a saving throw vs. spells to avoid the effect.

The change can be dispelled by a \iref[spell:Dispel Magic]{Dispel Magic} spell.

The target is given all special abilities of the new form, including thinking and behaving as the new form. It cannot create a duplicate of a specific individual, only a generic individual of the desired race or monster species; and if something is polymorphed into a human or demi-human then it will not have class levels.

If an \iref[chap:Immortals]{Immortal} is polymorphed by this spell, they change in outward form only. They retain their mind (and their aura), and do not get the special abilities of the new form. However, they can return to their normal form at any time.

\subsection{Polymorph Self}\index[spells]{Polymorph Self}\label{spell:Polymorph Self}
\statblock{\textit{Elf 4, Fey 3, Medicine Man 5,  Wizard 4}

\textbf{Target:} Caster

\textbf{Range:} Personal

\textbf{Duration:} 1 hour + 10 mins/level}

This spell allows the caster to change their shape into that of another race or species.

The caster can only change into a form that has no more hit dice than the caster’s normal form. The caster cannot take the form of a specific individual, only a generic individual of the desired race or species.

The caster’s basic statistics (armor class, hit points, attack rolls, and saving throws) do not change, and the caster does not get special or magical abilities of the new form (such as the breath of a dragon or hellhound, or the regeneration ability of a troll).

The caster does get the basic physical characteristics and physical attacks of the new form (such as a dragon’s flight and a hellhound’s bite). The caster cannot cast spells while polymorphed.

If the caster is killed while polymorphed, this spell ends and they revert back to their normal shape. It also ends if dispelled.

An \iref[chap:Immortals]{Immortal} under the effect of this spell retains their aura.

\subsection{Power Word Blind}\index[spells]{Power Word Blind}\label{spell:Power Word Blind}
\statblock{\textit{Fey 7, Wizard 8}

\textbf{Target:} One creature

\textbf{Range:} 120 ft.

\textbf{Duration:} Special}

This spell blinds the targeted creature with no saving throw allowed.

Creatures with 40 or fewer hit points are \iref[sec:Blinded]{Blinded} for 1d4 days. Creatures with 41-80 hit points are \iref[sec:Blinded]{Blinded} for 2d4 hours. Creatures with 81+ hit points are unaffected by the spell.

\subsection{Power Word Kill}\index[spells]{Power Word Kill}\label{spell:Power Word Kill}
\statblock{\textit{Wizard 9}

\textbf{Target:} One or more living creatures

\textbf{Range:} 120 ft.

\textbf{Duration:} Instant}

This spell can be cast on either a single living (not undead or non-living) creature or a group of creatures.

If cast on a single creature with 60 or fewer hit points the creature dies. If cast on a single creature with 61-100 hit points the creature is \iref[sec:Stunned]{Stunned}  for 1d4x10 minutes. Creatures with 101+ hit points are unaffected by the spell.

If cast on a group of up to five creatures, any of them with 20 or fewer hit points will die, but any with 21+ hit points will be unaffected by the spell.

The target or targets of this do not get a saving throw unless they are wizards or can cast wizard spells. Even if they can cast such spells, they must make a saving throw vs. spells at a penalty of -4 in order to avoid the effects of this spell.

\subsection{Power Word Stun}\index[spells]{Power Word Stun}\label{spell:Power Word Stun}
\statblock{\textit{Wizard 7}

\textbf{Target:} One creature

\textbf{Range:} 120 ft.

\textbf{Duration:} Special}

This spell stuns the targeted creature with no saving throw.

Creatures with 35 or fewer hit points are stunned for 2d6x10 minutes. Creatures with 36-70 hit points are stunned for 1d6x10 minutes. Creatures with 71+ hit points are unaffected by the spell.

\subsection{Precipitation}\index[spells]{Precipitation}\label{spell:Precipitation}
\statblock{\textit{Elf 1, Fey 1}

\textbf{Target:} 30 ft. + 10 ft./level diameter

\textbf{Range:} 10 ft./level

\textbf{Duration:} 1 round/level}

This spell absorbs all the water vapor in the area and turns it into light rain. If the rain comes into contact with freezing cold it will turn into sleet or snow. If the rain comes into contact with extreme heat, it will turn into obscuring fog.

\textbf{Reverse:} \hypertarget{spell:Evaporation}{Evaporation}\index[spells]{Evaporation} causes dampness to evaporate or it cancels the effects of a Precipitation spell. The spell is limited to evaporating 1 gallon of water per level of the caster.

This spell has a second reversible effect. \hypertarget{spell:Dehydration}{Dehydration}\index[spells]{Dehydration} inflict 1 hp of damage on every living creature in the area unless they can make a saving throw vs. spells.

\subsection{Predict Weather}\index[spells]{Predict Weather}\label{spell:Predict Weather}
\statblock{\textit{Dervish 1, Druid 1, Elf 2, Fey 1, Medicine Man 1}

\textbf{Target:} Caster

\textbf{Range:} Personal

\textbf{Duration:} Instant}

This spell lets the caster know exactly what the weather will be like for the next 12 hours, over a range of 1 mile per level of the caster.

\subsection{Prismatic Wall}\index[spells]{Prismatic Wall}\label{spell:Prismatic Wall}
\statblock{\textit{Wizard 9}

\textbf{Target:} Special

\textbf{Range:} Special

\textbf{Duration:} 1 hour}

This spell creates seven magical barriers, each a quarter of an inch apart, that between them form a 2-inch-thick magical wall.

The wall can either be created as a 10-foot radius sphere around the caster or as a vertical wall of up to 500 square feet within 60 feet of the caster.

Once the wall is in place, only the caster may move through it without effect, and the wall may not be moved by any force—not even a \iref[spell:Wish]{Wish} spell.

The wall must be created where there is room for it. If there are creatures or objects blocking the wall then it will form around them without affecting them.

Each magical barrier is a different color, starting with violet—the barrier closest to the caster—and ending with red.

Any creature trying to pass through the barrier will be affected by each layer in turn as they move through. A creature with an \iref[spell:Anti-Magic Shell]{Anti-Magic} Shell spell active (including the caster) cannot pass through the barriers but cannot be affected by them either.

Each barrier has a different effect, and can only be destroyed in a specific way or by a \iref[spell:Wish]{Wish} spell, which will destroy the nearest three barriers to the caster. In either case, only the barrier closest to the creature trying to destroy it can be destroyed. A barrier in the middle of the wall cannot be destroyed even in the listed manner while there is another barrier between it and the creature trying to destroy it.

The barriers extend into the \iref[sec:Ethereal Plane]{Ethereal Plane} and have their full effect there as well as on the Material Plane.

\textbf{Red:} This barrier inflicts 12 points of damage (no save allowed) to any creature that crosses it. It blocks all magical missiles, and can only be destroyed by taking (any amount of) magical cold damage.

\textbf{Orange:} This barrier inflicts 24 points of damage (no save allowed) to any creature that crosses it. It blocks all non-magical missiles, and can only be destroyed by taking (any amount of) magical lightning damage.

\textbf{Yellow:} This barrier inflicts 48 points of damage (no save allowed) to any creature that crosses it. It blocks all breath weapons, and can only be destroyed by a \iref[spell:Magic Missile]{Magic Missile} spell.

\textbf{Green:} This barrier kills any creature that crosses it unless they can make a saving throw vs. spells. It blocks all forms of magical detection, and can only be destroyed by a \iref[spell:Passwall]{Passwall} spell.

\textbf{Blue:} This barrier turns any creature that crosses it to stone unless they can make a saving throw vs. petrification. It blocks all poisons, gasses and gaze attacks, and can only be destroyed by a \iref[spell:Disintegrate]{Disintegrate} spell.

\textbf{Indigo:} This barrier transports any creature that touches it to a random outer plane unless they can make a saving throw vs. spells. It blocks all matter, and can only be destroyed by a \iref[spell:Dispel Magic]{Dispel Magic} spell.

\textbf{Violet:} This barrier knocks any creature that crosses it unconscious and sends them permanently insane, unless they make a saving throw vs. spells (only one saving throw is needed to avoid both effects at once). The insanity can only be cured by a \iref[spell:Heal]{Heal} spell or a \iref[spell:Wish]{Wish} spell. It blocks all magic, and can only be destroyed by a \iref[spell:Continual Light]{Continual Light} spell.

\subsection{Produce Fire}\index[spells]{Produce Fire}\label{spell:Produce Fire}
\statblock{\textit{Druid 2, Elf 2, Fey 2, Medicine Man 2}

\textbf{Target:} Caster

\textbf{Range:} Personal

\textbf{Duration:} 20 mins/level}

This spell causes a small ball of flame to appear in the caster’s hand. The flame sheds light as a torch, and does not burn the caster.

The caster can cause the flame to disappear or reappear by concentrating for a round at any time during the duration of the spell.

The fire can be thrown up to 30 feet by the caster and will set fire to particularly flammable objects that it hits or do 1d4 damage to a creature.

In either case, the flame will then disappear, although the caster can make it reappear again (back in their hand) as normal.

\textbf{Environmental Effect:} This spell has no effect if cast underwater.

\subsection{Projected Image}\index[spells]{Projected Image}\label{spell:Projected Image}
\statblock{\textit{Elf 6, Fey 6, Sorcerer 6, Wizard 6}

\textbf{Target:} Special

\textbf{Range:} Personal

\textbf{Duration:} 1 hour}

This spell causes an illusionary duplicate of the caster to appear within 240 feet of the real caster.

For the duration of the spell, the caster can control the image’s actions by concentrating (the caster can have the image walk as they walk) and any spells cast by the caster will appear to be cast by the image—although the caster must be able to see the targets of the spell as normal, the caster cannot see through the image’s eyes.

The duplicate is armor class 9, but will not be affected by spells or missile attacks. It will disappear if touched.

\subsection{Protection from Evil}\index[spells]{Protection from Evil}\label{spell:Protection from Evil}
\statblock{\textit{Cleric 1, Druid 1, Elf 1, Fey 1, Medicine Man 1, Shaman 1, Sorcerer 1, Wizard 1}

\textbf{Target:} Caster

\textbf{Range:} Personal

\textbf{Duration:} 2 hours}

This spell creates a barrier an inch away from the caster’s body that protects the caster from various creatures.

No creature that is magically summoned, controlled, charmed or possessed can touch the caster. Neither can any creature that can only be hit by magical weapons. Such creatures can still throw or shoot things at the caster.

Additionally, all attacks against the caster are at a -1 penalty to hit and the caster gets a +1 bonus to all saving throws for the duration of this spell.

If the caster attacks a creature that is being blocked by this spell from touching them, that individual creature is no longer blocked. The to-hit penalty still applies to the creature, however.

\subsection{Protection from Evil 10-foot radius}\index[spells]{Protection from Evil 10-foot radius}\label{spell:Protection from Evil 10-foot radius}
\statblock{\textit{Cleric 4, Druid 4, Elf 3, Fey 3, Wizard 3}

\textbf{Target:} 10-foot radius

\textbf{Range:} Personal

\textbf{Duration:} 2 hours}

This spell creates a 10-foot radius barrier around the caster’s body that protects all within it from various creatures.

No creature that is magically summoned, controlled, charmed or possessed can touch those within the barrier. Neither can any creature that can only be hit by magical weapons. Such creatures can still throw or shoot things at those within.

Additionally, all attacks against creatures within the barrier are at a -1 penalty to hit and those creatures get a +1 bonus to all saving throws while inside.

If anyone within the barrier attacks a creature that is being blocked by this spell from touching them, that individual creature is no longer blocked from touching anyone inside the barrier. The to-hit penalty still applies to the creature, however.

\subsection{Protection from Lightning}\index[spells]{Protection from Lightning}\label{spell:Protection from Lightning}
\statblock{\textit{Druid 4, Elf 5, Fey 5}

\textbf{Target:} One creature

\textbf{Range:} Touch

\textbf{Duration:} 10 mins/level}

This spell protects the target from lightning damage. The spell will stop a total of 1 die of damage (of whatever die type the attack uses) per level of the caster from attacks. In the case of dragon breath, each hit dice of the dragon counts as a damage die.

Dice that are stopped by this spell are removed before rolling the damage.

\subsection{Protection from Normal Missiles}\index[spells]{Protection from Normal Missiles}\label{spell:Protection from Normal Missiles}
\statblock{\textit{Elf 3, Fey 3, Wizard 3}

\textbf{Target:} One creature

\textbf{Range:} 30 ft.

\textbf{Duration:} 2 hours}

This spell stops all small non-magical missiles (arrows, bolts, sling stones, thrown weapons, etc.) from striking the target creature.

Large missiles such as those from siege weaponry and magical missiles are not blocked by this spell.

\subsection{Protection from Poison}\index[spells]{Protection from Poison}\label{spell:Protection from Poison}
\statblock{\textit{Druid 3, Elf 3, Fey 3}

\textbf{Target:} One creature

\textbf{Range:} Touch

\textbf{Duration:} 10 mins/level}

This spell makes the target completely immune to all non-magical poisons, and magical poison spells such as the \iref[spell:Cloudkill]{Cloudkill} spell. It also gives the target a +4 bonus to all saving throws against the poisonous breath weapons that some creatures possess.

\subsection{Purify Food and Water}\index[spells]{Purify Food and Water}\label{spell:Purify Food and Water}
\statblock{\textit{Cleric 1, Druid 1, Elf 2, Fey 2, Medicine Man 1}

\textbf{Target:} Special

\textbf{Range:} 10 ft.

\textbf{Duration:} Permanent}

This spell will purify spoiled or poisoned food and drink.

It will affect enough fresh food to feed a dozen people, or enough preserved food (making it fresh again) for one person, or enough water for six people.

The spell will purify muddy or otherwise dirty water by settling out the sediment, but will have no effect on water-based creatures.

This spell does not affect preserved food if cast by an elf.

\subsection{Quest}\index[spells]{Quest}\label{spell:Quest}
\statblock{\textit{Cleric 5, Dervish 5, Druid 5, Medicine Man 5}

\textbf{Target:} One living creature

\textbf{Range:} 30 ft.

\textbf{Duration:} Special}

This spell forces the target to perform a specific action. The target may make a saving throw vs. spells to escape the effect.

The action must be something that is possible, and can’t be something suicidal—for example you can’t quest someone into jumping off a cliff.

The target must perform the action, but they are not mind controlled in any way, and they are fully aware that they may only be performing the action in order to avoid the consequences of this spell.

If the target goes against the quest, they receive a Curse, as if by the reversed form of the \iref[spell:Remove Curse]{Remove Curse} spell. Neither this curse nor the quest itself can be dispelled or removed via a \iref[spell:Remove Curse]{Remove Curse} spell, although a \iref[spell:Dispel Evil]{Dispel Evil} will remove it. The curse will not lift until the quest is fulfilled.

This spell cannot affect an \iref[chap:Immortals]{Immortal}, even if cast by another \iref[chap:Immortals]{Immortal}.

\textbf{Reverse:} \hypertarget{spell:Remove Quest}{Remove Quest}\index[spells]{Remove Quest} will remove an unwanted quest, although for each level the caster of the quest is above the caster of the remove quest there is a 5\% chance of failure.

\subsection{Raise Dead}\index[spells]{Raise Dead}\label{spell:Raise Dead}
\statblock{\textit{Cleric 5, Druid 5}

\textbf{Target:} One human or demi-human

\textbf{Range:} 120 ft.

\textbf{Duration:} Permanent}

This spell raises the body of a human or demi-human from the dead.

If the body has been dead for more than four days per level of the caster above \nth{7} then this spell will not work.

The body must be reasonably whole for this spell to work, and severed or missing body parts will still be missing afterwards, possibly leading to disability.

The target returns to life in a weakened state that lasts until they have had two weeks of bed rest, being unable to move faster than a walk and unable to attack or cast spells or use class abilities. Additionally, the target has only 1 hit point and cannot be cured further until they have rested for the two weeks.

A \iref[spell:Heal]{Heal} spell can be used to remove the two-week rest period, but will not do this and also cure the target at the same time.

If a raise dead spell is cast on an undead creature with 9 or fewer hit dice, the creature must make a saving throw vs. spells or be destroyed. Vampires are not destroyed by this spell, but forced into gaseous form and forced to retreat to their coffins until the following night.

If Raise Dead is cast on an undead creature with more than 9 hit dice, the creature takes 3d10 damage, although it can make a saving throw vs. spells to take only half damage.

When a character becomes an \iref[chap:Immortals]{Immortal}, this spell will not bring their former mortal body back to life.

\textbf{Reverse:} \hypertarget{spell:Finger of Death}{Finger of Death}\index[spells]{Finger of Death} will kill any one living creature within 60 feet unless it can make a saving throw vs. spells. Undead targeted by this spell are cured of 3d10 damage.

\subsection{Raise Dead Fully}\index[spells]{Raise Dead Fully}\label{spell:Raise Dead Fully}
\statblock{\textit{Cleric 7, Druid 7}

\textbf{Target:} One dead creature

\textbf{Range:} 60 ft.

\textbf{Duration:} Permanent}

This spell raises the body of any formerly living creature from the dead.

If the body has been dead for more than four months per level of the caster above \nth{16} then this spell will not work.

If the body has been dismembered, eaten or otherwise damaged, then only a small piece of it—a lock of hair or a sliver of bone— is needed to cast this spell, and the whole body will re-form around that piece. The piece must have been recovered from the body after death.

If the target is a human or demi-human, they are raised back up to full health and can immediately use abilities and spells with no rest period required.

If the target is not a human or demi-human, they return to life in a weakened state that lasts until they have had two weeks of bed rest, being unable to move faster than a walk and unable to attack or cast spells or use class abilities. Additionally, the target has only 1 hit point and cannot be cured further until they have rested for the two weeks.

A \iref[spell:Heal]{Heal spell} can be used to remove the two-week rest period, but will not do this and also cure the target at the same time.

If this spell is cast on an undead creature with 7 or fewer hit dice, the creature is destroyed with no saving throw.

If this spell is cast on an undead creature with 8-14 hit dice, the creature must make a saving throw vs. spells with a -4 penalty or be destroyed.

If this spell is cast on an undead creature with 15+ hit dice, the creature takes 6d10 damage, although it can make a saving throw vs. spells to take only half damage.

When a character becomes an \iref[chap:Immortals]{Immortal}, this spell will not bring their former mortal body back to life.

\textbf{Reverse:} \hypertarget{spell:Obliterate}{Obliterate}\index[spells]{Obliterate} will kill any one living creature.

If an obliterate spell is cast on a living creature with 7 or fewer hit dice or levels, the creature is killed with no saving throw.

If an obliterate spell is cast on a living creature with 8-14 hit dice or levels, the creature must make a saving throw vs. spells with a -4 penalty or be killed.

If obliterate is cast on a living creature with 15+ hit dice or levels, the creature takes 6d10 damage, although it can make a saving throw vs. spells to take only half damage.

If obliterate is cast on an undead creature, it will cure nearly all damage from the target, leaving them with only 1d6 damage—although if the target is already healthier than that it won’t damage them.

Alternatively, the spell can be used on an undead creature as a \iref[spell:Remove Curse]{Remove Curse}, \iref[spell:Cure Disease]{Cure Disease} or \iref[spell:Cure Blindness]{Cure Blindness} spell, or it can be used to cure an undead creature of a \iref[spell:Feeblemind]{Feeblemind} spell. However, it will only cure one thing per casting.

\subsection{Read Languages}\index[spells]{Read Languages}\label{spell:Read Languages}
\statblock{\textit{Elf 1, Fey 1, Sorcerer 1, Wizard 1}

\textbf{Target:} Caster

\textbf{Range:} Personal

\textbf{Duration:} 20 minutes}

This spell lets the caster read (but not speak or write) any non-magical written language or code.

\subsection{Read Magic}\index[spells]{Read Magic}\label{spell:Read Magic}
\statblock{\textit{Elf 1, Sorcerer 1, Wizard 1}

\textbf{Target:} Caster

\textbf{Range:} Personal

\textbf{Duration:} 10 minutes}

This spell lets the caster read magical runes and writings. Once a particular magical inscription has been read by this spell, the caster can re-read it at any time without needing to cast this spell again.

The most common use of this spell is to read magical scrolls.

\subsection{Reincarnation}\index[spells]{Reincarnation}\label{spell:Reincarnation}
\statblock{\textit{Elf 6, Fey 7, Sorcerer 6, Wizard 6}

\textbf{Target:} One dead creature

\textbf{Range:} 10 ft.

\textbf{Duration:} Permanent}

This spell creates a new body—not necessarily of the same species—to house the life-force of a dead creature. There is no limit on how long the creature can have been dead for, but the caster must have part of the creature’s body to cast this spell.

If the body has been dismembered, eaten or otherwise damaged, then only a small piece of it—a lock of hair or a sliver of bone— is needed to cast this spell, and the new body will re-form around that piece. The piece must have been recovered from the body after death.

All creatures have an 80\% chance to come back in a body of the same gender, and a 20\% chance to come back in a body of the opposite gender.

A human who comes back in a human body has the same class and level as before they died.

A demi-human who comes back in a human body has the same experience points as before they died, and will be the class that is closest to their racial class (elf = wizard or fighter at the player’s choice, dwarf = fighter, gnome = fighter, halfling = fighter or rogue at the player’s choice).

A human or demi-human who comes back in a demi-human body has the same experience points as before they died, but now has the racial class that corresponds to their new race.

A humanoid that comes back in a human or demi-human body will be a normal commoner without a class or levels.

A human or demi-human that comes back in a humanoid body will be a normal member of that race and unable to gain further experience.

When a creature is reincarnated, consult \fullref{tab:Reincarnation} to determine the race or species of the new body.
\begin {table}[H]
  \caption{Reincarnation}\label{tab:Reincarnation}
  \begin{tabularx}{\columnwidth}{>{\bfseries}YYYYY}
	\thead{1d8} & \thead{Race/Species}\\
	1-2 & Original\\
	3-5 & Close (GM’s choice)\\
	6 -7 & Distant (GM’s choice)\\
	8 & Very Distant (GM’s choice)
  \end {tabularx}
\end {table}

\textbf{Close:} Similar in both form and outlook to the original race. For example, humans and demi-humans are close to each other, and goblins and hobgoblins are close to each other—but a human is not close to a goblin.

\textbf{Distant:} Similar in either form or outlook, but not both, to the original race. For example humans and kobolds are distant from each other, as are elves and treants.

\textbf{Very Distant:} Similar in neither form nor outlook, for example humans and badgers.

When a character becomes an \iref[chap:Immortals]{Immortal}, this spell will not bring their former mortal body back to life.

\subsection{Remove Curse}\index[spells]{Remove Curse}\label{spell:Remove Curse}
\statblock{\textit{Cleric 3, Druid 3, Elf 4, Fey 3, Medicine Man 3, Shaman 3, Sorcerer 4, Wizard 4}

\textbf{Target:} One creature or item

\textbf{Range:} Touch

\textbf{Duration:} Permanent}

This spell removes a curse from a creature or item. Some very powerful curses may not be removable with this spell.

\textbf{Reverse:} \hypertarget{spell:Curse}{Curse}\index[spells]{Curse} gives the target a curse unless they can make a saving throw vs. spells.

The exact nature of the curse is up to the caster, although the following are typical effects.

\begin{itemize}
	\item{Something that produces up to a -4 penalty to attacks (e.g. blindness)}
	\item{Something that produces up to a -2 penalty on saving throws (such as a susceptibility to poison)}
	\item{Something that produces up to a -2 penalty to social interactions (such as smelling like a corpse)}
	\item{Something that causes a halving of a single ability score (such as withering of a limb).}
\end{itemize}

Using the curse spell to inflict effects other than things in this list require the Game Master’s permission.

\subsection{Remove Fear}\index[spells]{Remove Fear}\label{spell:Remove Fear}
\statblock{\textit{Cleric 1, Druid 1}

\textbf{Target:} One living creature

\textbf{Range:} Touch

\textbf{Duration:} 20 minutes}

This spell makes the touched creature resist fear effects. Any fear effect that allows a saving throw will automatically be resisted by the target. The target is also allowed a saving throw against effects that do not normally allow saving throws, with a bonus on the saving throw equal to the caster’s level.

Remove fear can be cast on a target who is already afraid, and will either remove or allow an immediate saving throw against the fear as above.

This spell will also remove the terror effect of an \iref[chap:Immortals]{Immortal}’s aura, but only when cast by an \iref[chap:Immortals]{Immortal}.

\textbf{Reverse:} \hypertarget{spell:Cause Fear}{Cause Fear}\index[spells]{Cause Fear} can be cast on any living creature within 120 feet and will make it flee in terror for 20 minutes unless it can make a saving throw vs. spells. If the target is cornered, they will cower and fight only to defend themselves.

\subsection{Resist Cold}\index[spells]{Resist Cold}\label{spell:Resist Cold}
\statblock{\textit{Cleric 1, Druid 1, Elf 1}

\textbf{Target:} 30-foot radius

\textbf{Range:} Personal

\textbf{Duration:} 1 hour}

This spell protects all creatures in the area from frostbite in freezing temperatures, gives each creature a +2 bonus to saving throws made against cold based attacks, and reduces all magical cold damage by one point per die of damage (to a minimum of 1 damage per die). In the case of dragon breath, each hit dice of the dragon counts as a damage die.

\subsection{Resist Fire}\index[spells]{Resist Fire}\label{spell:Resist Fire}
\statblock{\textit{Cleric 2, Druid 2, Elf 2, Medicine Man 2}

\textbf{Target:} One creature

\textbf{Range:} 30 ft.

\textbf{Duration:} 20 minutes}

This spell protects the target from being burned by natural fires, gives them a +2 bonus to saving throws made against fire based attacks, and reduces all magical fire damage by one point per die of damage (to a minimum of 1 damage per die). In the case of dragon breath, each hit dice of the dragon counts as a damage die.

\subsection{Restore}\index[spells]{Restore}\label{spell:Restore}
\statblock{\textit{Cleric 7, Druid 7}

\textbf{Target:} One creature

\textbf{Range:} Touch

\textbf{Duration:} Permanent}

This spell restores one level that has been drained from the target by an \ilink{spell:Energy Drain}{Energy Drain} spell or an energy draining creature. The target is restored to the exact experience total that they had before they were energy drained. If the target has already gained more experience than that since the energy drain then this spell does not add extra experience.

Restore can also be used to remove a magical aging effect from a creature, restoring them to their normal age (plus any time that has passed since the magical aging happened). It will not remove natural aging.

Unless the caster of this spell is an \iref[chap:Immortals]{Immortal}, they temporarily lose a level when casting this spell as if they had been energy drained themselves. However, this level is recovered after 2d20 days of rest.

\textbf{Reverse:} \hypertarget{spell:Energy Drain}{Energy Drain}\index[spells]{Energy Drain} drains a single level from the target, who the caster must touch (make a normal attack), but who gets no saving throw.

If an \iref[chap:Immortals]{Immortal} casts energy drain at another \iref[chap:Immortals]{Immortal}, the target must make a saving throw vs. power or lose 5 pp. \iref[chap:Immortals]{Immortals} can not lose levels due to energy drain, even if they have no power points left.

\subsection{Reverse Gravity}\index[spells]{Reverse Gravity}\label{spell:Reverse Gravity}
\statblock{\textit{Wizard 7}

\textbf{Target:} 30-foot radius

\textbf{Range:} 90 ft.

\textbf{Duration:} 2 seconds}

This spell reverses the pull of gravity on all creatures within a 15-foot radius of the target point. The creatures will fall upwards for the duration of the spell, falling a maximum of 65 feet. Then, at the end of the spell’s duration, they will fall back to the floor again.

There is no saving throw against this spell.

Creatures who hit obstacles on either the upwards or downwards fall take 1d6 damage per 10 feet fallen.

\subsection{Sanctify}\index[spells]{Sanctify}\label{spell:Sanctify}
\statblock{\textit{Medicine Man }

\textbf{Target:} One creature or object

\textbf{Range:} None

\textbf{Duration:} Permanent}



\subsection{Second Sight}\index[spells]{Second Sight}\label{spell:Second Sight}
\statblock{\textit{Fey 4}

\textbf{Target:} Caster

\textbf{Range:} Personal

\textbf{Duration:} 1 day/level}

This spell allows the caster to see all invisible things and the true form of any shapechanged creatures.

\subsection{Shapechange}\index[spells]{Shapechange}\label{spell:Shapechange}
\statblock{\textit{Elf 9, Fey 6, Wizard 9}

\textbf{Target:} Caster

\textbf{Range:} Personal

\textbf{Duration:} 10 mins/level}

This spell allows the caster to change their shape into that of other races or species. The caster can change shape as often as they like during the spell’s duration, with each change taking a round.

The caster cannot take the form of a specific individual, only a generic individual of the desired race or species.

The caster’s basic statistics (armor class, attack rolls, number of attacks) change, although the caster’s hit points and saving throws do not.

The caster does get the special abilities of the new form (such as the breath of a dragon or hellhound, and the regeneration ability of a troll). This applies to flaws of the form as well as benefits.

The caster cannot cast spells while shapechanged, unless in the form of a bipedal humanoid. In any case, the caster can only cast their own spells, not spells that are innate to the form (such as a dryad’s innate Charm Person spell).

If the caster is killed while shapechanged, this spell ends and they revert back to their normal shape. It also ends if dispelled.

A shapechanged caster cannot pass through a \iref[spell:Protection from Evil]{Protection from Evil} spell or an \iref[spell:Anti-Magic Shell]{Anti-Magic Shell} spell.

A shapechanged \iref[chap:Immortals]{Immortal} retains their aura, and can cast spells regardless of form.

\subsection{Shield}\index[spells]{Shield}\label{spell:Shield}
\statblock{\textit{Fey 1, Wizard 1}

\textbf{Target:} Caster

\textbf{Range:} Personal

\textbf{Duration:} 20 minutes}

This spell creates a magical barrier an inch away from the caster’s body that shields the caster from harm.

While this spell is in effect, the caster has an armor class of 2 against missile attacks and an armor class of 4 against all other attacks rather than their normal armor class of 9.

This spell also grants the caster a saving throw vs. spells each time they would be hit by the missile from a \iref[spell:Magic Missile]{Magic Missile} spell. If the saving throw fails, the missile hits the caster normally. If the saving throw succeeds, the missile is blocked, but this spell is ends immediately.

When cast by an \iref[chap:Immortals]{Immortal}, this spell gives the caster a -4 AC bonus against missile attacks and a -2 AC bonus against other attacks.

\subsection{Shift Sand}\index[spells]{Shift Sand}\label{spell:Shift Sand}
\statblock{\textit{Dervish 3}

\textbf{Target:} 25 cubic feet of sand

\textbf{Range:} 30 ft.

\textbf{Duration:} Special}

This spell causes sand to flow like water, but at the caster's bidding. It can be used to tunnel passages, unbury objects, or excavate a ruin. When the spell ends, the sand returns to normal, flowing naturally with gravity. The sand will retain its shape for one day if it is thoroughly wetted before casting this spell.

\subsection{Shimmer}\index[spells]{Shimmer}\label{spell:Shimmer}
\statblock{\textit{Medicine Man }

\textbf{Target:}

\textbf{Range:}

\textbf{Duration:} }



\subsection{Silence}\index[spells]{Silence}\label{spell:Silence}
\statblock{\textit{Elf 2, Fey 2}

\textbf{Target:} One creature

\textbf{Range:} Touch

\textbf{Duration:} 2 hours}

This spell makes it impossible for any noise to be produced by the target creature. The creature may make a saving throw vs. spells to avoid the effect.

\subsection{Silence 15-foot radius}\index[spells]{Silence 15-foot radius}\label{spell:Silence 15-foot radius}
\statblock{\textit{Cleric 2, Druid 2, Fey 3}

\textbf{Target:} 15-foot radius

\textbf{Range:} 180 ft.

\textbf{Duration:} 2 hours}

This spell makes it impossible for any noise to be produced in the targeted area. However, noises from outside the area can still be heard from within it.

The spell can be cast at an area, in which case the effect is static until the duration ends or it is dispelled, or it can be cast at a creature, in which case the creature must make a saving throw vs. spells. If the saving throw fails, the spell moves with the creature. If the saving throw succeeds, the spell still works centered on the creature’s current position; but does not move with the creature.

\subsection{Silent Move}\index[spells]{Silent Move}\label{spell:Silent Move}
\statblock{\textit{Medicine Man }

\textbf{Target:}

\textbf{Range:}

\textbf{Duration:} }



\subsection{Sleep}\index[spells]{Sleep}\label{spell:Sleep}
\statblock{\textit{Fey 1, Sorcerer 1, Wizard 1}

\textbf{Target:} Creatures in a 20-foot radius

\textbf{Range:} 240 ft.

\textbf{Duration:} 4d4 x 10 minutes}

This spell puts one or more creatures to sleep.

Roll 2d8 to see how many hit dice worth of creatures are slept by the effect.

Go through all the living (not undead or non-living) creatures in the area with fewer than 5 hit dice or levels, starting with the closest to the target point of the spell. If there are enough hit dice left from the roll, that creature is slept with no saving throw and their hit dice are taken from the running total. Once there are no more creatures left with fewer (or equal) hit dice to the number of hit dice left over, the spell stops.

Falling to the ground when slept by this spell will not wake the target up, and neither will noise. Targets will awaken if kicked, shaken or otherwise physically disturbed.

Sleeping creatures are considered to be \iref[sec:Helpless]{Helpless} opponents.

\subsection{Snake Charm}\index[spells]{Snake Charm}\label{spell:Snake Charm}
\statblock{\textit{Cleric 2, Dervish 2, Druid 2, Shaman 2}

\textbf{Target:} One or more snakes

\textbf{Range:} 60 ft.

\textbf{Duration:} Special}

This spell charms 1 hit dice of snakes per level of the caster, with no saving throw allowed.

The affected snakes will rise up and sway, and will not attack any creature unless they are attacked themselves.

If this spell is cast on snakes that are attacking the caster, it lasts for 1d4+1 rounds. If cast on snakes that are not attacking the caster, it lasts for 10 minutes + 1d4 x 10 minutes.

After the spell duration has ended, the snakes return to their normal behavior.

\subsection{Speak with Animal}\index[spells]{Speak with Animal}\label{spell:Speak with Animal}
\statblock{\textit{Cleric 2, Dervish 2, Druid 2, Elf 3, Medicine Man 2, Shaman 2}

\textbf{Target:} Caster

\textbf{Range:} Personal

\textbf{Duration:} 1 hour}

This spell lets the caster communicate with a single species of animal for the duration. The species named must be a normal type of animal, not a magical or intelligent type.

The communication is limited by the intelligence of the animal being communicated with.

\subsection{Speak with Dead}\index[spells]{Speak with Dead}\label{spell:Speak with Dead}
\statblock{\textit{Cleric 3, Druid 3, Fey 6}

\textbf{Target:} One corpse

\textbf{Range:} 10 ft.

\textbf{Duration:} 1 rnd/level}

This spell lets the caster ask three questions to a dead body. The spirit of the deceased will is summoned and must answer the questions, although it cannot interact in any other way unless it is already an undead creature such as a ghost or spectre.

The caster must possess a part of the corpse, such as a lock of hair or a piece of bone. The age of the corpse that can be spoken with by use of this spell depends on the level of the caster as indicated on \fullref{tab:Speak with Dead}.

\begin {table}[H]
  \caption{Speak with Dead}\label{tab:Speak with Dead}
  \begin{tabularx}{\columnwidth}{>{\bfseries}YY}
	\thead{Level} & \thead{Age}\\
	6-7 & 4 days\\
	8-14 & 4 months\\
	15-20 & 4 years\\
	21+ & Unlimited
  \end {tabularx}
\end {table}

The spirit of the deceased will always reply in a language known to the caster, and must answer the questions truthfully, but a hostile spirit may equivocate and mislead if it chooses providing it does not directly lie.

The spirit will only have knowledge of things that it experienced when it was alive.

\subsection{Speak with Monsters}\index[spells]{Speak with Monsters}\label{spell:Speak with Monsters}
\statblock{\textit{Cleric 6, Dervish 6, Druid 6, Fey 6, Shaman 6}

\textbf{Target:} 30-foot radius

\textbf{Range:} Personal

\textbf{Duration:} 1 rnd/level}

This spell lets the caster communicate with any creature for the duration.

The depth of communication is limited by the intelligence of the creature being communicated with, although even unintelligent creatures which do not normally communicate can do so in a rudimentary fashion if this spell is used.

\textbf{Reverse:} Babble causes one target within 60 feet of the caster to be completely unable to communicate with other creatures for 10 minutes per caster level, unless they can make a saving throw vs. spells with a -2 penalty.

Any attempt to communicate with others—including hand signals, telepathy and writing will be garbled.

The target can still cast spells, but is unable to use magic items that require command words to activate them.

\subsection{Speak with Plants}\index[spells]{Speak with Plants}\label{spell:Speak with Plants}
\statblock{\textit{Cleric 4, Dervish 4, Druid 4, Medicine Man 4, Shaman 4}

\textbf{Target:} Caster

\textbf{Range:} Personal

\textbf{Duration:} 30 minutes}

This spell allows the caster to talk to plants, which will respond as if they were intelligent.

Normal plants will be friendly to the caster, and are able to move slowly while under the influence of this spell in order to obey simple commands from the caster, such as picking something up or leaning out of the way of a path.

The spell also allows the caster to talk to plant-like monsters, although it does not influence them.

\subsection{Spell Turning}\index[spells]{Spell Turning}\label{spell:Spell Turning}
\statblock{\textit{Medicine Man }

\textbf{Target:}

\textbf{Range:}

\textbf{Duration:} }


\subsection{Spirit Sending}\index[spells]{Spirit Sending}\label{spell:Spirit Sending}
\statblock{\textit{Medicine Man }

\textbf{Target:}

\textbf{Range:}

\textbf{Duration:} }


\subsection{Spirit Storm}\index[spells]{Spirit Storm}\label{spell:Spirit Storm}
\statblock{\textit{Medicine Man }

\textbf{Target:}

\textbf{Range:}

\textbf{Duration:} }


\subsection{Spirit Walk}\index[spells]{Spirit Walk}\label{spell:Spirit Walk}
\statblock{\textit{Medicine Man }

\textbf{Target:}

\textbf{Range:}

\textbf{Duration:} }



\subsection{Statue}\index[spells]{Statue}\label{spell:Statue}
\statblock{\textit{Elf 7, Fey 5, Wizard 7}

\textbf{Target:} Personal

\textbf{Range:} Caster

\textbf{Duration:} 20 mins/level}

This spell allows the caster to transform to or from a statue as often as they like during the duration. Each transformation takes a round. The caster can even turn back to normal after having been petrified by a monster’s attack or a spell, providing this spell was already active when the petrification happened.

While in stone form, the caster cannot move or cast spells (but can continue to concentrate on spells already cast). The caster is armor class -4, and cannot be hurt by non-magical weapons or by cold or fire attacks. The caster does not need to breathe while in statue form and is immune to poison, drowning, and gas-based attacks.

When this spell is active but the caster is in normal form, the caster gets a +2 bonus on initiative rolls if their action for the round is to turn to statue form.

If this spell is cast by an Elf or Fey, they instead turn into a rock similar to ones in the area.

\subsection{Steelform}\index[spells]{Steelform}\label{spell:Steelform}
\statblock{\textit{Wizard 8}

\textbf{Target:} None

\textbf{Range:} Touch

\textbf{Duration:} Permanent}

This spell creates a sheet of high quality steel up to 2 inches thick and up to 500 square feet in area. The sheet can be created flat or can be created pre-shaped. The steel does not appear instantly, but takes time to form, ranging from a single round for a simple sheet of steel to two hours to create something with a precise specification, such as a sword.

Whatever the complexity of the shape, the steel must form a single piece with no moving parts.

However, the caster can create the steel in a “rough” form, which can then have the steelform spell cast on it again in order to either add to the object or reshape it. If the caster does create the steel in rough form, then this spell is cast once again to “set” the steel in its final form so that other casters can’t cast this spell on it in order to modify it.

Note that whatever shape the sheet takes, it is always a maximum of 2 inches thick—so that if it is formed into a statue, for example, the statue would be hollow.

Once created, the steel is real and cannot be dispelled, and it will last until it is physically or magically destroyed.

\subsection{Sticks to Snakes}\index[spells]{Sticks to Snakes}\label{spell:Sticks to Snakes}
\statblock{\textit{Cleric 4, Druid 4}

\textbf{Target:} Up to 16 sticks

\textbf{Range:} 120 ft.

\textbf{Duration:} 1 hour}

This spell turns a number of normal sticks into snakes.

Roll 2d8. That many sticks in range turn into small snakes. Each snake has an equal chance to be a Poisonous or Racer snake.

The snakes obey the caster, and then turn back to sticks when either the duration ends or they are killed.

\textbf{Environmental Effect:} If this spell is cast underwater, the sticks are turned into eels rather than snakes. The spell may also be cast on seaweed fronds.

\subsection{Stone to Flesh}\index[spells]{Stone to Flesh}\label{spell:Stone to Flesh}
\statblock{\textit{Elf 6, Fey 6, Sorcerer 6, Wizard 6}

\textbf{Target:} One creature or object

\textbf{Range:} 120 ft.

\textbf{Duration:} Permanent}

This spell turns any one stone object (or a 10-by-10-by-10-foot section of stone wall) to flesh.

If the stone object is a petrified creature or part of one, then the creature will be restored by this spell (although they may immediately die if there are parts missing). If the stone object is not a petrified creature or part of one, then it turns into a gelatinous mass of shapeless skin, fat and dead flesh with tufts of coarse hair and the occasional eye. The lump of flesh has no proper internal structure or bones, and cannot support its own weight. It cannot be raised or reincarnated, since it has never had a life-force.

The meat is edible, although it smells and tastes foul.

\textbf{Reverse:} \hypertarget{spell:Flesh to Stone}{Flesh to Stone}\index[spells]{Flesh to Stone} will permanently turn one living creature (including all equipment carried) into stone unless the target makes a saving throw vs. petrification.

If an \iref[chap:Immortals]{Immortal} is petrified by this spell, it only lasts for one round per hit dice the \iref[chap:Immortals]{Immortal} has, and the target remains fully aware of their surroundings while petrified. An \iref[chap:Immortals]{Immortal} whose \iref[sec:Embodied Form]{Embodied Form} is petrified can still switch to Spirit Form.

If this spell is cast by an elf, it is not reversible.

\subsection{Stoneform}\index[spells]{Stoneform}\label{spell:Stoneform}
\statblock{\textit{Wizard 6}

\textbf{Target:} None

\textbf{Range:} Touch

\textbf{Duration:} Permanent}

This spell creates a mass of stone up to 1,000 cubic feet in area. The mass can be arranged in any manner the caster desires. The stone does not appear instantly, but takes time to form, ranging from a single round for a simple stone wall to two hours to create something with a precise specification, such as a statue.

The stone created by this spell can be as soft as chalk or as hard as granite, but cannot be a precious or semi-precious stone. The exception to this is that the spell can create clear (or tinted) lead crystal of a fine enough quality to be used for windows.

Whatever the complexity of the shape, the stone must form a single piece with no moving parts. However, the caster can create the stone in a “rough” form, which can then have the stoneform spell cast on it again in order to either add to the object or reshape it. If the caster does create the stone in rough form, then this spell is cast once again to “set” the stone in its final form so that other casters can’t cast this spell on it in order to modify it.

Note that whatever shape the stone takes, it cannot be created in the space where another object exists and must be created on a surface that can support its weight.

Once created, the stone is real and cannot be dispelled, and it will last until it is physically or magically destroyed.

\subsection{Strength of Mind}\index[spells]{Strength of Mind}\label{spell:Strength of Mind}
\statblock{\textit{Medicine Man }

\textbf{Target:}

\textbf{Range:}

\textbf{Duration:} }



\subsection{Striking}\index[spells]{Striking}\label{spell:Striking}
\statblock{\textit{Cleric 3, Druid 3}

\textbf{Target:} One weapon

\textbf{Range:} 30 ft.

\textbf{Duration:} 10 minutes}

This spell temporarily enchants a weapon to do extra damage. It can only be cast on hand-held weapons or missile weapons, not on the natural weaponry of a creature.

Until the duration ends, the weapon does an extra 1d6 damage to anything it hits, and can hit creatures that are only hit by magical weapons (although it does not grant the weapon any magical bonuses to hit).

\subsection{Summon Animal Spirit, Greater}\index[spells]{Summon Animal Spirit, Greater}\label{spell:Summon Animal Spirit, Greater}
\statblock{\textit{Medicine Man }

\textbf{Target:}

\textbf{Range:}

\textbf{Duration:} }


\subsection{Summon Animal Spirit, Lesser}\index[spells]{Summon Animal Spirit, Lesser}\label{spell:Summon Animal Spirit, Lesser}
\statblock{\textit{Medicine Man }

\textbf{Target:}

\textbf{Range:}

\textbf{Duration:} }


\subsection{Summon Animals}\index[spells]{Summon Animals}\label{spell:Summon Animals}
\statblock{\textit{Dervish 4, Druid 4, Elf 4, Fey 3, Medicine Man 4}

\textbf{Target:} Special

\textbf{Range:} 360 ft.

\textbf{Duration:} 30 minutes}

This spell summons one or more animals within range. It only affects normal, non-magical animals, and not normal insects and arthropods.

The caster may specify a particular species or group of animals (but not individual named animals) or may summon any animals in range.

The minimum number of animals that arrive will be none—if there are no animals within range—and the maximum number of animals is a number whose hit dice are equal to the caster’s level. Tiny animals (such as mice, songbirds, frogs, rabbits etc.) count as a tenth of a hit dice each.

When the animals arrive, they will understand the caster’s instructions and help the caster in any way they can, although if they are attacked by anything they will not fight to the death but will flee instead—although if the caster is already in combat when the animals arrive, they will fight to help the caster unless badly injured.

When deciding which animals answer the summons, animals that the caster can see will be affected first, even if they are currently hostile to the caster.

\subsection{Summon Elemental}\index[spells]{Summon Elemental}\label{spell:Summon Elemental}
\statblock{\textit{Druid 7}

\textbf{Target:} None

\textbf{Range:} 240 ft.

\textbf{Duration:} 1 hour}

When this spell is cast, a 16 hit dice Elemental (see \fullref{sec:Elemental}) will appear within 240 feet of the caster. If this spell is cast more than once during the same day, a different type of elemental must be summoned each time.

While the caster controls the elemental, they can make it do anything it is capable of doing, including fighting to the death on the caster’s behalf. The caster can also send the controlled elemental home.

The caster does not need to concentrate to keep controlling the elemental.

A summoned elemental is blocked by a Protection from Evil, and can be sent home by a Dispel Magic or a Dispel Evil.

\textbf{Environmental Effect:} If this spell is cast underwater, only an Earth Elemental or Water Elemental may be conjured. Earth Elementals must stay in contact with the ground, or they will be sent home in 1d4 rounds. There are no restrictions on consecutively summoning the same type of elemental.

\subsection{Summon Herd}\index[spells]{Summon Herd}\label{spell:Summon Herd}
\statblock{\textit{Medicine Man }

\textbf{Target:}

\textbf{Range:}

\textbf{Duration:} }



\subsection{Summon Object}\index[spells]{Summon Object}\label{spell:Summon Object}
\statblock{\textit{Elf 9, Fey 7, Wizard 7}

\textbf{Target:} One object in caster’s home

\textbf{Range:} Infinite

\textbf{Duration:} Instant}

This spell summons a non-living object weighing 50 pounds (500 cn) or less from the caster’s home to their hand, no matter how great the distance even if the caster is on another plane, but providing the caster is in the same Celestial Sphere as their home. The object must be no bigger than a staff or small chest.

The caster must know the exact location of the item, and must be familiar with it. The item must also have been prepared beforehand by sprinkling it with a special powder that costs 1,000 gold pieces per item. The powder evaporates as part of the preparation process, and does not affect the item in any way once the preparation is complete.

If the item is a container of some sort, it will appear without its contents—even if those contents have been independently prepared.

If another creature has taken the object from where the caster placed it, the object will not be summoned—but the caster will know roughly where the object is and who has it.

If an \iref[chap:Immortals]{Immortal} casts this spell while in a Mortal Form, it will not summon objects from the home of the \iref[chap:Immortals]{Immortal's} \iref[sec:Embodied Form]{Embodied Form}.

\subsection{Summon Weather}\index[spells]{Summon Weather}\label{spell:Summon Weather}
\statblock{\textit{Dervish 6, Druid 6, Fey 4}

\textbf{Target:} One weather phenomenon

\textbf{Range:} 5+ miles

\textbf{Duration:} 1 hour/level}

This spell summons a weather phenomenon within range that the caster is aware of (either by being able to see it or knowing about it through a \iref[spell:Predict Weather]{Predict Weather} spell or other form or scrying) to the caster’s current location.

The weather phenomenon must be within 5 miles, plus a mile for every caster level above \nth{15}.

Particularly powerful weather conditions such as hurricanes, tornadoes, etc. can only be summoned by a caster of \nth{21} level or higher.

\subsection{Survival}\index[spells]{Survival}\label{spell:Survival}
\statblock{\textit{Cleric 7, Dervish 7, Druid 7, Fey 6, Wizard 9}

\textbf{Target:} One creature

\textbf{Range:} Touch

\textbf{Duration:} 1 hour/level}

This spell protects the target from harmful environments.

The target is completely protected from normal heat and cold, and can survive without air, food, water or sleep.

This spell does not protect the target against attacks of any kind, only the natural environment. It does protect completely from the environment of other planes, and even the Luminiferous Aether.

\subsection{Sword}\index[spells]{Sword}\label{spell:Sword}
\statblock{\textit{Elf 9, Fey 6, Wizard 7}

\textbf{Target:} None

\textbf{Range:} 30 ft.

\textbf{Duration:} 1 rnd/level}

This spell creates a glowing sword made from magical force next to the caster.

The caster can make the sword attack any creature within 30 feet by concentrating. The sword flies to the target and attacks.

If the caster stops concentrating, the sword stops attacking, but remains in existence and returns to the caster’s side, following them for the duration.

The sword attacks twice per round, making attacks at the caster’s base attack bonus for 1d10 damage. The sword has no magical bonuses to its attack rolls, but can hit any target.

The sword cannot be physically damaged in any way, but may be dispelled.

If an elf cast this spell, they may choose to wield the sword. This allows them to use any special maneuvers that they would normally be able to use with a normal sword.

\subsection{Sword of Fire}\index[spells]{Sword of Fire}\label{spell:Sword of Fire}
\statblock{\textit{Medicine Man }

\textbf{Target:}

\textbf{Range:}

\textbf{Duration:} }



\subsection{Symbol}\index[spells]{Symbol}\label{spell:Symbol}
\statblock{\textit{Elf 8, Medicine Man 6, Wizard 8}

\textbf{Target:} Special

\textbf{Range:} Touch

\textbf{Duration:} Permanent}

This spell creates a glowing magical rune that can be traced either on an object or surface or even traced in the air.

The rune can have one of six different effects, chosen at the time of casting.

Any creature that passes through a rune in the air, or touches the object on which the rune is placed, is affected by the rune immediately with no saving throw allowed.

Merely seeing the rune is not enough to make it work, but reading the rune aloud will also trigger it with no saving throw.

A character using the \iref[spell:Read Magic]{Read Magic} spell can read a rune safely in order to identify which type of rune it is without triggering it.

A symbol can be triggered any number of times, and will remain active until dispelled.

If placed on an object, the symbol must be exposed for it to work. A symbol will not work if it is covered over and hidden.

\textbf{Symbol of Death:} Kills any creature with 75 hit points or fewer. Creatures with more than 75 hit points are unaffected.

\textbf{Symbol of Discord:} Confuses the creature permanently, as if the \iref[spell:Confusion]{Confusion} spell had been cast upon them. The confusion can be cured by a \iref[spell:Heal]{Heal} spell or a \iref[spell:Wish]{Wish} spell.

\textbf{Symbol of Fear:} Causes the target to flee for 30 rounds.

\textbf{Symbol of Insanity:} Causes the victim to be driven permanently insane and unable to cast spells. The insanity can only be cured by a \iref[spell:Heal]{Heal} spell or a \iref[spell:Wish]{Wish} spell.

\textbf{Symbol of Sleep:} Sends the creature into a magical sleep for 1d10+10 hours, which they cannot be woken from unless the sleep is dispelled.

\textbf{Symbol of Stunning:} Stuns any creature with 150 or fewer hit points for 2d6x10 minutes.

\iref[chap:Immortals]{Immortals} are immune to the effects of symbol spells, even if cast by other \iref[chap:Immortals]{Immortals}.

\subsection{Telekinesis}\index[spells]{Telekinesis}\label{spell:Telekinesis}
\statblock{\textit{Elf 5, Fey 4, Wizard 5}

\textbf{Target:} One object or creature

\textbf{Range:} 120 ft.

\textbf{Duration:} 6 rounds}

This spell lets the caster move an object or creature weighing up to 20 pounds (200 cn) per level of the caster by concentrating on it.

The object can be moved at a speed of 20 feet per round.

If the target is a creature, it can avoid the effect by making a saving throw vs. spells.

If the target is an object being held by a creature, such as a weapon in the creature’s hand, the creature can make a saving throw vs. spells with a -2 penalty in order to keep hold of the object—but if the caster is high enough level to move the weight of both the object and creature together, this will result in the creature being dragged along with the object unless they let go.

If the target is an object being carried (but not held) by a creature, such as a weapon in the creature’s scabbard, the creature can make a saving throw vs. spells with a -5 penalty in order to keep hold of the object—but if the caster is high enough level to move the weight of both the object and creature together, this will result in the creature being dragged along with the object unless they let go.

If the target is an object being worn by a creature, such as a backpack, it cannot be moved unless the caster is high enough level to move the weight of both the object and creature together, this will result in the creature being dragged along with the object.

\subsection{Teleport}\index[spells]{Teleport}\label{spell:Teleport}
\statblock{\textit{Elf 5, Wizard 5}

\textbf{Target:} One living creature

\textbf{Range:} 10 ft.

\textbf{Duration:} Instant}

This spell instantly transports a living (not undead or non-living) creature to another place on the same plane and within the same Celestial Sphere. The caster can use this spell to transport themselves.

The destination can be any distance from the target’s current location, but it cannot be a place occupied by a solid object and must be a place with a solid ground or floor.

If the target is unwilling to be teleported, they may make a saving throw vs. spells to avoid the effect.

There is a chance that this spell will result in the target appearing too high or too low. A character appearing in solid matter (usually because they appeared to low) is instantly killed.

The chance of failure depends on how familiar the caster is with the destination as indicated on \fullref{tab:Teleport}.

\begin {table}[H]
  \caption{Teleport}\label{tab:Teleport}
	\begin{tabularx}{\columnwidth}{>{\bfseries}Y>{\bfseries}Y>{\bfseries}Yc}
		\multicolumn{3}{c}{\thead{Familiarity (1d100)}} & \thead{}\\
	\thead{Unfamiliar} & \thead{Familiar} & \thead{Very Familiar} & \thead{Result}\\
	01-50 & 01-80 & 01-95 & On Target\\
	51-75 & 81-90 & 96-99 & 1d10x10 ft. too high\\
	76-00 & 91-00 & 00 & 1d10x10 ft. too low
  \end {tabularx}
\end {table}

\textbf{Unfamiliar:} The caster has been to the location but spent less than a day there, or the caster is currently scrying the location but has not studied it.

\textbf{Familiar:} The caster has spent more than a day at the destination, or has spent several weeks scrying on the destination.

\textbf{Very Familiar:} The caster can see the destination or the caster has spent several weeks there.

\subsection{Teleport Any Object}\index[spells]{Teleport Any Object}\label{spell:Teleport Any Object}
\statblock{\textit{Elf 7, Fey 7, Wizard 7}

\textbf{Target:} One object or creature

\textbf{Range:} Touch

\textbf{Duration:} Instant}

This spell instantly transports one creature or object that weighs up to 50 pounds (500 cn) per level of the caster to another place on the same plane and within the same Celestial Sphere. The spell can be used to transport a 10-by-10-by-10-foot section of rock such as from a wall or floor, but can not be used in this way to transport only part of a creature.

The caster can use this spell to transport themselves, and if doing so there is no chance of error.

The destination can be any distance from the target’s current location, but it cannot be a place occupied by a solid object and must be a place with a solid ground or floor.

If the target is unwilling to be teleported, or unwilling to have an object that they are holding or carrying teleported, they may make a saving throw vs. spells with a -2 penalty to avoid the effect.

Unless the caster is teleporting themselves, there is a chance that this spell will result in the target appearing too high or too low. A character appearing in solid matter (usually because they appeared to low) is instantly killed.

The chance of failure depends on how familiar the caster is with the destination as indicated on \fullref{tab:Teleport}.

\subsection{Thunder Drum}\index[spells]{Thunder Drum}\label{spell:Thunder Drum}
\statblock{\textit{Medicine Man 3}

\textbf{Target:} One animal

\textbf{Range:} 120 ft.

\textbf{Duration:} 1 round}



\subsection{Timestop}\index[spells]{Timestop}\label{spell:Timestop}
\statblock{\textit{Elf 9, Fey 7, Wizard 9}

\textbf{Target:} Caster

\textbf{Range:} Personal

\textbf{Duration:} 1d4+1 rounds}

This spell speeds up the caster so much that from their perspective time seems to have stopped.

From the caster’s point of view, time has effectively stopped, and they have 1d4+1 rounds during which everything else is frozen.

The extra rounds that a caster has during this spell are not affected by \iref[spell:Haste]{Haste} and \ilink{spell:Slow}{Slow} spells. The caster always gets the normal one action per round.

During the spell, the caster is still harmed by magical fire, cold, gas, poison and so on; but not by their non-magical equivalents.

The caster can move freely through air and water during the timestop, but cannot affect creatures or normal objects (other than those the caster was carrying when the spell was cast) in any way.

The caster can cast other spells during the timestop. Those of instant duration such as a \iref[spell:Fireball]{Fireball} spell will have no effect on anything other than the caster, and those with a non-instant duration will not come into effect until the timestop runs out.

The caster cannot pass through a Protection from Evil or an Anti-Magic Shell during the timestop.

\iref[chap:Immortals]{Immortals} within 50 feet of the caster when the spell is cast (or whom the caster approaches within 50 feet during the spell’s duration) are sped up alongside the caster and can interact with the caster and each other normally until they are no longer within 50 feet.

\subsection{Totem Mastery}\index[spells]{Totem Mastery}\label{spell:Totem Mastery}
\statblock{\textit{Medicine Man }

\textbf{Target:}

\textbf{Range:}

\textbf{Duration:} }


\subsection{Trance}\index[spells]{Trance}\label{spell:Trance}
\statblock{\textit{Medicine Man }

\textbf{Target:}

\textbf{Range:}

\textbf{Duration:} }


\subsection{Transport Through Plants}\index[spells]{Transport Through Plants}\label{spell:Transport Through Plants}
\statblock{\textit{Druid 6, Elf 7, Fey 5}

\textbf{Target:} Caster and up to two others

\textbf{Range:} Touch

\textbf{Duration:} Instant}

This spell allows the caster to magically enter a plant large enough to hold them within range that is and emerge from a similar plant at their chosen destination.

The caster can either specify an exact plant that they wish to emerge from, or specify a general location; in which case they will emerge from a random plant at that location. There is no limit to the distance that can be traveled using this spell, although both plants must be alive and both plants must be on the same plane and within the same Celestial Sphere.

If either plant is dead, or if there is no plant large enough to hold the caster at the specified location, the spell fails.

The caster can take up to two creatures with them, providing each creature is small enough to fit into the plants and each creature is willing.

This spell can only be cast once per day.

\subsection{Travel}\index[spells]{Travel}\label{spell:Travel}
\statblock{\textit{Cleric 7, Druid 7, Elf 8, Wizard 8}

\textbf{Target:} Special

\textbf{Range:} Personal

\textbf{Duration:} 10 mins/level}

This spell allows the caster, to travel quickly, even from plane to plane.

The caster can fly at a rate of 120 feet per round, and can enter another plane (adjacent to the caster’s current plane) by concentrating for a round. The caster can only enter one plane per 10 minutes.

The caster can also use this spell to open up a hole in a celestial shell in order to travel from a Celestial Sphere to the Luminiferous Aether or vice versa. The hole is 100 feet in diameter, and lasts for 10 minutes before closing. Not even a \iref[spell:Wish]{Wish} spell can hold such a hole open for longer.

The caster can also turn to smoke, and fly at double the normal rate (240 feet per round).

While in smoke form the caster cannot use items or cast spells, but cannot be hurt by non-magical means. Also, while in smoke form, the caster cannot pass through a \iref[spell:Protection from Evil]{Protection from Evil} spell or an \iref[spell:Anti-Magic Shell]{Anti-Magic Shell}.

The caster can bring up to one other creature with them for each five levels. All passengers must be touching the caster at all times, and the caster controls the direction and form of travel. Unwilling passengers may make a saving throw vs. spells to avoid the effect.

The caster must travel with all passengers, they cannot send passengers on while remaining behind.

When an \iref[chap:Immortals]{Immortal} is traveling under the effect of this spell, it costs them 5 pp to cross a planar boundary, or 10 pp if the planes either side of the boundary are both Outer Planes. Opening a hole in a Celestial Sphere costs an \iref[chap:Immortals]{Immortal} 100 pp.

\subsection{Truesight}\index[spells]{Truesight}\label{spell:Truesight}
\statblock{\textit{Cleric 5, Druid 5, Elf 7, Fey 6, Medicine Man 5}

\textbf{Target:} Caster

\textbf{Range:} Personal

\textbf{Duration:} 10 mins + 1 rnd/level}

This spell allows the caster to see all ethereal, hidden and invisible things within 120 feet.

The caster also sees the true form of any disguised, polymorphed and shapechanged creatures.

Finally, the caster can tell the alignment and level or hit dice of any creature by looking at them.

\subsection{Truth or Else}\index[spells]{Truth or Else}\label{spell:Truth or Else}
\statblock{\textit{Dervish 3}

\textbf{Target:} One living creature

\textbf{Range:} Touch

\textbf{Duration:} Instant}

A creature who swears an oath to perform an action or declares a statement as true will be inflicted by the \ilink{spell:Curse}{Curse} spell if they do not complete the action or if their statement was intentionally falsified.

It is the Game Master's decision on whether or not the action was completed or if the statement was true. It is also their decision on when the curse takes effect and the type of curse inflicted.

A target of this spell can make a saving throw vs. spells at a -4 penalty to avoid the curse.

\subsection{Turn Wood}\index[spells]{Turn Wood}\label{spell:Turn Wood}
\statblock{\textit{Druid 6, Elf 7}

\textbf{Target:} Special

\textbf{Range:} 360 ft.

\textbf{Duration:} 10 mins/level}

This spell creates an invisible force field 120 feet wide and 60 feet tall anywhere within 30 feet of the caster.

The field moves away from the caster at 10 feet a round until it reaches a maximum distance of 360 feet from the caster. The caster can stop the movement of the field at a shorter distance, but if they do so they can not re-start it again.

The field pushes all wood and wooden objects, including those held by creatures but excluding those securely fastened down, away. The objects aren’t harmed by the pushing—if pushed against a wall they won’t be damaged—but can’t be moved back through the field and are therefore probably unusable.

The caster can cancel the spell at any time, or it can be dispelled. Otherwise, the force field remains until the end of the duration.

\subsection{Uncontrollable Hideous Laughter}\index[spells]{Uncontrollable Hideous Laughter}\label{spell:Uncontrollable Hideous Laughter}
\statblock{\textit{Wizard 2}

\textbf{Target:} One creature

\textbf{Range:} 60 ft.

\textbf{Duration:} 1 rnd/level}

The target of this spell collapses into an uncontrollable fit of laughter. While in this state, the target suffers a -4 to their armor class and may not take any actions.

Creatures with an intelligence score less than 2 are not affected by this spell.

\subsection{Ventriloquism}\index[spells]{Ventriloquism}\label{spell:Ventriloquism}
\statblock{\textit{Elf 1, Fey 1, Wizard 1}

\textbf{Target:} One object or location

\textbf{Range:} 60 ft.

\textbf{Duration:} 20 minutes}

This spell allows the caster to make the sound of their voice come from the targeted object or location rather than from their own mouth.

The caster can still cast spells normally while this spell is in effect.

\subsection{Wall of Fire}\index[spells]{Wall of Fire}\label{spell:Wall of Fire}
\statblock{\textit{Elf 4, Sorcerer 4, Wizard 4}

\textbf{Target:} Special

\textbf{Range:} 60 ft.

\textbf{Duration:} Concentration}

This spell creates a vertical wall of fire of any shape with a total area of up to 1,200 square feet. The entire wall must be within 60 feet of the caster. The wall cannot be created in the space occupied by creatures or objects.

The wall blocks sight, and lasts as long as the caster concentrates.

The wall cannot be pushed through by creatures with less than 4 hit dice or levels. Creatures with 4 or more hit dice or levels can push through the wall but take 1d6 damage while doing so (cold based creatures and undead take double damage). Pushing through does break the wall or leave a hole in it.

\textbf{Environmental Effect:} If this spell is cast underwater, the duration is quartered and it causes the water around it to coming to a steaming boil, obscuring vision in the area.

\subsection{Wall of Iron}\index[spells]{Wall of Iron}\label{spell:Wall of Iron}
\statblock{\textit{Sorcerer 6, Wizard 6}

\textbf{Target:} Special

\textbf{Range:} 120 ft.

\textbf{Duration:} Permanent}

This spell creates a wall of solid iron, up to 500 square feet in area and 2 feet thick. The entire wall must be within 120 feet of the caster.

The wall cannot be created in the space occupied by creatures or objects, and it must rest on the ground or similar support.

The wall is magical and lasts until it is dispelled, disintegrated or physically destroyed.

If the wall is toppled over, it will cause 10d10 damage to those it lands on, and then it crumbles to rust colored dust and disappears. Any iron chipped or hacked off the wall also crumbles to rust colored dust and disappears.

\subsection{Wall of Stone}\index[spells]{Wall of Stone}\label{spell:Wall of Stone}
\statblock{\textit{Sorcerer 5, Wizard 5}

\textbf{Target:} Special

\textbf{Range:} 60 ft.

\textbf{Duration:} Permanent}

This spell creates a wall of solid stone, up to 500 square feet in area and 2 feet thick. The entire wall must be within 60 feet of the caster. The wall cannot be created in the space occupied by creatures or objects, and it must rest on the ground or similar support.

The wall is magical and lasts until it is dispelled, disintegrated or physically destroyed.

If the wall is toppled over, it will cause 10d10 damage to those it lands on, and then it crumbles to sand. Any stone chipped or hacked off the wall also crumbles to sand and disappears.

\subsection{War Paint}\index[spells]{War Paint}\label{spell:War Paint}
\statblock{\textit{Medicine Man 4}

\textbf{Target:} Body paints

\textbf{Range:} None

\textbf{Duration:} 1 day}



\subsection{Warp Wood}\index[spells]{Warp Wood}\label{spell:Warp Wood}
\statblock{\textit{Druid 2, Elf 2, Fey 2}

\textbf{Target:} One or more wooden weapons

\textbf{Range:} 240 ft.

\textbf{Duration:} Permanent}

This spell causes one or more wooden weapons within range to bend and twist and become unusable.

The spell will affect one arrow per level of the caster. Spears, javelins or magical wands count as two arrows each, and clubs, bows, axe or mace shafts, or staffs count as four arrows each.

Weapons wielded by creatures will be unaffected if their holders make a saving throw vs. spells. Weapons merely carried by creatures get no such saving throw.

Magic items have a 10\% chance per magical plus (rolled independently of the saving throw, if any) to be unaffected.

\subsection{Watcher}\index[spells]{Watcher}\label{spell:Watcher}
\statblock{\textit{Elf 1, Fey 1}

\textbf{Target:} One plant or animal

\textbf{Range:} 10 ft.

\textbf{Duration:} 2-8 turns + 1 turn/level}

This spell causes one plant or local animal to react to the presence of any living creature larger than 1/2 cubic feet or 3 lbs in weight. As soon as a living creature moves past, touches, or otherwise disturbs the target, the target will begin emitting a loud keening sound that can be heard within a 60' radius. Interposing doors reduce this radius by 10' and thick walls reduce it by 20'. The sound last for 1 round.

\subsection{Water Breathing}\index[spells]{Water Breathing}\label{spell:Water Breathing}
\statblock{\textit{Druid 3, Elf 3, Fey 3, Sorcerer 3, Wizard 3}

\textbf{Target:} One creature

\textbf{Range:} 30 ft.

\textbf{Duration:} 1 day}

This spell lets the target creature breathe water for the duration of the spell. It does not affect movement in any way.

\textbf{Reverse:} Air Breathing lets the target creature breathe air for the duration of the spell. It does not affect movement in any way.

\subsection{Weather Control}\index[spells]{Weather Control}\label{spell:Weather Control}
\statblock{\textit{Dervish 7, Druid 7, Elf 6, Fey 5, Medicine Man 8, Wizard 6}

\textbf{Target:} 720-foot radius around caster

\textbf{Range:} Personal

\textbf{Duration:} Concentration}

This spell allows the caster to control the weather in the surrounding area. The spell only works outdoors, and lasts as long as the caster concentrates.

The caster may make any of the following conditions, and can change the conditions by one minute of concentration:

\textbf{Clear:} This nullifies any existing weather conditions, and has no special effect of its own.

\textbf{Fog:} Reduces visibility to 20 feet and halves the movement rate of all creatures.

\textbf{Gales:} Halves the movement rate of all creatures, and makes missile fire and flying impossible. Speeds up the movement of sailing ships by 50\%. In a desert, sandstorms reduce visibility to 20 feet.

\textbf{Heatwave:} Halves the movement rate of all creatures and after 30 minutes dries up rain, mud and snow.

\textbf{Rain:} Gives a -2 penalty to attack rolls with missile weapons. After 30 minutes, the ground turns muddy and non-flying movement is halved.

\textbf{Snow:} Visibility is reduced to 20 feet and all non-flying creatures have their movement halved. After 30 minutes, standing water and slow rivers and streams freeze. When the snow thaws, the ground turns muddy and non-flying movement is halved.

\textbf{Tornado:} Treat as a 12 hit dice Air Elemental (see \fullref{sec:Air Elemental}) under the caster’s control.

\subsection{Web}\index[spells]{Web}\label{spell:Web}
\statblock{\textit{Elf 2, Fey 2, Sorcerer 2, Wizard 2}

\textbf{Target:} 10 ft. x 10 ft. x 10 ft. cube

\textbf{Range:} 10 ft.

\textbf{Duration:} 8 hours}

This spell creates a mass of sticky web in the area, blocking the area from movement and trapping all those within it. Creatures caught in the web can still defend themselves but cannot move.

To break through or out of the web takes 2d4x10 minutes for creatures of human strength. Creatures with giant \iref[sec:Strength]{Strength} (21+) can break through or out in two rounds, and creatures with ogre \iref[sec:Strength]{Strength} (18+) can break through or out in four rounds.

The web is highly flammable and if touched by a flame it will burn away in two rounds doing 1d6 damage to all creatures within it.

\subsection{Wish}\index[spells]{Wish}\label{spell:Wish}
\statblock{\textit{Cleric 7, Druid 7, Elf 9, Wizard 9}

\textbf{Target:} Varies

\textbf{Range:} Varies

\textbf{Duration:} Varies}

This particularly powerful spell can only be cast by a \nth{36} level caster with 18 or higher \iref[sec:Wisdom]{Wisdom}.

The caster speaks a wish out loud and the universe itself—a power beyond even the \iref[chap:Immortals]{Immortals}, although \iref[chap:Immortals]{Immortals} are certainly capable of casting this spell—will rearrange matters to make the wish come true within limits.

If the wish is overly powerful, then it may be only partially granted or may be granted in a way that fits the caster’s literal wording but goes against the caster’s intent.

However, usage of this spell should not be allowed to degenerate into a contest between the player and the Game Master where the player tries to make the wording of the wish as airtight as possible in order to achieve an effect that the Game Master would otherwise not allow and the Game Master tries their best to find loopholes in the wording that they can use to mess up the player’s intent.

Instead, if the Game Master is not happy with the wording of the wish because the effect is too powerful they should simply say that it does not work, and explain what it is about it that they think is too powerful.

The exact limits of this spell are left to the Game Master’s discretion, although some default limitations and examples are given below:

A wish can duplicate the effects of any other wizard spell of \nth{8} level or lower, or any other cleric or druid spell of \nth{6} level or lower.

A wish cannot give experience points or levels.

A wish can gain the caster treasure, goods or magic items worth up to 50,000 gp, but at the cost of an equal number of experience points.

A wish cannot hurt or kill another creature (unless it is duplicating a lower level spell that might have that effect), but can inconvenience them such as by transporting them away.

A wish can change the species or race of an intelligent creature as if it had been killed and raised by the Reincarnation spell, with the caster choosing the new race. An unwilling target of this can make a saving throw vs. spells at a -4 penalty to avoid this effect.

Multiple wishes can permanently raise an ability score by one point, up to a maximum of 18. It takes a number of wishes equal to the new value of the ability score, and they must all be cast within a week by the same caster.

Some other spells, items or abilities specifically say that their effects can be altered by a wish or that their effects cannot be altered by a wish. Those effects override the limitations and permitted usages listed here.

When cast by an \iref[chap:Immortals]{Immortal}, this spell costs the \iref[chap:Immortals]{Immortal} 100,000 experience points, and this spell can not be used to increase an \iref[chap:Immortals]{Immortal}’s ability scores.

\subsection{Wizard Eye}\index[spells]{Wizard Eye}\label{spell:Wizard Eye}
\statblock{\textit{Elf 4, Fey 4, Wizard 4}

\textbf{Target:} One invisible eye

\textbf{Range:} 240 ft.

\textbf{Duration:} 1 hour}

This spell creates an invisible magical eye the size of a human eye. It can see as well as a human and has \iref[sec:Infravision]{Infravision}.

The caster can see through and/or move the eye (40 feet per round) by concentrating, although it remains in existence for the duration even while the caster is not concentrating; returning to the caster’s side and following them.

\subsection{Wizard Lock}\index[spells]{Wizard Lock}\label{spell:Wizard Lock}
\statblock{\textit{Fey 2, Wizard 2}

\textbf{Target:} One portal or other lock

\textbf{Range:} 10 ft.

\textbf{Duration:} Permanent}

This spell magically holds closed any door, gate or other portal; or magically holds locked any lock.

The caster can open the door at any time, but otherwise the door can only be forced open by creatures with at least three more levels or hit dice than the caster or by a \iref[spell:Knock]{Knock} spell. In either case, the door re-locks as soon as it is allowed to close, and this spell resumes until dispelled.

An \iref[chap:Immortals]{Immortal} can always open a door that is being held closed by a mortal’s casting of this spell.

\subsection{Wizardry}\index[spells]{Wizardry}\label{spell:Wizardry}
\statblock{\textit{Cleric 7, Druid 7}

\textbf{Target:} One magical item

\textbf{Range:} Personal

\textbf{Duration:} 10 minutes}

This spell allows the caster to use one magic item that is normally useable only by wizards or elves (e.g. a wand or staff), or read one scroll containing a wizard spell of up to third level.

If used to read a scroll, the scroll (but not others with the same spell) is considered “known” to the caster, who can use it at a later time after this spell has run out. The caster is treated as if they are a wizard of the minimum level needed to cast the spell on the scroll.

\subsection{Woodform}\index[spells]{Woodform}\label{spell:Woodform}
\statblock{\textit{Wizard 5}

\textbf{Target:} None

\textbf{Range:} Touch

\textbf{Duration:} Permanent}

This spell creates a mass of wood up to 1,000 cubic feet in area. The mass can be arranged in any manner the caster desires. The wood does not appear instantly, but takes time to form, ranging from a single round for a simple wooden wall to two hours to create something with a precise specification, such as a staircase.

Whatever the complexity of the shape, the wood must form a single piece with no moving parts. However, the caster can create the wood in a “rough” form, which can then have the woodform spell cast on it again in order to either add to the object or reshape it. If the caster does create the wood in rough form, then this spell is cast once again to “set” the wood in its final form so that other casters can’t cast this spell on it in order to modify it.

Note that whatever shape the wood takes, it cannot be created in the space where another object exists and must be created on a surface that can support its weight.

Once created, the wood is real and cannot be dispelled, and it will last until it is physically or magically destroyed.

\subsection{Word of Recall}\index[spells]{Word of Recall}\label{spell:Word of Recall}
\statblock{\textit{Cleric 6, Dervish 6, Druid 6, Shaman 6}

\textbf{Target:} Caster

\textbf{Range:} Personal

\textbf{Duration:} Instant}

This spell instantly teleports the caster and their equipment (but no other creatures) back to the caster’s home, providing the caster is on the same plane and within the same Celestial Sphere.

The caster must have a permanent home for this spell to work, and must have a private bedroom or meditation chamber within that home.

The caster automatically wins initiative in the round that they cast this spell, although it cannot be cast in a surprise round.

\subsection{Wrath of Amerind}\index[spells]{Wrath of Amerind}\label{spell:Wrath of Amerind}
\statblock{\textit{Medicine Man 6}

\textbf{Target:} None

\textbf{Range:} 240 ft.

\textbf{Duration:} 1 turn}


\end{multicols*}

