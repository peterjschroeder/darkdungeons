\chapter[green]{Combat}
\label{chap:Combat}
\chapterimage[Combat (Bordered)]
\thispagestyle{plain}

\begin{multicols*}{2}
When a fight breaks out, or is about to break out, between two groups of characters or monsters, timekeeping in the game switches to round-by-round timekeeping and the rules in this chapter are followed.

Although in reality combat is fluid with actions happening simultaneously, in Dark Dungeons the action is split into a number of discrete rounds during which each combatant (usually) gets one action. Within the round, the action of each combatant is handled one at a time, in order of their initiative.

\section{The Combat Round}\index[general]{The Combat Round}
Each combat round is a period of ten seconds. During this time, each combatant will normally perform a single action and possibly also move. The round is split up into three phases, which are always performed in order:

\begin{enumerate}
	\item{Statement of Intent}
	\item{Initiative Roll}
	\item{Actions (in initiative order)}
\end{enumerate}

When all phases have been performed, a new round starts with the first phase again. This continues until there is no more combat or round-by-round action (such as chasing fleeing combatants) happening.

\subsection{Statement of Intent}\index[general]{Statement of Intent}
At the start of each round, each player must announce what their characters are intending to do in the round, and the Game Master announces what the monsters will do.

The statement of intent phase is split into three segments, which proceed in order.

Firstly, players may announce what actions their characters will be doing this round, if they wish their characters to do such actions urgently. If a player announces their character’s action at this time, their character is assumed to be pressing on with that action quickly, and the player will get a +1 bonus on their initiative roll this round. However, the disadvantage of announcing at this time is that their intent is obvious to their enemies who may decide how to respond accordingly.

Secondly, the Game Master announces what actions the monsters will be doing this round, taking into account the fact that the monsters will be aware of the intentions of the players that have already announced such intentions.

Thirdly, players who wish their characters to be fighting in a more cautious manner must announce what their characters will be doing this round. They have the advantage of not declaring (or deciding) until after they know what the monsters are doing, but pay for this hesitancy by having a -1 penalty on their initiative roll this round.

When announcing their actions, people must specify whether they are going to attack (including target and whether a special attack such as a \iref[sec:Charge]{Charge} will be used), run (including intended destination), cast a spell (including which spell and which targets), or do another action.

\subsection{Initiative}\index[general]{Initiative}
Once everyone has announced their actions for the round, everyone rolls for initiative, in order to see who manages to complete their actions first.

The basic roll for initiative is 1d6. A player who declared a statement of intent before the monsters did gets +1. A player who waited to see what the monsters were doing before declaring a statement of intent gets a -1. This roll may be further modified by such things as spells, class selection, and high or low \iref[sec:Dexterity]{Dexterity}.

In some cases, an item or ability will specifically indicate that a character or monster will either automatically win initiative or automatically lose initiative.

If there is only one combatant using such an ability in a round, then the effect is straightforward. The combatant does not need to roll for initiative, and instead automatically wins or automatically loses depending on the ability.

If there is more than one combatant who “automatically wins” initiative then all those combatants will act before everyone else, but they should roll initiative normally in order to determine the order in which they go in relation to each other.

Similarly, if there is more than one combatant who “automatically loses” initiative then all those combatants will act after everyone else, but they should roll initiative normally in order to determine the order in which they go in relation to each other.

When two or more combatants roll the same initiative total, their actions should take place simultaneously with the results of both actions being resolved after both actions have taken place. Common sense should prevail here, although if both make attacks on each other, then it should be possible for both to kill each other simultaneously.

When rolling for initiative, the players should each roll individually for their characters. The Game Master should roll once per type of monster that the players are fighting, and roll separately for leaders and/or other special monsters.

\example{Elfstar and Aloysius are fighting some zombies. Elfstar has already Turned as many as she can, and Aloysius has run out of spells, so they are both resorting to melee attacks.

At the beginning of the round, Debbie knows that zombies are slow and always lose initiative. Therefore, during the statement of intent phase she waits to see what the zombies are doing—knowing that even with the -1 penalty to initiative rolls Elfstar will still act before they do.

Andy, on the other hand, knows that Aloysius’s staff is a two handed weapon, and therefore also always loses initiative; so he is going to have to roll against the zombies. Wanting to finish off the zombie that is attacking him before it gets another blow, he declares during the first part of the statement of intent phase that Aloysius is hitting that zombie with his staff. Because Andy declared before the zombies, Aloysius will get a +1 bonus to his initiative roll against them.

The Game Master then gives the statement of intent for the zombies.

The one that is attacking Aloysius will continue to attack him, and the two that are attacking Elfstar will continue to attack her. The other zombie—which is too far away to attack anyone—will use its full movement to close to melee range with Aloysius.

Debbie now gives her statement of intent for Elfstar, which is to make a melee attack on one of the zombies.

Initiative is rolled. Debbie doesn’t bother rolling because everyone except for her automatically loses initiative, so she automatically acts first.

Andy and the Game Master both roll 1d6. Andy rolls a 4, which—with his +1 bonus for making an early statement of intent—gives him an initiative of 5. The Game Master rolls a 2 for the zombies.

Then everyone takes their actions: Elfstar first, then Aloysius, then the zombies.}

\subsection{Actions}\index[general]{Actions}\label{sec:Actions}
The following actions are commonly used by combatants during Dark Dungeons combat. The list is not exhaustive, as unusual situations may require unusual actions to be performed, such as breaking down a door.

In these cases, extrapolate from the listed actions in order to determine when the action can be done, how it affects initiative, and whether a character can also move in the same round.

\textbf{Activate Magic Item:}\index[general]{Activate Magic Item} A combatant who declares that they are activating a magic item (such as a wand or scroll) must declare which item they are activating, which of the item’s powers they wish to use, and who the targets are (if any).

Only some magic items (see \fullref{sec:Magic Items}) need to be activated in this way.

The combatant is considered to be in the process of activating the item from the start of the round until their action is resolved. If they take any damage before their turn (because someone who beat their initiative attacked them, for example) the activation is disrupted.

If the character has any deflect abilities or armor class bonuses from their weapon feats, they may not use them during a round in which they have declared an activate magic item action without voluntarily (and immediately) allowing the activation to be disrupted if it is not yet complete.

If the activation is disrupted, the item still counts as having been used. Depending on the item and power being activated, this may result in charges or ‘per day’ usages being used up, or even the destruction of the item if it was a single use item such as a scroll.

A combatant may abandon their activation action entirely (for example if their chosen target is no longer valid or if the activation got disrupted) but may not otherwise change the target, item or power during their action.

\textbf{Attack:}\index[general]{Attack}\label{sec:Attack} A combatant who declares that they are making an attack (whether in melee, by throwing something, or by firing a missile weapon) must declare who they are attacking during the statement of intent phase.

A combatant who declares that they are attacking with a two handed melee weapon automatically loses initiative.

A combatant who is attacking can move their normal per-round movement distance (40 feet for an unencumbered character) before making the attack, but may not move after the attack.

Normally a combatant can make only a single attack per attack action, but some combatants are capable of making multiple attacks. These multiple attacks occur as part of the same action and on the same initiative, and the combatant cannot move between attacks. If a combatant has multiple attacks, then they must declare the target for each attack during the statement of intent phase.

If any of the attacks are disarm attacks, this must also be declared during the statement of intent phase.

If the character has any deflect abilities or armor class bonuses from their weapon feats, they may use them at any time during a round in which they have declared an attack action.

When taking their action, the combatant must move toward and attack the target(s) that they declared attacks on. They cannot change targets during the round, although they can simply abandon either the movement or the attack or both, and simply not make one or the other if they choose.

If a combatant abandons the attack, they may not change their action.

\example{During the statement of intent phase, Marcie declares that Black Leaf is going to stab the goblin that is guarding the door. The Game Master declares that the goblin is going to try to run away.

When initiative is rolled, Marcie rolls a 1 for Black Leaf and the Game Master rolls a 5 for the goblin. Even with Black Leaf’s initiative bonuses for her high \iref[sec:Dexterity]{Dexterity} and for declaring first, the lucky goblin still beats her initiative roll and acts before her.

On the goblin’s action, it runs away from Black Leaf as fast as it can—which is at three times its normal per-round movement speed (i.e. 3 x 30 feet = 90 feet), shouting for reinforcements to come and help fight the adventurers.

On Black Leaf’s turn, she can move her normal movement rate (40 feet) towards the goblin and attack. She cannot reach the goblin with this move, but decides to make it anyway. Since she is not within melee range, she cannot make her melee attack so must abandon it.}

\textbf{Cast Spell:}\index[general]{Cast Spell} A combatant who declares that they are casting a spell must declare which spell they are casting and who the targets are (if any).

The magical special abilities of monsters are considered spells for this purpose, even if they do not exactly match the description of a standard spell.

In order to cast a spell, a caster must be able to speak and must have at least one hand free to gesture.

The caster is considered to be in the process of casting the spell from the start of the round until their action is resolved. If they take any damage before their turn (because someone who beat their initiative attacked them, for example) the spellcasting is disrupted.

If the character has any deflect abilities or armor class bonuses from their weapon feats, they may not use them during a round in which they have declared a cast spell action without voluntarily (and immediately) allowing their spell to be disrupted if casting is not yet complete.

If the spell is disrupted, the spell slot is still used up.

A caster may abandon their spellcasting action entirely (for example if their chosen target is no longer valid or if the spell got disrupted) but may not otherwise change the target or spell during their action.

\textbf{Charge:}\index[general]{Charge}\label{sec:Charge} A character can only charge if they are using a weapon with that ability and if they are mounted.

A combatant who declares that they are making a charge must declare the target during the statement of intent phase.

The character moves up to their mount’s normal per-round movement speed, and makes a single attack against their target the end of the movement.

If the attack hits, it does double the normal damage.

If the character has any deflect abilities or armor class bonuses from their weapon feats, they may use them at any time during a round in which they have declared a charge action.

\textbf{Concentrate:}\index[general]{Concentration} Some spells or other effects require ongoing concentration.

A combatant who declares that they are concentrating to maintain an effect must declare what the effect is that they are concentrating on, and if the effect is one that can be changed or moved by concentration they must also declare how they are changing or moving it. If the combatant also wishes to move in the round that they are concentrating, they must also declare where they are moving to.

A combatant who is concentrating may move up to half their normal per-round movement speed during their action (usually 20 feet for an unencumbered character).

The concentration is assumed to last for the entire round, so if the combatant who is concentrating takes any damage during the round they will lose their concentration and the effect that requires concentration to maintain will end.

If the character has any deflect abilities or armor class bonuses from their weapon feats, they may not use them during a round in which they have declared a concentrate action without voluntarily (and immediately) allowing their concentration to be disrupted.

A combatant whose concentration has been disrupted before their action may still make their declared movement.

\textbf{Fighting Withdrawal:}\index[general]{Fighting Withdrawal} This action may only be declared if the combatant is in melee at the start of the round.

This is similar to a normal attack action in that the character can move their normal per-round movement rate and then make one or more attacks.

However, instead of being committed to attacking their target, and moving if necessary to reach the target; the combatant is instead committed to moving away from their target.

If the target acts before the character doing the fighting withdrawal, the withdrawing character gets their full defenses against any attacks the target might do.

If the target acts after the character doing the fighting withdrawal, and follows them in order to attack them, the withdrawing character interrupts the attacking character after movement but before their attack in order to make their own attack.

If the character has any deflect abilities or armor class bonuses from their weapon feats, they may use them at any time during a round in which they have declared a fighting withdrawal action.

\textbf{Run:}\index[general]{Run} A combatant using the run action can move up to three times their normal per-round movement speed (usually 120 feet for an unencumbered character).

The combatant must declare where they are running to during the statement of intent phase—although this may be towards a moving target such as towards another combatant.

A character who chooses the run action may not change where they are running to, but may stop running at any time short of their intended destination.

A combatant who is running does not count their shield bonus towards their armor class.

If the character has any deflect abilities or armor class bonuses from their weapon feats, they may not use them during a round in which they have declared a run action.

\textbf{Set Spear:}\index[general]{Set Spear}\label{sec:Set Spear} A character can only set a spear if they are using a weapon with that ability.

A combatant who declares that they are setting a spear against possible charges does not need to specify targets.

The character braces their weapon against the ground for the whole round, and waits for incoming attacks.

If, at any point during the round, the combatant is attacked by someone using the charge action, they may interrupt the charging character’s action after movement but before their attack in order to make their own attack.

If this attack hits the charging opponent, it does double damage, the effects of which are resolved before the charging opponent gets their attack.

If the character has any deflect abilities or armor class bonuses from their weapon feats, they may not use them during a round in which they have declared a set spear action.

\textbf{Use Non-Activatable Item:}\index[general]{Use Non-Activable Item} A combatant who declares that they are using a non-activatable item (such as a ring or potion) must declare which item they are using, which of the item’s powers they wish to use, and who the targets are (if any). If the combatant also wishes to move in the round that they are using the item, they must also declare where they are moving to.

Only some magic items (see \fullref{sec:Magic Items}) can be used without activation in this way.

A combatant who is using a non-activatable item can move their normal per-round movement distance (40 feet for an unencumbered character) before using the item, but may not move after using it.

If the character has any deflect abilities or armor class bonuses from their weapon feats, they may use them at any time during a round in which they use a non-activatable item.

When taking their action, the combatant cannot change targets during the round, although they can simply abandon either the movement or the usage or both, and simply not make one or the other if they choose.

If a combatant abandons the attack, they may not change their action.

\section{Attack Bonus}\index[general]{Attack Bonus}\label{sec:Attack Bonus}
A character or creature’s attack bonus represents their combat skill. For player characters it is based on their level and class (see the class progression tables in \fullref{chap:Classes}). For monsters, it is based on their Hit Dice (see \fullref{tab:Base Monster Abilities}). Attack bonuses start at +0, which represents a person or monster who is completely unskilled and unused to combat, and increase with increasing ability, to a maximum of +50 or more.

\section{Attack Rolls}\index[general]{Attack Rolls}
When a combatant makes an attack, their base chance to hit an opponent is determined by adding the defender’s Armor Class to the attacker’s Attack Bonus. Either of these may be modified by such things as spells, magical items, and high or low ability scores. The total of these is called the To-Hit Value.

If the attacker is a character, their attack bonus is based on their class and level. See \fullref{chap:Classes} for details on level based character abilities.

If the attacker is a monster, their attack bonus is based on their hit dice (see \fullref{chap:Monsters}).

To determine if an attack hits, take the to-hit value and add a roll of 1d20 to it. If the total of the value plus roll is greater than or equal to 20, then the attack hits; otherwise it misses.

Rolling a 1 on the d20 before modifiers (called a “natural 1”) is always a miss, regardless whether the total is greater than 20 or not.

Rolling a 20 on the d20 before modifiers (called a “natural 20”) is always a hit regardless whether the total is greater than 20 or not.

If the to-hit value is already greater than 20 before adding the d20 roll, the attack will do extra damage if it hits. Each two points (round odd points up) that the to-hit value exceeds 20 by means that the attack will do 1 extra point of damage.

\example{A \nth{3} level fighter has a base attack bonus of +2, and has a +3 bonus to hit from various sources. They are attacking a target that is armor class 6. Therefore, the fighter’s to-hit value is 2+3+6 = 11. If the fighter rolls 8 or less on their to-hit roll they will miss their target since 11+8 is less than the 20 that they need. If the fighter rolls 9 or higher on their to-hit roll they will hit their target since 11+9=20.}

\example{A 1 hit dice creature has an attack bonus of +1, and is attacking a target that is armor class -8. The monster has no other bonuses to hit. Therefore, the monster has a to-hit value of -7. If the monster rolls a 19 or less than it will miss its target since -7+19 is less than 20. If the monster rolls a 20 then it will hit its target since although -7+20 is also less than 20, a natural 20 always hits.}

\example{A \nth{36} level fighter has a base attack bonus of +23, and a +13 bonus to hit from various sources. When attacking a target that is armor class 1, the fighter has a to-hit value of 23+13-1 = 35. If the fighter rolls a 1 on their to-hit roll they will miss their target since a natural 1 always misses.

If the fighter rolls 2 or higher on their to-hit roll they will hit their target since 2+35>20. Since the to-hit value is more than 20 even before adding the d20 roll, the fighter will do extra damage on a hit. Specifically, since it is 15 more, the fighter will do +8 damage on a hit.}

\section{Armor Class}\index[general]{Armor Class}\label{sec:Armor Class}
A character or creature’s armor class (abbreviated to “AC”) represents how hard they are to hit in combat. A “hit” in combat does not represent a single solid blow with a weapon but instead represents one or more potentially lethal blows.

The armor class of an unarmored human character will normally be 9. That is the default value for an average person. Monsters and demi-human characters may have better (i.e. lower) armor classes than that because of their tough hides, better-than-human agility, or a combination of the two.

Armor class may be modified by such things as armor, shields, magic items, and high or low \iref[sec:Dexterity]{Dexterity} score.

\section{Saving Throws}\index[general]{Saving Throws}\label{sec:Saving Throws}
In some situations something might have a harmful effect on a creature other than direct damage (for example the petrifying gaze of a basilisk), or it might have a damaging effect that does not rely on an attack hitting the creature (for example a dragon’s fiery breath filling an area). In these cases, player characters and monsters often have a chance to partially or fully avoid the effect by rolling a saving throw on a d20.

There are six types of saving throws that between them cover nine of these situations: death, death rays and poison; magic wands; paralysis and petrification; breath weapons; rods, staffs, and spells. The difficulty—the number which needs to be equaled or exceeded on the d20 roll—is usually based on the level of the defender (or the number of hit dice in the case of monsters), although there may rarely be modifiers.

Although rods, staffs and spells are covered by the same saving throw, player characters only add their \iref[sec:Wisdom]{Wisdom} bonus (or penalty) when this saving throw is used against spells.

\section{Projectiles}\index[general]{Projectiles}
Projectiles consist of missile, hurled, and thrown weapons.

If a character is in melee with other combatants when their action occurs, they can not use a missile weapon. Thrown and hurled weapons may still be used in this situation.

If a projectile is used at short range for that weapon, the attacker has a +1 bonus to hit with the attack. If it is made at long range, the attacker has a -1 penalty to hit with the attack. \fullref{tab:Weapon Feats} for weapon ranges.

If the target of a projectile is partially or wholly hidden behind an object (e.g. a parapet or a table, or is behind an arrow slit), the attacker gets a penalty as shown on \fullref{tab:Cover}. Soft cover is cover that blocks sight of the target but will allow attacks through (such as smoke or a curtain). Hard cover is cover that will block both sight and attacks (such as a wall or an overturned table).

\begin {table}[H]
	\caption{Cover}\label{tab:Cover}
  \begin{tabularx}{\columnwidth}{>{\bfseries}cY}
	\thead{Type of Cover} & \thead{To-Hit Modifier}\\
	Soft cover up to knees & -1\\
	Soft cover up to waist & -2\\
	Looking around or through soft cover & -3\\
	Fully behind soft cover & -4\\
	Hard cover up to knees & -2\\
	Hard cover up to waist & -4\\
	Looking around or through hard cover & -6\\
	Fully behind hard cover & Can’t Attack
  \end {tabularx}
\end {table}

\section{Two Weapon Fighting}\index[general]{Two Weapon Fighting}
When a character wields a weapon in either hand, they make one extra attack with their off hand weapon in addition to however many attacks they get with their primary weapon.

If the weapon being used in the off hand does not have the “Off Hand” ability, then the attacker is treated as having one fewer weapon feat with the weapon for all purposes, and there is an additional -4 penalty to hit.

The additional off hand attack is not modified by the number of attacks gained at high level and is not affected by the \iref[spell:Haste]{Haste} or \ilink{spell:Slow}{Slow} spells.

\example{Oeric is a \nth{25} level fighter and is fighting a creature that he only needs to roll a 2 to hit, and so he normally gets three attacks per round. He is wielding a normal sword in his main hand and a dagger in his off hand. He is a grand master with both weapons.

He is also hasted.

Each round, Oeric gets 6 attacks with his sword (3 per round doubled for the \iref[spell:Haste]{Haste} spell) plus a single attack with his dagger. The sword attacks are done at grand master level, and the dagger attack is done at master level with an additional -4 penalty to hit.}

\section{Hit Points}\index[general]{Hit Points}\label{sec:Hit Points}
The ability of a character or creature to avoid potentially lethal damage is represented by their hit points. These hit points indicate a combination of skill, luck, divine favor, and sheer determination. A heroic character with many hit points will be able to keep fighting and keep dodging potentially lethal blows for a long time, whereas a character with few hit points is inexperienced and is likely to be killed rather quickly by the first or second such blow.

As characters avoid more and more potentially lethal blows, they will still pick up nicks, bruises and scrapes; and they will become more and more fatigued. Therefore, when something potentially lethal hits a character, they take Damage.

Damage reduces the number of hit points a character has left, and if a character takes enough damage they will run out of hit points and be knocked unconscious or killed.

Hit points lost to damage can be recovered by either time, the application of first aid, or magical healing.

Monsters have a number of hit dice, which shows how many d8’s should be rolled to determine their hit points. Characters get extra hit points each level, at lower levels the additional hit points are rolled on a die (of varying type depending on the character’s class) and the character’s \iref[sec:Constitution]{Constitution} bonus or penalty is added to each roll. At higher levels, characters gain a fixed number of hit points per level (again depending on their class) and no longer also add their \iref[sec:Constitution]{Constitution} bonus or penalty.

\section{Damage}\index[general]{Damage}
When an attack hits, it will usually do damage to its target, reducing the target’s hit points.

When player characters hit with attacks, the amount of damage that they do is based on their level of proficiency with the weapon that they are using (see \fullref{chap:Weapon Feats}).

When a monster attacks, the amount of damage it does with each attack will be listed in the monster’s description.

The amount of damage done by an attack may be changed by various things such magical weapon modifiers and low or high \iref[sec:Strength]{Strength} score.

\subsection{Healing Damage}\index[general]{Healing Damage}
All characters heal one hit point per day if active, or two hit points per day if resting. If the characters are adventuring over an extended period, this healing should take place each morning when the characters wake up.

Characters may also be healed by other methods such as magic items, spells, and the First Aid skill.

\section{Helpless Targets}\index[general]{Helpless Targets}
A target who is completely helpless because they are paralysed, sleeping or unconscious may be given a Coup de Grace with any edged weapon.

This will immediately knock them unconscious (if they weren’t already) and make them start dying as if they had run out of hit points, but will not actually cause them to lose any hit points.

\section{Dying and Death}\index[general]{Dying and Death}\label{sec:Dying and Death}
When a character runs out of hit points, they fall unconscious and can take no more actions.

At the end of the round in which they fell unconscious, the character must make a saving throw vs. death ray in order to stay alive.

If the saving throw fails, the character dies.

The saving throw must be repeated at the end of each subsequent round until either the character dies or they either have their wounds tended by a character who successfully uses the First Aid skill on them or are given magical healing.

However, each saving throw after the first gets a cumulative -1 penalty.

\example{Gretchen and Elfstar are fighting a giant. Unfortunately, Gretchen only has 7 hit points left, and the giant hits her for 21 damage.

Gretchen now has 0 hit points (the extra damage is ignored) and falls unconscious.

At the end of the round, Gretchen must roll a saving throw vs. death ray. She makes the roll and survives the round.

The following round, Elfstar is stuck, unable to tend to her friend because of the giant. Instead, she attacks the giant and hurts it badly.

At the end of the round, Gretchen rolls a second saving throw vs. death ray, this time at a -1 penalty. She makes this one too.

In the third round, Elfstar again attacks the giant, and manages to kill it.

At the end of this round, Gretchen makes her third saving throw with a -2 penalty this time. Again, she makes it.

The fight with the giant is now over, but since Gretchen is in danger of bleeding to death, the Game Master continues to use round-by-round timekeeping.

In the fourth round, Elfstar doesn’t want to risk trying to bandage Gretchen’s wounds in case she fails the First Aid check, Instead she casts a \iref[spell:Cure Light Wounds]{Cure Light Wounds} spell on Gretchen.

Debbie, Elfstar’s player, rolls 1d6+1 for the spell, and gets a 4. Gretchen is healed back up to 4 hit points, and does not need to make any more saving throws to avoid dying.

Now that the immediate danger is over, Gretchen and Elfstar start to bandage their wounds.}

\section{Structures in Combat}\index[general]{Structures in Combat}
Sometimes combat will not just involve creatures, but will also involve structures such as buildings and/or ships taking damage.

This may be incidental to the fight, or one or both sides in the fight may be deliberately targeting structures.

While a full siege is dealt with later in this chapter, the following rules can be used when there is a simple attack; such as a tribe of goblins using a battering ram to break down a town gate, or two ships exchanging cannon fire.

Attacking a structure is just the same as attacking a creature—the attacker rolls a to-hit roll based on their attack bonus and the structure’s armor class.

However, damage is handled differently, since structures are much tougher than creatures but don’t get fatigued.

Normal hand held weapons (including hand held missile weapons) do no damage to structures. While it’s possible to totally destroy a wooden building with an axe, it’s simply not possible to do it during the course of a few combat rounds.

Attacks from ogre sized or larger creatures, siege weapons and magic spells do affect structures.

Wooden structures lose 1 structure point for each 2 hit points of damage done by such attacks, although creatures which eat wood do full damage.

Stone structures lose 1 structure point for each 5 points of damage done by such attacks, although creatures who can burrow through rock do full damage.

\begin {table}[H]
  \caption{Simple Building/Structure Combat Ratings}
	\begin{tabularx}{\columnwidth}{>{\bfseries}M{1in}YYY}
	\thead{Type of Structure} & \thead{Armor Class vs. Missile} & \thead{Armor Class vs. Melee} & \thead{Structure Points}\\
	Simple Wooden Building & -4 & 6 & 40\\
	Simple Stone Building & -4 & 6 & 60\\
	Reinforced Wooden Stockade Wall & -4 & 6 & 300\\
	Barred Wooden Palisade Gate & -8 & 2 & 100\\
	Reinforced Stone Castle Wall & -4 & 6 & 500\\
	Reinforced Iron Door & -10 & 2 & 35\\
	Iron Portcullis & -4 & 6 & 150\\
	Wooden Ship & See \fullref{sec:Ship to Ship Combat} & See \fullref{sec:Ship to Ship Combat} & See \fullref{sec:Ship to Ship Combat}
  \end {tabularx}
\end {table}

\section{Morale}\index[general]{Morale}\label{sec:Morale}
Although players will always decide whether to stand and fight or to retreat when a fight seems to be going against them, sometimes the Game Master needs to quickly determine whether an NPC or a monster will run or fight.

In the case of monsters, each monster listing in \fullref{chap:Monsters} has a base morale score. In the case of hirelings employed by PCs, their base morale score will be based on the \iref[sec:Charisma]{Charisma} of the designated party leader. \fullref{tab:Ability Score Bonuses and Penalties} for \iref[sec:Charisma]{Charisma} bonuses and \fullref{sec:Hirelings} for more information on employing hirelings.

When a fight appears to be going against an individual or a group, the Game Master may make a morale check for them.

A morale check is made by rolling 2d6 and comparing it to the base morale score of the individual or group. If the roll is less than or equal to their base morale score then they will continue to fight, but if it is greater than their base morale score then they will either flee, surrender, or attempt to halt the fight and parley.

Characters with extremely high \iref[sec:Charisma]{Charisma} scores may provide their followers with such a high base morale score that they will never fail a morale check even with extreme situational penalties.

Morale checks should be made at the beginning of the Statement of Intent phase of combat, before the monsters or NPCs decide on their action for the round.

The exact times when a morale check is needed may vary from fight to fight, but can include such times as:

\begin{itemize}
	\item{Opponents start a fight when the group does not wish to fight.}
	\item{Opponents display vastly superior magic or fighting ability.}
	\item{Half the group is slain or incapacitated.}
	\item{Members of the group have already fled.}
	\item{The group’s leader is slain or incapacitated.}
	\item{Opponents kill a significant number of the group in a single round.}
	\item{Opponents display willingness to escalate the fight (killing in a fight that was previously non-lethal).}
	\item{Reinforcements arrive to shore up the opponents’ numbers.}
	\item{An individual is badly wounded (less than half hit points).}
	\item{Opponents make an offer to accept a surrender.}
\end{itemize}
Although there are many possible situations listed above that might require morale checks, such checks should not be overused. Creatures should not be checking morale more than two or three times in a fight at the most.

The Game Master should also bear in mind what happens after death in their campaign setting. If the existence of life after death or some other form of continued consciousness is a known fact in the setting rather than a mere matter of faith then intelligent creatures will be more likely to fight to the death than to surrender to possible maltreatment or torture. Similarly, intelligent creatures who have good reason to think that they will be raised from the dead by their employers or priests will be more inclined to fight to the death.

The above factors should be taken into account and should give situational modifiers to the morale checks made by intelligent creatures.

Other factors that may give situational modifiers to morale checks for intelligent and/or unintelligent creatures include:

\begin{itemize}
	\item{Fighting with no escape route.}
	\item{Fighting to defend one’s home or lair.}
	\item{Fighting to defend loved ones or innocents.}
	\item{The expectation that the enemy will slay incapacitated prisoners if victorious.}
	\item{The expectation that the enemy will torture prisoners if victorious.}
	\item{The expectation that the enemy will be merciful if victorious.}
	\item{The knowledge that if the combatant is incapacitated but their side wins the fight they will be healed.}
	\item{The fear of being executed (or worse) for cowardice if they run.}
	\item{A creature is fighting for reasons of desperation (e.g. extreme hunger or maddening pain).}
	\item{A previous offer to surrender has not been accepted.}
\end{itemize}

When an individual or group fails a morale check, it is up to the Game Master how they behave.

In the case of unintelligent creatures, this will almost always involve a fighting retreat. Intelligent creatures may retreat or it may try to stop the fight by either surrendering or otherwise parleying with the attacking force.

In extreme cases where intelligent creatures think that escape is likely to be impossible and that the consequences of losing the fight and surviving would be worse than death, it may even include suicide.

\section{Ship to Ship Combat}\index[general]{Ship to Ship Combat}\label{sec:Ship to Ship Combat}
When the crew of two ships wish to fight, they can do so in three ways.

Firstly, if their ship is equipped with catapults or cannons, it can keep its distance from the enemy and try to sink it or drive it away.

Secondly, if the ship has ship’s rams attached, it can try to ram the enemy ship in order to sink it.

And finally, the ship can pull up alongside the enemy and grapple it, so that the crew can cross between the ships and fight hand-to-hand.

All of this combat is done using the normal combat rules. The captain of each ship declares what action the ship will perform, and the ships act in initiative order.

\subsection{Boarding Actions}
If two ships pull alongside each other, (within 50 feet) either because one is in the process of ramming the other or because the captains wish to grapple and board, then either crew can attempt to grapple the other ship.

If both crews wish to grapple, then it is automatically successful. If only one crew wishes to grapple, then the other crew can roll 1d6; and on a 1-4, they manage to repel the grapple attempt by cutting and casting free the grappling hooks and lines.

If the grapple is successful, both ships are pulled tight together and crew can pass from one to the other in order to fight hand-to-hand.

Any character crossing between the two ships has difficulty maneuvering due to having to climb over rails and ropes, and takes a +2 penalty to armor class and a -2 penalty to all attacks during the round in which they cross.

\subsection{Damage to Ships}
Ships that are damaged lose 10\% of their speed for every 10\% of their structure points that they have lost.

Rowed ships also lose 10\% of their speed for every 10\% of their rowers that are missing.

Once a ship has lost three quarters of its structure points, it is dead in the water and can no longer sail under its own power.

When a ship has lost all of its structure points, it sinks over the course of the next 1d10 rounds.

\subsection{Repairing Ships}
Makeshift repairs can repair up to half the damage that a ship has taken while at sea, providing there are at least five crew assigned to repair duty; with one structure point being repaired per ten minutes. Multiple five-person crews can repair a ship simultaneously.

These jury rigged repairs will only last for 6d6 days before coming irreparably apart.

To permanently and fully repair a ship it must either be docked or magic must be used.

\section{Skysailing Combat}\index[general]{Skysailing Combat}
Because of the speed of skysailing, combats are rare when a ship is flying at cruising speed.

Most natural creatures can’t keep up with one, and the speeds mean that two ships won’t even be in missile range of each other for a whole round before zooming off in different directions.

However, a ship that is traveling at maneuvering speed is much more vulnerable to—and capable of—attack.

Flying ships in combat are treated just like normal ships in combat, and can grapple, board and ram each other.

Like normal ships they lose 10\% of their speed for each 10\% of their structure points that are missing, and when they have lost 75\% of their structure points they are reduced to maneuvering speed.

When a ship has lost 100\% of its structure points, it can no longer fly and will fall to the ground.

\section{Mass Combat}\index[general]{Mass Combat}\label{sec:Mass Combat}
There are times when relations between two factions have broken down to the point where war is the only answer. Such a war could be between the armies of rival dominions, or between the collected armies of rival countries. It could even be between two other humanoid races. Whoever the sides are, the result is one or more battles between armies.

The normal combat system is designed for small numbers of combatants. For large battles with hundreds or even thousands of combatants on either side, the system would be completely unworkable due to the amount of time and book-keeping involved.

Therefore, this section presents rules for battles between armies. This battle system is not designed for detailed simulation of a battle with the complexity of a war game, and lots of cardboard chits or counters (representing squads or units) being moved around the battlefield. Such simulations are slow and rely on the tactical skill of the players.

Instead, each clash between two armies is resolved in a single roll.

\subsection{Basics}
Each army has three scores associated with it. It has a Troop Rating, which measures the experience and/or innate toughness of the troops, and which varies only when the troops gain in experience; a Quality, which is based on the Troop Rating but includes modifiers for mounts and special or magical abilities which may change between battles; and a size, which is simply the number of troops in the army.

When two armies enter battle with each other, the controller of each army decides on a tactic for the army to use in this battle, and then a Battle Score is calculated for each one. This score is based on the Quality of the army, modified by factors unique to the battle, such as the effectiveness of the tactics by each side against each other, the terrain and location in which the battle is fought, and how much one side outnumbers the other.

The results of the battle are found by adding 1d100 to each Battle Score, and seeing which side gets the higher total. The amount by which one side or the other wins determines how many casualties each side takes and whether or not the losing army must retreat or even be routed.

\subsubsection{Multiple Armies}
If the two sides in a fight have multiple armies that take the field, the armies pair off and battle each other in pairs, with army with the highest Quality score selecting an opposing army to engage, then the unengaged army with the next highest Quality score, and so on until all armies are engaged.

Should one side have more armies than the other (which doesn’t necessarily mean they have more troops), the side with the fewest armies must split one or more of their armies until each side has the same number. Splitting an army in this way does not affect the Quality of the army, only the size.

Then the armies on both sides pair off as normal.

After each pair of armies has fought, one army from each pair will have been forced to leave the field of battle. If the armies remaining on the field are all on the same side, then the battle ends. Otherwise, the remaining armies re-maneuver and pair up again—again this may involve one or more armies splitting up so there are equal numbers of armies on each side.

\example{Baroness Black’s dominion is under attack from a goblin horde. The goblin horde consists of two armies: a group of 500 skirmishers, and a group of 150 wolf riders. Baroness Black’s dominion is protected by a single army of 600 foot soldiers. Deciding that the wolf riders are a bigger threat, Baroness Black splits her army into a 250 strong force and a 350 strong force.

The army with the highest Quality score is the wolf riders, and they attack the 250 troops. Baroness Black’s armies share the next highest Quality score (they both have the Quality of her original army, just smaller sizes), so in theory her 350 troop army chooses next.

There is only one other unengaged army to choose, so the 350 troops face off against the 500 goblin skirmishers.

After the Battle Scores are added up for each battle, it turns out that the 250 troops force the 150 wolf riders to retreat, and only take 10\% casualties while doing so. There are now 225 of them left.

The 500 goblin skirmishers force the 350 troop army to retreat, but take 40\% casualties while doing so. There are now only 300 skirmishers left.

Since there are still armies belonging to both sides of the conflict on the field, the battle continues; with the 225 humans fighting against the 300 goblin skirmishers.

The goblin leader, not fancying these odds much, raises a flag of truce in order to try to negotiate.}

\subsection{Troop Rating}
The troop rating of an army is based on the amount of training, experience, and general toughness that an army has. The troop rating of an army may range from “Untrained” to “Elite”. See \fullref{tab:Army Quality} for the list of possible troop ratings.

\subsubsection{Humans and Demi-Humans}
For human and demi-human troops, the initial troop rating of a force gathered from peasant militia will be either “Poor” if comprised of 10\% of the peasants in an area or “Untrained” if comprised of 20\% of the peasants.

If mercenaries or other professional soldiers are hired, the initial troop rating will be “Below Average”.

For each year that the army spends active without disbanding, it gains a level of troop rating, to a maximum rating of “Average”, which is the highest rating available to troops that have not seen combat experience.

After the army has won (not merely fought) its first battle, the troop rating immediately increases by one level, and can now (by further years of training) reach “Elite”.

Any time an army is routed as the result of a battle, its troop rating immediately drops by one level.

\subsubsection{Humanoids and Other Monsters}
Because humanoids and other monsters do not normally gain in experience and levels, their troop rating is simply based on their hit dice as indicated on \fullref{tab:Troop Rating}.

\begin {table}[H]
	\caption{Troop Rating}\label{tab:Troop Rating}
  \begin{tabularx}{\columnwidth}{>{\bfseries}YY}
	\thead{Hit Dice} & \thead{Troop Rating}\\
	Less than 1 & Untrained\\
	1 & Poor\\
	1+-2 & Below Average\\
	2+-3 & Fair\\
	3+-5 & Average\\
	5+-7 & Good\\
	7+-9 & Excellent\\
	9+ & Elite
  \end {tabularx}
\end {table}

\subsubsection{Mixing Troops}
There are three cases when troops of two different troop ratings will be combined to form a single army. Firstly, two smaller armies may be being combined into a single army. Secondly, new recruits may be joining an experienced army to replace combat losses. Thirdly, new recruits may be joining an experience army to simply increase its size. Note that in this latter case, it may or may not be strategically better to keep the recruits separate and maintain two smaller armies with differing ratings than to maintain a single larger army.

In each of these cases, the combined army starts at the troop rating of its best troops, and loses one rating per 20\% of the combined army that has come from the less good troops. This reduction cannot reduce the army to a lower troop rating than the less good troops were before the merge.

Additionally, if the troops are human or demi-human and 50\% or more of the combined army has not yet won a battle, then the whole army is considered to no longer have combat experience and cannot rise above “Average” until it wins a battle.

\example{Baroness Black has an army of 400 mercenaries that have defended her dominion for the past four years, although they have not seen a battle in that time. They were “Below Average” when she hired them, so after four years together they should have become “Excellent”.

However, their lack of actual battle experience limits them to “Average”.

When she received news that there was a goblin horde on its way to attack her dominion, she decided to bolster her army by recruiting another 100 mercenaries. As with her initial army, these new recruits start with a troop rating of “Below Average”.

Her combined army now consists of 500 troops, 80\% of which were “Average” and 20\% of which were “Below Average”.

Since 20\% of her combined army is from the troops of lesser quality, her army is now considered to be one level less than the quality of her best troops—i.e. “Fair”.

As more than 50\% of her combined army has not yet won a battle (actually, none of her army has), the entire army is considered to be lacking in real combat experience and is limited to a maximum rating of “Average”.}

\subsection{Quality}
To determine the quality of such an army, consult \fullref{tab:Army Quality}.

Each troop rating has a base quality and a bonus. To determine the quality of the army, start with the base quality, and for each of the following statements that is true, add the bonus:

\begin{itemize}
	\item{At least 20\% of the army is mounted.}
	\item{At least 50\% of the army is mounted.}
	\item{At least 1\% of the army can fly.}
	\item{At least 20\% of the army can fly.}
	\item{The average movement rate of the army is at least 35 feet per round.}
	\item{At least 20\% of the army can have missile weapons.}
	\item{At least 20\% of the army have missile weapons with a range of at least 100 feet.}
	\item{At least 1\% of the army have magical abilities (breath weapons, Energy Drain, poison, Regeneration, etc.)}
	\item{At least 20\% of the army have magical abilities (breath weapons, Energy Drain, poison, Regeneration, etc.)}
	\item{100\% of the army have magical abilities (breath weapons, Energy Drain, poison, Regeneration, etc.)}
	\item{At least 5\% of the army are spellcasters.}
	\item{At least 30\% of the army are spellcasters.}
\end{itemize}

\example{Lucy the Game Master is making notes for the following evening’s game. She knows that the goblin tribes that Baroness Black has been encouraging adventurers to raid are forming a horde to come and attack her.

The horde will consist of two armies; an army of 500 skirmishers and an army of 150 wolf riders.

Looking at the wolf riders, Lucy sees that the army will actually be a combination of 70 wolves and 70 goblins. The goblins are less than 1 hit die each, so rank as “Untrained” troops. The wolves are 2+ hit dice each so rank as “Fair”. Therefore, the army as a whole will be two ranks below “Fair”, i.e. “Poor”. Lucy combines the troop types like this because the wolves are combatant in their own right.

Had the goblins been riding ponies or horses, then they would simply have counted as 150 “Untrained” goblins.

This gives the army a basic quality of 28, with a +3 bonus for each statement from the list is true.

They are all mounted, so they get two +3 adds for that. Their movement rate is at least 35 feet per round, so they get another +3 add. They are the only adds that apply, so the total quality of the wolf rider army is: 28 + (3 x 3 ) = 37}

\begin {table}[H]
	\caption{Army Quality}\label{tab:Army Quality}
  \begin{tabularx}{\columnwidth}{>{\bfseries}YYY}
	\thead{Troop Rating} & \thead{Base Quality} & \thead{Bonus}\\
	Untrained & 10 & 1\\
	Poor & 28 & 3\\
	Below Average & 45 & 5\\
	Fair & 63 & 7\\
	Average & 75 & 8\\
	Good & 90 & 9\\
	Excellent & 112 & 12\\
	Elite & 140 & 14
  \end {tabularx}
\end {table}

\subsection{Resolving a Battle}
Once both sides of the battle have ensured that they each have the same number of armies and have sorted out which armies will be engaging one another, the battles between individual armies can commence.

Firstly, determine if either side is In Defense. A side is considered to be in defense if it occupies the battlefield before the other army arrives and waits for the other army to come to it. If both armies meet each other together, then neither is considered to be in defense.

Resolving each battle has three steps: Deciding on tactics, calculating Battle Score, and then rolling for the battle itself.

\subsubsection{Tactics}
The commander of each side must decide on the tactic that their armies will use in the day’s battles. Note that this is a single decision made for all the armies together, not a decision made on an army by army basis. All armies on one side work together to achieve the goals of the tactic decided by the commander.

The six tactics are:

\textbf{Attack:} This is the most basic tactic. The armies simply move forward and engage the enemy. Although effective against withdrawing or holding armies, armies using this tactic can find themselves suffering if the enemy tries to envelop them or lure them into a trap. However, overall it is still one of the less risky tactics.

\textbf{Envelop:} The armies try to surround the enemy and attack from all sides. It is particularly effective against enemies who are holding position, but leaves the armies vulnerable to direct attacks and attempts to overrun.

\textbf{Hold:} The armies try to hold position, letting the enemy come to them and engaging them when they do. This tactic is a good defense against a normal attack or an attempt to lure the armies into a trap, but it is easily enveloped and can be counterproductive when the enemy tries to overrun the holding armies.

\textbf{Overrun:} This is an all out attack, charging the enemies and attempting to punch through the front ranks to attack the more vulnerable troops. It is great for punching a hole through troops that are trying to envelop you or for running down troops that are trying to withdraw, but it can be a costly tactic in terms of casualties.

\textbf{Trap:} The armies try to lure the enemy into making costly lunges and flanking maneuvers, before attacking those troops committed to such maneuvers. This tactic is effective against attacking troops and can be devastating against troops that are trying to overrun you; but is very weak against armies that are reluctant to engage directly, such as those trying to hold position or withdraw from combat completely.

\textbf{Withdraw:} This is simply an attempt to leave the battlefield with as little fighting as possible. It is a risky maneuver against enemies which are attacking directly or trying to overrun you, but can often avoid a fight completely if the enemy is holding position.

The player and Game Master (or the two players) should each write down the tactic that their armies are using, before revealing them simultaneously.

\fullref{tab:Tactics} shows the effect that each tactic has on the battle, based on the tactic that it is facing. Each side should consult this table separately. The effect will either be a modification in the number of casualties taken, a modification to the army’s Battle Score, no combat taking place, or no effect.

\example{Baroness Black thinks that goblins are likely to attack directly, so decides that the best tactic is to try to lead them into traps and ambushes. Sure enough, the Game Master decides that the goblins are going to throw subtlety to the wind and try to overrun Baroness Black’s troops.

	Marcie checks \autoref*{tab:Tactics}, and sees that using the trap tactic against an overrun will give each of Baroness Black’s armies a +20 to their Battle Score.

The Game Master also checks \autoref*{tab:Tactics}, and sees that using the overrun tactic into a trap will give each of the goblin armies +20\% casualties.}

\end{multicols*}
\begin {table}[H]
	\caption{Tactics}\label{tab:Tactics}
  \begin{tabularx}{\columnwidth}{>{\bfseries}YYYYYYY}
	\thead{} & \multicolumn{6}{c}{\thead{Enemy Tactic}}\\
	\thead{Tactic Chosen} & \thead{Attack} & \thead{Envelop} & \thead{Hold} & \thead{Overrun} & \thead{Trap} & \thead{Withdraw}\\
	Attack & +10\% Cas & +10\% Cas & - & -20 BS & +10\% Cas & +10 BS\\
	Envelop & -10 BS & - & +20 BS & +10\% Cas & -10\% Cas & +10 BS\\
	Hold & -10\% Cas & +20\% Cas & No Combat & -25 BS & -10\% Cas & No Combat\\
	Overrun & +20\% Cas & +10 BS & +20\% Cas & +20\% Cas & +20\% Cas & +20 BS\\
	Trap & +10 BS & -20 BS & -20 BS & +20 BS & - & -10\% Cas\\
	Withdraw & +20\% Cas & -10\% Cas & No Combat & +30\% Cas & -10\% Cas & No Combat
  \end {tabularx}
\end {table}
\begin{multicols*}{2}

\subsubsection{Battle Score}
The basic battle score of an army is equal to its Quality. This basic score is increased by a fixed amount for each of the following statements that is true:

\begin{itemize}
	\item{+15 if the army outnumbers its opponent by at least 1.5 to 1 but less than 2 to 1.}
	\item{+30 if the army outnumbers its opponent by at least 2 to 1 but less than 3 to 1.}
	\item{+45 if the army outnumbers its opponent by at least 3 to 1 but less than 4 to 1.}
	\item{+60 if the army outnumbers its opponent by at least 4 to 1 but less than 5 to 1.}
	\item{+70 if the army outnumbers its opponent by at least 5 to 1 but less than 6 to 1.}
	\item{+80 if the army outnumbers its opponent by at least 6 to 1 but less than 7 to 1.}
	\item{+90 if the army outnumbers its opponent by at least 7 to 1 but less than 8 to 1.}
	\item{+100 if the army outnumbers its opponent by at least 8 to 1 but less than 11 to 1.}
	\item{+110 if the army outnumbers its opponent by at least 11 to 1 but less than 16 to 1.}
	\item{+120 if the army outnumbers its opponent by at least 16 to 1 but less than 21 to 1.}
	\item{+130 if the army outnumbers its opponent by at least 21 to 1 but less than 31 to 1.}
	\item{+140 if the army outnumbers its opponent by at least 31 to 1 but less than 41 to 1.}
	\item{+150 if the army outnumbers its opponent by at least 41 to 1 but less than 51 to 1.}
	\item{+160 if the army outnumbers its opponent by at least 51 to 1.}
	\item{+10 if the army is in the dominion of their liege.}
	\item{+10 if the army have beaten this enemy before.}
	\item{+10 if the troop class of the army is at least two levels higher than that of their enemy.}
	\item{+30 if ambushing an enemy while the enemy is marching.}
	\item{-10 if any allied force has routed.}
	\item{+20 if the battle is at night and the entire army has \iref[sec:Infravision]{Infravision}.}
	\item{+20 if attacking from higher ground.}
	\item{+20 for a \iref[class:Halfling]{Halfling} army in fields or woods.}
	\item{+10 for an \iref[class:Elf]{Elf} army in woods or forest.}
	\item{+10 for a \iref[class:Dwarf]{Dwarf} or \iref[class:Gnome]{Gnome} army in hills or mountains.}
	\item{-20 for mounted troops in mountains, woods, or at a stronghold.}
	\item{-20 for combat in swamp unless at least half the army can fly.}
	\item{-10 for combat in snow or sand unless at least half the army can fly.}
	\item{+10 if the army is In Defense.}
	\item{+50 if In Defense of a bridge, narrow pass, or gorge.}
	\item{+40 if In Defense and the attacker must cross deep water.}
	\item{+20 if In Defense of mountains, hills, or a town.}
	\item{+50 if In Defense of a stronghold.}
	\item{+30 if the army has more (by value) siege weaponry than its enemy does.}
	\item{+50 if at least 1\% of the army is immune to the enemy’s attacks.}
	\item{+50 if the whole army is immune to at least 80\% of the enemy’s attacks.}
	\item{+50 if the whole army is immune to all of the enemy’s attacks.}
	\item{-10 if the army has medium fatigue.}
	\item{-30 if the army has severe fatigue.}
\end{itemize}

\subsubsection{Rolling for the Battle}
Once both armies who are engaging each other have had their final battle scores calculated, the controller of each one rolls 1d100 and adds it to their army’s battle score. Whichever side gets the highest total wins the battle, and the other side loses the battle.

The Game Master subtracts the total of the loser from the total of the winner, and checks on \fullref{tab:Battle Results} to see what the effect is on each army.

\autoref*{tab:Battle Results} has three columns for each army, detailing casualties, location and fatigue. The results for each army work in the same manner.

\end{multicols*}
\begin {table}[H]
  \caption{Battle Results}\label{tab:Battle Results}
  \begin{tabularx}{\columnwidth}{>{\bfseries}YYYYYYY}
	\thead{} & \multicolumn{3}{c}{\thead{Winner}} & \multicolumn{3}{c}{\thead{Loser}}\\
	\thead{Difference in Battle Scores} & \thead{Casualties} & \thead{Location} & \thead{Fatigue} & \thead{Casualties} & \thead{Location} & \thead{Fatigue}\\
	1-8 & 0\% & Hold & None & 10\% & Retreat 1 & None\\
	9-15 & 0\% & Hold & None & 20\% & Retreat 1 & None\\
	16-24 & 10\% & Hold & None & 20\% & Retreat 1 & Medium\\
	25-30 & 10\% & Hold & None & 30\% & Retreat 2 & Medium\\
	31-38 & 20\% & Retreat 1 & Medium & 40\% & Retreat 1 & Severe\\
	39-50* & 0\% & Hold & None & 30\% & Retreat 3 & Severe\\
	51-63 & 20\% & Advance 1 & Medium & 50\% & Retreat 4 & Severe\\
	64-80 & 30\% & Advance 1 & Medium & 60\% & Retreat 5 & Severe\\
	81-90 & 10\% & Advance 3 & None & 50\% & Retreat 4 & Severe\\
	91-100 & 0\% & Advance 3 & None & 30\% & Rout & Rout\\
	101-120 & 10\% & Advance 3 & None & 70\% & Rout & Rout\\
	121-150 & 10\% & Advance 5 & None & 70\% & Rout & Rout\\
	151+ & 10\% & Advance 5 & None & 100\% & Rout & Rout
  \end {tabularx}
	* Maximum possible result if the winner was using the “Hold” tactic.
\end {table}
\begin{multicols*}{2}

\textbf{Casualties:} The size of the army is reduced by the given percentage. Remember that this percentage may be modified up or down by the tactics that the army used.

It is not possible for an army to take more than 100\% casualties.

\textbf{Location:} After the battle is finished, this shows the location of the army. It may hold the battlefield, have been forced to retreat one or more miles, or have advanced one or more miles in pursuit of the enemy. Note that advancing in this manner is compulsory in order to inflict the casualties on the enemy army. Additionally, any army that was using the “Withdraw” tactic may move an extra mile (but only away from the enemy, not towards it).

\textbf{Fatigue:} This shows how fatigued the battle makes the army. Fatigue affects the battle score of the army in future battles that occur on the same day. Fatigue is removed from an army by spending a single day neither moving nor fighting.

\textbf{Rout:} An army that is routed ceases to exist as a unit. Whichever troops survive the fight will scatter, and slowly return to their homes, arriving 1d10 weeks after the battle.

\subsection{Aftermath}
When all the pairs of armies involved in the battle have finished fighting, the result will be that some will have held the field and some will have moved away—either in retreat or in pursuit of retreating armies.

If all the armies that still hold the field are on the same side, the battle is finished for the day. If armies of both sides still hold the field, the battle continues, with the armies first splitting if necessary to ensure that there are equal numbers on each side once more.

\subsection{Strongholds in Battle}
When a stronghold is under attack, it has the following effect on the battle:

\begin{itemize}
	\item{When calculating troop ratios to see who gets a bonus to their Battle Score, treat the defending armies as having four times as many troops as they actually have.}
	\item{The defender only takes half the indicated casualties.}
	\item{The defender ignores “Retreat” or “Rout” results.}
\end{itemize}
The defender only gets these bonuses if they use the “Hold” tactic.

If an attacker chooses to besiege the stronghold instead of attacking, the attacker gains a +5 cumulative bonus to their Battle Score per week of siege, and if the defenders run out of food they will take 10\% casualties per week of siege. Remember that high level clerics can create enough food to feed many people.

The besieged defenders can, of course, attack the sieging army at any time using any tactic except “Hold”. If they choose to do so, they gain a +20 bonus to their Battle Score for the element of surprise.

\end{multicols*}

