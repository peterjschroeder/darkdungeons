\chapter[purple]{Treasure}
\label{chap:Treasure}
\chapterimage[Treasure (Bordered)]
\thispagestyle{plain}

\begin{multicols*}{2}
While some adventurers go out and do heroic deeds purely through a sense of altruism and honor, others do it for the money.

However, even the most pure and noble adventurer will smile when they discover that the rampaging dragon that they have just killed happened to have a huge hoard of treasure which is now theirs for the taking (or re-distributing to the poor if they’re that way inclined).

Dragons are not the only creatures that have treasure. Most sapient creatures know the value of gold and silver and use it to trade in whatever economies they are part of; and many non-sapient creatures will collect shiny baubles and things that capture their interest.

Even completely mindless creatures may incidentally end up with treasure—as the inedible stuff their victims were carrying accumulates in their lairs.

\section{Treasure Types}\index[general]{Treasure Types}\label{sec:Treasure Types}
Treasure in Dark Dungeons generally comes in three categories-coins and gems; jewelry; and magic items.

However, not all creatures will possess all those types of treasure in equal measure. For example a dragon’s hoard will be very different to the contents of a goblin’s belt pouch.

Dark Dungeons handles this by having twenty-two treasure types labeled from A to V. The first fifteen of these (A-O) usually represent large treasure caches found in the lairs of creatures. The other seven (P-V) represent small amounts of treasure carried by individual creatures.

The monster descriptions in \fullref{chap:Monsters} show the treasure type that each type of creature may have either individually or in its lair.

When checking for a group of monsters, lair treasure should be checked once for the whole group but individual treasure should be checked for each monster separately.

In the case of lair treasure, the treasure may not all be in one place. Some or all of it (especially magic items) may be distributed amongst the creatures rather than simply in a vault or spoil heap.

Once you know the treasure type that you are checking, check the actual contents of the treasure by looking it up on \fullref{tab:Treasure Types}. The table shows the percentage chance of each type of coin being present, along with the number of coins of that type that there will be if they are present.

Similarly, the table also has columns for gems, jewelery, special treasures (which are miscellaneous items of value), and magic items.

For the larger lair treasures, the table also shows the average monetary value gained from the treasure type (excluding magic items) as a guideline for if you are in a hurry and want to simply assign a value rather than roll for each type of coin and item separately.

Remember that the average value is an indication of the likely results from the rest of the table, and is not supposed to be given as well as rolling on the rest of the table.

\end{multicols*}
\begin {table}[H]
  \caption{Treasure Types}\label{tab:Treasure Types}
	\begin{tabularx}{\columnwidth}{>{\bfseries}M{.4in}M{.48in}M{.48in}M{.48in}M{.48in}M{.48in}M{.2in}M{.4in}M{.4in}YM{.45in}}
	\thead{Treasure Type} & \thead{Copper Pieces} & \thead{Silver Pieces} & \thead{Electrum Pieces} & \thead{Gold Pieces} & \thead{Platinum Pieces} & \thead{Gems} & \thead{Jewelery} & \thead{Special Treasure} & \thead{Magic Items} & \thead{Average Value}\\
	A & 25\% (1d6x1,000) & 30\% (1d6x1,000) & 20\% (1d4x1,000) & 35\% (2d6x1,000) & 25\% (1d2x1,000) & 50\% (6d6) & 50\% (6d6) & 10\% (1d2) & 30\% (any 3) & 17,000 gp\\
	B & 50\% (1d8x1,000) & 25\% (1d6x1,000) & 25\% (1d4x1,000) & 35\% (1d3x1,000) & - & 25\% (1d6) & 25\% (1d6) & - & 10\% (1 sword, misc weapon, or armor) & 2,000 gp\\
	C & 20\% (1d12x1,000) & 30\% (1d4x1,000) & 10\% (1d4x1,000) & - & - & 25\% (1d4) & 25\% (1d4) & 5\% (1d2) & 10\% (any 2) & 750 gp\\
	D & 10\% (1d8x1,000) & 15\% (1d12x1,000) & - & 60\% (1d6x1,000) & - & 30\% (1d8) & 30\% (1d8) & 10\% (1d2) & 15\% (any 2, 1 potion) & 4,000 gp\\
	E & 5\% (1d10x1,000) & 30\% (1d12x1,000) & 25\% (1d4x1,000) & 25\% (1d8x1,000) & - & 10\% (1d10) & 10\% (1d10) & 15\% (1d2) & 25\% (any 3, 1 scroll) & 2,500 gp\\
	F & - & 10\% (2d10x1,000) & 20\% (1d8x1,000) & 45\% (1d12x1,000) & 30\% (1d3x1,000) & 20\% (2d12) & 10\% (1d12) & 20\% (1d3) & 30\% (1 potion, 1 scroll, 3 any but weapons) & 7,600 gp\\
	G & - & - & - & 50\% (1d4x10,000) & 50\% (1d6x1,000) & 25\% (3d6) & 25\% (1d10) & 30\% (1d3) & 35\% (any 4, 1 scroll) & 25,000 gp\\
	H & 25\% (3d8x1,000) & 50\% (1d100x1,000) & 50\% (1d4x10,000) & 50\% (1d6x10,000) & 25\% (5d4x1,000) & 50\% (1d100) & 50\% (1d4x10) & 10\% (1d2) & 15\% (any 4, 1 potion, 1 scroll) & 60,000 gp\\
	I & - & - & - & - & 30\% (1d8x1,000) & 50\% (2d6) & 50\% (2d6) & 5\% (1d2) & 15\% (any 1) & 7,500 gp\\
	J & 25\% (1d4x1,000) & 10\% (1d3x1,000) & - & - & - & - & - & - & - & 25 gp\\
	K & - & 30\% (1d6x1,000) & 10\% (1d2x1,000) & - & - & - & - & - & - & 250 gp\\
	L & - & - & - & - & - & 50\% (1d4) & - & - & - & 225 gp\\
	M & - & - & - & 40\% (2d4x1,000) & 50\% (3d10x1,000) & 55\% (5d4) & 45\% (2d6) & - & - & 50,000 gp\\
	N & - & - & - & - & - & - & - & 10\% (1d2) & 40\% (2d4 potions) & -\\
	O & - & - & - & - & - & - & - & 10\% (1d3) & 50\% (1d4 scrolls) & -\\
	P & 100\% (3d8) & - & - & - & - & - & - & - & - & -\\
	Q & - & 100\% (3d6) & - & - & - & - & - & - & - & -\\
	R & - & - & 100\% (2d6) & - & - & - & - & - & - & -\\
	S & - & - & - & 100\% (2d4) & - & 5\% (1) & - & - & - & -\\
	T & - & - & - & - & 100\% (1d6) & 5\% (1) & - & - & - & -\\
	U & 10\% (1d100) & 10\% (1d100) & - & 5\% (1d100) & - & 5\% (1d2) & 5\% (1d4) & 2\% (1) & 2\% (any 1) & -\\
	V & - & 10\% (1d100) & 5\% (1d100) & 10\% (1d100) & 5\% (1d100) & 10\% (1d2) & 10\% (1d2) & 5\% (1) & 5\% (any 1) & -
  \end {tabularx}
\end {table}
\begin{multicols*}{2}

\section{Items of Value}\index[general]{Items of Value}
The treasure table gives three types of valuable item other than coins: gems, jewelry and special items. When the treasure table indicates that one or more of these items is present, check each item in turn to find out its value.

If there are many items of the same basic type, they may be checked in batches or small groups to save time.

\subsection{Gems}\index[general]{Gems}
The value of each gem can simply be determined by rolling on \fullref{tab:Gems}. The table also gives examples of the types of gem that are likely to be worth the given value.

If the monster whose treasure is being determined has less than 9 hit dice, subtract 10 from the roll, treating results less than 01 as if they were 01.

In terms of encumbrance, each gem is roughly equivalent to a coin; and therefore it is extremely unlikely that gems will be significant when it comes to calculating a characters encumbrance and movement rate.

\begin {table}[H]
  \caption{Gems}\label{tab:Gems}
  \begin{tabularx}{\columnwidth}{>{\bfseries}ccY}
	\thead{d100*} & \thead{Value} & \thead{Examples}\\
	01-03 & 10 gp & Agate, quartz, turquoise\\
	04-10 & 50 gp & Jasper, onyx\\
	11-25 & 100 gp & Amber, amethyst, garnet, jade\\
	26-46 & 500 gp & Aquamarine, pearl, topaz\\
	47-71 & 1,000 gp & Carbuncle, opal\\
	72-90 & 5,000 gp & Emerald, ruby, sapphire\\
	91-97 & 10,000 gp & Diamond, jacinth\\
	98-00 & x2 & Flawless or unusually cut gem, roll again for type\
  \end {tabularx}
	* If the monster whose treasure this is has less than 9 hit dice, subtract 10 from the roll.
\end {table}

\subsection{Jewelry}\index[general]{Jewelry}
To determine the value of a piece of jewelry, roll on \fullref{tab:Jewelery Value}. If the monster whose treasure is being determined has less than 9 hit dice, subtract 10 from the roll, treating results less than 01 as if they were 01.

Once the value of the piece of jewelry has been determined, roll on \fullref{tab:Jewelery Form} to determine the exact form of the piece.

In terms of encumbrance, jewelry will vary tremendously. A good rule of thumb is that common jewelry weighs 10 cn, uncommon jewelry weighs 25 cn and rare jewelry weighs 50 cn.

\begin {table}[H]
  \caption{Jewelery Value}\label{tab:Jewelery Value}
  \begin{tabularx}{\columnwidth}{>{\bfseries}YYY}
	\thead{d100*} & \thead{Value} & \thead{Jewelery Type}\\
	01 & 100 gp & Common\\
	02-03 & 500 gp & Common\\
	04-06 & 1,000 gp & Common\\
	07-10 & 1,500 gp & Common\\
	11-16 & 2,000 gp & Common\\
	17-24 & 2,500 gp & Common\\
	25-34 & 3,000 gp & Common\\
	35-45 & 4,000 gp & Uncommon\\
	46-58 & 5,000 gp & Uncommon\\
	59-69 & 7,500 gp & Uncommon\\
	70-78 & 10,000 gp & Uncommon\\
	79-85 & 15,000 gp & Rare\\
	86-90 & 20,000 gp & Rare\\
	91-94 & 25,000 gp & Rare\\
	95-97 & 30,000 gp & Rare\\
	98-99 & 40,000 gp & Rare\\
	00 & 50,000 gp & Rare\
  \end {tabularx}
	* If the monster whose treasure this is has less than 9 hit dice, subtract 10 from the roll.
\end {table}

\begin {table}[H]
  \caption{Jewelery Form}\label{tab:Jewelery Form}
  \begin{tabularx}{\columnwidth}{>{\bfseries}YYYY}
	\thead{1d10} & \thead{Common} & \thead{Uncommon} & \thead{Rare}\\
	1 & Anklet & Armband & Amulet\\
	2 & Beads & Belt & Crown\\
	3 & Bracelet & Collar & Diadem\\
	4 & Brooch & Earring & Medallion\\
	5 & Buckle & Heart & Orb\\
	6 & Cameo & Leaf & Ring\\
	7 & Chain & Necklace & Scarab\\
	8 & Clasp & Pendant & Scepter\\
	9 & Locket & Rabbit’s Foot & Talisman\\
	10 & Pin & Torc & Tiara
  \end {tabularx}
\end {table}

\subsection{Special Items}\index[general]{Special Items}
Special items are other potentially valuable goods in the possession of the monsters.

They may be works of art, rare books, trade goods such as spices or silks, expensive perfumes and incenses, furs, or almost anything else.

Since these items can be so varied, it is not possible to create an exhaustive list or table of possibilities.

Simply roll 1d100 x 10 to find out the value (in gold pieces) of each special item, and then decide what that value represents.

Encumbrance will vary wildly depending on the item. A 500 gp statue may be anywhere from 6 inches to 6 feet tall. Similarly, 1,000 gp of trade goods might be a single sack of rare spices or might be a whole cart full of fine clothing.

\section{Magic Items}\label{sec:Magic Items}
Not all treasure has purely monetary value. Many items have potent magical powers that will aid an adventurer.

When the treasure table indicates that one or more magical items are present, it may also indicate the type of magical item or items. If the type of a magic item is not predetermined, roll on \fullref{tab:Magic Item Type} to find out what type of magic item each one is. Unlike gems and jewelry, magic items should be rolled for individually rather than in groups.

Unlike the gem and jewelry tables, there are no specific guidelines in the magic item tables for high or low level creatures. This means that in theory a bunch of goblins are as likely to have come across a +5 Sword of Slicing as an angel is. The Game Master has the option to reject a rolled item if they think it is too out-of-place or too powerful for the party of adventurers.

However, being too stingy and rejecting any kind of useful item can be very frustrating for the players. Remember, for example, that when it comes down to it, the difference between a +1 sword and a +3 sword is fairly insignificant compared to even a single increase in weapon expertise level.

\begin {table}[H]
  \caption{Magic Item Type}\label{tab:Magic Item Type}
  \begin{tabularx}{\columnwidth}{>{\bfseries}YY}
	\thead{d100} & \thead{Type}\\
	01-25 & Potion\\
	26-37 & Scroll or Map\\
	38-46 & Wand, Staff or Rod\\
	47-52 & Ring\\
	53-62 & Wondrous Item\\
	63-72 & Armor or Shield\\
	73-83 & Missile Weapon or Armor\\
	84-92 & Sword\\
	93-00 & Miscellaneous Weapon
  \end {tabularx}
\end {table}

\subsection{Restricted Items}
Many magical items can be used by any character. However, some may only be used by characters of certain classes. These are marked by abbreviations as follows:

\begin{itemize}
 \item{Clerics only (‘C’)}
 \item{Druids only (‘Dr’)}
 \item{Elves only (‘E’)}
 \item{Wizards only (‘W’)}
 \item{Non-spellcasters only (‘N’)}
 \item{Spellcasters only (‘S’)}
 \item{One type of spellcaster only (‘S*’)}
\end{itemize}

These indicators may be combined; for example an item marked with (C, Dr) can be used by clerics or druids.

\subsection{Potions}\index[general]{Potions}
A potion is a magical liquid, usually contained in a vial, that must be drunk for its magical effect to take place. Normally, this involves drinking the entire potion—meaning that each potion can only be used once.

Unlike most magic items, potions do not need to be identified with an Analyze spell. A potion can be identified by taking a small sip—although this does run the risk of it being poison.

Drinking a potion in combat or feeding it to an unconscious person requires a Use Non-Activatable Item action.

Unless otherwise indicated, the magical effect of a potion lasts for 1 hour + 1d6 x 10 minutes, and (if applicable) will be as if cast by a \nth{6} level spellcaster. The drinker of the potion will have no advance warning of exactly when it will run out.

If someone drinks a potion while an existing potion is in effect, they will become sick and unable to take any actions for 30 minutes (no save), and neither potion will have its effect.

Potions which allow the drinker to control others require the drinker to be able to see all controlled creatures and concentrate for the duration. Controlled creatures cannot be made to kill themselves. Targeted creatures may make a saving throw vs. spells in order to avoid being controlled, but the drinker can simply try the control again the following round.

\begin {table}[H]
  \caption{Potions}
  \begin{tabularx}{\columnwidth}{>{\bfseries}YY}
	\thead{d100} & \thead{Potion}\\
	01-02 & Agility\\
	03 & Animal Control\\
	04-06 & Antidote\\
	07-08 & Blending\\
	09-10 & Bug Repellent\\
	11-12 & Clairaudience\\
	13-14 & Clairvoyance\\
	15-16 & Climbing\\
	17-18 & Defense\\
	19-22 & Delusion\\
	23-24 & Diminution\\
	25 & Dragon Control\\
	26-27 & Dreamspeech\\
	28 & Elasticity\\
	29-30 & Elemental Form\\
	31-32 & ESP\\
	33 & Ethereality\\
	34-36 & Fire Resistance\\
	37-39 & Flying\\
	40-41 & Fortitude\\
	42 & Freedom\\
	43-45 & Gaseous Form\\
	46 & Giant Control\\
	47-49 & Giant Strength\\
	50-51 & Growth\\
	52-57 & Healing\\
	58-60 & Heroism\\
	61 & Human Control\\
	62-64 & Invisibility\\
	65-66 & Invulnerbility\\
	67-68 & Levitation\\
	69-70 & Longevity\\
	71 & Luck\\
	72 & Merging\\
	73-74 & Plant Control\\
	75-77 & Poison\\
	78-80 & Polymorph Self\\
	81-82 & Sight\\
	83-84 & Speech\\
	85-88 & Speed\\
	89-90 & Strength\\
	91-93 & Super Healing\\
	94-96 & Swimming\\
	97 & Treasure Finding\\
	98 & Undead Control\\
	99-00 & Water Breathing
  \end {tabularx}
\end {table}

\textbf{Agility:}\index[magicitems]{Potion of Agility} The drinker’s \iref[sec:Dexterity]{Dexterity} score becomes 18 for the duration of the potion.

\textbf{Animal Control:}\index[magicitems]{Potion of Animal Control} The drinker may control one or more animals with a total hit dice of 3d6 or less. When the effect of the potion wears off, the animals will flee.

\textbf{Antidote:}\index[magicitems]{Potion of Antidote} The drinker becomes immune to all poisons of up to a certain strength for the duration of the potion.

The strength of the potion should be determined randomly as indicate on \fullref{tab:Antidote}.

\begin {table}[H]
	\caption{Antidote}\label{tab:Antidote}
  \begin{tabularx}{\columnwidth}{>{\bfseries}YY}
	\thead{1d10} & \thead{Strength}\\
	1-4 & Poison from 3 HD creatures\\
	5-7 & Poison from 7 HD creatures\\
	8-9 & Poison from 15 HD creatures\\
	10 & All poison
  \end {tabularx}
\end {table}

Poison from non-creature sources (e.g. a Potion of Poison) should be considered to be as if from a 7 HD creature.

\textbf{Blending:}\index[magicitems]{Potion of Blending} The drinker may change color to blend in with their surroundings. When hiding, the drinker will have a 90\% chance to go unnoticed unless the viewer can see invisible creatures.

\textbf{Bug Repellent:}\index[magicitems]{Potion of Bug Repellent} Any normal or giant bug (an insect, spider, scorpion, centipede, or other arthropod) will completely ignore the drinker unless magically controlled.

If the bugs are magically controlled to attack the drinker, the drinker gets a +4 bonus to any saving throws against the controlling effect that allow the damage done by the insects to be reduced.

\textbf{Clairaudience:}\index[magicitems]{Potion of Clairaudience} While concentrating, the drinker may listen as if at any point within 60 feet of their current location.

\textbf{Clairvoyance:}\index[magicitems]{Potion of Clairvoyance} While concentrating, the drinker may see as if at any point within 60 feet of their current location.

\textbf{Climbing:}\index[magicitems]{Potion of Climbing} The drinker may walk on walls and ceilings as if a spider.

\textbf{Defense:}\index[magicitems]{Potion of Defense} This potion only lasts 10 minutes. The drinker gets a bonus to their armor class based on the power of the potion. The power of the potion is determined randomly as indicated on \fullref{tab:Potion of Defense}.

\begin {table}[H]
  \caption{Potion of Defense}\label{tab:Potion of Defense}
  \begin{tabularx}{\columnwidth}{>{\bfseries}YY}
	\thead{1d10} & \thead{Power}\\
	1-3 & +1\\
	4-5 & +2\\
	6-7 & +3\\
	8-9 & +4\\
	10 & +5
  \end {tabularx}
\end {table}

\textbf{Delusion:}\index[magicitems]{Potion of Delusion} This potion will have no effect when drunk. However, if tasted it will falsely give the taster the impression that it is another type of potion from this list. Multiple tasters will all get the same impression. An Analyze spell will correctly identify this potion as one of Delusion rather than as the type it tastes like.

\textbf{Diminution:}\index[magicitems]{Potion of Diminution} The drinker shrinks down to 6 inches in height. While in this state they cannot hurt creatures larger than 1 foot tall with physical attacks, and have a 90\% chance of being able to hide. If this potion is drunk while a Potion of Growth is in effect, they will simply cancel each other rather than making the drinker sick.

\textbf{Dragon Control:}\index[magicitems]{Potion of Dragon Control} The drinker of this potion can control up to three dragons of sub-adult or younger. The dragons do not get saving throws, but older dragons are not affected.

Each potion of dragon control will only control one type of dragon, which is determined randomly as indicated on \fullref{tab:Potion of Dragon Control}.

\begin {table}[H]
  \caption{Potion of Dragon Control}\label{tab:Potion of Dragon Control}
  \begin{tabularx}{\columnwidth}{>{\bfseries}YY}
	\thead{1d10} & \thead{Dragon}\\
	1-2 & Black\\
	3-4 & Blue\\
	5-6 & Green\\
	7-8 & Red\\
	9-10 & White
  \end {tabularx}
\end {table}

The controlled dragons will do anything that is commanded (other than suicidal commands) although if dragon queens, they are not capable of casting spells.

When the control ends, the dragons will become hostile to the drinker and will either flee or attack depending on their impression of the drinker’s power.

\textbf{Dreamspeech:}\index[magicitems]{Potion of Dreamspeech} This potion only lasts for 10 minutes. The drinker can speak to one paralyzed, petrified or sleeping creature within 30 feet (only one creature can be spoken with per potion) and can hear the responses of the creature via ESP. The drinker will automatically understand the language of the target, but the target is under no compulsion to speak the truth (or even respond at all) if they do not wish to.

\textbf{Elasticity:}\index[magicitems]{Potion of Elasticity} This potion only lasts for 10 minutes. The drinker may stretch and deform themselves and their equipment to fit through gaps as small as an inch and reach as far as 30 feet. The drinker cannot cast spells or attack while stretched, and items carried cannot be used or dropped unless they are in normal form. However, while stretched the drinker only takes half damage from blunt and bashing attacks.

\textbf{Elemental Form:}\index[magicitems]{Potion of Elemental Form} This potion only lasts for 10 minutes. For the duration of the potion, the drinker may transform to an elemental of a certain type and back. Each transformation takes 1 round.

The drinker’s hit points do not change while in elemental form, but their other abilities become the same as an elemental with the same number of hit dice as the drinker’s level.

The type of elemental that the potion allows the drinker to change to is determined randomly as indicated on \fullref{tab:Potion of Elemental Form}.

\begin {table}[H]
  \caption{Potion of Elemental Form}\label{tab:Potion of Elemental Form}
  \begin{tabularx}{\columnwidth}{>{\bfseries}YY}
	\thead{1d4} & \thead{Element}\\
	1 & Air\\
	2 & Earth\\
	3 & Fire\\
	4 & Water
  \end {tabularx}
\end {table}

\textbf{ESP:}\index[magicitems]{Potion of ESP} This potion has the same effect as the spell \iref[spell:ESP]{ESP}.

\textbf{Ethereality:}\index[magicitems]{Potion of Ethereality} This potion gives the drinker the ability to shift from the \ilink{sec:Prime Plane}{Prime Plane} to the \iref[sec:Ethereal Plane]{Ethereal Plane}. The drinker may shift at any time before the potion’s duration runs out, and may then spend up to 24 hours on the \iref[sec:Ethereal Plane]{Ethereal Plane} before shifting back.

Once the drinker has shifted to the \iref[sec:Ethereal Plane]{Ethereal Plane} and back, the potion’s duration immediately expires.

\textbf{Fire Resistance:}\index[magicitems]{Potion of Fire Resistance} The drinker becomes immune to normal fire, gains a +2 bonus on all saving throws against fire or heat based attacks, and takes -1 point of damage per die (to a minimum of 1 point per die) from magical fire attacks.

\textbf{Flying:}\index[magicitems]{Potion of Flying} This potion has the same effect as the spell \iref[spell:Fly]{Fly}.

\textbf{Fortitude:}\index[magicitems]{Potion of Fortitude} The drinker’s \iref[sec:Constitution]{Constitution} score becomes 18 for the duration of the potion, possibly giving them extra hit points. When the drinker takes damage, it comes off these extra hit points first. Damage already taken before the extra hit points were applied remains unless cured.

\textbf{Freedom:}\index[magicitems]{Potion of Freedom} The drinker is immune to paralysis and to all forms of Hold spell or effect.

\textbf{Gaseous Form:}\index[magicitems]{Potion of Gaseous Form} This potion has the same effect as the spell \iref[spell:Gaseous Form]{Gaseous Form}.

\textbf{Giant Control:}\index[magicitems]{Potion of Giant Control} The user may control up to four giants of a certain type, but each one gets a saving throw. The giants will normally be hostile once control ends. The type of giant affected is determined randomly as indicated on \fullref{tab:Potion of Giant Control}.

\begin {table}[H]
  \caption{Potion of Giant Control}\label{tab:Potion of Giant Control}
  \begin{tabularx}{\columnwidth}{>{\bfseries}YY}
	\thead{1d6} & \thead{Giant}\\
	1 & Cloud\\
	2 & Fire\\
	3 & Frost\\
	4 & Hill\\
	5 & Stone\\
	6 & Storm
  \end {tabularx}
\end {table}

\textbf{Giant Strength:}\index[magicitems]{Potion of Giant Strength} The drinker gains the \iref[sec:Strength]{Strength} of a giant, and can do double damage with melee weapons or throw rocks (range: 60/130/200) for 3d6 damage. This potion does not stack with other \iref[sec:Strength]{Strength} enhancing items such as Gauntlets of Ogre Power.

\textbf{Growth:}\index[magicitems]{Potion of Growth} The drinker grows to twice normal size, and be able to do double damage with melee attacks. If this potion is drunk while a Potion of Diminution is in effect, they will simply cancel each other rather than making the drinker sick.

\textbf{Healing:}\index[magicitems]{Potion of Healing}\label{mitem:Potion of Healing} This potion will either cure 1d6+1 hit points of damage to the drinker or cure their paralysis. This potion has an instant effect, rather than a duration.

\textbf{Heroism:}\index[magicitems]{Potion of Heroism} This potion will only affect commoners, fighters, dwarfs, gnomes or halflings. The drinker will temporarily gain one or more levels as indicated on \fullref{tab:Potion of Heroism}.

\begin {table}[H]
  \caption{Potion of Heroism}\label{tab:Potion of Heroism}
  \begin{tabularx}{\columnwidth}{>{\bfseries}YY}
		\thead{Level} & \thead{Levels Gained}\\
	Commoner & 4 (fighter)\\
	1-3 & 3\\
	4-7 & 2\\
	8-10 & 1\\
	11+ & None
  \end {tabularx}
\end {table}

All damage (including \iref[sec:Energy Drain]{Energy Drains}) is taken from the extra levels and hit points first.

\textbf{Human Control:}\index[magicitems]{Potion of Human Control} The drinker may control up to 6 levels (commoners count as half a level each) of humans, similar to a Charm Person spell. The targets can only be controlled while within 60 feet of the drinker, and the effect lasts only for the duration of the potion.

\textbf{Invisibility:}\index[magicitems]{Potion of Invisibility} This potion has the same effect as the spell \iref[spell:Invisibility]{Invisibility}.

\textbf{Invulnerability:}\index[magicitems]{Potion of Invulnerability} The drinker gains a +2 bonus to armor class and all saving throws for the duration of the potion. If a second potion of invulnerability is drunk within a week, the only effect is sickness.

\textbf{Levitation:}\index[magicitems]{Potion of Levitation} This potion has the same effect as the spell \iref[spell:Levitate]{Levitation}.

\textbf{Longevity:}\index[magicitems]{Potion of Longevity} The drinker immediately becomes 10 years younger. The effect is an instant one rather than an ongoing one, and therefore cannot be dispelled. This potion will not reduce the age of the drinker below 15 years old.

\textbf{Luck:}\index[magicitems]{Potion of Luck} This potion lasts only for 1 hour. The potion makes the drinker extremely lucky.

The player of the drinking character may choose any one roll that they make on behalf of the drinking character within the duration and simply place the dice on the result of their choice rather than having to actually roll them.

\textbf{Merging:}\index[magicitems]{Potion of Merging} This potion allows the drinker to merge other creatures into their own body. The drinker can merge up to seven other creatures, and both the drinker and the other creatures must be willing. Creatures simply step “into” the drinker and disappear along with their items and equipment. While merged with the drinker, the creatures do not take damage if the drinker is hit, and they can not take any actions other than speaking. The merged creatures can step “out of” the drinker at any time.

\textbf{Plant Control:}\index[magicitems]{Potion of Plant Control} The drinker may control all mundane plants and all plant-like creatures in a 30-by-30-foot area up to 60 feet away. Mundane plants may entangle creatures in their area, but cannot otherwise attack.

\textbf{Poison:}\index[magicitems]{Potion of Poison} This potion looks like any other, but it is poisonous. Anyone taking even a taste of the potion must save vs. poison or die.

\textbf{Polymorph Self:}\index[magicitems]{Potion of Polymorph Self} This potion has the same effect as the spell \iref[spell:Polymorph Self]{Polymorph Self}.

\textbf{Sight:}\index[magicitems]{Potion of Sight} This potion lasts for 10 minutes. While this potion is in effect, the drinker can temporarily see despite any blindness conditions. Additionally, the drinker may see invisible things.

\textbf{Speech:}\index[magicitems]{Potion of Speech} The drinker can understand all languages that they hear, and can respond in those same languages. This potion does not give the drinker the power to speak a language unless they first hear it spoken.

\textbf{Speed:}\index[magicitems]{Potion of Speed} This potion has the same effect as a Haste spell.

\textbf{Strength:}\index[magicitems]{Potion of Strength} The drinker’s \iref[sec:Strength]{Strength} score becomes 18 for the duration of the potion.

\textbf{Super Healing:}\index[magicitems]{Potion of Super Healing} This potion will either cure 3d6+3 hit points of damage to the drinker.

This potion has an instant effect, rather than a duration.

\textbf{Swimming:}\index[magicitems]{Potion of Swimming} This potion lasts for 8 hours. The drinker floats on top of water or other liquid, and can’t be weighed down unless by over 3,000 cn of weight. Additionally, the drinker can swim at a speed of 60 feet per round.

\textbf{Treasure Finding:}\index[magicitems]{Potion of Treasure Finding} When the drinker concentrates, they can detect the distance and direction to the largest amount of treasure within 360 feet. The drinker gains no insight about the nature of the treasure or how to get to it.

\textbf{Undead Control:}\index[magicitems]{Potion of Undead Control} The drinker can control up to 18 HD in total of undead creatures of 9 HD or less. The undead will be hostile when the duration ends.

\textbf{Water Breathing:}\index[magicitems]{Potion of Water Breathing} This potion lasts for four hours, and has the same effect as the spell \iref[spell:Water Breathing]{Water Breathing}.

\subsection{Scrolls and Maps}\index[general]{Scrolls and Maps}
A scroll is a piece of parchment or paper with magical writings on it. The scroll is used by unrolling it and reading aloud the writing. If used in combat, this takes an Activate Magic Item action, and requires there to be enough light to read by.

Scrolls and maps normally do not need to be identified with an Analyze spell. They can be identified simply by reading them. However, while a scroll containing spells can be identified as such by reading, the actual spells themselves may need a Read Magic spell to identify them.

This section also contains maps that may be found in treasure. These are not magical, and are not usually used in combat.

Some scrolls can be used by anyone, but others can only be used by certain types of character. Scrolls marked with an (S) can only be used by spellcasters.

\begin {table}[H]
  \caption{Scrolls}
  \begin{tabularx}{\columnwidth}{>{\bfseries}YY}
	\thead{d100} & \thead{Scrolls and Maps}\\
	01-03 & Communication\\
	04-05 & Creation\\
	06-13 & Curse\\
	14 & Delay (S*)\\
	15-17 & Equipment\\
	18-19 & Illumination\\
	20-21 & Mage’s (S)\\
	22-25 & Map to Treasure (Normal Treasure)\\
	26-28 & Map to Treasure (Magical Treasure)\\
	29-30 & Map to Treasure (Combined Treasure)\\
	31 & Map to Treasure (Special Treasure)\\
	32-34 & Mapping\\
	35-36 & Portals\\
	37-42 & Protection from Elementals\\
	43-50 & Protection from Lycanthropes\\
	51-54 & Protection from Magic\\
	55-61 & Protection from Undead\\
	62-63 & Questioning\\
	64 & Repetition (S*)\\
	65-66 & Seeing\\
	67-68 & Shelter\\
	69-71 & Spell Catching\\
	72-96 & Spells (S*)\\
	97-98 & Trapping\\
	99-00 & Truth
  \end {tabularx}
\end {table}

\textbf{Communication:}\index[magicitems]{Scrolls of Communication} This is a pair of matching scrolls. Any message (of up to 100 words) that is written on one scroll will also appear on the other, providing they are on the same plane as each other.

This scroll can be re-used, but each time the writing is erased and replaced by a different message there is a 5\% chance that the magic will stop working.

\textbf{Creation:}\index[magicitems]{Scroll of Creation} Once per day, the owner of this scroll can draw an item on it, and then pick up the drawing off the paper and it will become a real item that lasts for 24 hours. The item can be up to 10 by 5 by 1 foot in size, and cannot weigh more than 5,000 cn or be worth more than 500 gp. Magical or living items may not be created.

\textbf{Curse:}\index[magicitems]{Scroll of Curse} Anyone who reads this scroll, even only glancing to check what is on it, is immediately cursed with no saving throw allowed. For typical curses, check the reversed version of the Remove Curse spell.

The curse lasts until removed by a Remove Curse spell.

\textbf{Delay (S*):}\index[magicitems]{Scroll of Delay} This is a Spells Scroll containing a single spell. However, when the spell is cast from the scroll, the caster may choose to delay the effects of the spell anywhere from 0 to 12 rounds.

If the caster still has the scroll when the spell goes off, the caster chooses whatever parameters are needed by the spell (e.g. targets) at the time it goes off. If the caster has let go of the scroll when the spell goes off, it goes off affecting the location of the scroll itself or the nearest valid targets (within the normal spell range).

\textbf{Equipment:}\index[magicitems]{Scroll of Equipment} This scroll has the names of six mundane items written on it. When any of the names is read aloud, the name vanishes and the item appears. The item will remain for 24 hours, and then disappear, with the name reappearing on the scroll at the same time.

No more than three items can be brought forth per day.

\textbf{Illumination:}\index[magicitems]{Scroll of Illumination} If the writing on this scroll is read out and the scroll is rolled up tightly, it will burst into flame and act as a torch. It will burn for a total of 6 hours per day, and the torch flame will not harm the scroll—although it will set light to other things.

If the scroll is unrolled, the torch flame will immediately go out. Otherwise, no amount of wind or rain will put it out, although it will go out if immersed in water.

\textbf{Mage’s (S):}\index[magicitems]{Mage’s (S)} A Mage’s scroll may only be used by a spellcaster. Once per day, its user may command it to identify a magical effect. The name of the effect and the caster level will appear on the scroll. If the magical effect is a non-standard one, the Game Master should invent an informative name for it; but should not give a full description of exactly what the effect does unless it is extremely straightforward.

\textbf{Map to Treasure:}\index[magicitems]{Map to Treasure} These non-magical scrolls are simply treasure maps indicating the location of some kind of treasure.

\textbf{Mapping:}\index[magicitems]{Scroll of Mapping} Once per day, this scroll may be called upon to draw its surrounding area.

The scroll will reproduce an accurate map of everything within a 100-foot radius. Each secret door has a 1-in-6 chance of being drawn, although the presence of some secret doors may be inferred by what is drawn behind them.

\textbf{Portals:}\index[magicitems]{Scroll of Portals} Twice per day, this scroll can be placed on a wall and commanded to create a Passwall effect like the spell \iref[spell:Passwall]{Passwall}. The scroll will then disappear, and the Passwall tunnel will last for 30 minutes, before the tunnel closes and the scroll re-appears.

\textbf{Protection from Elementals:}\index[magicitems]{Scroll of Protection from Elementals} This scroll can only be used once. When read aloud, it produces a 10-foot radius zone of protection. Elementals cannot enter the zone, but can use missile and spell attacks against those inside the zone. The zone lasts for 20 minutes, or until someone inside the zone attacks an elemental in hand-to-hand combat.

\textbf{Protection from Lycanthropes:}\index[magicitems]{Scroll of Protection from Lycanthropes} This scroll can only be used once. When read aloud, it produces a 10-foot radius zone of protection. Lycanthropes cannot enter the zone, but can use missile and spell attacks against those inside the zone. The zone lasts for 60 minutes, or until someone inside the zone attacks a lycanthrope in hand-to-hand combat.

\textbf{Protection from Magic:}\index[magicitems]{Scroll of Protection from Magic} This scroll can only be used once. When read aloud, it produces a 10-foot radius zone of protection. Mortal level magic (whether from spells or items) cannot enter or leave the zone. The zone lasts for 60 minutes, and can only be broken by a Wish spell.

\textbf{Protection from Undead:}\index[magicitems]{Scroll of Protection from Undead} This scroll can only be used once. When read aloud, it produces a 10-foot radius zone of protection. Undead cannot enter the zone, but can use missile and spell attacks against those inside the zone. The zone lasts for 60 minutes, or until someone inside the zone attacks an undead in hand-to-hand combat.

\textbf{Questioning:}\index[magicitems]{Scroll of Questioning} This scroll enables the user to ask questions of inanimate objects and receive answers. The objects will answer as if they were living beings with normal human senses. The user can ask three questions per day.

This scroll cannot be used to question living beings or magical objects.

\textbf{Repetition (S*):}\index[magicitems]{Scroll of Repetition} This scroll appears to be a Spells Scroll containing a single spell. However, ten minutes after the spell is cast from the scroll, it will cast itself a second time, centered on the scroll or affecting the nearest valid target (within normal spell range).

\textbf{Seeing:}\index[magicitems]{Scroll of Seeing} Once per day, this scroll can be commanded to draw the creatures that are within 100 feet of the user within a single direction. The scroll will draw up to four types of creature, starting with the largest and working its way smaller.

\textbf{Shelter:}\index[magicitems]{Scroll of Shelter} This scroll contains a drawing of a 10-foot square room with two beds, a table and two chairs. The table is shown laden with food, and there are swords and shields hung on the wall.

Once per day, the scroll can be placed against a vertical surface it will grow to life-size, and the room can be entered. Anyone walking into the room disappears and a drawn version of them appears on the scroll.

To people in the scroll, the items in the room are all mundane and usable, although none of them can be removed from the room.

The scroll will stay in place for 12 hours, or until it is taken down by someone.

If the scroll is taken down, anyone still in the room is trapped inside it. The air and food replenish themselves, and the room is a comfortable—although boring—place to live for an extended period.

While the scroll is not in place, the only way out of the room is via a Wish spell.

\textbf{Spell Catching:}\index[magicitems]{Scroll of Spell Catching} This scroll can be used to catch spells cast at its user. It can only cast actual spells cast by spellcasters or from scrolls. It can’t catch the spell-like effects produced by items.

The exact level of spell that the scroll can catch is determined randomly as indicated on \fullref{tab:Scroll of Spell Catching}.

\begin {table}[H]
	\caption{Scroll of Spell Catching}\label{tab:Scroll of Spell Catching}
  \begin{tabularx}{\columnwidth}{>{\bfseries}YY}
	\thead{1d10} & \thead{Spell Level}\\
	1-4 & \nth{1}-\nth{2}\\
	5-7 & \nth{1}-\nth{4}\\
	8-9 & \nth{1}-\nth{6}\\
	10 & \nth{1}-\nth{8}
  \end {tabularx}
\end {table}

The user of the scroll must make a saving throw vs. spells with a +4 bonus when a spell is cast at them in order to catch it in the scroll. If this saving throw is successful, the spell does not have its normal effect, but is caught on the scroll instead.

The scroll can only contain one spell at a time, and once a spell is in the scroll it works just like a normal Spells Scroll until the spell has been cast from it.

Anyone can use the scroll to catch a spell, but casting the caught spell from the scroll has the normal restrictions that a Spells Scroll containing the same spell would have.

\textbf{Spells (S*):}\index[magicitems]{Scroll of Spells} A spell scroll will contain one or more spells. Use \autoref*{tab:Spell Scroll Type} to \autoref*{tab:Spell Scroll Wizard/Elf Spell Level} to determine the exact contents of the scroll.

Firstly find out which type of spell is on the scroll, then how many spells, then for each spell roll a random level and determine randomly which spell of that level is on the scroll.

A spell scroll can be identified as such by simple reading, but the type and identity of the spells on it can only be recognized by casters of the correct type. Additionally, wizards and elves will be able to recognize that the spells on a scroll are wizard/elf spells but will not be able to read those spells without using a Read Magic spell.

Each spell on the scroll can only be used once, and they may only be used by characters of a class that can cast the spell (elves and wizard share the same type of scrolls and clerics and druids share the same type of scrolls with the exception that clerics cannot cast druid-only spells).

If the character is high enough level to cast the spell, the spell will automatically work, but if the character is not high enough level to cast the spell then there is a 10\% chance that the spell will misfire. Offensive spells that misfire will go off centered on the caster and other spells will simply fizzle and be wasted.

\iref[class:Rogue]{Rogues} of \nth{10} level an above may decipher Spell Scrolls containing wizard/elf spells, but is always considered to be of insufficient level to cast the spell and therefore always has the 10\% chance of the spell misfiring.

Spells cast from scrolls are always treated as having a caster of the minimum level needed to cast the spell, not as having a caster of the reader’s level.

When each spell is cast from the scroll it will fade and disappear, leaving part of the scroll blank.

\begin {table}[H]
  \caption{Spell Scroll Type}\label{tab:Spell Scroll Type}
  \begin{tabularx}{\columnwidth}{>{\bfseries}YY}
	\thead{d100} & \thead{Type of Spells}\\
	01-70 & Wizard/Elf\\
	71-95 & Cleric\\
	96-00 & Druid
  \end {tabularx}
\end {table}

\begin {table}[H]
  \caption{Spell Scroll Number}
  \begin{tabularx}{\columnwidth}{>{\bfseries}YY}
	\thead{d100} & \thead{Number of Spells}\\
	01-50 & 1\\
	51-83 & 2\\
	84-00 & 3
  \end {tabularx}
\end {table}

\begin {table}[H]
  \caption{Spell Scroll Cleric/Druid Spell Level}
  \begin{tabularx}{\columnwidth}{>{\bfseries}YY}
	\thead{d100} & \thead{Spell Level}\\
	01-34 & 1\\
	35-58 & 2\\
	59-76 & 3\\
	77-88 & 4\\
	89-95 & 5\\
	96-99 & 6\\
	00 & 7
  \end {tabularx}
\end {table}

\begin {table}[H]
  \caption{Spell Scroll Wizard/Elf Spell Level}\label{tab:Spell Scroll Wizard/Elf Spell Level}
  \begin{tabularx}{\columnwidth}{>{\bfseries}YY}
	\thead{d100} & \thead{Spell Level}\\
	01-28 & 1\\
	29-49 & 2\\
	50-64 & 3\\
	65-75 & 4\\
	76-84 & 5\\
	85-91 & 6\\
	92-95 & 7\\
	96-99 & 8\\
	00 & 9
  \end {tabularx}
\end {table}

\textbf{Trapping:}\index[magicitems]{Scroll of Trapping} This scroll can only be used once. It is placed against a smooth hard surface, and the writing on it is read out.

If it was placed on a floor, it will disappear and be replaced by a 10-by-10-foot covered pit 20 feet deep with poison spikes.

If it was placed on a ceiling, it will disappear and be replaced by a poisoned blade held ready to scythe down.

If it was placed on a wall, it will be replaced by a dart trap that fires poisoned darts.

In each case, the trap is non-magical once created, and can be detected and disarmed with normal chances.

If any of the traps are set off, they will do 2d6 damage to their victim, plus if the victim fails a saving throw vs. poison then the victim will be killed.

\textbf{Truth:}\index[magicitems]{Scroll of Truth} Once per day, the user of this scroll can ask a question of any sapient being within 30 feet.

The complete and true answer to the question (or at least what the subject thinks is the complete and true answer to the question) will be read from the subject’s mind via ESP and will appear on the scroll.

An unwilling subject may make a saving throw vs. spells to avoid having the answer drawn from them, in which case the scroll will remain blank, but may not give a false answer.

\subsection{Wands, Staves and Rods}\index[general]{Wands, Staves, and Rods}
Wands, staves and rods are magical devices that contain spell-like enchantments.

Wands are normally thin sticks around 18 inches long. They can only be used by wizards or elves. A wand will hold 2d10 charges, and each time it is used one charge will be used up. Once all the charges are used up, a wand is simply a non-magical stick.

Rods are larger than wands, often 2-3 feet long, and usually made of metal with some kind of protrusion on the end like a scepter. Most rods can be used by anyone. Unlike wands and staves, most rods do not use charges.

Staves are the bigger cousins of wands. They are normally 5-6 feet long, and made of wood. Staves tend to be more versatile and powerful than wands, although they still use charges. A staff will have 3d10 charges when found, and each use of the staff may use one of more of these charges. Staves are often restricted in terms of who can use them.

As with wands, when a staff runs out of charges it is just a stick. This applies even if the staff formally had powers that did not use charges (such as acting as a magical weapon).

A staff can be used in melee as if a quarterstaff.

Activating a wand, staff or rod requires the speaking of a command word and the use of an Activate Magic Item action. The functions of a wand, staff or rod, and the command words to activate those functions, can be discovered by an Analyze spell.

Unless otherwise noted, for purposes of dispelling treat the effects produced by all wands, staffs and rods to be as if made by a \nth{6} level caster. Saving throws against spells and effects created by wands, staves and rods are always made vs. wands unless the description of the item says otherwise.

\begin {table}[H]
  \caption{Wands, Staves and Rods}
  \begin{tabularx}{\columnwidth}{>{\bfseries}YYY}
	\thead{d100} & \thead{Wand}, \thead{Staff or Rod}\\
	01-05 & Wand of Cold (W)\\
	06-10 & Wand of Enemy Detection (W)\\
	11-14 & Wand of Fear (W)\\
	15-19 & Wand of Fireballs (W)\\
	20-23 & Wand of Illusion (W)\\
	24-28 & Wand of Lightning Bolts (W)\\
	29-33 & Wand of Magic Detection (W)\\
	34-38 & Wand of Metal Detection (W)\\
	39-42 & Wand of Negation (W)\\
	43-47 & Wand of Paralyzation (W)\\
	48-52 & Wand of Polymorphing (W)\\
	53-56 & Wand of Secret Door Detection (W)\\
	57-60 & Wand of Trap Detection (W)\\
	61 & Staff of Commanding (S)\\
	62-63 & Staff of Dispelling\\
	64-66 & Staff of the Druids (Dr)\\
	67-69 & Staff of an Element (W)\\
	70-71 & Staff of Harming (C)\\
	72-78 & Staff of Healing (C)\\
	79 & Staff of Power (W)\\
	80-82 & Snake Staff (C, Dr)\\
	83-85 & Staff of Striking (S)\\
	86-87 & Staff of Withering (C)\\
	88 & Staff of Wizardry (W)\\
	89-90 & Rod of Cancellation\\
	91 & Rod of Dominion\\
	92 & Rod of Health (C)\\
	93-94 & Rod of Inertia (N)\\
	95 & Rod of Parrying\\
	96 & Rod of Victory\\
	97-99 & Rod of Weaponry (N)\\
	00 & Rod of the Wyrm
  \end {tabularx}
\end {table}

\textbf{Wand of Cold (W):}\index[magicitems]{Wand of Cold} Each time a charge is expended, this wand creates a cone of cold, 60 feet long and 30 feet wide at the end. All creatures within the cone must take 6d6 damage.

If they can make a saving throw vs. wands, they take only half damage.

\textbf{Wand of Enemy Detection (W):}\index[magicitems]{Wand of Enemy Detection} Each time a charge is expended, all creatures with hostile intent towards the user within 60 feet will glow as if on fire. This includes invisible or hidden enemies; in which case the glow may give away their positions. The glow lasts for 10 minutes before fading.

\textbf{Wand of Fear (W):}\index[magicitems]{Wand of Fear} Each time a charge is expended, this wand creates a cone of fear, 60 feet long and 30 feet wide at the end. All creatures within the cone must make a saving throw vs. wands or flee in terror for 5 minutes.

\textbf{Wand of Fireballs (W):}\index[magicitems]{Wand of Fireballs} Each time a charge is expended, this wand creates a \iref[spell:Fireball]{Fireball} up to 240 feet away. Anyone in the area of effect takes 6d6 damage. If they can make a saving throw vs. wands they take only half damage.

\textbf{Wand of Illusion (W):}\index[magicitems]{Wand of Illusion} Each time a charge is expended, this wand creates a \iref[spell:Phantasmal Force]{Phantasmal Force} as the spell \iref[spell:Phantasmal Force]{Phantasmal Force}. The caster must concentrate to maintain the illusion.

\textbf{Wand of Lightning Bolts (W):}\index[magicitems]{Wand of Lightning Bolts} Each time a charge is expended, this wand creates a \iref[spell:Lightning Bolt]{Lightning Bolt} up to 240 feet away and then extending 60 feet from that point. Anyone in the path of the lightning takes 6d6 damage. If they can make a saving throw vs. wands they take only half damage.

\textbf{Wand of Magic Detection (W):}\index[magicitems]{Wand of Magic Detection} Each time a charge is expended, all magical items and active spells within a 20-foot radius will glow for 6 rounds (1 minute).

\textbf{Wand of Metal Detection (W):}\index[magicitems]{Wand of Metal Detection} Each time a charge is expended, the wand will point towards a type of metal named by the user if there is at least 1,000 cn of it within 20 feet.

\textbf{Wand of Negation (W):}\index[magicitems]{Wand of Negation} Each time a charge is expended, this wand will cancel the effects of another wand or staff. If the effect that is being negated is an effect with a duration, it will be negated for a single round.

\textbf{Wand of Paralyzation (W):}\index[magicitems]{Wand of Paralyzation} Each time a charge is expended, this wand creates a cone 60 feet long and 30 feet wide at the end. All creatures within the cone must make a saving throw vs. wands or be paralyzed for 1 hour.

\textbf{Wand of Polymorphing (W):}\index[magicitems]{Wand of Polymorphing} Each time a charge is expended, this wand produces the effect of either a Polymorph Self on the user of the wand or a Polymorph Other on a target that the wand is pointed at. Unwilling targets may make a saving throw vs. wands to avoid the effect.

\textbf{Wand of Secret Door Detection (W):}\index[magicitems]{Wand of Secret Door Detection} When this wand is activated, it will point the user towards all secret doors within 20 feet. This expends one charge per secret door revealed.

\textbf{Wand of Trap Detection (W):}\index[magicitems]{Wand of Trap Detection} When this wand is activated, it will point the user towards all traps within 20 feet. This expends one charge per trap revealed.

\textbf{Staff of Commanding (S):}\index[magicitems]{Staff of Commanding} This staff can be used by any spellcaster. When the staff is activated and a charge is expended, it allows the user to act as if they had drunk a Potion of Animal Control, Potion of Human Control, or Potion of Plant Control. However, unlike the potions, this gives the user only a single attempt to establish control per use.

\textbf{Staff of Dispelling:}\index[magicitems]{Staff of Dispelling} This staff is usable by any character. When the staff is used and a charge is expended, it will cast Dispel Magic as if by a \nth{15} level caster, except with a range of only touch.

The staff can be used to dispel magic items. When it is used against an item and a charge is expended it will automatically destroy any potion or scroll; and will cause any other magic item to temporarily cease working for 1d4 rounds. Note that magic weapons and armor still work as mundane weapons and armor even when their magical properties are temporarily suppressed.

\textbf{Staff of the Druids (Dr):}\index[magicitems]{Staff of the Druids} This staff may only be used by druids. If a druid activates the staff and expends a charge while preparing their spells for the day, they will be able to prepare an extra spell of each level that they can cast. Additionally, this staff counts as a +3 weapon when wielded by a druid.

\textbf{Staff of an Element (W):}\index[magicitems]{Staff of an Element} This staff may only be used by magic users. The type of staff is determined randomly as indicated on \fullref{tab:Staff of an Element Type}.

\begin {table}[H]
  \caption{Staff of an Element Type}\label{tab:Staff of an Element Type}
  \begin{tabularx}{\columnwidth}{>{\bfseries}YY}
	\thead{1d100} & \thead{Type}\\
	01-21 & Staff of Air\\
	22-42 & Staff of Earth\\
	43-63 & Staff of Fire\\
	64-84 & Staff of Water\\
	85-91 & Staff of Air and Water\\
	92-98 & Staff of Earth and Fire\\
	99-00 & Staff of All Elements
  \end {tabularx}
\end {table}

When the wielder of the staff is on the \ilink{sec:Prime Plane}{Prime Plane}, the staff confers the following powers to its wielder:

\begin{itemize}
 \item{+4 to saving throws against attacks based on the staff’s element.}
 \item{Immunity to attacks by elementals of the staff’s element.}
 \item{One charge can be expended per day to summon an 8 hit dice elemental of the staff’s element.}
 \item{One charge can be expended to cast a spell as if a \nth{10} level spellcaster as indicated on \fullref{tab:Staff of an Element Spells}.}
\end{itemize}

\begin {table}[H]
  \caption{Staff of an Element Spells}\label{tab:Staff of an Element Spells}
  \begin{tabularx}{\columnwidth}{>{\bfseries}YY}
	\thead{Type} & \thead{Spells}\\
	Air & Lightning Bolt, Cloudkill\\
	Earth & Web, Wall of Stone\\
	Fire & Fireball, Wall of Fire\\
	Water & Ice Storm/Wall of Ice
  \end {tabularx}
\end {table}

When the wielder of the staff is on the Elemental Plane corresponding to the staff’s element, the staff instead has the following powers:

\begin{itemize}
 \item{Prevents environmental damage from the plane such as burning or choking on dust (but does not provide air for the wielder to breathe).}
 \item{Allows the wielder to fly and swim at a speed of 40 feet per round.}
 \item{Allows the caster to speak and understand the languages used by natives of the plane.}
 \item{Gives the caster a +4 bonus to armor class against natives of the plane.}
\end{itemize}

Additionally, a summoned elemental of a type not matching the staff’s element can be dismissed back to its home plane (with no saving throw) at the cost of one charge if it was summoned by a magical device such as a similar staff or two charges if it was summoned by a spell.

The staff also counts as a +2 staff.

Finally, if the staff is taken to an elemental plane other than that of the staff’s element, it will immediately explode doing 20 points of lightning damage plus 1d8 damage per charge remaining to all creatures within a 60-foot radius. The wielder of the staff gets no saving throw, but other creatures in the area can make a saving throw vs. spells with a -4 penalty to take half damage.

\textbf{Staff of Harming (C):}\index[magicitems]{Staff of Harming} This staff may only be used by clerics. Whenever a creature is struck by the staff, the wielder may expend one charge to inflict an additional 1d6+1 damage to the creature in addition to the normal damage that the staff does. This does not require the normal Activate Magic Item action, but is done as part of the Attack action.

The wielder of the staff can also activate it as normal to use any of the effects listed on \fullref{tab:Staff of Harming} as if casting the reverse of the curing spells (using up a variable number of charges).

\begin {table}[H]
  \caption{Staff of Harming}\label{tab:Staff of Harming}
  \begin{tabularx}{\columnwidth}{>{\bfseries}YY}
	\thead{Charges} & \thead{Spell}\\
	2 & Cause Disease\\
	3 & Cause Serious Wounds\\
	2 & Curse (Blindness)\\
	4 & Create Poison
  \end {tabularx}
\end {table}

\textbf{Staff of Healing (C):}\index[magicitems]{Staff of Healing} This staff can only be used by clerics. The wielder may activate the staff to cure 1d6+1 damage to a target. This does not use a charge, but each target can only be cured once per day by the staff.

The staff can also be activated to cure other things as indicated on \fullref{tab:Staff of Healing}, although this uses charges.

\begin {table}[H]
  \caption{Staff of Healing}\label{tab:Staff of Healing}
  \begin{tabularx}{\columnwidth}{>{\bfseries}YY}
	\thead{Charges} & \thead{Spell}\\
	1 & Cure Blindness\\
	1 & Cure Disease\\
	2 & Cure Serious Wounds\\
	2 & Neutralize Poison
  \end {tabularx}
\end {table}

\textbf{Staff of Power (W):}\index[magicitems]{Staff of Power} This staff can only be used by wizards. Whenever a creature is struck by the staff, the wielder may expend one charge to inflict an additional 2d6 damage to the creature in addition to the normal damage that the staff does. This does not require the normal Activate Magic Item action, but is done as part of the Attack action.

The wielder of the staff can also activate it as normal and expend a single charge to use any of the following effects as if casting the spells as a \nth{6} level caster: Fireball, Lightning Bolt, Ice Storm, and Continual Light, and Telekinesis (up to 2,400 cn).

\textbf{Snake Staff (C, Dr):}\index[magicitems]{Snake Staff} This staff can only be used by clerics and druids. The staff is a +1 weapon.

Whenever a creature is struck by the staff, the wielder may command the staff to turn into a snake and coil around and hold the target. This does not require the normal Activate Magic Item action, but is done as part of the Attack action.

The snake will hold any target that is human-sized or smaller, unless the target can make a saving throw vs. wands to avoid the effect. The target will be held for 1d4x10 minutes, or until the wielder orders the snake to release the target.

If the snake fails to hold the target, or when the target escapes or is released, the snake will attempt to return to the wielder of the staff and return to staff form. Should it be attacked before this happens, it has an armor class of 5, 3 hit dice (20 hit points), and moves at 20 feet per round. The snake will neither attack nor attempt to defend itself, being concerned only with returning to its owner.

Once the snake returns to staff form, any damage it may have taken is healed.

The staff neither has nor uses charges, but if the snake is killed before it can return to staff form the staff is ruined.

\textbf{Staff of Striking (S):}\index[magicitems]{Staff of Striking} This staff can only be used any spellcaster. Whenever a creature is struck by the staff, the wielder may expend one charge to inflict an additional 2d6 damage to the creature in addition to the normal damage that the staff does. This does not require the normal Activate Magic Item action, but is done as part of the Attack action.

\textbf{Staff of Withering (C):}\index[magicitems]{Staff of Withering} This staff is usable only by clerics. The wielder of the staff may activate the staff and expend a charge to touch a target and cause the target to make a saving throw vs. wands or age 10 years. This staff does not work on undead.

\textbf{Staff of Wizardry (W):}\index[magicitems]{Staff of Wizardry} This staff is only usable by wizards. Whenever a creature is struck by the staff, the wielder may expend one charge to inflict an additional 2d6 damage to the creature in addition to the normal damage that the staff does. This does not require the normal Activate Magic Item action, but is done as part of the Attack action.

The wielder of the staff can also activate it as normal and expend a single charge to use any of the following effects as if casting the spells as a \nth{6} level caster: Fireball, Lightning Bolt, Ice Storm, Continual Light, Telekinesis (up to 2,400 cn), Invisibility, Passwall, Web, and Conjure Elemental.

The user may also use the staff as if it were a Wand of Paralyzation.

Additionally, the wielder may choose to break the staff to cause an explosion that does 8 points of damage per charge remaining in the staff to all within 30 feet. The wielder of the staff gets no saving throw, but other creatures in the area can make a saving throw vs. wands to take half damage.

\textbf{Rod of Cancellation:}\index[magicitems]{Rod of Cancellation} This rod may be used by any character, but may only be used once. When activated, it will permanently drain any magical item (except an artifact.) that is touched by it of all power.

The Game Master may require an attack roll to touch a magic item carried or worn by another creature, normally against armor class 9.

If an item with a +5 bonus is being wielded or worn by a creature, that creature may make a saving throw vs. wands for their item to resist the effect. If the item resists, the rod of cancellation is not expended and may be used again.

\textbf{Rod of Dominion:}\index[magicitems]{Rod of Dominion} This rod may be used by any character. If the ruler of a dominion carries it while parading around the dominion, it will give a bonus to the dominion’s Confidence Rating. This rod does not have charges, but it can only be used once per year.

To determine the effect of the rod, refer to \fullref{tab:Rod of Dominion} each time it is paraded to see what proportion of the population view it, and therefore what bonus it gives.

\begin {table}[H]
  \caption{Rod of Dominion}\label{tab:Rod of Dominion}
  \begin{tabularx}{\columnwidth}{>{\bfseries}YY}
	\thead{1d100} & \thead{Bonus}\\
	01-50 & +10\\
	51-75 & +20\\
	76-90 & +30\\
	91-99 & +40\\
	00 & +50
  \end {tabularx}
\end {table}

\textbf{Rod of Health (C):}\index[magicitems]{Rod of Health} This rod is only usable by clerics. The rod functions as a Staff of Healing, except that none of the functions use charges. However, the rod can only heal a creature once per day regardless of the type of healing bestowed.

\textbf{Rod of Inertia (N):}\index[magicitems]{Rod of Inertia} This unusually long rod is only usable by non-spellcasters. It has a blade on its end and is weighted for throwing, and can therefore be wielded in combat as if a +3 javelin.

The wielder of the rod may give it a command at any time to stop, and the rod will instantly become completely immobile, and cannot be moved by any means short of a Wish spell.

When the wielder gives a second command, the rod will continue moving on its original trajectory as if nothing had happened.

Commanding the rod to stop or start can be done very quickly and does not require an action in combat. The wielder can, for example, command the rod to stop while they are falling and then hang on to the rod.

\textbf{Rod of Parrying:}\index[magicitems]{Rod of Parrying} This rod may be used by any character. The rod is heavy and scepter-like, and can be used in melee as if it were a +5 mace. When held by a character with the fighter's Parry ability, the rod provides its +5 magical bonus to the wielder’s armor class. This is in addition to the normal -4 penalty to the attacker’s to-hit roll caused by parrying.

\textbf{Rod of Victory:}\index[magicitems]{Rod of Victory} This rod may be used by any character. When the wielder of the rod commands an army in battle, the rod gives the army a +25 bonus to their roll for the battle, and prevents them from losing the battle roll by more than 100 points.

Any loss greater than this is treated as a 100-point loss when determining casualties and post-battle tactical positioning.

\textbf{Rod of Weaponry (N):}\index[magicitems]{Rod of Weaponry} This rod can be used by any non-spellcaster.

Upon command, this rod will extend into a +5 staff or retract back into rod form.

While in staff form, the wielder may also command the +5 staff to split into two +2 staves, and each of those may be commanded to split into two +1 staves.

The staves will not split accidentally, and they can be re-joined by simply placing them together.

\textbf{Rod of the Wyrm:}\index[magicitems]{Rod of the Wyrm} This rod may be used by any character.

The rod has a carved dragon’s head on top, making it look somewhat reminiscent of a hobby-horse. This head will be of a random color as indicated on \fullref{tab:Rod of the Wyrm}.

\begin {table}[H]
  \caption{Rod of the Wyrm}\label{tab:Rod of the Wyrm}
  \begin{tabularx}{\columnwidth}{>{\bfseries}YY}
	\thead{1d10} & \thead{Color}\\
	1-2 & Black\\
	3-4 & Blue\\
	5-6 & Green\\
	7-8 & Red\\
	9-10 & White
  \end {tabularx}
\end {table}

The rod will function as a +5 mace in combat.

When the wielder commands it, the rod will turn into a young dragon of the same color as its head. This dragon can only be hit by magical weapons.

The dragon will serve the wielder of the rod as a messenger, steed or bodyguard to the best of its ability; and will sacrifice itself for the wielder if necessary.

The dragon is not healed by being returned to rod form and will not heal or age naturally, but may be healed as normal by spells and potions.

If the dragon is ever killed, it cannot be raised and cannot be turned back to rod form.

\subsection{Rings}\index[general]{Rings}
Magical rings are usually either constant in effect or are activated by a Use Non-Activatable Item action.

A magical ring must be worn on a finger or thumb to operate, and a character can only wear one magical ring per hand. If a second ring is put on the same hand as an existing ring, neither ring will function (with the exception of a Ring of Weakness).

\begin {table}[H]
  \caption{Rings}
  \begin{tabularx}{\columnwidth}{>{\bfseries}YY}
	\thead{d100} & \thead{Ring Type}\\
	01-02 & Animal Control\\
	03-08 & Delusion\\
	09 & Djinn Summoning\\
	10-13 & Ear\\
	14-17 & Elemental Adaption\\
	18-23 & Fire Resistance\\
	24-26 & Holiness (C, Dr)\\
	27 & Human Control\\
	28-32 & Invisibility\\
	33-35 & Life Protection\\
	36-38 & Memory (S)\\
	39-40 & Plant Control\\
	41-45 & Protection +1\\
	46-48 & Protection +2\\
	49-50 & Protection +3\\
	51 & Protection +4\\
	52-55 & Quickness\\
	56 & Regeneration\\
	57-59 & Remedies\\
	60-61 & Safety\\
	62-64 & Seeing\\
	65-67 & Spell Eating\\
	68-69 & Spell Storing\\
	70-71 & Spell Turning\\
	72-75 & Survival\\
	76-77 & Telekinesis\\
	78-81 & Truth\\
	82-84 & Truthfulness\\
	85-86 & Truthlessnesss\\
	87-91 & Water Walking\\
	92-96 & Weakness\\
	97-98 & Wishes\\
	99-00 & X-Ray Vision
  \end {tabularx}
\end {table}

\textbf{Animal Control:}\index[magicitems]{Ring of Animal Control} This ring may be used once per turn. It acts as if the wearer had drunk a Potion of Animal Control with the exception that the wearer only gets a single attempt to control creatures per use.

\textbf{Delusion:}\index[magicitems]{Ring of Delusion} This ring will appear to function as a different type of ring if an Analyze spell is used on it. Once the ring is worn, it will be completely non-functional, but cannot be removed without a Remove Curse being cast on it.

\textbf{Djinn Summoning:}\index[magicitems]{Ring of Djinn Summoning} Once per week, the wearer of this ring can summon a djinn, who will serve them for a day. If the djinn is killed, the ring loses its magical power.

\textbf{Ear:}\index[magicitems]{Ear Ring} Three times per day this ring can be removed from its wearer’s finger and placed against any surface. Until the wearer recovers the ring, they will be able to hear everything that happens around the ring as if their ear were in the ring’s location.

\textbf{Elemental Adaption:}\index[magicitems]{Ring of Elemental Adaption} The exact type of ring found is determined randomly as indicated on \fullref{tab:Ring of Elemental Adaptation}.

\begin {table}[H]
  \caption{Ring of Elemental Adaptation}\label{tab:Ring of Elemental Adaptation}
  \begin{tabularx}{\columnwidth}{>{\bfseries}YY}
	\thead{1d100} & \thead{Type}\\
	01-21 & Air\\
	22-42 & Earth\\
	43-63 & Fire\\
	64-84 & Water\\
	85-91 & Air and Water\\
	92-98 & Earth and Fire\\
	99-00 & All Elements
  \end {tabularx}
\end {table}

The wearer of the ring is protected from environmental hazards in the elemental plane that matches the elemental alignment of the ring, and is able to breathe in those planes despite lack of air.

\textbf{Fire Resistance:}\index[magicitems]{Ring of Fire Resistance} This ring makes its wearer immune to natural fire, gives a +2 bonus to all saving throws against magical fire, and reduces all magical fire damage done to the wearer by one point per die (to a minimum of one point per die).

\textbf{Holiness (C, Dr):}\index[magicitems]{Ring of Holiness} This ring can only be used by a cleric or druid. When worn by a cleric or druid, the wearer will be able to prepare an extra spell of each level from one to three. The ring will not allow the wearer to learn spells of a level that is too high for them to learn without it.

If the ring is removed, the extra spells are immediately lost.

Additionally, if worn by a cleric, the ring gives a +1 bonus to all rolls to turn undead, both the rolls for success and the rolls for the number of hit dice of undead creatures turned.

\textbf{Human Control:}\index[magicitems]{Ring of Human Control} This ring may be used once per turn. It acts as if the wearer had drunk a Potion of Human Control with the exception that the wearer only gets a single attempt to control people per use.

\textbf{Invisibility:}\index[magicitems]{Ring of Invisibility} This ring may only be used once per turn. When activated it affects the wearer as if they had cast an Invisibility spell on themselves.

\textbf{Life Protection:}\index[magicitems]{Ring of Life Protection} This ring will protect its wielder from 1d6 levels worth of \iref[sec:Energy Drain]{Energy Drain}. If overloaded (for example if it has only a single level of protection left and the wearer is drained two levels) then it will successfully protect the caster and then turn to dust. If it is exactly used up without being overloaded, it will become a Ring of Protection +1 once its levels of protection have been used up.

\textbf{Memory (S):}\index[magicitems]{Ring of Memory} This ring can only be used by a spellcaster. Once per day, the caster may activate it to recover one spell that they have cast within the previous ten minutes. That spell will be available for casting again as if freshly prepared.

\textbf{Plant Control:}\index[magicitems]{Ring of Plant Control} This ring may be used once per turn. It acts as if the wearer had drunk a Potion of Plant Control with the exception that the wearer only gets a single attempt to control creatures per use.

\textbf{Protection:}\index[magicitems]{Ring of Protection} This ring gives a bonus to its wearers armor class and to all saving throws equal to its magical bonus. If a character wears a ring of protection on either hand, only the larger of the two bonuses applies.

\textbf{Quickness:}\index[magicitems]{Ring of Quickness} Once per day this ring allows the wearer to move and attack as if they had cast a Haste spell which lasts 10 minutes.

\textbf{Regeneration:}\index[magicitems]{Ring of Regeneration} The wearer of this ring will recover 1 hit point per ten minutes, and can slowly re-grow lost body parts. A limb will re-grow over the course of a week, whereas a finger or ear would re-grow over the course of a single day.

This ring will not stop working when the wearer is on 0 hit points, and will not prevent the wearer from dying. It will also not heal damage from fire or acid (although will re-grow limbs lost to fire or acid).

\textbf{Remedies:}\index[magicitems]{Ring of Remedies} Once per day, this ring will duplicate the effect of a Cure Blindness, Cure Disease, Remove Curse or Neutralize Poison spell as if cast by a \nth{25} level cleric. The spell can be cast on the wearer or on a target that the wearer touches.

\textbf{Safety:}\index[magicitems]{Ring of Safety} This ring acts like a Potion of Luck except that it has 1d4 charges and uses up a charge each time a die is placed instead of rolled. Once all the charges have been used up, this ring becomes non-magical.

\textbf{Seeing:}\index[magicitems]{Ring of Seeing} Once per day this ring allows the wearer to see as if they had cast a Truesight spell which lasts 30 minutes.

\textbf{Spell Eating:}\index[magicitems]{Ring of Spell Eating} This ring appears to Analyze spells to be a Ring of Spell Turning and operates as one.

However, if the wearer of the ring casts a spell themselves, the ring immediate “eats” all the remaining prepared spells that the caster has. The ring can then no longer be removed except by the use of a Remove Curse cast by a \nth{25} level caster.

Once the curse has been activated, the wearer can prepare new spells as normal, but will lose them again when they next cast one.

\textbf{Spell Storing:}\index[magicitems]{Ring of Spell Storing} This ring will contain 1d6 different spells (determined randomly as if spells on a Spells Scroll) when found. The wearer of the ring can cast these spells once each, even if not a spellcaster. The spells will always be cast as if by a caster of the minimum level needed to cast them, even if the wearer of the ring is a caster of higher level.

Each spell can be recharged by having a spellcaster cast it directly into the ring. The ring cannot “catch” spells generally cast at the wearer; the spells must be cast directly at the ring with the purpose of recharging it.

\textbf{Spell Turning:}\index[magicitems]{Ring of Spell Turning} Each day, this ring will reflect the first 2d6 spells (roll each day) cast at the wearer back at their casters. This only affects actual spells, not the spell-like abilities of monsters or magic items.

Once it has absorbed as many spells as it can, the ring has no other powers until the following day.

\textbf{Survival:}\index[magicitems]{Ring of Survival} This ring will contain 1d100+100 charges when found. Each charge spent will allow the wearer to go for 24 hours without food and drink or for 1 hour without breathing.

When the ring is down to its last five charges, it will turn black; and when it runs out of charges it will crumble to dust.

\textbf{Telekinesis:}\index[magicitems]{Ring of Telekinesis} This ring can be activated to produce an effect identical to the Telekinesis spell, capable of moving objects weighing up to 2,000 cn.

\textbf{Truth:}\index[magicitems]{Ring of Truth} Three times per day, this ring can be activated to telepathically warn its wearer whether a spoken statement that they have just heard is true in the opinion of the speaker. Note that there is a difference between the speaker being untruthful and the speaker merely being honestly wrong about something.

\textbf{Truthfulness:}\index[magicitems]{Ring of Truthfulness} This ring appears to be a Ring of Truth when examined with an Analyze spell, and will work as a Ring of Truth. However, once worn it cannot be removed except by a Remove Curse spell cast by a \nth{26} level caster, and it compels its wearer to always speak the truth.

The wearer may not knowingly lie while wearing the ring (but the ring doesn’t prevent them from accidentally being wrong).

\textbf{Truthlessnesss:}\index[magicitems]{Ring of Truthlessnesss} This ring appears to be a Ring of Truth when examined with an Analyze spell, and will work as a Ring of Truth. However, once worn it cannot be removed except by a Remove Curse spell cast by a \nth{26} level caster, and it compels its wearer to always speak lies. The wearer may not knowingly make any true statement while wearing the ring (but the ring does not prevent them from making a statement that is accidentally true if they genuinely don’t know that it is).

\textbf{Water Walking:}\index[magicitems]{Ring of Water Walking} This ring allows its wearer to walk on the surface of any liquid without sinking.

\textbf{Weakness:}\index[magicitems]{Ring of Weakness} 1d6 rounds after this ring is worn, it will immediately lower its wearer’s \iref[sec:Strength]{Strength} score to 3. The ring can not be removed until it has had a Remove Curse spell cast on it.

\textbf{Wishes:}\index[magicitems]{Ring of Wishes} This ring will grant from one to four wishes (as the Wish spell) before crumbling into dust. The number of wishes contained in the ring are determined randomly as indicated on \fullref{tab:Ring of Wishes}.

\begin {table}[H]
  \caption{Ring of Wishes}\label{tab:Ring of Wishes}
  \begin{tabularx}{\columnwidth}{>{\bfseries}YY}
	\thead{1d10} & \thead{\# of Wishes}\\
	1-4 & 1\\
	5-7 & 2\\
	8-9 & 3\\
	10 & 4
  \end {tabularx}
\end {table}

\textbf{X-Ray Vision:}\index[magicitems]{Ring of X-Ray Vision} The wearer of this ring can see up to 30 feet through stone or up to 60 feet through wood. It cannot be used to see through metal.

To use the ring, the wearer must stand still and concentrate, and can view a 10-by-10-foot area per use. It takes 10 minutes to scan such an area, and the ring can only be used once per hour.

\subsection{Wondrous Items}\index[general]{Wondrous Items}

\begin {table}[H]
  \caption{Wondrous Items}
  \begin{tabularx}{\columnwidth}{>{\bfseries}YY}
	\thead{d100} & \thead{Item}\\
	01-02 & Amulet of Scrying Protection\\
	03-04 & Bag of Devouring\\
	05-09 & Bag of Holding\\
	10-12 & Boat, Undersea\\
	13-15 & Boots of Leaping\\
	16-17 & Boots of Levitation\\
	18-19 & Boots of Speed\\
	20 & Bowl of Water Elementals\\
	21 & Brazier of Fire Elementals\\
	22-23 & Broom of Flying\\
	24 & Censer of Air Elementals\\
	25-27 & Chime of Time\\
	28-29 & Crystal Ball (E, M)\\
	30 & Crystal Ball with Clairaudience  (W)\\
	31 & Crystal Ball with ESP (W)\\
	32-33 & Displacer Cloak\\
	34 & Drums of Panic\\
	35 & Efreeti Bottle\\
	36-38 & Egg of Wonder\\
	39-40 & Elven Boots\\
	41-42 & Elven Cloak\\
	43 & Flying Carpet\\
	44-45 & Gauntlets of Ogre Power\\
	46-47 & Girdle of Giant Strength\\
	48-49 & Helm of Blindness\\
	50-51 & Helm of Reading\\
	52 & Helm of Telepathy\\
	53 & Helm of Teleportation (W)\\
	54 & Horn of Blasting\\
	55-56 & Hurricane Lamp\\
	57-59 & Lamp of Long Burning\\
	60-61 & Medallion of ESP (30 ft. range)\\
	62 & Medallion of ESP (90 ft. range)\\
	63 & Mirror of Life Trapping\\
	64-66 & Muzzle of Training\\
	67-68 & Nail, Finger\\
	69-71 & Nail of Pointing\\
	72-76 & Ointment\\
	77-79 & Pouch of Security\\
	80-82 & Quill of Copying (S)\\
	83-86 & Rope of Climbing\\
	87-88 & Scarab of Protection\\
	89-91 & Slate of Identification (S)\\
	92 & Stone of Earth Elementals\\
	93-94 & Talisman of Travel (M, Dr)\\
	95-97 & Wheel of Floating\\
	98-00 & Wheel, Square
  \end {tabularx}
\end {table}

\textbf{Amulet of Scrying Protection:}\index[magicitems]{Amulet of Scrying Protection} This item protects its wearer from being scried on via a Crystal Ball, and makes them immune to all types of ESP.

\textbf{Bag of Devouring:}\index[magicitems]{Bag of Devouring} This bag looks like a normal sack, but any non-living item placed entirely within it disappears from view and weighs nothing while in the bag. It will hold items up to 10,000 cn in weight, providing the items fit wholly within it (i.e. they are 5 by 1 by 1 foot or smaller).

Items placed within the bag can be found by touch by anyone reaching into the bag and can be withdrawn.

However, any item left in the bag for more than two hours will vanish completely. This will not affect living creatures that are stuffed into the bag.

\textbf{Bag of Holding:}\index[magicitems]{Bag of Holding} This bag looks like a normal sack, but any non-living item placed entirely within it disappears from view and weighs nothing while in the bag. It will hold items up to 10,000 cn in weight, providing the items fit wholly within it (i.e. they are 5 by 1 by 1 foot or smaller).

Items placed within the bag can be found by touch by anyone reaching into the bag and can be withdrawn.

\textbf{Boat, Undersea:}\index[magicitems]{Undersea Boat} This boat can be operated as a fully functional skiff. However, when a command word is given, it will dive under the water while simultaneously radiating a Water Breathing effect that protects all passengers who are touching the boat. The pilot of the boat can control its movement underwater as if on the surface.

\textbf{Boots of Leaping:}\index[magicitems]{Boots of Leaping} The wearer of these boots can make leaps of up to 10 feet vertically and up to 30 feet horizontally.

\textbf{Boots of Levitation:}\index[magicitems]{Boots of Levitation} The wearer of these boots may Levitate as per the spell \iref[spell:Levitate]{Levitation}. There is no limit to the duration of the levitation.

\textbf{Boots of Speed:}\index[magicitems]{Boots of Speed} When traveling in the wilderness, the wearer of these boots travels at the speed of a riding horse. However, the wearer can only move at this speed for a single day and then must rest for a day.

\textbf{Bowl of Water Elementals:}\index[magicitems]{Bowl of Water Elementals} Once per day, this bowl can be filled with water and used to cast a Conjure Elemental spell except that it will only conjure a 12 hit dice water elemental. Conjuring the elemental takes 10 minutes.

\textbf{Brazier of Fire Elementals:}\index[magicitems]{Brazier of Fire Elementals} Once per day, this brazier can be lit and used to cast a Conjure Elemental spell except that it will only conjure a 12 hit dice fire elemental. Conjuring the elemental takes 10 minutes.

\textbf{Broom of Flying:}\index[magicitems]{Broom of Flying}\label{mitem:Broom of Flying} This broom will carry its owner through the air at a speed of 80 feet per round. The owner of the broom must concentrate to move, and the broom will hover if the owner stops concentrating.

The broom can also carry a passenger, but in this case its speed is reduced to 60 feet per round.

\textbf{Censer of Air Elementals:}\index[magicitems]{Censer of Air Elementals} Once per day, this censer can be filled with incense and used to cast a Conjure Elemental spell except that it will only conjure a 12 hit dice air elemental. Conjuring the elemental takes 10 minutes.

\textbf{Chime of Time:}\index[magicitems]{Chime of Time} This simple metal chime can be commanded to keep track of time. It will sound every hour until commanded to stop, and can be clearly heard within a 60-foot radius.

If the chime is in a Silence 15-foot radius spell when it is time to chime, it will automatically dispel the spell as it chimes.

The chime can also be used like a sand-timer. If commanded, it will slowly change color from one end to the other taking exactly one hour to do so.

\textbf{Crystal Ball (E, M):}\index[magicitems]{Crystal Ball} A crystal ball is a scrying device that can only be used by an elf or by a wizard. The crystal ball can be used three times per day to see any place or object that they desire; and a current image of that place or object will appear and last for 10 minutes. The clarity of the image will be based on the familiarity that the user has with the object or area.

\textbf{Crystal Ball with Clairaudience (W):}\index[magicitems]{Crystal Ball with Clairaudience} This works just as a normal Crystal Ball, except that by concentrating the user can hear what is going on at the far end as if through the ears of any living creature shown in the ball. Only a wizard can use this item.

\textbf{Crystal Ball with ESP (W):}\index[magicitems]{Crystal Ball with ESP} This works just as a normal Crystal Ball, except that by concentrating the user can read the thoughts of the main subject of the vision if it is a living creature. Only a wizard can use this item.

\textbf{Displacer Cloak:}\index[magicitems]{Displacer Cloak} This item makes the wearer appear to be 5 feet from their actual position. This makes all attacks against the wearer take a -2 penalty to their to-hit rolls, and gives the wearer a +2 bonus to all saving throws vs. spells, vs. wands, staves and rods, and vs. petrification.

\textbf{Drums of Panic:}\index[magicitems]{Drums of Panic} These are a pair of large kettle drums. When played, they have no effect on creatures within 10 feet of them, but creatures from 10-240 feet from them must make a saving throw vs. spells or flee in terror for 30 minutes.

\textbf{Efreeti Bottle:}\index[magicitems]{Efreeti Bottle} This 3-foot-tall heavy jug contains an efreeti. The stopper may be opened once per day, and the efreeti will come forth and serve the opener. If the efreeti is slain, the bottle becomes non-magical.

The efreeti is reluctant to serve, and will do its best to use loopholes in the commands that it is given in order to cause harm to the owner of the bottle.

\textbf{Egg of Wonder:}\index[magicitems]{Egg of Wonder} This is a painted hollow eggshell. When dropped or throw (up to 60 feet) it will break and an animal will emerge and grow to full size. For the next hour, the animal will obey the user of the egg to the best of its ability; and then it will disappear.

The type of animal that is summoned is determined randomly as indicated on \fullref{tab:Egg of Wonder}.

\begin {table}[H]
  \caption{Egg of Wonder}\label{tab:Egg of Wonder}
  \begin{tabularx}{\columnwidth}{>{\bfseries}YY}
	\thead{1d12} & \thead{Animal}\\
	1 & Ape (Rock Baboon)\\
	2 & Bat (Giant)\\
	3 & Bear (Black)\\
	4 & Bear (Grizzly)\\
	5 & Boar\\
	6 & Cat (Mountain Lion)\\
	7 & Cat (Panther)\\
	8 & Ferret (Giant)\\
	9 & Lizard (Giant Gecko)\\
	10 & Lizard (Giant Draco)\\
	11 & Snake (Racer)\\
	12 & Wolf
  \end {tabularx}
\end {table}

\textbf{Elven Boots:}\index[magicitems]{Elven Boots} These boots give their wearer a Move Silently ability of 75\% like a rogue.

\textbf{Elven Cloak:}\index[magicitems]{Elven Cloak} This cloak is just like a Ring of Invisibility except that its invisibility is not perfect. When the wearer is in the presence of creatures that might notice them, their faint outline will be seen if a 1 is rolled on 1d6.

\textbf{Flying Carpet:}\index[magicitems]{Flying Carpet}\label{mitem:Flying Carpet} This carpet will carry one passenger at a speed of 100 feet per round, two at a speed of 80 feet per round, three at a speed of 60 feet per round, four at a speed of 40 feet per round, or five at a speed of 20 feet per round.

The owner of the carpet must concentrate to make it move, and the carpet will hover in place if the owner stops concentrating.

\textbf{Gauntlets of Ogre Power:}\index[magicitems]{Gauntlets of Ogre Power} The owner of these gauntlets has a \iref[sec:Strength]{Strength} of 18 while wearing the gauntlets, but taking them off return’s the owner’s \iref[sec:Strength]{Strength} back to its normal value.

\textbf{Girdle of Giant Strength:}\index[magicitems]{Girdle of Giant Strength} Anyone who wears this girdle does double damage with whatever melee attacks they make.

\textbf{Helm of Blindness:}\index[magicitems]{Helm of Blindness} Anyone wearing this helmet is immediately made blind. The blindness cannot be cured until the helmet has been removed, and the helmet cannot be removed without a Remove Curse spell being cast on it.

\textbf{Helm of Reading:}\index[magicitems]{Helm of Reading} This helmet allows the wearer to read any language or cipher, and allows them to identify magical scrolls. However, it does not allow the wearer to use Spells Scrolls if they are not normally able to do so.

This helmet is rather delicate, and has a 10\% chance of being broken each time its wearer is struck in combat.

\textbf{Helm of Telepathy:}\index[magicitems]{Helm of Telepathy} The wearer of this helmet can transmit their thoughts to any creature within 60 feet. The target creature will understand the thoughts of the wearer despite language differences. The wearer may also use the ESP spell to read the thoughts of others.

\textbf{Helm of Teleportation (W):}\index[magicitems]{Helm of Teleportation} This helm is only usable by wizards. The wearer of the helm may use the Teleport spell as often as they like to teleport themselves, with the normal chances of failure.

The wearer may also use the helm to Teleport another creature (again, as if casting the spell). However, doing this discharges the helmet and it can no longer be used for any teleporting until it is recharged by having a Teleport spell cast into it.

\textbf{Horn of Blasting:}\index[magicitems]{Horn of Blasting} This horn can be blown once per ten minutes. When it is blown, it creates a cone of sound 100 feet long and 20 feet wide at the end. Everyone in the area must take 2d6 damage and make a saving throw vs. spells or be deafened for ten minutes. Buildings and ships in the area of effect take 1d8 damage.

\textbf{Hurricane Lamp:}\index[magicitems]{Hurricane Lamp} When this lamp is opened for the first time in a day, a hurricane of wind and rain emerge. Everyone within 30 feet of the lamp is knocked to the floor for three rounds, and must make a saving throw vs. spells to avoid having all loose items or items held or carried by them (including weapons, but not including clothing) is scattered throughout the area.

Once the hurricane has ended, the lamp may be used as a Lamp of Long Burning, although the hurricane will re-set each day and must be triggered again before the lamp can be used.

\textbf{Lamp of Long Burning:}\index[magicitems]{Lamp of Long Burning} This lamp must be filled with oil and lit like any other lantern, but it will burn forever without using any of the oil that it contains.

The lamp can be put out and re-lit repeatedly like any other lamp.

If the lamp is ever completely submerged in water while lit, it will immediately stop working and lose its magical properties.

\textbf{Medallion of ESP:}\index[magicitems]{Medallion of ESP} This medallion allows the wearer to use the ESP spell at will, with a range as given in the item listing (either 30 feet or 90 feet).

However, each time it is used, there is a 1 in 6 chance that instead of letting its wearer read minds, it will broadcast its all of its wearer’s thoughts out loud for the next ten minutes.

\textbf{Mirror of Life Trapping:}\index[magicitems]{Mirror of Life Trapping} If the owner of this mirror presents it to a creature of human size or smaller and activates it, the creature must make a saving throw vs. spells or be sucked into the mirror, complete with everything they are wearing and carrying.

The mirror can hold up to 20 creatures, who exist in a state of suspended animation while in the mirror, not needing food, drink or air and completely unable to take any actions.

Anyone looking into the mirror can see the reflections of the faces of all those trapped inside, and can wake any of them up temporarily to talk with them. No special communication powers are granted, so the viewer and victim need to share a common language to talk in.

If the mirror is broken, all the creatures inside are instantly freed. However, the only way to free a single creature without breaking the mirror is to use a Wish spell.

\textbf{Muzzle of Training:}\index[magicitems]{Muzzle of Training} This muzzle will automatically reshape and resize itself to fit any creature with a mouth. When put on a creature and locked with a command word, the creature is magically prevented from biting, talking and casting spells; but it can still eat, drink and breathe normally.

The owner of the muzzle can unlock it with a second command word, but otherwise it is locked with the equivalent of a Wizard Lock cast by a \nth{15} level caster.

\textbf{Nail, Finger:}\index[magicitems]{Finger Nail} This appears to be a Nail of Pointing if checked with an Analyze spell. However, when someone tries to use it as such it will disappear. The next time that person tries to be unnoticed (by disguise, Invisibility or hiding), an illusion of a large glowing hand will appear above their head pointing them out.

After that first instance, there is a 25\% chance each time the person tries to go un-noticed that the hand will re-appear. This will continue to happen until the victim has a Remove Curse spell cast on them.

\textbf{Nail of Pointing:}\index[magicitems]{Nail of Pointing} This appears to be a simple 6-inch iron nail. Once per day, it can be commanded to point to a mundane (not magical or living) object, and for the next ten minutes the nail will point at the closest instance of that type of object (distance is irrelevant).

\textbf{Ointment:}\index[magicitems]{Ointment} This creamy white salve is normally found in small jars.

To use the ointment, the whole jar must be applied to someone’s skin. When the ointment is applied in this manner, it will have a random effect as indicated on \fullref{tab:Ointment}.

\begin {table}[H]
  \caption{Ointment}\label{tab:Ointment}
  \begin{tabularx}{\columnwidth}{>{\bfseries}cY}
	\thead{1d6} & \thead{Effect}\\
	1 & The user gets a +2 bonus to armor class and saving throws for the next ten minutes.\\
	2 & The user is healed 2d6+2 points of damage.\\
	3 & The user must make a saving throw vs. poison or die.\\
	4 & The user takes 2d6 points of acid damage.\\
	5 & The user is cured of all burn damage, whether magical or normal.\\
	6 & The user’s skin turns bright orange for the next 1d4 months.
  \end {tabularx}
\end {table}

\textbf{Pouch of Security:}\index[magicitems]{Pouch of Security} If this large sack is stolen, it will shout “Help! I am being stolen!” in common repeatedly for the next hour. The sack’s owner can command it to be quiet before the end of the duration.

\textbf{Quill of Copying (S):}\index[magicitems]{Quill of Copying} When supplied with 1,000 gp worth of rare inks, any spell user can command this quill to copy a Spells Scroll containing one or more spells of a type that they can cast onto a blank parchment.

There is a 75\% chance that the spells will be copied correctly, creating two identical Spells Scrolls, and a 25\% chance that the quill will burst, becoming useless and depositing its ink over both parchments (both ruining the original and failing to make a copy).

\textbf{Rope of Climbing:}\index[magicitems]{Rope of Climbing} On command, this 50-foot rope will wriggle along the ground like a snake, and even up walls and on ceilings. It can fasten itself onto any solid protrusion on a surface that it is climbing up, and will support up to 10,000 cn of weight.

A second command will cause the rope to loosen itself and re-coil.

\textbf{Scarab of Protection:}\index[magicitems]{Scarab of Protection} This charm has 2d6 charges. Each time a curse is placed on the wearer, it will immediately target the wearer with a Remove Curse as if by a \nth{36} level caster. Each Remove Curse uses up one charge.

Additionally, it will block any Finger of Death or Obliterate spell cast at the wearer; and this also uses up one charge.

When the scarab has run out of charges, it crumbles to dust.

\textbf{Slate of Identification (S):}\index[magicitems]{Slate of Identification (S)} This 3-square-foot framed sheet of slate is used to identify magic items.

The slate has ten charges per day, and is used by placing it on a horizontal surface and then placing a magic item on it. When activated, the slate will identify the magic item (if it has enough charges left to do so) and chalk writing will appear on the slate indicating type of magic item and what command words it has (if any).

If asked to identify an artifact, the slate will shatter.

However, the slate is incapable of identifying cursed items, even those that fail to fool an Analyze spell (such as a Potion of Poison).

If it is asked to identify a cursed item, the slate will identify it as being a random type of similar but useful item.

The number of charges needed to identify different types of item are listed on \fullref{tab:Slate of Identification}.

\begin {table}[H]
  \caption{Slate of Identification}\label{tab:Slate of Identification}
  \begin{tabularx}{\columnwidth}{>{\bfseries}YY}
	\thead{Item} & \thead{\# of Charges}\\
	Potion & 2\\
	Scroll & 3\\
	Ammunition & 3\\
	Wand & 4\\
	Staff & 5\\
	Weapon & 6\\
	Armor or Shield & 7\\
	Ring or Rod & 8\\
	Wondrous Item & 9
  \end {tabularx}
\end {table}

\textbf{Stone of Earth Elementals:}\index[magicitems]{Stone of Earth Elementals} Once per day, this stone can be buried in the earth and used to cast a Conjure Elemental spell except that it will only conjure a 12 hit dice earth elemental. Conjuring the elemental takes 10 minutes.

\textbf{Talisman of Travel (M, Dr):}\index[magicitems]{Talisman of Travel (M, Dr)} This talisman allows the wearer to cast either a Conjure Elemental or Summon Elemental spell in reverse. Instead of the spell summoning an elemental to the caster, the reversed spell will transport the caster to the elemental plane of their choice.

While the caster continues to wear the talisman, they are protected from environmental damage on the elemental plane, such as choking or burning, and can breathe normally.

\textbf{Wheel, Floating:}\index[magicitems]{Wheel of Floating} When put on a cart or wagon, this wheel will allow the cart to be pulled on the surface of water rather than sinking.

A single wheel of floating will hold up a vehicle of up to 10,000 cn in weight, with every additional wheel of floating adding 5,000 cn to that weight.

Although these wheels will keep the vehicle above water, they do not affect whatever animal or animals are pulling the vehicle.

\textbf{Wheel, Square:}\index[magicitems]{Square Wheel} This wheel is the size of a wagon or cart wheel, but is square rather than round.

It cannot be used on normal terrain, but when fitted to a wagon or cart it will allow the cart to travel over desert or mountain as if being pulled along a road.

\subsection{Armor and Shields}\index[general]{Magic Armor and Shields}
Magic armor and shields will have a numerical bonus from +1 to +5. This bonus applies to the wearer’s armor class when the armor or shield is worn.

Like other armor class bonuses, the bonus is subtracted from the armor class of the wearer, not added to it.

The magical bonuses of armor and shields stack with each other if both are used.

Some magical armor or shields also have additional powers beyond a simple bonus to armor class. If these powers require activation in combat, then the wearer must take an Activate Magic Item action.

When found, a piece of magical armor or a magical shield may be cursed. The chance of the item being cursed is 10\%. If an item is cursed, this will be revealed by an Analyze spell.

A cursed set of armor or cursed shield will add its bonus to the wearer’s armor class rather than subtracting it—i.e. it will make the wearer’s armor class worse. Additionally, the extra property of cursed armor will not function.

The wearer of cursed armor or a cursed shield is compelled to always use the item in a combat situation, even though they may know that other armor would be better for them. Similarly, if the cursed item is taken away from them they are compelled to do their best to retrieve the item—even being prepared to kill friends if the friends are withholding it.

A Remove Curse spell will break the compulsion, although if the character wears or uses the item again then the curse will return.

If the Remove Curse spell is cast by a caster of \nth{26} level or higher, it will remove the curse from the item completely, causing it to act as normal magical suit or armor or shield.

\begin {table}[H]
  \caption{Armor or Shield Type}
  \begin{tabularx}{\columnwidth}{>{\bfseries}YY}
	\thead{d100} & \thead{Type}\\
	01-13 & Leather Armor\\
	14-22 & Scale Mail\\
	23-38 & Chain Mail\\
	39-50 & Banded Mail\\
	51-64 & Plate Mail\\
	65-72 & Suit Armor\\
	73-00 & Shield
  \end {tabularx}
\end {table}

\begin {table}[H]
  \caption{Armor or Shield Bonus}
  \begin{tabularx}{\columnwidth}{>{\bfseries}YY}
	\thead{d100*} & \thead{Bonus}\\
	01-54 & +1\\
	55-57 & +1 of Curing\\
	58-59 & +1 of Shocking\\
	60 & +1 of Reflection\\
	61-78 & +2\\
	79-80 & +2 of Curing\\
	81 & +2 of Gaseous Form\\
	82-89 & +3\\
	90-91 & +3 of Absorption\\
	92 & +3 of Remove Curse\\
	93-96 & +4\\
	97 & +4 of Flying\\
	98 & +4 of Haste\\
	99 & +5\\
	00 & +5 of Ethereality\
  \end {tabularx}
* If the monster whose treasure this is has less than 9 hit dice, subtract 10 from the roll.
\end {table}

\textbf{Absorption:}\index[magicitems]{Armor of Absorption} If the wearer of this item or shield is struck by an \iref[sec:Energy Drain]{Energy Drain}, the item will absorb it. The wearer will still take normal physical damage (if any) from the attack, but will not lose any levels.

Each time this item absorbs an \iref[sec:Energy Drain]{Energy Drain}, it loses one point of magical “plus” (i.e. after absorbing one \iref[sec:Energy Drain]{Energy Drain} the item will only have a +2 bonus; after absorbing a second it will only have a +1 bonus).

When the item has absorbed three \iref[sec:Energy Drain]{Energy Drains} it will crumble to dust.

\textbf{Curing:}\index[magicitems]{Armor of Curing} Once per day, the wearer of this armor or shield may activate it in order to heal half of the damage that they have currently taken.

\textbf{Ethereality:}\index[magicitems]{Armor of Ethereality} Once per day, the wearer of this armor or shield can activate it in order to move from the \ilink{sec:Prime Plane}{Prime Plane} to the \iref[sec:Ethereal Plane]{Ethereal Plane}.

Additionally, once per day the wearer of this armor or shield can activate it in order to move from the \iref[sec:Ethereal Plane]{Ethereal Plane} to the \ilink{sec:Prime Plane}{Prime Plane}.

\textbf{Flying:}\index[magicitems]{Armor of Flying} Once per day, the wearer of this armor or shield may activate it, and it will cast a Fly spell on them that lasts for 2 hours.

\textbf{Gaseous Form:}\index[magicitems]{Armor of Gaseous Form} Once per day, the wearer of this armor or shield may activate it in order to turn into gaseous form for up to one hour, including all their equipment and carried items. The drinker keeps control of their body, and can move through any non-airtight barrier.

While in gaseous form, the drinker cannot attack, but has an armor class of -2 and can only be hit by magical weapons.

\textbf{Haste:}\index[magicitems]{Armor of Haste} Once per day, the wearer of this armor or shield may activate it and it will cast a Haste spell on them that lasts for 10 minutes.

\textbf{Reflection:}\index[magicitems]{Armor of Reflection} If any Light or Continual Light spell (or their reverse) is cast at the wearer of this armor or shield, the spell is reflected back to the caster.

Additionally, the wearer of this armor or shield may used as if it were a mirror in order to either attempt to reflect the gaze of a creature with a gaze attack (such as a basilisk) or to fight such a creature without meeting its gaze (in which case the normal -2 penalty for fighting while using a mirror does not apply).

\textbf{Remove Curse:}\index[magicitems]{Armor of Remove Curse} This armor or shield will never be cursed when found. Additionally, it has three charges, and can be activated to expend a charge in order to cast a Remove Curse on the wearer as if from a \nth{36} level caster.

\textbf{Shocking:}\index[magicitems]{Armor of Shocking} The wearer of this armor or shield can activate it, causing it to charge with energy. The next creature to attack the wearer in melee will be hit for 6d6 lightning damage, unless it can make a saving throw vs. spells to take half damage.

If the creature striking the wearer is using a melee weapon rather than striking with natural weaponry then it gets a +4 bonus to the saving throw.

The armor can be activated and de-activated as often as the wearer desires, but can only release its charge once per day. Once the charge has been released, the armor deactivates and cannot be re-activated until the following day.

\subsection{Weapons and Ammunition}\index[general]{Magic Weapons and Ammunition}
Magic weapons and ammunition will have a numerical bonus from +1 to +5. This bonus applies both to the wearer’s to-hit roll when using the weapon or ammunition and to the damage done by the weapon.

The magical bonuses of missile weapons and their ammunition stack with each other if both are used.

Some magical weapons or ammunition also have additional powers beyond a simple bonus to attack and damage. If these powers require activation in combat, then the wearer must take an Activate Magic Item action.

When a magical weapon or some magical ammunition is found, roll on the relevant \autoref*{tab:Missile Weapons and Ammunition Type} to \autoref*{tab:Miscellaneous Weapon Type} in order to see exactly what type of weapon or ammunition it is, and then in order to find out what magical bonuses or other properties it has, either roll on \fullref{tab:Ammunition Bonus} if it is ammunition or roll on \fullref{tab:Weapon Bonus} if it is a weapon (including a missile weapon).

When found, a magical weapon (but not magical ammunition) may be cursed. The chance of the weapon being cursed is 10\%. If an item is cursed, this will be revealed by an Analyze spell.

A cursed weapon will subtract its bonus from the wearer’s to-hit and damage rolls rather than adding it—i.e. it will make the wearer’s to-hit and damage worse. Additionally, the extra property of a cursed weapon will not function.

The wielder of a cursed weapon is compelled to always use the item in a combat situation, even though they may know that other weapons would be better for them.

Similarly, if the cursed item is taken away from them they are compelled to do their best to retrieve the item—even being prepared to kill friends if the friends are withholding it.

A Remove Curse spell will break the compulsion, although if the character wears or uses the item again then the curse will return.

If the Remove Curse spell is cast by a caster of \nth{26} level or higher, it will remove the curse from the item completely, causing it to act as a normal magical weapon.

Some weapons have a bonus that varies depending on the type of creature that is being attacked.

Magical ammunition can only be used for a single shot, and is then broken, bent or otherwise useless. However, such ammunition is normally found in groups rather than as single items. \fullref{tab:Ammunition Bonus}  shows the number of pieces of ammunition that are found together.

Some of the thrown weapons on \fullref{tab:Miscellaneous Weapon Type} are listed as (Returning). These weapons will magically return to their owner’s hand after being thrown.

In the case of a bolas or net, if the weapon successfully entangles an opponent it will not return to its owner until after the opponent has freed themselves.

\begin {table}[H]
  \caption{Missile Weapons and Ammunition Type}\label{tab:Missile Weapons and Ammunition Type}
  \begin{tabularx}{\columnwidth}{>{\bfseries}YY}
	\thead{d100} & \thead{Type}\\
	01-30 & Arrows\\
	31-48 & Bullets\\
	49-59 & Quarrels\\
	60-70 & Sling Stones\\
	71-72 & Blowgun\\
	73-79 & Bow, Long\\
	80-83 & Bow, Short\\
	84-85 & Crossbow, Heavy\\
	86-87 & Crossbow, Light\\
	88-91 & Gun, Pistol\\
	92-94 & Gun, Smoothbore\\
	95-00 & Sling
  \end {tabularx}
\end {table}

\begin {table}[H]
  \caption{Sword Type}
  \begin{tabularx}{\columnwidth}{>{\bfseries}YY}
		\thead{d100} & \thead{Type}\\
	01-65 & Sword, Normal\\
	66-84 & Sword, Short\\
	85-92 & Sword, Two-Handed\\
	93-00 & Sword, Bastard
  \end {tabularx}
\end {table}

\begin {table}[H]
  \caption{Miscellaneous Weapon Type}\label{tab:Miscellaneous Weapon Type}
  \begin{tabularx}{\columnwidth}{>{\bfseries}YY}
	\thead{d100} & \thead{Type}\\
	01-09 & Axe, Battle\\
	10-15 & Axe, Hand\\
	16-17 & Axe, Hand (Returning)\\
	18-20 & Blackjack\\
	21-22 & Bolas\\
	23 & Bolas (Returning)\\
	24-28 & Club\\
	29-40 & Dagger\\
	41-43 & Dagger (Returning)\\
	44-45 & Halberd\\
	46-55 & Hammer, Throwing\\
	56-58 & Hammer, War (Returning)\\
	59-61 & Javelin\\
	62 & Javelin (Returning)\\
	63-66 & Lance\\
	67-78 & Mace\\
	79-80 & Net\\
	81 & Net (Returning)\\
	82-84 & Poleaxe\\
	85-94 & Spear\\
	95-97 & Spear (Returning)\\
	98-00 & Whip
  \end {tabularx}
\end {table}

\begin {table}[H]
  \caption{Ammunition Bonus}\label{tab:Ammunition Bonus}
  \begin{tabularx}{\columnwidth}{>{\bfseries}YYY}
	\thead{d100*} & \thead{Bonus} & \thead{Number}\\
	01-54 & +1 & 2d10\\
	55-57 & Of Speaking & 2d10\\
	58-59 & Of Seeking & 2d10\\
	60 & +1 of Disarming & 2d10\\
	61-78 & +2 & 2d6\\
	79-80 & +2 of Screaming & 2d6\\
	81 & +2 of Dispelling & 2d6\\
	82-89 & +3 & 2d4\\
	90-91 & +3 of Sinking & 2d4\\
	92 & +3 of Biting & 2d4\\
	93-96 & +4 & 1d4\\
	97 & +4 of Stunning & 1d4\\
	98 & +4 of Teleporting & 1d4\\
	99 & +5 & 1\\
	00 & +5 of Slaying & 1\
  \end {tabularx}
	* If the monster whose treasure this is has less than 9 hit dice, subtract 10 from the roll.
\end {table}

\begin {table}[H]
  \caption{Weapon Bonus}\label{tab:Weapon Bonus}
  \begin{tabularx}{\columnwidth}{>{\bfseries}YY}
	\thead{d100*} & \thead{Bonus}\\
	01-54 & +1\\
	55-57 & +1 of Watching\\
	58-59 & +1 of Light\\
	60 & +1 of Finding\\
	61-62 & +1, +3 vs. Undead\\
	63-64 & +1, +3 vs. Animals\\
	65-66 & +1, +3 vs. Lycanthropes\\
	67-78 & +2\\
	79-80 & +2 of Flaming\\
	81 & +2 of Silence\\
	82-83 & +2, +4 vs. Undead\\
	84 & +2, +4 vs. Giants\\
	85-89 & +3\\
	90-91 & +3 of Speed\\
	92 & +3 of Hiding\\
	93 & +3, +5 vs. Dragons\\
	94-96 & +4\\
	97 & +4 of Slowing\\
	98 & +4 of Flying\\
	99 & +5\\
	00 & +5 of Slicing\
  \end {tabularx}
	* If the monster whose treasure this is has less than 9 hit dice, subtract 10 from the roll.
\end {table}

\textbf{Biting:}\index[magicitems]{Ammunition of Biting} When a piece ammunition of biting hits an opponent, it turns into a snake which bites the target. In addition to the normal damage done by the attack, the target must also make a saving throw vs. poison or die.

\textbf{Disarming:}\index[magicitems]{Ammunition of Disarming} When this ammunition hits a target that is wielding an item in their hand, the target must make a saving throw vs. spells or drop the item. If the target is holding an item in either hand, their primary hand will be the one affected.

\textbf{Dispelling:}\index[magicitems]{Ammunition of Dispelling} When this ammunition hits a target, it immediately casts a Dispel Magic effect centered on the target as if cast by a \nth{15} level caster.

\textbf{Finding:}\index[magicitems]{Weapon of Finding} Once per day, the wielder of this weapon may activate it to cast a Locate Object spell.

\textbf{Flaming:}\index[magicitems]{Flaming Weapon} The wielder of this weapon can activate it at will and it will burst into flames that burn without fuel until the weapon is deactivated again.

The flaming weapon has an additional +2 bonus against trolls and against feathered creatures; and an additional +3 bonus against creatures made of wood or undead. If a creature comes under more than one of these categories, these bonuses do not stack.

\textbf{Flying:}\index[magicitems]{Weapon of Flying} If a melee weapon has this power, once per day it will attack by itself. The weapon must be wielded in combat for at least one round, then its wielder can let go of it and it will continue to attack the same opponent for three rounds, as if its wielder were still wielding it. After three rounds (or when its wielder commands, whichever is sooner) the weapon will return to its wielder’s hand.

If a missile weapon has this power, then it never needs reloading and never runs out of ammunition. Whenever it is fired, it will create its own ammunition (and Red Powder if it is a gun). However, the wielder may still choose to load the weapon with magical ammunition if they prefer.

\textbf{Hiding:}\index[magicitems]{Weapon of Hiding} Once per day, this weapon can be activated to cast an Invisibility spell on its wielder.

Additionally, three times per day, the weapon can be activated to cast an Invisibility spell on itself. If the weapon turns itself invisible while being carried, the touch of its wielder will not turn it visible again, but the touch of any other creature (whether intelligent or not) will.

\textbf{Light:}\index[magicitems]{Weapon of Light} Once per day, this weapon can be activated to cast a Light spell lasting 1 hour.

\textbf{Screaming:}\index[magicitems]{Ammunition of Screaming} When this ammunition is fired, whether it hits or misses its target, it will emit a piercing shriek causing all opponents within 30 feet of its path to make a saving throw vs. spells or retreat in fear for 1d8 rounds.

\textbf{Seeking:}\index[magicitems]{Ammunition of Seeking} This ammunition will not hit a living creature. However, it will unerringly hit whatever object (a button, a lever, a trip-wire, etc.) it is fired at.

If fired at an object held by a creature, treat the ammunition as being +1 Ammunition of Disarming.

\textbf{Silence:}\index[magicitems]{Weapon of Silence} Once per day, this weapon can be activated to cast a Silence 15-foot radius spell.

\textbf{Sinking:}\index[magicitems]{Ammunition of Sinking} This ammunition causes 1d10+10 structure points of damage to any ship or wooden structure that it hits.

\textbf{Slaying:}\index[magicitems]{Ammunition of Slaying} Whatever is hit by this ammunition is affected as if hit by a Disintegrate spell.

\textbf{Slicing:}\index[magicitems]{Slicing} This property can only be found on edged melee weapons such as swords, axes or pole arms.

Whenever a living creature (not a construct or undead) is hit by the wielder of this weapon’s player rolling a natural 19 or 20, it must make a saving throw vs. death or have its head cut off. This will normally kill the creature, unless it is a hydra or other multi-headed creature.

If the creature makes its saving throw, it still takes triple damage from the attack.

\textbf{Slowing:}\index[magicitems]{Weapon of Slowing} Once per day, after hitting a creature with this weapon, the wielder may activate it (this doesn’t take an action) in order to affect the creature as if hit by a Slow spell. The creature may make a saving throw vs. spells to avoid the effect.

\textbf{Speaking:}\index[magicitems]{Ammunition of Speaking} This ammunition may be given any message of twenty words or less, and then given a target which may be any location or object (but not a creature) within ten miles.

When fired, the ammunition will automatically hit the object or land on the floor in the location, and then speak its message out loud twice.

\textbf{Speed:}\index[magicitems]{Weapon of Speed} Once per day, this weapon can be activated in order to cast a Haste spell that affects only its wielder.

\textbf{Stunning:}\index[magicitems]{Ammunition Stunning} Any creature hit by this ammunition must make a saving throw vs. spells or be \iref[sec:Stunned]{Stunned}  for 1d6 rounds.

A stunned opponent cannot attack or cast spells and can only move at 1/3 normal speed. \iref[sec:Stunned]{Stunned} opponents also have a +2 penalty to armor class and a -2 penalty to all saving throws.

\textbf{Teleporting:}\index[magicitems]{Ammunition of Teleporting} Any creature hit by this ammunition must make a saving throw vs. spells or be teleported 1d100 miles in a random direction. The target will always land safely after the teleport.

\textbf{Watching:}\index[magicitems]{Weapon of Watching} Once per day, the wielder of this weapon can command it to watch for a particular type of creature. The weapon will continue to watch for that type of creature until commanded to watch for a different type.

If any creature of the specified type comes within 60 feet of the weapon, it will glow softly and vibrate to alert its wielder.

The wielder must name a race or type of monster, not the name of a specific individual.

\subsection{Magic Item Values}\index[general]{Magic Item Values}
Dark Dungeons does not give specific rules for the creation of magic items, neither does it give strict prices for buying and selling them.

However, the cost of the Immortal level spell Create Mundane Object is based on the value of the item created.

For purposes of the Create Mundane Object spell, use (or approximate) the following formula:

\subsubsection{Armor}
Multiply the standard price for a non-magical suit of the appropriate armor (in gold pieces) by the encumbrance of the armor (in coins) and divide the total by three.

That value is the value per “plus” of the armor.

If the armor has an additional power, add 5,000 gp to the value of the armor per level of the spell which the power is based on.

\subsubsection{Weapons and Shields}
Multiply the standard price for a non-magical weapon or shield of the appropriate type (in gold pieces) by the encumbrance of the weapon or shield (in coins) and multiply the total by five.

That value is the value per “plus” of the weapon or shield. If the item is a weapon which has varying “pluses” against different opponents, use the average of the pluses to determine value.

If the weapon or shield has an additional power, add 5,000 gp to the value of the armor per level of the spell which the power is based on.

\subsubsection{Other Magic Items}
For each power that the item has, find the spell that most resembles the power, and multiply that spell’s level by 100 gp if there is no restriction on the power’s use, 75 gp if the power can only be used once per day, or 70 gp if the power can only be used once per month.

If the item has only a single charge (such as a scroll or a potion), then that is the value of the item.

If the item has multiple charges (such as a wand or a staff), multiply the value by the number of charges. For staffs with multiple powers and a single set of charges, add the costs of the powers together and then multiply the total by the number of charges.

If the item has a permanent effect or unlimited uses (such as most rings, rods or wondrous items) multiply the value by 50. Again, for items with multiple powers, add the costs of the powers together and then multiply the total by the number of charges.

\subsection{Making Magic Items}\index[general]{Making Magic Items}
It should be stressed that the values given above are only for \iref[chap:Immortals]{Immortals} bringing magic items into existence with the Create Mundane Object spell.

When mortals try to create magic items, the process is much more difficult and time consuming, and isn’t guaranteed to work.

If a mortal wishes to create a magic item, you should apply the following rules:

\begin{itemize}
 \item{The mortal must be a spellcaster able to cast each spell that the item will have as a power.}
 \item{The mortal must use materials costing the value of the item (as defined above). These materials are used up regardless of whether creating the item is successful or not.}
 \item{The mortal must also quest for and use one rare ingredient (determined by the Game Master) that must be personally gathered and used while fresh—it is not possible to simply buy a rare ingredient or to gather them in bulk in advance. The difficulty and danger of the quest for the rare ingredient should depend on the power of the item that is being made.}
 \item{The mortal must spend one week plus one day per 1,000 gp value of the item working (8 hours per day) to create the item.}
\end{itemize}

Once the mortal spellcaster has followed all those requirements, the percentage chance of successfully making the magical item is determined by adding together the caster’s level and either the caster’s \iref[sec:Intelligence]{Intelligence} or \iref[sec:Wisdom]{Wisdom} score (whichever is higher), doubling the result, and subtracting three times the level of the spell or number of “pluses” they are trying to put into the item.

If the caster is trying to make an item with both “pluses” and a power, or is trying to make an item with multiple powers, then each should be rolled separately, and all must succeed for the item to be made.

If any creation roll fails, the item will be ruined and all time and materials (including the rare ingredient) will have been wasted.

\example{Elfstar wishes to make a Potion of Healing. This item is effectively a single use of a \nth{1} level spell (Cure Light Wounds), so it will cost only 100 gp in ingredients and take a week to make.

The Game Master decides that for such a low level item, the rare ingredient is not going to be difficult to find, so specifies that Elfstar needs to find a rare herb that is known to grow in a nearby enchanted forest.

Elfstar gathers the herb and starts to make the potion. Since Elfstar is currently \nth{5} level and has a Wisdom of 17, her chance of success is: (5+17)x2 - (1x3) = 41\%

Elfstar’s player rolls 1d100 and gets a 19. At the end of the week, Elfstar has successfully made a Potion of Healing.}

\example{Aloysius wishes to make a +3 Sword of Speed.

The base value per plus for a normal sword is determined by multiplying its price (10 gp) by its weight (60 cn) and multiplying the result by five. This gives a total of: (10x60)x5 = 3,000 gp per plus.

The sword is going to be a +3 sword, so the total cost is 9,000 gp.

However, the sword is also going to be Of Speed. This is the equivalent of a third level spell (Haste), that can be used once per day but has unlimited charges. Therefore, the cost is: (3x75)x50 = 11,250 gp

The total cost to make the sword will therefore be: 11,250 gp+9,000 gp = 20,250 gp

Since this is a fairly powerful item, the Game Master decides that it needs a fairly difficult to find rare ingredient—the tail feather of a cockatrice.

Aloysius manages to find a suitable tail feather, and spends 27 days in his workshop making the sword.

At the end of that period, he must make two rolls; one for the +3 bonus and one for the “Of Speed” ability.

Since he is \nth{23} level, and has an Intelligence of 16, his chances are: (23+16)x2 - (3x3) = 69\% -and- (23+16)x2 - (3x3) = 69\%

Aloysius’s player rolls 1d100 twice. His first roll is a 41, but his second roll is an 80. He has failed to make the sword properly and his 20,250 gp worth of materials (and his 27 days) have been wasted.}

\subsection{Trading Magic Items}
When it comes to trading magical items, the above rules for their value do not directly apply for four reasons.

Firstly, the value of an item doesn’t take into account how easy it is to make. Items that are difficult to make are likely to need multiple attempts before the creator is successful, making the actual price of making such an item in real terms more than the calculated value.

Secondly, magic items rarely get destroyed. There have been people making magic swords for hundreds if not thousands of years, so there are lots out there lying around in monster lairs and tombs and so on. In the case of some common items, the ease of finding one or taking one from monsters means that the market price would be brought down; possibly even to the point where it is not economically viable to make one since you can buy an old one more cheaply.

Thirdly, magic items are only worth what people will pay for them. A Ring of Water Walking costs more to make than a Ring of Invisibility for example, but the invisibility ring will be worth far more than the water walking ring to the average adventurer.

Fourthly, there is a limited market for magic items. Your average farmer has little use for a Potion of Flying or a Ring of Life Protection, and even less use for a Wand of Fireballs that they can’t even use. The fact that the people who want magic items—adventurers—are the very people who are most likely to find their own and not need to buy them means that it is very much a buyer’s market.

For these four reasons, Dark Dungeons does not give hard prices for the buying and selling of magic items. It is suggested that most trading of magic items will be simply done in the form of like-for-like barter if items between adventurers, and that there should not be merchants or shops that buy and sell such items.

In particular, the magic item list should never be simply treated as a shopping list with price tags attached.

\subsection{Making Constructs}\index[general]{Making Constructs}\label{sec:Making Constructs}
Many of the creatures in \fullref{chap:Monsters} have the Construct keyword. These creatures are artificial beings created and animated by magic.

It is suggested that these creatures are treated as if they were magic items for purposes of creation by both \iref[chap:Immortals]{Immortals} and mortal spellcasters.

The base value of a construct should be 2,000 gp per hit die, plus 5,000 gp per asterisk that the creature has on its hit dice in its monster entry.

When a mortal wishes to make a construct, it should be just like making a magic item; they need to use the value in materials, find a special rare ingredient, and spend a week plus a day per 1,000 gp of value making it.

The percentage chance of successfully making a construct is determined by adding together the caster’s level and either their \iref[sec:Intelligence]{Intelligence} or \iref[sec:Wisdom]{Wisdom} score (whichever is higher), doubling the result, and subtracting the sum of the construct’s hit dice and asterisks.

\example{Aloysius wishes to make a manikin construct. A manikin construct has a hit dice value of 6**.

The cost of making the construct is therefore 6x2,000 gp plus 2x5,000 gp—for a total of 22,000 gp.

The construct will take 29 days to make, and the chance of success will be: (23+16)x2 - (6+2) = 70\%}

Once a construct is made, it will obey the commands of its creator, who becomes its first owner.

Ownership of a construct may be transferred to someone else at any time by the current owner, although once a construct becomes ownerless (because its old owner died without passing on ownership) nothing short of a Wish can take ownership of it again.

Usually, constructs are unintelligent, and will therefore continue to operate under their current instructions indefinitely if ownerless. However, occasionally the inability to carry out its instructions combined with the lack of an owner to give it new instructions will cause a construct to break its magical programming and go rogue.

Most rogue constructs simply go berserk and attack any creature they encounter. Rarely, however, one will develop a free will and intelligence of its own and develop a personality.

\section{Artifacts}\index[general]{Artifacts}\label{sec:Artifacts}
Artifacts are powerful magical items that only \iref[chap:Immortals]{Immortals} can create. Unlike normal magic items, which come in fairly standard types, artifacts. are all unique.

\iref[chap:Immortals]{Immortals} create artifacts. by using the Create Artifact spell. This spell enables them to put part of their life force into an object and make it into an artifact level magical item.

While this is much more expensive for the \iref[chap:Immortals]{Immortal} doing the creating than simply using the Create Mundane Object spell to create a normal magic item, artifacts. are much more powerful than normal magic items. In fact artifacts. are so powerful that they cannot be safely used by mortals. Their power is simply too great for mortals to be able to control.

This danger doesn’t stop mortals from actually using them, of course.

\subsection{Finding an Artifact}\index[general]{Finding an Artifact}
Given the rare and unique nature of artifacts, they do not appear in the standard treasure tables. Basically, if the Game Master wants the party to find an artifact, the Game Master first needs to spend time using the rules in this chapter designing the artifact, and then decide where it will be found.

When designing it, the Game Master should have an idea of who created it and why they did. Creating an artifact is a significant thing to do, and not something that most \iref[chap:Immortals]{Immortals} would do on a whim.

On the other hand; once created, artifacts. hang around for a long time. An artifact that might have been created and specially placed for a mortal to find may have been found and used by that mortal. And then it might have been lost, or stolen, or otherwise changed hands many times.

While \iref[chap:Immortals]{Immortals} often try to keep track of artifacts. that they have made, they can’t be everywhere at once; and artifacts. do sometimes end up being simply lost or end up sitting in a dragon’s hoard for centuries on end.

Unless the \iref[chap:Immortals]{Immortal} happens to be watching via a Detect Immortal Magic spell when an artifact gets used (and gets lucky), they may never find it again.

From the point of view of player characters; unexpectedly finding an artifact is a mixed blessing. Notwithstanding the inherent danger involved in using such a powerful item, the characters can never be sure that the find was truly accidental. Was the artifact genuinely lost, or was it put there specifically for them to find as part of some \iref[chap:Immortals]{Immortal’s} plot? Or worse—was it put there specifically for someone else to find as part of some \iref[chap:Immortals]{Immortal’s} plot, and now they’re getting in the way?

\subsection{Using an Artifact}\index[general]{Using an Artifact}
An artifact will normally have a basic form. That form might be that of a weapon, shield or armor, in which case the artifact will probably have magical pluses when used in combat. Alternately it might be shaped like any other object.

Regardless of its form, an artifact can be detected with a Detect Magic spell and identified with an Analyze spell. However, an Analyze spell will only describe the powers of an artifact, not its handicaps and penalties. The handicaps and penalties of the artifact must be found out by trial and error.

Once an artifact has been identified, any character can use it unless it is in the shape of armor or a weapon or shield that the character can’t use.

Each artifact has a Power Reserve just like an \iref[chap:Immortals]{Immortal}, and this power reserve is the source of Power Points that fuel the artifact’s powers.

The wielder of the artifact can activate any of its powers by taking an Activate Magic Item action; and if the artifact has enough remaining power points then the power will function.

An artifact’s power reserve refreshes each morning.

Any powers of an artifact that duplicate spell effects work as if cast by a \nth{40} level mortal caster; and artifacts can affect \iref[chap:Immortals]{Immortals}. An \iref[chap:Immortals]{Immortal} hit by a weapon shaped artifact takes normal damage (unless they can save vs. physical attacks) rather than minimum damage, and an \iref[chap:Immortals]{Immortal} hit by a spell cast from an artifact will be affected normally as if the spell had been cast by another \iref[chap:Immortals]{Immortal} (although they still get their Anti-Magic and their save vs. spell attacks).

An artifact that has run out of power points keeps its weapon, shield or armor bonus.

\subsection{Destroying an Artifact}\index[general]{Destroying an Artifact}
Artifacts are incredibly hard to destroy. They are immune to most damage, and will very rarely be accidentally destroyed.

An artifact has an armor class of -20 and a number of hit points equal to its power reserve. It can only be damaged by +5 weapons, or spells of \nth{5} level and higher; and makes saving throws as if a \nth{36} level fighter.

Although artifacts are not intelligent, they do have a basic self-defense mechanism. An artifact will ignore accidental damage and simply repair itself, but if an artifact is deliberately attacked directly it will use whatever power points it has left to instinctively use its offensive powers against its attacker and use its movement powers to escape if it can.

A damaged artifact can repair itself at a cost of 1 power point per hit point regained.

When an artifact is destroyed, the power used to create it is lost forever. There is no way to recover it, although if the creator of the artifact is that upset with its loss they can always simply create a new artifact with an identical form and powers.

\subsection{Creating an Artifact}\index[general]{Creating an Artifact}
An \iref[chap:Immortals]{Immortal} creates an artifact by casting the Create Artifact spell. This will cost them experience points based on the artifact created. There is no chance for error, and creating the artifact takes only a single round (although making the decisions about exactly what it should do and what it should look like can take far longer).

The amount of experience it costs to create an artifact depends on both the physical form and the amount of power reserve that the artifact will have.

\subsubsection{Physical Form}
The physical form of an artifact must be created as part of the creation process. This form may be something mundane, like a simple wooden bowl; something ostentatious, like a golden crown; or something practical like a sword.

The basic cost for this physical form is 1 experience point per 1 gold piece value of the form—with a minimum cost of 100 XP; so in the examples above the bowl would cost the minimum of 100 XP, as would the sword. The crown could cost anywhere from 1,000 XP to around 50,000 XP depending on how big and valuable it is.

If the physical form is a weapon, there is an additional cost based on its other statistics:

\begin{itemize}
 \item{+10,000 XP per two points of weapon damage done by the form using basic weapon expertise.}
 \item{+30,000 XP if the form a missile weapon.}
 \item{+50,000 XP if the form is a commonly thrown weapon.}
 \item{+100,000 XP per magical “plus” the weapon has.}
\end{itemize}

Similarly, if the physical form is a shield or a suit of armor there is an additional cost based on its other statistics:

\begin{itemize}
 \item{+10,000 XP per point of base armor class below 9 if the form is armor.}
 \item{+10,000 XP if the form is a shield}
 \item{+50,000 XP per magical “plus” the armor or shield has.}
\end{itemize}

\example{An artifact that is in the form of a +5 dagger will cost: 100 XP (because a dagger is worth 3 gp) + 20,000 XP (because a dagger does 1d4 damage) + 50,000 XP (because a dagger can be thrown) + 500,000 XP (because it is a +5 weapon) = 570,100 XP in total.

An artifact that is a suit of chain mail +3 will cost: 100 XP (because chain mail is worth 40 gp) + 40,000 XP (because chain mail provides a base armor class of 5) + 300,000 XP (because it is +3 chain mail) = 340,100 XP in total.

An artifact that is a +4 shield will cost: 100 XP (because a shield is worth 10 gp.) + 10,000 XP (because it is a shield) + 400,000 XP (because it is a +4 shield) = 410,100 XP in total.

An artifact that is a large flawless ruby (worth 15,000 gp) will cost: 15,000 XP (because the ruby is worth 15,000 gp) = 15,000 XP in total.}

\subsubsection{Power Reserve}
Additionally, the \iref[chap:Immortals]{Immortal} must pay 10,000 experience points for each point of power reserve that the artifact is intended to have.

An artifact must be given a power reserve of at least 50 points, and can be given a power reserve of up to 750 points.

As well as determining the number of power points that the artifact can spend each day on powers, the amount of power reserve an artifact is given also determines how many hit points the artifact will have and how many powers and drawbacks it will have.

\fullref{tab:Artifact Power Levels} shows the maximum number of powers of each type that an artifact can have based on its power reserve. The \iref[chap:Immortals]{Immortal} creating the artifact is free to choose these powers from the lists in this chapter.

An artifact does not have to have as many powers as it can, in fact it doesn’t have to have any powers at all. However, since it doesn’t cost the creating \iref[chap:Immortals]{Immortal} any extra experience to put more powers in, most artifacts. are created with the maximum number of powers that their power reserve will allow.

Similarly, \autoref*{tab:Artifact Power Levels} also shows how many handicaps and penalties an artifact will have based on its power reserve. These handicaps and penalties do not manifest when an \iref[chap:Immortals]{Immortal} uses an artifact, only when a mortal uses it. They are a side effect of the mortal using such a powerful device, and as such they are not chosen by the \iref[chap:Immortals]{Immortal} creating the artifact.

Handicaps and penalties are instead chosen by the Game Master. When choosing handicaps and penalties, the Game Master should try to be fair and both try to choose those that fit the “feel” or “theme” of the artifact (if there is one) and also try not to choose those that would render the artifact useless by completely going against its purpose (again, if there is one).

\begin {table}[H]
  \caption{Artifact Power Levels}\label{tab:Artifact Power Levels}
  \begin{tabularx}{\columnwidth}{>{\bfseries}Ycccc}
	\thead{Power Reserve} & \thead{50-100} & \thead{101-250} & \thead{251-500} & \thead{501-750}\\
	Maximum Attack Powers & 2 & 3 & 4 & 4\\
	Maximum Transform Powers & 2 & 2 & 3 & 5\\
	Maximum Defense Powers & 3 & 4 & 4 & 5\\
	Maximum Misc. Powers & 1 & 2 & 3 & 4\\
	Number of Handicaps & 1 & 2 & 3 & 4\\
	Handicap Duration & 30 days & 60 days & 120 days & 240 days\\
	Number of Penalties & 1 & 3 & 5 & 8
  \end {tabularx}
\end {table}

\subsection{Artifact Powers}
There are many, many powers that an artifact can have, and these powers are chosen by the \iref[chap:Immortals]{Immortal} creating the artifact.

Each power has a cost in power points which must be spent when the artifact is used. If the artifact does not have enough power points left to use the power, it will not function. However, even an artifact with no power points left will still function as its basic form, including any magical pluses that form may have as a weapon, shield or suit of armor.

The possible artifact powers are listed in \autoref*{tab:Attack Powers (Direct Physical)} to \autoref*{tab:Miscellaneous Powers (Encumbrance Offset)}.

Most of these powers are simply duplications of mortal level spells, and are cast as if by \nth{40} level mortal spellcasters (with the exception that \iref[chap:Immortals]{Immortals’} immunity to mortal level magic does not work against them). However, some of them are unique powers—and those powers are explained below.

\begin {table}[H]
  \caption{Attack Powers (Direct Physical)}\label{tab:Attack Powers (Direct Physical)}
  \begin{tabularx}{\columnwidth}{>{\bfseries}YY}
	\thead{Power} & \thead{Cost (PP)}\\
	Cause Light Wounds & 10\\
	Magic Missile & 15\\
	Flaming* & 20\\
	Cause Disease & 25\\
	Cause Serious Wounds & 30\\
	Extinguishing* & 30\\
	Cause Critical Wounds & 35\\
	Bearhug* & 35\\
	Create Poison & 40\\
	Dispel Evil & 40\\
	Electricity* & 40\\
	Cloudkill & 45\\
	Ice Storm & 45\\
	Death Spell & 50\\
	Finger of Death & 50\\
	Poison Gas Breath* & 50\\
	Slicing* & 50\\
	Fireball & 55\\
	Fire Breath* & 60\\
	Ice Breath* & 60\\
	Lightning Bolt & 60\\
	Acid Breath* & 65\\
	Delayed Blast Fireball & 65\\
	Explosive Cloud & 75\\
	Disintegrate & 80\\
	Power Word Kill & 85\\
	Obliterate & 90\\
	Meteor Swarm & 100\
  \end {tabularx}
	*See description in this chapter
\end {table}

\begin {table}[H]
  \caption{Attack Powers (Direct Mental)}
  \begin{tabularx}{\columnwidth}{>{\bfseries}YY}
	\thead{Power} & \thead{Cost (PP)}\\
	Cause Fear & 10\\
	Sleep & 15\\
	Charm Person & 20\\
	Confusion & 25\\
	Charm Monster & 30\\
	Calm* & 30\\
	Control Plants* & 35\\
	Feeblemind & 40\\
	Charm Plant & 45\\
	Geas & 50\\
	Control Animals* & 60\\
	Control Lesser Undead* & 70\\
	Charm, Mass & 75\\
	Open Mind & 80\\
	Control Giants* & 85\\
	Control Greater Undead* & 90\\
	Control Dragons* & 95\\
	Control Humans* & 100\
  \end {tabularx}
	*See description in this chapter
\end {table}

\begin {table}[H]
	\caption{Attack Powers (Trapping)}
  \begin{tabularx}{\columnwidth}{>{\bfseries}YY}
  \thead{Power} & \thead{Cost (PP)}\\
	Web & 10\\
	Hold Animal & 15\\
	Hold Person & 20\\
	Slow & 25\\
	Hold Monster & 35\\
	Turn Wood & 45\\
	Flesh to Stone & 50\\
	Power Word Stun & 60\\
	Dance & 75\\
	Power Word Blind & 85\\
	Life Trapping* & 100\\
	Maze & 100\\
	Immortal Life Trapping* & 500\
  \end {tabularx}
	*See description in this chapter
\end {table}

\begin {table}[H]
  \caption{Attack Powers (Bonuses)}
  \begin{tabularx}{\columnwidth}{>{\bfseries}YY}
  \thead{Power} & \thead{Cost (PP)}\\
	Bless & 10\\
	Weapon Damage Bonus +2* & 15\\
	Attack Roll Bonus +2* & 20\\
	Turn Undead Bonus +2* & 20\\
	Leap 30 ft. (+2 bonus)* & 25\\
	Weapon Damage Bonus +3* & 25\\
	Weapon Damage Bonus +1* & 25\\
	Attack Roll Bonus +3* & 30\\
	Spell Damage Bonus +1* & 30\\
	Striking & 30\\
	Weapon Damage Bonus +4* & 35\\
	Attack Roll Bonus +4* & 40\\
	Turn Undead Bonus +4* & 40\\
	Weapon Strength Bonus +2* & 40\\
	Weapon Damage Bonus +5* & 45\\
	Attack Roll Bonus +5* & 50\\
	Leap 60 ft. (+2 bonus)* & 50\\
	Spell Damage Bonus +2* & 55\\
	Weapon Damage Bonus +3* & 55\\
	Attack Roll Bonus +6* & 60\\
	Weapon Damage Bonus Double* & 70\\
	Weapon Strength Bonus +4* & 70\\
	Leap 90 ft. (+6 bonus)* & 75\\
	Spell Damage Bonus +3* & 80\\
	Smash Attack* & 85\\
	Weapon Strength Bonus +5* & 85\\
	Weapon Damage Bonus Triple* & 90\\
	Spell Damage Bonus +4* & 100\
  \end {tabularx}
*See description in this chapter
\end {table}

\begin {table}[H]
  \caption{Attack Powers (Other)}
  \begin{tabularx}{\columnwidth}{>{\bfseries}YY}
  \thead{Power} & \thead{Cost (PP)}\\
	Blight & 10\\
	Darkness & 15\\
	Light & 20\\
	Set Normal Trap 50\%* & 20\\
	Turn Undead (\nth{6} Level)* & 20\\
	Curse & 25\\
	Disarm Attack* & 25\\
	Continual Darkness & 30\\
	Pick Pockets 50\%* & 30\\
	Draining (1 Level)* & 35\\
	Set Normal Trap 70\%* & 40\\
	Silence 15-foot radius & 40\\
	Polymorph Other & 45\\
	Turn Undead (\nth{12} Level)* & 45\\
	Babble & 50\\
	Flying* & 50\\
	Dispel Magic & 55\\
	Pick Pockets 75\%* & 55\\
	Appear & 60\\
	Set Normal Trap 90\%* & 65\\
	Draining (2 Levels)* & 70\\
	Turn Undead (\nth{24} Level)* & 70\\
	Polymorph Any Object & 75\\
	Pick Pockets 100\%* & 80\\
	Anti-Magic Ray* & 90\\
	Blasting* & 100\\
	De-Power* & 250\
  \end {tabularx}
	*See description in this chapter
\end {table}

\begin {table}[H]
  \caption{Transform Powers (Creations \& Summonings)}
  \begin{tabularx}{\columnwidth}{>{\bfseries}YY}
  \thead{Power} & \thead{Cost (PP)}\\
	Produce Fire & 15\\
	Create Water & 20\\
	Summon Animals & 30\\
	Create Food & 35\\
	Create Normal Animals & 40\\
	Create Normal Monsters & 45\\
	Animate Dead & 50\\
	Animate Objects & 50\\
	Sword & 70\\
	Create Normal Objects* & 75\\
	Clone & 80\\
	Create Magical Monsters & 90\\
	Create Any Monster & 100\
  \end {tabularx}
	*See description in this chapter
\end {table}

\begin {table}[H]
  \caption{Transform Powers (Static Changes)}
  \begin{tabularx}{\columnwidth}{>{\bfseries}YY}
  \thead{Power} & \thead{Cost (PP)}\\
	Purify Food and Water & 10\\
	Repair Normal Objects* & 10\\
	Change Odors* & 10\\
	Change Tastes* & 10\\
	Hold Portal & 20\\
	Remove Traps 50\%* & 30\\
	Wizard Lock & 30\\
	Create Magical Aura* & 35\\
	Magic Door & 40\\
	Repair Temporary Magical Object* & 40\\
	Rulership* & 50\\
	Magic Lock & 60\\
	Remove Traps 75\%* & 60\\
	Remove Barrier & 70\\
	Repair Permanent Magical Object* & 70\\
	Victory* & 75\\
	Metal to Wood & 80\\
	Close Gate & 85\\
	Permanence & 90\\
	Remove Traps 100\%* & 90\\
	Gate & 95\\
	Timestop & 100\\
	Spell Generation* & 250\
  \end {tabularx}
	*See description in this chapter
\end {table}

\begin {table}[H]
  \caption{Transform Powers (Dynamic Changes)}
  \begin{tabularx}{\columnwidth}{>{\bfseries}YY}
  \thead{Power} & \thead{Cost (PP)}\\
	Open Locks 60\%* & 10\\
	Warp Wood & 15\\
	Growth of Animal & 20\\
	Knock & 20\\
	Growth of Plants & 25\\
	Heat Metal & 25\\
	Open Locks 70\%* & 25\\
	Shrink Plants & 25\\
	Control Winds & 30\\
	Harden & 30\\
	Control Temperature 10-foot radius & 35\\
	Dissolve & 35\\
	Lower Water & 40\\
	Open Locks 80\%* & 40\\
	Passwall & 45\\
	Move Earth & 50\\
	Open Locks 90\%* & 55\\
	Summon Weather & 55\\
	Reverse Gravity & 60\\
	Open Locks 100\%* & 70\\
	Weather Control & 80\\
	Open Locks 110\%* & 85\\
	Earthquake & 90\\
	Open Locks 120\%* & 95\\
	Wish & 100\
  \end {tabularx}
	*See description in this chapter
\end {table}

\begin {table}[H]
  \caption{Defense Powers (Cures)}
  \begin{tabularx}{\columnwidth}{>{\bfseries}YY}
  \thead{Power} & \thead{Cost (PP)}\\
	Remove Fear & 10\\
	Cure Light Wounds & 15\\
	Cure Blindness & 20\\
	Cure Disease & 20\\
	Free Person & 25\\
	Cure Serious Wounds & 25\\
	Neutralize Poison & 30\\
	Cure Critical Wounds & 35\\
	Free Monster & 40\\
	Remove Geas & 45\\
	Stone to Flesh & 50\\
	Raise Dead & 60\\
	Remove Curse & 70\\
	Raise Dead Fully & 85\\
	Restore & 90\\
	Regeneration* & 95\\
	Heal & 100\\
	Automatic Healing* & 100\
  \end {tabularx}
	*See description in this chapter
\end {table}

\begin {table}[H]
  \caption{Defense Powers (Personal Bonuses)}
  \begin{tabularx}{\columnwidth}{>{\bfseries}YY}
  \thead{Power} & \thead{Cost (PP)}\\
	Prepare Bonus Spells/Level +1* & 10\\
	Armor Class Bonus -2* & 20\\
	Ability Score Bonus (1 Ability)* & 20\\
	Prepare Bonus Spells/Level +2* & 20\\
	Parry* & 25\\
	Saving Throw Bonus +1* & 25\\
	Hit Point Bonus +1* & 30\\
	Prepare Bonus Spells/Level +3* & 30\\
	Dodge Normal Missiles* & 35\\
	Size Control 3 in. to 18 ft.* & 35\\
	Ability Score Bonus (2 Abilities)* & 40\\
	Armor Class Bonus -4* & 40\\
	Prepare Bonus Spells/Level +4* & 40\\
	Elasticity* & 45\\
	Dodge Any Missiles* & 50\\
	Prepare Bonus Spells/Level +5* & 50\\
	Saving Throw Bonus +4* & 50\\
	Ability Score Bonus (3 Abilities)* & 60\\
	Armor Class Bonus -6* & 60\\
	Hit Point Bonus +2* & 60\\
	Prepare Bonus Spells/Level +6* & 60\\
	Dodge Directional Attacks* & 65\\
	Polymorph Self & 65\\
	Prepare Bonus Spells/Level +7* & 70\\
	Saving Throw Bonus +6* & 75\\
	Ability Score Bonus (4 Abilities)* & 80\\
	Armor Class Bonus -8* & 80\\
	Prepare Bonus Spells/Level +8* & 80\\
	Inertia Control* & 85\\
	Hit Point Bonus +3* & 90\\
	Prepare Bonus Spells/Level +9* & 90\\
	Ability Score Bonus (All Abilities)* & 100\\
	Armor Class Bonus -10* & 100\\
	Prepare Bonus Spells/Level +10* & 100\\
	Shapechange & 100\
  \end {tabularx}
	*See description in this chapter
\end {table}

\begin {table}[H]
  \caption{Defense Powers (Personal Protections)}
  \begin{tabularx}{\columnwidth}{>{\bfseries}YY}
  \thead{Power} & \thead{Cost (PP)}\\
	Shield & 10\\
	Anti-Magic 10\%* & 15\\
	Mindmask & 15\\
	Reflection* & 15\\
	Water Breathing & 15\\
	Defending* & 20\\
	Invisibility & 20\\
	Immune to Disease* & 20\\
	Invisibility 10-foot radius & 25\\
	Immune to Paralysis* & 30\\
	Security* & 30\\
	Anti-Magic 20\%* & 35\\
	Immune to Poison* & 40\\
	Immune to Aging Attacks* & 50\\
	Anti-Magic 30\%* & 55\\
	Invisibility, Mass & 60\\
	Survival & 65\\
	Statue & 70\\
	Anti-Magic 40\%* & 75\\
	Immune to Energy Drain* & 80\\
	Mind Barrier & 80\\
	Immune to Magical Detection* & 85\\
	Anti-Magic 50\%* & 95\\
	Anchoring* & 100\\
	Luck* & 100\\
	Immunity & 100\\
	Immune to Breath Weapons* & 100\
  \end {tabularx}
	*See description in this chapter
\end {table}

\begin {table}[H]
  \caption{Defense Powers (Misdirection)}
  \begin{tabularx}{\columnwidth}{>{\bfseries}YY}
  \thead{Power} & \thead{Cost (PP)}\\
	Ventriloquism & 10\\
	Confuse Alignment & 15\\
	Obscure & 20\\
	Mirror Image & 25\\
	Hide in Shadows 30\%* & 30\\
	Massmorph & 30\\
	Hallucinatory Terrain & 35\\
	Merging* & 40\\
	Hide in Shadows 50\%* & 45\\
	Phantasmal Force & 50\\
	Hide in Shadows 70\%* & 60\\
	Projected Image & 70\\
	Blend with Surroundings* & 90\
  \end {tabularx}
	*See description in this chapter
\end {table}

\begin {table}[H]
  \caption{Defense Powers (Barriers)}
  \begin{tabularx}{\columnwidth}{>{\bfseries}YY}
  \thead{Power} & \thead{Cost (PP)}\\
	Resist Cold & 10\\
	Protection from Evil & 10\\
	Resist Fire & 15\\
	Protection from Normal Missiles & 20\\
	Protection from Some Creatures* & 20\\
	Protection from Evil 10-foot radius & 25\\
	Bug Repellent* & 25\\
	Wall of Ice & 25\\
	Wall of Fire & 25\\
	Anti-Plant Shell & 30\\
	Protection from Poison & 30\\
	Wall of Stone & 35\\
	Shelter* & 35\\
	Protection from Lightning & 40\\
	Anti-Animal Shell & 45\\
	Wall of Iron & 50\\
	Protection from Most Creatures* & 60\\
	Barrier & 70\\
	Anti-Magic Shell & 75\\
	Force Field & 80\\
	Protection from All Creatures* & 85\\
	Prismatic Wall & 100\
  \end {tabularx}
	*See description in this chapter
\end {table}

\begin {table}[H]
  \caption{Miscellaneous Powers (Aids to Normal Senses)}
  \begin{tabularx}{\columnwidth}{>{\bfseries}YY}
  \thead{Power} & \thead{Cost (PP)}\\
	Detect New Construction* & 10\\
	Read Languages & 10\\
	Read Magic & 10\\
	Timekeeping* & 10\\
	Detect Slopes* & 15\\
	Speak with Animal & 15\\
	Infravision & 20\\
	Hear Noise 50\%* & 25\\
	Speak with Dead & 25\\
	Speak with Plants & 30\\
	Tracking, Lesser* & 30\\
	Communication, Lesser* & 30\\
	Find Secret Doors* & 50\\
	Communication, Greater* & 50\\
	Hear Noise 90\%* & 50\\
	Lie Detection* & 50\\
	Speak with Monsters & 60\\
	Tracking, Greater* & 70\\
	Hear Noise 140\%* & 75\\
	X-Ray Vision* & 80\
  \end {tabularx}
	*See description in this chapter
\end {table}

\begin {table}[H]
  \caption{Miscellaneous Powers (Additional Senses)}
  \begin{tabularx}{\columnwidth}{>{\bfseries}YY}
  \thead{Power} & \thead{Cost (PP)}\\
	Find Traps 50\%* & 10\\
	Predict Weather & 10\\
	Detect Magic & 15\\
	Dispel Evil & 15\\
	Find Traps 60\%* & 20\\
	Know Alignment & 20\\
	Locate Object & 20\\
	Clairvoyance & 25\\
	ESP & 25\\
	Find Traps 70\%* & 30\\
	Wizard Eye & 30\\
	Find Traps 75\% & 35\\
	Detect Invisible & 35\\
	Detect Danger & 40\\
	Find Traps 80\%* & 40\\
	Choose Best Option* & 45\\
	Find Traps 90\%* & 50\\
	Truesight & 50\\
	Mapmaking* & 55\\
	Find Traps 100\%* & 60\\
	Treasure Finding* & 60\\
	Find Traps 110\%* & 70\\
	Lore & 70\\
	Find the Path & 80\
  \end {tabularx}
	*See description in this chapter
\end {table}

\begin {table}[H]
  \caption{Miscellaneous Powers (Aids to Movement)}
  \begin{tabularx}{\columnwidth}{>{\bfseries}YY}
  \thead{Power} & \thead{Cost (PP)}\\
	Climb Walls 70\%* & 10\\
	Levitate & 15\\
	Tree Movement* & 15\\
	Climb Walls 80\%* & 20\\
	Plant Door & 20\\
	Climb Walls 90\%* & 25\\
	Dimension Door & 25\\
	Fly & 25\\
	Gaseous Form* & 30\\
	Haste & 30\\
	Move Silently 50\%* & 35\\
	Pass Plant & 35\\
	Web Movement* & 35\\
	Climb Walls 100\%* & 40\\
	Telekinesis & 40\\
	Transport Through Plants & 45\\
	Teleport & 50\\
	Climb Walls 110\%* & 55\\
	Move Silently 70\%* & 55\\
	Burrowing* & 60\\
	Plane Travel* & 65\\
	Climb Walls 120\%* & 70\\
	Move Silently 90\%* & 75\\
	Travel & 80\\
	Teleport Any Object & 85\\
	Word of Recall & 90\\
	Time Travel & 100\
  \end {tabularx}
	*See description in this chapter
\end {table}

\begin {table}[H]
  \caption{Miscellaneous Powers (Encumbrance Offset)}\label{tab:Miscellaneous Powers (Encumbrance Offset)}
  \begin{tabularx}{\columnwidth}{>{\bfseries}YY}
  \thead{Power} & \thead{Cost (PP)}\\
	Container: 5,000 cn* & 10\\
	Floating Disc & 10\\
	Buoyancy: 10,000 cn* & 15\\
	Container: 10,000 cn* & 20\\
	Container: 15,000 cn* & 30\\
	Buoyancy: 20,000 cn* & 30\\
	Container: 20,000 cn* & 40\\
	Buoyancy: 40,000 cn* & 45\\
	Container: 25,000 cn* & 50\\
	Container: 30,000 cn* & 60\\
	Buoyancy: 80,000 cn* & 60\\
	Container: 35,000 cn* & 70\\
	Buoyancy: Any Weight* & 75\\
	Container: 40,000 cn* & 80\\
	Container: 50,000 cn* & 90\
  \end {tabularx}
	*See description in this chapter
\end {table}

\textbf{Ability Score Bonus:} When activated, one or more of the wielder’s randomly determined ability scores become 18, with all attendant bonuses, for a period of one hour.

\textbf{Acid Breath:} When activated, the wielder breathes an acid breath weapon 30 feet long and 5 feet across which does damage equal to one half of the wielder’s current hit points (to a maximum of 70 damage). Creatures hit may make a saving throw vs. breath weapon to take half damage.

\textbf{Anchoring:} When activated, the wielder becomes anchored (see \fullref{sec:The Anchored, Drifters and Alts}) against any changes in the true time line. If someone changes history they will not be affected by the change.

\textbf{Anti-Magic:} When activated, the wielder gains Anti-Magic (see \fullref{sec:Anti-Magic}) at the noted percentage for one hour.

\textbf{Anti-Magic Ray:} When activated, the artifact projects an Anti-Magic ray identical to that of a Gazer. The ray lasts until the wielder stops concentrating or for 10 minutes, whichever is sooner.

\textbf{Armor Class Bonus:} When activated, the wielder gains the given bonus to their armor class for 1 hour.

\textbf{Attack Roll Bonus:} When activated, the wielder gains the given bonus to all attack rolls for 10 minutes.

\textbf{Automatic Healing:} When activated, the artifact produces a Heal spell on either its wielder or a creature that the wielder touches. If the wielder’s hit points reach 0 and the artifact has enough power left, it will automatically activate itself.

\textbf{Bearhug:} When activated, the artifact gives the user the power to make an attack with either arm (the user must be unarmed) that does no damage. However, if both attacks hit, the wielder can squeeze the target for 2d6 damage per round until the target is able to make a saving throw vs. death in order to escape. The power to make these hugs lasts for 10 minutes.

\textbf{Blasting:} When this artifact is activated, it produces a cone of sound 60 feet long and 20 feet wide at the end. All within the blast take 2d6 damage and must make a saving throw vs. spells or be deafened for 10 minutes. Building and ships in the area take 1d8 damage.

\textbf{Blend with Surroundings:} When activated, the wielder will blend with their surroundings, making them completely undetectable except via touch or via magical means. The wielder can move around while blended and stay hidden, but if they attack or cast a spell they will become temporarily visible for the round.

\textbf{Bug Repellent:} When activated, any normal or giant bug (an insect, spider, scorpion, centipede, or other arthropod) will completely ignore the wielder unless magically controlled.

If the bugs are magically controlled to attack the wielder, the wielder gets a +4 bonus to any saving throws against the controlling effect that allow the damage done by the insects to be reduced.

The protection lasts for 8 hours.

\textbf{Buoyancy:} When activated, the artifact, wielder, and anything the wielder holds will float on any liquid providing they do not weigh more than the given weight.

The wielder is not given any special power to move across the surface of the liquid.

The buoyancy lasts for 4 hours.

\textbf{Burrowing:} When activated, the wielder gains the ability to burrow through loose sand or earth at 60 feet per round, or through hard packed earth at 30 feet per round, or through solid rock at 10 feet per round for the next hour. The wielder cannot burrow through metal, and the tunnel they leave closes after an hour.

\textbf{Calm:} When activated, this artifact produces a wave of calmness that soothes the tensions of up to 40 hit dice of creatures within 120 feet. Creatures subjected to the calming effect must make an immediate reaction roll, with a +4 bonus.

\textbf{Change Odors:} When activated, this artifact will change the odors and smells in an area of 30 by 30 by 30 feet.

Poisonous vapors in the area will have their scent masked but will still be effective. The odors will fade in 1d6 x 10 minutes indoors, or 1d6 rounds outdoors.

\textbf{Change Tastes:} When activated, this artifact will change the taste of any quantity of food or liquid within 20 feet. Poisonous foods or liquids will have their taste masked but will still be effective.

The change is permanent.

\textbf{Choose Best Option:} When activated, the wielder can think of two possibilities of action and ask the artifact which is the “best” of those two.

The criteria for “best” is up to the wielder, and can be “fastest” or “least dangerous” or “likely to gain me the most money” or any other criteria the wielder decides on. The artifact will reveal the answer, taking into account the likely results of those actions over the next 10 minutes.

The artifact will not communicate anything beyond the simple choice, and will use no criteria other than the one given by the wielder. It can see no further into the future.

\textbf{Climb Walls:} When activated, the wielder gains the Climb Walls ability of a rogue at the given chance of success. This ability lasts for 2 hours.

\textbf{Communication, Greater:} When activated, the wielder of the artifact may concentrate on any living or undead creature, regardless of distance.

The target is made aware of the wielder’s desire to communicate, and—if willing—may accept contact and may converse telepathically with the wielder of the artifact for 10 minutes.

If the target is not willing to communicate, the wielder may not try to contact the same target until 24 hours have passed.

\textbf{Communication, Lesser:} As Communication, Greater except that the only creature that can be contacted is the \iref[chap:Immortals]{Immortal} who made the artifact.

\textbf{Container:} This power may not be given to the same artifact as the Life Trapping or Shelter powers.

When activated, the artifact can store non-living items that are not artifacts. and are not being touched by any living being. Any combination of items up to the given weight may be stored inside the artifact by touching the artifact to them and giving a command word. A second word will bring an item back out of the artifact into the wielder’s hand.

This power lasts for six hours when activated, and at the end of that time the wielder has the choice of either activating the power again or allowing all the objects currently being stored to re-appear.

\textbf{Control Animals:} When activated, this affects the wielder as if they had drunk a Potion of Animal Control, except that the duration is 3 hours.

\textbf{Control Dragons:} When activated, this affects the wielder as if they had drunk a Potion of Dragon Control of a type of the wielder’s choosing, except that the duration is 3 hours.

\textbf{Control Giants:} When activated, this affects the wielder as if they had drunk a Potion of Giant Control of a type of the wielder’s choosing, except that the duration is 3 hours.

\textbf{Control Greater Undead:} When activated, this affects the wielder as if they had drunk a Potion of Undead Control, except that the duration is 3 hours.

\textbf{Control Humans:} When activated, this affects the wielder as if they had drunk a Potion of Human Control, except that the duration is 3 hours and up to 40 HD of humans can be controlled, as long as no individual is over 7 HD.

\textbf{Control Lesser Undead:} When activated, this affects the wielder as if they had drunk a Potion of Undead Control, except that the duration is 3 hours and up to 40 HD of undead can be controlled.

\textbf{Control Plants:} When activated, this affects the wielder as if they had drunk a Potion of Plant Control, except that the duration is 3 hours.

\textbf{Create Magical Aura:} When activated, the artifact will bestow a magical aura on one object or one area of up to 40 by 40 by 40 feet. The aura will cause the object or area to show up on Detect Magic spells for the next 30 minutes.

\textbf{Create Normal Objects:} When activated, the artifact will create a non-magical object with a maximum weight of 1,000 cn and a maximum value of 500 gp. The object can be any mundane item that the wielder desires, and will last for 24 hours before vanishing.

\textbf{Defending:} When activated, the wielder of this artifact can apply its magical bonus to either to-hit rolls and damage or armor class for the next hour. The wielder can move the bonuses back and forth each round during the Statement of Intent phase.

This power only can only be given to artifacts if their form is that of a weapon with a magical bonus.

\textbf{De-Power:} This power can only be placed on an item if at least five \iref[chap:Immortals]{Immortals} of \nth{31} level or higher create the artifact together. Similarly, it can only be activated if at least five \iref[chap:Immortals]{Immortals} of \nth{31} level or higher all activate it at once.

When activated, the artifact can be held against an \iref[chap:Immortals]{Immortal} and that \iref[chap:Immortals]{Immortal} will be drained of 1,000,000 experience points per round with no saving throw allowed. If the target runs out of experience points, they die.

Artifacts with the de-power ability are incredibly rare, and are usually used to punish \iref[chap:Immortals]{Immortals} who have severely broken the social rules of \iref[chap:Immortals]{Immortal} society by openly meddling on the prime plane or murdering other \iref[chap:Immortals]{Immortals} or a similarly heinous crime.

\textbf{Detect New Construction:} When activated, the wielder of the artifact gains the dwarven ability to detect newly constructed stonework and traps and secret doors that involve moving blocks of stone for the next 6 hours.

The wielder does not have to roll, as this ability is automatically successful.

\textbf{Detect Slopes:} When activated, the wielder of the artifact gains the dwarven ability to detect gently sloping stonework for the next 6 hours.

Unlike a dwarf, the wielder does not have to roll. The ability is automatically successful.

\textbf{Disarm Attack:} When activated, the wielder of this artifact gains the ability to make Disarm attacks (See \fullref{sec:Weapon Abilities}) with whatever weapon they are using.

\textbf{Dodge Any Missiles:} When activated, the artifact grants the wielder the power to dodge missiles for the next 10 minutes.

Any missiles can be dodged, including missile weapons, thrown weapons, siege missiles and even Magic Missile spells.

To dodge missiles, the wielder must take a Use Non-Activatable Item action, and may then dodge up to 6 missile attacks during the round by making a saving throw vs. wands against each one.

\textbf{Dodge Directional Attacks:} When activated, the artifact grants the wielder the power to dodge missiles, rays, breath attacks and other cone-shaped attacks for the next 10 minutes.

To dodge attacks, the wielder must take a Use Non-Activatable Item action, and may then dodge up a single attack during the round by making a saving throw vs. wands against it.

This saving throw is in addition to any normal saving throw that the wielder might get against the attack.

\textbf{Dodge Normal Missiles:} When activated, the artifact grants the wielder the power to dodge missiles for the next 10 minutes.

Only normal missiles or thrown weapons can be dodged. Siege missiles and Magic Missile spells may not be dodged.

To dodge missiles, the wielder must take a Use Non-Activatable Item action, and may then dodge up to 6 missile attacks during the round by making a saving throw vs. wands against each one.

\textbf{Draining:} When activated, the artifact will drain the given number of levels from any mortal target it touches.

\iref[chap:Immortals]{Immortal} targets are drained of 15 power points per level that would be drained from a mortal.

\textbf{Elasticity:} When activated, the artifact will affect the wielder as if the wielder has drunk a Potion of Elasticity except that the duration is 2 hours.

\textbf{Electricity:} When activated, the wielder of the artifact becomes charged with energy. The next creature to attack the wielder in melee will be hit for 6d6 lightning damage, unless it can make a saving throw vs. spells to take half damage. If the creature striking the wielder is using a melee weapon rather than striking with natural weaponry then it gets a +4 bonus to the saving throw.

The artifact can be activated and de-activated as often as the wearer desires providing it still has enough power points left to release a charge, but will only actually expend power points when it releases the charge. Once the charge has been released, the artifact deactivates and must be re-activated for the charge to be used again.

\textbf{Extinguishing:} When activated, this artifact will immediately douse all non-magical fires in a 500-foot radius and prevent further fires from being lit in the area for 1 hour. It has no effect on magical fires.

If the artifact is in the form of a weapon, it does double damage against fire-based creatures while this power is active.

\textbf{Find Secret Doors:} When activated, this will allow the wielder to automatically find all secret doors that they encounter for the next hour.

\textbf{Find Traps:} When activated, the wielder gains the Find Traps ability of a rogue at the given chance of success. This ability lasts for 2 hours.

\textbf{Fire Breath:} When activated, the wielder breathes a fire breath weapon 30 feet long and 5 feet across which does damage equal to one half of the wielder’s current hit points (to a maximum of 70 damage). Creatures hit may make a saving throw vs. breath weapon to take half damage.

\textbf{Flaming:} This power can only be given to an artifact that is in the form of a weapon.

When activated, the artifact becomes a flaming weapon for the next hour.

During that time, the wielder of the artifact can activate it at will and it will burst into flames that burn without fuel until the artifact is deactivated again.

The flaming artifact has an additional +2 bonus against trolls and against feathered creatures; and an additional +3 bonus against creatures made of wood or undead. If a creature comes under more than one of these categories, these bonuses do not stack.

\textbf{Flying:} This power can only be given to an artifact that is in the form of a weapon.

When activated, the artifact becomes a flying weapon for the next hour.

During that time, the artifact must be wielded in combat for at least one round, then its wielder can let go of it and it will continue to attack the same opponent for three rounds, as if its wielder were still wielding it. After three rounds (or when its wielder commands, whichever is sooner) the artifact will return to its wielder’s hand.

\textbf{Gaseous Form:} When activated, the artifact will affect the wielder as if the wielder has drunk a Potion of Gaseous Form.

\textbf{Hear Noise:} When activated, the wielder gains the Hear Noise ability of a rogue at the given chance of success. This ability lasts for 2 hours.

\textbf{Hide in Shadows:} When activated, the wielder gains the Hide in Shadows ability of a rogue at the given chance of success. This ability lasts for 2 hours.

\textbf{Hit Point Bonus:} When activated, the artifact grants its wielder the given bonus number of hit points per hit die for 10 minutes. Any damage taken comes off the extra hit points first.

\textbf{Ice Breath:} When activated, the wielder breathes an ice breath weapon 30 feet long and 5 feet across which does damage equal to one half of the wielder’s current hit points (to a maximum of 70 damage). Creatures hit may make a saving throw vs. breath weapon to take half damage.

\textbf{Immortal Life Trapping:} This power may not be given to the same artifact as the Container or Shelter powers.

When activated, any single creature touched will be sucked into the artifact along with whatever items they are wearing and carrying. Mortal creatures may make a saving throw vs. spells to avoid the effect. \iref[chap:Immortals]{Immortals} may automatically avoid the effect if they are unwilling.

Only one creature can be held in the artifact at one time. If a second creature is trapped, this frees the first creature.

The creature trapped in the artifact can take no actions and does not age or die. However, the victim can bring an image of their face to the artifact’s surface and see and hear out of it.

While at the surface of the artifact, the creature can also speak.

\textbf{Immune to Aging Attacks:} When activated, the artifact makes its wielder immune to aging attacks from all sources for 3 hours.

\textbf{Immune to Breath Weapons:} When activated, the artifact makes its wielder immune to all breath weapons for 10 minutes.

\textbf{Immune to Disease:} When activated, the artifact makes its wielder immune to diseases from all sources for 3 hours.

\textbf{Immune to Energy Drain:} When activated, the artifact makes its wielder immune to \iref[sec:Energy Drain]{Energy Drain} attacks from all sources for 1 hour.

\textbf{Immune to Magical Detection:} When activated, the artifact makes its wielder and all the wielder’s equipment immune to any form of magical detection except Detect Immortal Magic for 1 hour.

\textbf{Immune to Paralysis:} When activated, the artifact makes its wielder immune to paralysis from all sources for 1 hour.

\textbf{Immune to Poison:} When activated, the artifact makes its wielder immune to poison attacks from all sources for 3 hours.

\textbf{Inertia Control:} When activated, artifact can make any non-living object stop. The object will be frozen in place and cannot be moved by any means short of a Wish spell.

The object will remain frozen in place for 4 hours or until the wielder of the artifact deactivates it. In either case, once the object is no longer frozen it will continue moving on its last trajectory.

\textbf{Leap:} When activated, the artifact allows its wielder to make great leaps of up to the specified distance for 10 minutes. If the wielder leaps into combat while under the effect of this power, they gain the stated bonus on their to-hit roll.

\textbf{Lie Detection:} When activated, the artifact allows the wielder to concentrate on any one mortal creature within 120 feet.

While the wielder concentrates, they will be able to tell if the creature knowingly lies.

Note that the creature saying something that is accidentally incorrect because the creature is genuinely mistaken does not count as a lie.

Once activated, this power lasts for 30 minutes, and the wielder may stop and re-start concentrating on the same mortal creature or a different one during that time.

\textbf{Life Trapping:} This power may not be given to the same artifact as the Container or Shelter powers.

When activated, any single creature touched will be sucked into the artifact along with whatever items they are wearing and carrying. Mortal creatures may make a saving throw vs. spells to avoid the effect. \iref[chap:Immortals]{Immortals} are immune to the effect even if they are willing.

Only one creature can be held in the artifact at one time. If a second creature is trapped, this frees the first creature.

The creature trapped in the artifact can take no actions and does not age or die. However, the victim can be contacted via ESP or Telepathy.

\textbf{Luck:} When activated, the wielder gains supernatural luck for 10 minutes. The player of the wielding character may choose any one roll that they make on behalf of the wielding character within that duration and simply place the dice on the result of their choice rather than having to actually roll them.

\textbf{Mapmaking:} When activated, the artifact will reproduce an accurate map of everything within a 100-foot radius. Each secret door has a 1-in-6 chance of being drawn, although the presence of some secret doors may be inferred by what is drawn behind them.

\textbf{Merging:} When activated, this artifact allows the wielder to merge other creatures into their own body. The wielder can merge up to seven other creatures, and both the wielder and the other creatures must be willing. Creatures simply step “into” the wielder and disappear along with their items and equipment.

While merged with the wielder, the creatures do not take damage if the wielder is hit, and they can not take any actions other than speaking. The merged creatures can step “out of” the wielder at any time.

\textbf{Move Silently:} When activated, the wielder gains the Move Silently ability of a rogue at the given chance of success. This ability lasts for 2 hours.

\textbf{Open Locks:} When activated, the wielder gains the Open Locks ability of a rogue at the given chance of success. This ability lasts for 2 hours.

\textbf{Parry:} When activated, the wielder gains use of the Parry ability of a fighter. This ability lasts for 1 hour.

\textbf{Pick Pockets:} When activated, the wielder gains the Pick Pockets ability of a rogue at the given chance of success. This ability lasts for 2 hours.

\textbf{Plane Travel:} When activated, the wielder and all equipment carried (but not other creatures) moves from the plane they are on to another adjacent plane.

\textbf{Poison Gas Breath:} When activated, the wielder breathes a poison gas breath weapon 30 feet long and 5 feet across which does damage equal to one half of the wielder’s current hit points (to a maximum of 70 damage). Creatures hit may make a saving throw vs. breath weapon to take half damage.

\textbf{Prepare Bonus Spells/Level:} When activated during the wielder’s normal spell preparation time, the wielder can prepare a number of extra spells as indicated of each level that the wielder can cast for the day.

Wielders who are not spellcasters gain no benefit from activating this power.

\textbf{Protection from All Creatures:} When activated, the wielder of the artifact can not be touched by any mortal creature for 1 hour.

Creatures can still use missile attacks and spells against the wielder.

\textbf{Protection from Most Creatures:} When activated, the wielder of the artifact can not be touched by any mortal creature with 15 or fewer hit dice for 1 hour.

Creatures can still use missile attacks and spells against the wielder.

\textbf{Protection from Some Creatures:} When activated, the wielder of the artifact can not be touched by any mortal creature with 5 or fewer hit dice for 1 hour.

Creatures can still use missile attacks and spells against the wielder.

\textbf{Reflection:} When cast, the wielder is protected from Light spells and gaze attacks for 1 hour. If any Light or Continual Light spell (or their reverse) is cast at the wielder during this time, the spell is reflected back to the caster.

Additionally, the wielder is treated as if they are holding a mirror for the duration in order to either attempt to reflect the gaze of a creature with a gaze attack (such as a basilisk) or to fight such a creature without meeting its gaze (in which case the normal -2 penalty for fighting while using a mirror does not apply).

\textbf{Regeneration:} When activated, the wielder re-gains 3 lost hit points per round for 10 minutes (30 rounds).

This regeneration will not help the wielder if they die, and will not re-grow body parts.

\textbf{Remove Traps:} When activated, the wielder gains the Remove Traps ability of a rogue at the given chance of success. This ability lasts for 2 hours.

\textbf{Repair Normal Objects:} When activated, the artifact will repair one mundane (not magical) object weighing up to 1,000 cn.

All parts of the object must be present for the repair to take place.

\textbf{Repair Permanent Magical Object:} When activated, the artifact will repair one permanent magic object such as a magical weapon or shield weighing up to 1,000 cn. This power will not repair an artifact.

All parts of the object must be present for the repair to take place.

\textbf{Repair Temporary Magical Object:} When activated, the artifact will repair one temporarily magic object such as a scroll or wand weighing up to 1,000 cn. This power will not repair an artifact.

All parts of the object must be present for the repair to take place.

\textbf{Rulership:} When activated and paraded through a dominion, the artifact acts as a Rod of Dominion.

\textbf{Saving Throw Bonus:} When activated, the wielder gains the given bonus to all saving throws for 1 hour.

\textbf{Security:} When activated, this artifact will temporarily enchant up to five objects to magically shout for help when stolen as if they are a Pouch of Security.

The enchantment fades after 24 hours.

\textbf{Set Normal Trap:} When activated, the artifact will create a trap with a given chance of working. The trap can be one doing up to 6d6 damage to a victim or one doing up to 3d6 damage to a victim and entangling them.

The artifact can only form the trap out of existing materials. It cannot create a trap out of nothing.

\textbf{Shelter:} This power may not be given to the same artifact as the Container or Life Trapping powers.

When activated, the wielder will be sucked into the artifact along with whatever items they are wearing and carrying.

While in the artifact, the wielder can take no actions and does not age or die; but they can rest and sleep. The wielder can bring an image of their face to the artifact’s surface and see and hear out of it. While at the surface of the artifact, the wielder can also speak.

The wielder can emerge from the artifact whenever they like, but no other power can force the wielder out or force a way in.

After 24 hours inside the artifact, the wielder must either come out or activate this power again (which they can do from the inside).

\textbf{Size Control 3 in. to 18 ft.:} When activated, the wielder can control their size for the next hour.

The wielder can shrink to as small as 3 inches tall, as if having drunk a Potion of Diminution; or grow up to 18 feet tall, as if having drunk a Potion of Growth.

\textbf{Slicing:} This power can only be given to an artifact that is in the form of an edged weapon.

When activated, the artifact gains the properties of a Weapon of Slicing for one hour.

\textbf{Smash Attack:} When activated, the wielder gains use of the Smash ability of a fighter. This ability lasts for 1 hour.

\textbf{Spell Damage Bonus:} When activated while the wielder is casting a spell that does damage (but not a spell-like power of an item the wielder carries), the spell cast by the wielder will do the given additional damage per die of damage that it does.

The wielder of the artifact does not need to take a special action to use this power. It is used as part of the normal Cast Spell action.

\textbf{Spell Generation:} This power automatically activates itself at the start of any day when the creator of the artifact is unable to provide spells to their clerics. The artifact provides the spells instead.

This means that the \iref[chap:Immortals]{Immortal’s} clerics will continue to gain spells while the \iref[chap:Immortals]{Immortal} is recovering from having an \iref[sec:Embodied Form]{Embodied Form} killed, and theoretically it means that the clerics will continue to gain spells even if the \iref[chap:Immortals]{Immortal} truly dies.

\textbf{Timekeeping:} When activated, the artifact starts tracking time. At any point it can be asked how long has passed since it was activated and it will inform the user the exact duration.

The artifact can track time from up to three activations at once.

\textbf{Tracking, Greater:} When activated, the artifact will cause a set of tracks less than 24 hours old to glow so that they can be easily followed.

The glow can be seen by the wielder of the artifact or by anyone using a Detect Magic spell. There is a 10\% chance every 240 feet (indoors) or half mile (outdoors) that the artifact will lose the trail. This chance is not affected by weather or other conditions.

The glow will last until the tracks are more than 24 hours old.

\textbf{Tracking, Lesser:} When activated, the artifact will cause a set of tracks less than 24 hours old to glow so that they can be easily followed. The glow can be seen by the wielder of the artifact or by anyone using a Detect Magic spell. There is a 50\% chance every 240 feet (indoors) or 10\% chance every half mile (outdoors) that the artifact will lose the trail. This chance is not affected by weather or other conditions.

The glow will last until the tracks are more than 24 hours old.

\textbf{Treasure Finding:} When activated, the wielder will be telepathically informed of the distance and direction to the largest amount of treasure within 360 feet. The wielder gains no insight about the nature of the treasure or how to get to it.

\textbf{Tree Movement:} When activated, the wielder will be able to swing through trees like an ape or monkey at full speed (given normal encumbrance penalties) for the next 2 hours.

\textbf{Turn Undead:} When activated, the wielder gains the Turn Undead ability of a cleric of the given level. This ability lasts for 30 minutes.

\textbf{Turn Undead Bonus:} When activated, the wielder gains the given bonus on rolls to Turn Undead for the next 10 minutes. The bonus applies to both the initial attempt to turn and the roll for how many hit dice of undead are affected.

If the wielder cannot turn undead, this power is of no use to them.

\textbf{Victory:} When activated while the wielder is in command of an army in battle, the artifact gives the army a +25 bonus to their roll for the battle, and prevents them from losing the battle roll by more than 100 points.

Any loss greater than this is treated as a 100-point loss when determining casualties and post-battle tactical positioning.

\textbf{Weapon Damage Bonus:} When activated, the wielder does the given amount of additional damage with any weapon they use for the next 10 minutes. This includes unarmed attacks.

\textbf{Weapon Strength Bonus:} When activated, any weapon the wielder uses for the next ten minutes is treated as if it had the given number of extra magical “pluses”, to a maximum of +5. This includes unarmed attacks, and it does make a mundane weapon count as magical for the purposes of what creatures it can affect.

\textbf{Web Movement:} When activated, the wielder may move freely through webs of any kind without being stuck to them for the next 2 hours.

The wielder is not given any special ability to walk up vertical surfaces or walk on ceilings just because they are webbed, unless the wielder has the ability to walk on walls or ceilings from another source.

\textbf{X-Ray Vision:} When activated, the wielder can see up to 30 feet through stone or up to 60 feet through wood. It cannot be used to see through metal.

To use the power, the wearer must stand still and concentrate, and can view a 10-by-10-foot area per activation. It takes 10 minutes to scan such an area, and the power can only be activated once per hour.

\subsection{Handicaps and Penalties}
When used by mortals, all artifacts. have handicaps and penalties. These are not part of the design of the artifact, they are merely a side effect of a mortal trying to use such a powerful item.

The creator of the artifact does not choose what handicaps and penalties will be associated with an artifact. Instead they are chosen by the Game Master.

The Game Master should try to be fair when assigning handicaps and penalties to an artifact, and assign those that seem to fit its “theme” and will not render it completely useless.

As a general rule, the stronger the theme of an artifact, the weaker the handicaps and penalties should be; whereas artifacts. that have no theme and are simply an item with collection of completely unrelated powers should be given harsher handicaps and penalties.

\subsubsection{Handicaps}
A handicap is a long-term adverse effect that affects the mortal wielder of an artifact.

The Game Master should decide when the handicap comes into play. This can be either:

\begin{itemize}
	\item{When the wielder claims the artifact as their own.}
	\item{When the wielder first uses the artifact.}
	\item{When the wielder first uses the artifact in a particular manner (e.g. first activates a particular power, or first uses it against a particular type of opponent).}
\end{itemize}

Once the handicap is active, it will remain active until a certain time (given on \fullref{tab:Artifact Power Levels}) after the wielder has lost or given away the artifact.

The Game Master should decide the exact nature of the handicap or handicaps of the artifact, and it is not possible to list all possibilities here. However, here are some suggestions:

\textbf{Doom:} The wielder of the artifact undergoes some kind of transformation or extra-dimensional imprisonment (leaving the artifact behind for another to claim) until the duration is up.

The fate that has befallen the wielder can only be discovered by a Wish, and it is not possible to recover them early.

Naturally, this handicap is suitable only for the most powerful artifacts. that mortals shouldn’t be messing with or for when an artifact is used in a way completely against its purpose.

\textbf{Transform:} The wielder of the artifact is transformed (either immediately or as a gradual process) into another type of creature; perhaps a race that the creator of the artifact is a patron of, or perhaps a race that suits the theme of the artifact particularly well.

This change cannot be dispelled, and if the wielder is polymorphed back into their original race the change will start happening again.

\textbf{Lameness:} The wielder of the artifact partially loses the use of a limb for some reason associated with the artifact. Losing the use of an arm means that the wielder may only perform actions that use one arm, and losing the use of a leg means that the wielder moves at half speed.

\textbf{Magic Disruption:} The wielder of the artifact has a chance of any spell they cast or any magical item they activate (except the artifact itself) failing. The chance could be anywhere from 10\% to 80\%.

\textbf{Operating Cost:} The artifact may require some kind of sacrifice (of magical items, of gems, or even a blood sacrifice) before a new wielder can use it for the first time, or start to use it again after having misused it.

\textbf{Recharge Cost:} The artifact’s Power Reserve will not recharge on its own. Some kind of sacrifice (magical items, gems or even a blood sacrifice) must be made each time it is to be recharged.

\subsubsection{Penalties}
Penalties are instant adverse effects that may affect the wielder of the artifact when they activate its powers. Unlike handicaps, they have no long term effect on the wielder, and can be avoided simply by avoiding activating the power or powers that trigger them.

Some penalties may be activated whenever a particular power of the artifact is used, others may only have a chance of activation each use—for example the percentage chance of activation could be equal to the power point cost of the power being used.

As with handicaps, the Game Master should decide the exact nature of penalties and when they come into play; and again it is not possible to list the endless effects that could be possible here, so only a few examples are given:

\textbf{Wounding:} The wielder takes an amount of damage.

\textbf{Aging:} The wielder is aged a number of years.

\textbf{Energy Drain:} The wielder loses one or more levels.

\textbf{Ability Score Penalty:} One of the wielder’s ability scores is temporarily reduced to 3.

\textbf{Spell Loss:} The wielder loses one or more prepared spells as if they had been cast.

\textbf{Activation Error:} The power that the wielder is activating either fails to go off or goes off on the wrong target (possibly the user themselves).

\textbf{Polymorph:} The wielder is affected as if by a Polymorph Other spell.

\textbf{Saving Throw Penalty:} The wielder takes a penalty to all saving throws for a duration after activating the power.

\end{multicols*}
