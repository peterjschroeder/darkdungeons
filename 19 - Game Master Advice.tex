\chapter[purple]{Game Master Advice}
\label{chap:Game Master Advice}
\chapterimage[Game Master Advice (Bordered)]
\thispagestyle{plain}


\begin{multicols*}{2}
Although the rest of the chapters of this book contain rules for most things that may arise during the course of a game, there may be some situations that are not covered.

This chapter contains guidelines, advice, and miscellaneous rules for handling those situations.

\section{Character Alignment}\index[general]{Character Alignment}
When players create their characters, they will usually have an idea of what that character’s personality will be like. As part of that character creation process, each player must choose an alignment for their character.

Of course, during play the character’s personality might turn out rather different than originally imagined. This may be accidental, or it may be a deliberate intent of the player to have their character’s personality shift over time. Whatever the reason, a character can end up not actually acting like their alignment states.

There is nothing wrong with this, and the player should not be penalized for playing their character “wrongly”. Instead, you should simply talk to the player and between you decide whether to change the alignment on the character sheet so that it matches the way the character is being played.

You need to be careful not to be too enthusiastic about changing the alignment of characters.

Nobody is completely consistent in their behavior, and an occasional bit of unusual behavior is normal, so you should be wary of changing a character’s alignment too often over individual acts.

In particular, if you find that a character’s alignment constantly seems to be flip-flopping between lawful and chaotic, it’s probably the best just to change the character’s alignment to neutral and leave it there.

\section{Character Conditions}\index[general]{Character Conditions}
Various effects can alter the condition of a character's body. This section describes the effects of each of these conditions.

\subsection{Blinded}\label{sec:Blinded}
Characters that are blind suffer a -4 penalty to all saving throws, -6 penalty to attack rolls, and a +4 penalty to their armor class. They can also only move at 1/3 their normal speed unless assisted by a sighted character, in which case they can move at 2/3 their normal speed.

\subsection{Deafened}
Characters who are unable to hear suffer a -1 to initiative and can not listen for danger or attack invisible foes.

\subsection{Dehydration}
After a day without water, a character will become dehydrated. For each day the character is dehydrated, they will suffer 1d8 points of damage. These hit points can not be healed in any way until the character is no longer dehydrated.

\subsection{Invisible}\label{sec:Invisible}
Most characters can not see invisible creatures without magical assistance. The character must rely on their hearing to pinpoint the location of the creature. When attacking an invisible creature, the character suffers a -6 to their attack roll.

\subsection{Helpless}\label{sec:Helpless}
Characters who are completely helpless because they are paralyzed, sleeping or unconscious may be given a Coup de Grace with any edged weapon.

This will immediately knock them unconscious (if they weren’t already) and make them start dying as if they had run out of hit points, but will not actually cause them to lose any hit points.

\subsection{Prone}\label{sec:Prone}
Characters who have fallen to the ground are easier to hit. Attackers gain a +4 bonus to their attack roll when attacking a target on the ground. Fallen characters also suffer a -4 penalty to all saving throws and -2 penalty to their attack rolls while on the ground.

A character who has fallen may spend a round standing back up. While standing up, the character may take no other actions and still suffers penalties as if they were still on the ground.

\subsection{Sluggish}\label{sec:Sluggish}
Characters who have become sluggish suffer a -2 to initiative and can only move at half their speed.

\subsection{Starving}
After a day without food, a character will begin to starve. For each day the character is starving, they will suffer 1d2 points of damage. These hit points can not be healed in any way until the character is no longer starving.

A starving character also requires more rest than usual and suffers penalties to their attack rolls. The amounts depend on the percentage of hit points the character has lost by starving and are indicated on \fullref{tab:Starving}.

\begin {table}[H]
  \caption{Starving}\label{tab:Starving}
  \begin{tabularx}{\columnwidth}{>{\bfseries}YYY}
		\thead{Percentage of HP Lost} & \thead{Rest Required} & \thead{Attack Roll Penalty}\\
		00-24 & 6 hours & None\\
		25-49 & 8 hours & -2\\
		50-74 & 10 hours & -4\\
		75-99 & 12 hours & -6\
  \end {tabularx}
\end {table}

\subsection{Stunned}\label{sec:Stunned}
Characters that are stunned cannot attack or cast spells and can only move at 1/3 normal speed. They also have a +2 penalty to armor class and a –2 penalty to all saving throws. A stunned character can make a saving throw vs. death ray each round to shake off the stun.

\section{Creating a Setting}\index[general]{Creating a Setting}
The Dark Dungeons rules are not tied to a particular setting. The exact setting of the game—and in particular the towns and countries that exist and so forth—is left for the Game Master to decide.

This may seem a daunting task, but it can be done piecemeal as a campaign is run. For example at very low levels, all that is needed is some kind of town for the adventurers to start in and a few adventure locations around it. It is only when the adventurers reach a higher level and have a need or a wish to go exploring that the Game Master needs to know what lies beyond the mountains.

However, the Dark Dungeons rules do make various assumptions about the setting that the Game Master must bear in mind, since although the player characters might not interact with them much at lower levels, their presence in the world deeply affects the way the world works and the effects of that should be visible to the players in the background.

If this is not taken into account, the sudden introduction of elements such as skysailing or \iref[chap:Immortals]{Immortals} into the campaign because the adventurers are now “ready for them” will be jarring, as such things should have been around all the time.

In particular, the following key parts of the assumed setting may have a large influence on the feel of the campaign:

\begin{itemize}
	\item{Guns exist using \iref[eq:Red Powder]{Red Powder} to fire.}
	\item{Ships equipped with a \iref[eq:Sail of Skysailing]{Sail of Skysailing} can fly at high speed, connecting major cities around the world in a trade network.}
	\item{\iref[chap:Immortals]{Immortals} play an active, yet subtle, part in the world’s politics and events.}
	\item{There is likely to be more than one inhabited planet in the \iref[sec:The Celestial Sphere]{Celestial Sphere}, and ships can fly between them.}
	\item{There are an effectively infinite number of \iref[sec:The Celestial Sphere]{Celestial Spheres} out there to explore, each containing its own worlds.}
	\item{Travelers, settlers or invaders of unusual or never-before-seen races can come from other worlds.}
\end{itemize}

All these parts of the assumed setting have been deliberately designed to be modular, so that the Game Master can choose not to make them part of their specific setting.

For example, if the Game Master doesn’t want travel through the \iref[sec:Luminiferous Aether]{Luminiferous Aether} to be part of the game, they can simply say that only one Celestial Sphere exists and that there is no way to leave it. \iref[sec:The Celestial Sphere]{The Celestial Sphere} and its associated \iref[sec:The Inner Planes]{Inner Planes} and \iref[sec:The Outer Planes]{Outer Planes} are effectively the whole multi-verse.

Similarly, the Game Master could go a step further and simply say that travel through the \iref[sec:The Void]{Void} is impossible—effectively limiting the campaign world to the single planet; or even say that a \iref[eq:Sail of Skysailing]{Sail of Skysailing} don’t exist in the world, limiting large scale movement and communication around the world and effectively reducing the campaign setting to a single continent or part of a continent.

Either \iref[chap:Immortals]{Immortals} or \iref[eq:Red Powder]{Red Powder} can be dropped from the game very easily if the Game Master doesn’t like their effect on the tone of the campaign. In the case of \iref[chap:Immortals]{Immortals}, this would simply mean ruling that clerics get their power straight from whatever gods exist, or that they get their power from the strength of their faith.

However, in any of these cases, the decision to drop the item from the setting should be taken (and should be discussed with the players in order to manage their expectations of the campaign) before the campaign starts.

Although these elements of the setting may see little use in a low level game (low level characters are unlikely to travel off the planet, for example) their presence—or lack of it—will shape the social and political structures of the world to an extent, and should therefore be consistent throughout the campaign.

\end{multicols*}

