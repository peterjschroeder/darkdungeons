\chapter[green]{Other Worlds}
\label{chap:Other Worlds}
\chapterimage[Other Worlds (Bordered)]
\thispagestyle{plain}

\begin{multicols*}{2}
The mundane world of castles, cities and forests is not the sum total of existence.

While it may seem so to the average commoner, seasoned adventurers know that there are a multitude of other places out there, from other planets to alternate realities. High level adventurers will likely have fought or dealt with creatures from such places, and may well—with the right magic—visit those places themselves.

This chapter is reliant on Game Master’s discretion more than most. Things like the number of planets orbiting the sun and the number of Outer Planes (both of which are described in this chapter) are very dependent on the campaign setting, and therefore this chapter must be necessarily vague about such things.

\section{Overview}
The planet that adventurers live on is just that—a planet. It flies through the Void as it orbits the sun. Like many planets, it’s basically a big rock surrounded by air and a bit of water. At least, that’s the assumption that these rules make. Individual Game Masters can set campaigns on very different planets if they want to.

Of course, the adventurers do not live on the only planet out there. Dark Dungeons assumes that there are a few planets orbiting the sun. This being a magical world and not our real world, some of these planets are almost always inhabitable (although not always inhabited).

The whole solar system sits in the Void—an empty nothingness much like our real-world space—but the Void does not stretch forever. The solar system is inside a giant sphere known as a Celestial Sphere. The Celestial Sphere is hundreds of millions of miles across, capable of fitting the orbits of all the planets inside it.

The Celestial Sphere appears to be made of a dark, smoky crystal or glass that is completely impervious to any physical force. Dotted about the sphere are huge transparent “windows” which let in light.

These windows are visible from the surface of the planets inside the sphere as stars. These windows, however, are just as impervious as the rest of the sphere.

Huge though it is, the sphere is not the only nearby “place”. There are also parallel realities called Planes.

The normal world is often called the \hypertarget{sec:Prime Plane}{Prime Plane}, since it is the only one which is infinite in size, and it is the plane on which all others are anchored, directly or indirectly.

The other planes are split into two categories—Inner Planes and Outer Planes.

The Inner Planes are truly parallel to the Prime Plane, or at least to the Celestial Sphere. Each inner plane is exactly the same size as the sphere, and has a sun and planets in the exact same positions and orbits as those in the Prime Plane sphere. There are five of these Inner Planes—four Elemental Planes (Fire, Earth, Air and Water) and an \iref[sec:Ethereal Plane]{Ethereal Plane}. These planes have a one-to-one correspondence with the Prime Plane in that every point on the Prime Plane has a corresponding point on each of the five Inner Planes. If you travel from the Prime Plane to The Elemental Plane of Water, then sail 20 miles, and then travel back to the Prime Plane; you’ll end up 20 miles from where you started.

Outer Planes, on the other hand, are different. These planes are created by \iref[chap:Immortals]{Immortals}, and can be as varied as their creators’ whims dictate.

One thing they have in common, however, is that each one has a single point where it is “anchored” onto another plane (often the Prime Plane, but theoretically any plane will do). The anchor point is the only point with a fixed correspondence to the plane that it is anchored to. If you travel to an outer plane at the anchor point and then walk, fly or sail 20 miles, you simply won’t be able to travel back without either returning to the anchor point or using magic such as a \iref[spell:Gate]{Gate} spell.

And that is not all...

The Celestial Sphere itself can be penetrated with powerful magic. Outside the sphere is a strange glowing substance, if substance is the right word, called the Luminiferous Aether. Sail or fly through this for long enough and you’ll encounter other Celestial Spheres—each of which will contain its own unique set of planets and will have its own Inner Planes and Outer Planes attached.

However, all planes are attached to the Prime Plane inside Celestial Spheres. In the Luminiferous Aether there are no other planes.

The rest of this chapter looks at these different locations and phenomena, how to survive in them, and how to travel to and through them.

\section{The Void}\index[general]{The Void}\label{sec:The Void}
The void is the empty space between worlds. As the name suggests, it is completely empty, not even containing air.

\subsection{Gravity}
There is no gravity in the void away from large masses, so any unsecured object will simply drift randomly. Because of the lack of air, creatures with wings will find it impossible to fly; although magical flight still works.

Any mass in the void, from the smallest pebble to the largest planet or sun will have its own gravity. However, unlike our world this gravity does not always pull in the direction of the center of the mass and is not directly proportional to the mass of the mass(es) involved.

Instead, the Strength of the gravity around an object or group of objects is always the same; the Distance over which that gravity acts is limited and based on the size of the object or group of objects; and the Direction of the gravity is based on the shape of the object.

\textbf{Strength:} The strength of gravity is always simply the normal strength of gravity that is found on the real-world Earth. Within the gravity envelope of a planet, big or small, things fall just as expected.

\textbf{Distance:} The distance away from an object to which its gravity extends is based on the width of the object in the direction in which the distance is being measured. That’s less complicated than it sounds. Basically it means that the gravity envelope for an object stretches as far above the object as the object is tall, as far to either side of the object as the object is wide, and as far in front of and behind the object as the object is deep. The gravity envelope is therefore the same shape as the object but three times the size (and 27 times the volume).

However, there is a limit to the size of the gravity envelope produced by an object. The envelope will never be more than fifty miles deep no matter how large the width of the object is.

\example{The Game Master has decided that the campaign is primarily set on a planet the size of Earth—a spherical planet of approximately 8,000 miles diameter. In theory, the gravity envelope of this planet would therefore also be spherical and stretch 8,000 miles in every direction. However, gravity envelopes cannot be more than 50 miles deep, so instead it spherical and stretches 50 miles in every direction.

The party are traveling through the void in The Black Swan—a skiff that has been equipped with a Sail of Skysailing. The Black Swan is 45 feet long, 15 feed wide and 8 feet tall. The gravity envelope of this ship is therefore an area 135 feet long, 45 feet wide and 24 feet tall. In theory, there is a thin bit of the envelope that sticks up in the middle where the mast of the ship sticks up, but practically that can be ignored for most purposes.}

\textbf{Direction:} The direction of gravity within a gravity envelope is determined by the shape of the object that is responsible for the envelope. Each gravity envelope will have a consistent direction of gravity throughout, and if the shape of the object responsible for the envelope changes there may be a sudden flip from one gravity direction to another. Different parts of the gravity envelope will never have different directions.

The three most common directions are point, plane and line gravity.

If the object responsible for the gravity field is roughly spherical, such as a planet, then throughout the gravity field gravity pulls towards a point in the center of the object. This means that it is possible to walk all round the sphere without falling off.

If the object responsible for the gravity field is roughly cylindrical, then throughout the gravity field gravity pulls towards the central line of the cylinder. This means that it is possible to walk all around the cylinder without falling off, but it is possible to fall off either end of the cylinder. Anyone doing so would oscillate back and forth until they settled next to the end of the cylinder lined up with the mid point. The gravity envelope would prevent them from “falling” further, but there would be nothing to stop them drifting away in a direction perpendicular to the cylinder’s end.

If the object responsible for the gravity envelope is roughly flat, such as a ship, then throughout the gravity envelope gravity pulls towards a plane that cuts through the middle of the object. This means that it is possible to walk around on the upper deck(s) of the ship without falling off, and it is also possible to walk around on the underneath of the ship’s hull without falling off. However, it is still possible to fall off the side of the ship, and someone doing so would oscillate back and forth until they settled next to the side of the ship lined up with the mid plane of the ship. The gravity envelope would prevent them from “falling” further, but there would be nothing to stop them drifting either along or away from the ship whilst staying on that same plane.

In the case of large objects with unusual shapes, these may have different gravity directions at the Game Master’s discretion. For example a large hollow sphere might have gravity pointing towards the middle of its thickness, meaning that the inside and outside of the sphere can both be walked on; or a large doughnut-shaped object might have gravity pointing to a ring through its center, so that it can be walked around without falling off.

If in doubt, use the direction of gravity that seems the most sensible and convenient.

In any case, when a small object enters the gravity envelope of a larger object, the smaller object takes on the gravity envelope of the larger object.

\example{The Black Swan, like most ships, is roughly flat; so its gravity points in the direction of its central plane. People can stand on both its deck and the underneath of its hull without falling off.

When the ship approaches a planet in order to land, it will take on the gravity envelope of the planet. The safest way for Aloysius to manage this transition without everyone falling off is to maneuver the ship so that its hull is facing the planet. That way gravity will still be pointing “down” through the hull and people on the decks will not fall off.}

\subsection{Air and Breathing}
The void itself contains no air of any kind. However, the objects within the void will normally be surrounded by an envelope of air that clings to them because of their gravity envelope. The air envelope is generally the same size as the gravity envelope.

Unfortunately for travelers, the air around a creature or object will go “bad” and become unbreathable if it is not replenished regularly. Large planets don’t have this problem—partly because of the sheer size of the air envelope around them and partly because they have whole ecosystems constantly replenishing the air.

Air quality is divided into three levels of quality: fresh, fouled, and dead.

\textbf{Fresh Air:} Fresh air is healthy and normal. Creatures can survive and operate in it without problem.

\textbf{Foul Air:} Foul air is unhealthy, humid and smells bad. Creatures can survive in it, but will often be short of breath and take a -2 penalty to all actions that require rolls.

\textbf{Dead Air:} Dead air can no longer support creatures at all. Creatures trapped in dead air will suffocate to the point of unconsciousness over the course of 2d6 rounds and then die in another 1d4 minutes. Before unconsciousness sets in, creatures in dead air take a -4 penalty to all actions that require rolls.

A character who is about to knowingly enter an area of dead air can hold their breath for a number of rounds equal to their \iref[sec:Constitution]{Constitution} before symptoms start.

The exact time it takes for a creature to foul its own air envelope to the point where it drops a level in quality (if drifting in the void) depends on a multitude of factors and variables; and is best abstracted to a roll of 2d10x10 minutes. Larger creatures take more air with them, but also use more air; so the result is the same.

Undead, constructs, golems, and \iref[chap:Immortals]{Immortals} (unless in Mortal Form) do not foul the air that they carry around, since they do not need to breathe, and they also take no penalties for being in fouled or dead air.

In the case of people on a rock or on a ship, the object that they are on will provide a much bigger air envelope, so it will last much longer before becoming foul. Although the amount of air in the envelope would be most accurately determined by the exact shape and size of the object and many other factors; it is best abstracted by the following (particularly for ships equipped with a Sail of Skysailing):

For each ton of weight of the object, it’s air envelope will last one person 120 days. Horses and large creatures use the same amount of air as two people (or even more in the case of extremely large creatures).

\example{The party traveling in the Black Swan consists of four people. The ship weighs 5 tons (it’s a skiff), and can thus support 5 x 120 = 600 person-days. Therefore, it will take 600/4 = 150 days for the air to become fouled and a further 150 days for the air to become dead. The ship cannot safely travel for more than 300 days (with this crew) before landing on a planet to refresh its air supply.}

The \iref[spell:Create Air]{Create Air} spell always maintains fresh air within its area of effect for the duration of the spell, but once the duration runs out, the air returns to its former fouled or dead state.

A person under the effects of a \iref[spell:Survival]{Survival} spell does not suffer the effects of foul or dead air for the duration of the spell.

\subsection{Movement and Travel}
Getting to the void is easy. It can be done by simply flying up for long enough to reach the edge of the planet’s air and gravity envelope.

However, assuming that the campaign is set on an Earth sized planet, this envelope will be 50 miles deep; so while creatures flying under their own power may be able to fly that distance they are unlikely to fly that high by accident.

Ships equipped with a Sail of Skysailing can easily fly such distances. However, they cannot necessarily fly at full cruising speed since they are fighting gravity all the way. It takes a ship a full hour to reach the edge of the gravity envelope regardless of the effective level of the pilot.

While doing such a take-off or landing, the ship can do other navigation at the same time. It can travel around the planet at normal cruising speed while ascending or descending.

Once outside of a gravity envelope, any form of magical flight will work as normal, although winged flight will not unless the creature’s description specifically says that it can fly through the void. Winged creatures that do fly high enough to reach the edge of the gravity and air envelopes will be able to feel that they have reached the edge and non-intelligent ones will instinctively go no higher.

Once in the void, the main difficulty with traveling through it is that the distances are so immense. The distance from one planet to another may be anywhere from 36 million miles to 3.5 billion miles or more; and the radius of a Celestial Sphere can be up to 7 billion miles. Getting from planet to planet is therefore only possible with magic designed for that purpose. Normal magical flying effects or items are far too slow to even attempt the journey. Even a simple trip to a planet’s moon is likely to be at least 200,000 miles.

Ships equipped with a Sail of Skysailing are one of the few commonly found magical effects designed for such long distance travel.

The speeds listed for such ships in \fullref{sec:Waterborne Movement} are the speeds within a gravity envelope. Once a skysailing ship has escaped the gravity envelopes of nearby planets and is in the open void, it can accelerate to Voidspeed. Unlike the speeds achieved in atmosphere, voidspeed is not dependent on the effective level of the pilot of the ship. It is fixed at 100 million miles per standard 8-hour travel day. As with air travel, the pilot of the ship can do a “double shift” if the need arises. See \fullref{sec:Waterborne Movement} for more details.

This immense voidspeed can only be maintained in a straight line. Any need to maneuver will cause the ship to drop to normal air speed. Similarly, entering the gravity envelope of another object will cause the ship to drop to normal air speed.

No encounters are normally had in the void, since the distances and speeds involved are so huge that the odds of two ships actually coming close enough to even detect each other are astronomically low.

When traveling from planet to planet, it is not necessary to calculate the exact orbits of each planet in order to find out the exact distance between them. Instead, simply use a standard “average” distance between each planet.

\example{Aloysius is piloting the Black Swan home after visiting another planet. The two planets are the same distance apart as the Earth and Mars—about 50 million miles.

Aloysius flies the Black Swan straight towards home at voidspeed, and arrives at the edge of the gravity envelope (50 miles above the surface) in 4 hours. He then maneuvers the ship so that its hull is facing the planet, and lowers it into the atmosphere. The time taken to descend through atmosphere is always 1 hour, so it takes a further hour to bring the Black Swan down to within a few hundred feet of the ground.

While descending, Aloysius checks on his maps to see whereabouts on the planet he is, and then sets off at cruising speed to get to the skyport of his choice. When he has both arrived and finished descending (whichever takes longer) he will switch to maneuvering speed in order to land carefully.}

\section{The Celestial Sphere}\index[general]{The Celestial Sphere}\label{sec:The Celestial Sphere}
Although incredibly huge, the void is not endless. It is bounded by a shell of crystalline material called the Celestial Sphere.

The sphere is impervious to any physical or magical damage, even from \iref[chap:Immortals]{Immortals}.

Assuming your campaign world is not terribly unusual, the sphere will contain either a sun in the center with a number of planets orbiting it, or a central planet that is orbited by a sun and one or more other moons and planets.

In either case, the radius of the sphere will always be at least twice the radius at which the outermost planet orbits.

\example{If the campaign was set on a planet like Earth (third planet of nine from the sun, if we include Pluto), the radius of the Celestial Sphere should be at least twice the radius at which Pluto orbits.

Pluto orbits at a radius of 3.6 billion miles from the sun, so the sphere would have a radius of at least 7.2 billion miles.

At standard Voidspeed, it would take a ship 72 days to reach the edge on average.}

Dotted around the sphere are glowing points of light that can be seen as stars from the various planets in the sphere. These are a part of the sphere, and cannot be moved or damaged in any way. However, they are useful for navigation, and provide a dim light in the absence of a sun.

\subsection{Astronomical Bodies}\index[general]{Astronomical Bodies}
The astronomical bodies—planets and sun(s)—in the sphere can be loosely categorized into four types, corresponding to the four states of matter:

\textbf{Gaseous Body:} A gaseous body is a gas planet. The body of the planet is made of air or some similar gas, and it is not usually possible to tell where the air envelope stops and the planet itself starts. A gaseous body may have a small core at the center, composed of debris that has accumulated there, but generally it is entirely composed of air. A gaseous planet usually has extremely violent weather patterns.

\textbf{Solid Body:} A solid body is a solid planet like our own Earth. It will be primarily composed of rock, and have a normal atmosphere. Dark Dungeons assumes that the campaign is set on a solid body.

\textbf{Radiating Body:} A radiating body is a sun. There is usually one per sphere, and it is often in the center. A sun provides light for the whole sphere; without a sun the sphere is dark and lit only by the stars. However, a sun is not necessary to provide heat in the Void. A sphere without a sun will be cold enough to freeze water, but will still be warm enough to survive in.

If the gravity and air envelope of a radiating body is entered, treat it as if on the Elemental Plane of Fire, with the exception that there is no ground. The planet is usually fire all the way to the center, since any debris that would accumulate there will typically burn up.

\textbf{Liquid Body:} A liquid planet is basically a large blob of water or a similar liquid, forming a planet that is entirely sea. It may have a small core made of debris, and may also have small islands of floating debris, possibly even with settlers or refugees living on them. The atmosphere of a water planet is normal.

In a sphere without a sun, a liquid planet will be a frozen ball of ice and snow rather than liquid water.

\section{The Inner Planes}\index[general]{The Inner Planes}\label{sec:The Inner Planes}
Inside the Celestial Sphere, there is not just the mundane world (known as the \ilink{sec:Prime Plane}{Prime Plane}). There are also parallel worlds that take up the same space as the mundane world. Such parallel worlds are called Planes, and the five that always take up the same space as the inside of a sphere are referred to as the inner planes. Each of them is finite in size and surrounded by the same crystalline sphere.

There is an exact mapping between the inner planes and the prime plane, in that each inner plane has its own copies of the astronomical bodies contained in the prime plane, and these bodies contain the same rough geographical features (mountain ranges, seas, etc.) as those on the prime plane. However, artificial structures and vegetation will not be mirrored.

The correspondence between the planes extends to movement as well. If someone travels from the prime plane to the \iref[sec:Ethereal Plane]{Ethereal Plane}, travels north for a mile, and then returns to the prime plane; they will arrive one mile north of their starting position.

\section{The Ethereal Plane}\index[general]{The Ethereal Plane}\label{sec:Ethereal Plane}
The Ethereal Plane is the plane most often visited by adventurers. It directly touches the prime plane at all points, which makes travel relatively easy (for example an adventurer under the influence of a \iref[spell:Travel]{Travel} spell can move between the ethereal and prime planes at any location).

Everything on the Ethereal Plane is made from ether, which is a kind of sticky gray ectoplasm that looks like dense smoke and feels cold and clammy to the touch. Ether can only exist on the Ethereal Plane. If it is brought off that plane onto any other plane it simply evaporates and vanishes leaving no trace.

The Ethereal Plane is constantly lit by a dim light, although visibility is equivalent to shadowy torchlight. There are no actual shadows, since the light penetrates the entire plane, but any ability that requires shadows to work will work in the dimly lit ethereal. There is no color in the Ethereal Plane as the ether is all gray; although denser ether is more whitish. The only colors that can be seen are on objects or creatures that have entered the Ethereal Plane from other planes. This often makes such creatures and objects stand out vividly despite the low light.

While all the Inner Planes touch the prime plane, the Ethereal Plane actually overlaps to some extent. The ether is attracted to matter on the prime plane, and becomes more dense in the ethereal equivalent of the location, being least dense where there is Void on the prime plane and most dense where there is metal.

Because of this effect, it is possible to “see” the shapes of things that are on the prime plane from the Ethereal Plane by seeing the patches of dense ether that correspond to them.

\example{Black Leaf is scouting well ahead of the rest of the party when she triggers a trap. The door to the room she is in slams shut and spikes protrude from the ceiling, which then starts inexorably lowering.

After failing to pick the lock on the door, Black Leaf decides that it’s too risky to wait and see if the rest of the party can rescue her, and she drinks a Potion of Ethereality that she has been saving for emergencies.

Black Leaf enters the Ethereal Plane, and finds herself in what appears to be the same room with the same descending spiked ceiling, except that everything is made out of ectoplasm. The ectoplasmic ceiling continues to descend, following the descent of the real ceiling on the prime plane, but Black Leaf simply pushes through it unharmed.

Satisfied that she has escaped danger, she wades slowly through the ectoplasm back to the door. The real door is still locked, but once again she can push through the ectoplasmic copy of it and emerge back outside in the corridor.

Black Leaf then returns to the prime plane, and appears unharmed in the corridor outside the room. To any observers who can’t see invisible creatures, she would have apparently disappeared when drinking the potion and then reappeared outside a minute or so later.

Creatures who could see invisible would have been able to see her shadowy form as she moved around on the Ethereal Plane.}

\subsection{Air and Breathing}
Although there is no actual air or water on the Ethereal Plane. both air and water breathers alike can breathe the etheric ectoplasm with no ill effect.

\subsection{Movement and Travel}
Getting to the Ethereal Plane sually requires a \iref[spell:Travel]{Travel} spell (Ethereal Plane is adjacent to the prime plane) or a \iref[spell:Gate]{Gate} spell. However, there are certain magic items that provide a specialized form of the \iref[spell:Travel]{Travel} spell that allows access to the Ethereal Plane but no other planes.

Movement in the Ethereal Plane is tricky at best, since even the most solid “ground” still gives. Walking is possible, but can only be done at half speed, since walking on the soft ether is like trudging through loose sand.

However, this softness does have its advantages. It is possible for a solid being visiting the Ethereal Plane to push through the ethereal representations of solid prime plane objects. Pushing through dense ether in this way reduces movement speed to one quarter of normal.

Flying (both winged and magical) can be done in the ether at normal rates through the less dense ether that corresponds to prime plain water air or Void, and can be done at half speed through the denser ether that corresponds to solid prime plane matter.

A second consequence of the overlap between the prime plane and the Ethereal Plane is that creatures able to see invisible things (whether through an innate ability or through a \iref[spell:Detect Invisible]{Detect Invisible} spell) can see creatures or objects on the Ethereal Plane.

\section{The Elemental Planes}\index[general]{The Elemental Planes}\label{sec:The Elemental Planes}
In addition to the \iref[sec:Ethereal Plane]{Ethereal Plane}, there are four elemental planes touching the prime plane within the Celestial Sphere. These are the elemental planes of air, earth, fire and water.

As with the \iref[sec:Ethereal Plane]{Ethereal Plane}, there is a direct mapping between all points on each of these planes and the corresponding points on the prime plane.

Each elemental plane contains Void just like the prime plane, and contains copies of the same astronomical bodies as the prime plane. However, unlike the prime plane versions of these bodies, all matter in an elemental plane is composed of only a single element, in different states that simulate the other elements as far as possible. In all cases however, the elemental version of the Void is still simply Void.

Each of these elemental versions of the astronomical bodies has the same basic geographic features (mountains, rivers, seas and so on) as the equivalent prime body; but vegetation and artificial structures are not represented.

\example{Elfstar is standing by her house, which is by a stream in a forest clearing. She casts a Survival spell followed by a Travel spell and moves to the Elemental Plane of Water.

When she arrives, she is standing on ice rather than soil, and neither the house nor the trees are there.

The stream is still there, however, and through the water vapor that makes up the sky Elfstar can see a brightly shining white sun that appears to be made of steam.}

The elemental equivalents of the other elements are as follows:

\textbf{Air:}\label{sec:Elemental Plane of Air} The ground is made from soft but solid clouds, which halve the movement rates of any land based creature. However, these clouds are solid enough that they can’t simply be pushed through.

The atmosphere is clear air, and bodies of water are made from a smoky vapor that settles in depressions like liquid. The “liquid air” is viscous like water, and will support swimming creatures—although it can be breathed by air breathing creatures (but not water breathing creatures) without drowning.

The sun and other large natural fire sources are made from balls of lightning.

The soft nature of the ground and the fact that air breathing creatures cannot drown in the seas and rivers actually makes the Elemental Plane of Air a fairly safe place.

\textbf{Earth:}\label{sec:Elemental Plane of Earth} The ground is, naturally, normal earth and rock. The atmosphere is made entirely of floating dust motes, which force most travelers from the prime plane to have to cover their mouths and noses with scarves or wraps to avoid choking. Even without choking on the dust, it still cannot be breathed, so travelers need to either be able to go without air or to have some means of creating air.

The seas on the Elemental Plane of Earth are composed of a fine silty sand that behaves much like a liquid. Needless to say, it is just as inhospitable to air (and water) breathers as the atmosphere is.

The sun and other large natural fire sources are composed of shining crystals, which—although they glow brightly—do not give out appreciable heat.

The biggest problem for travelers to the Elemental Plane of Earth is breathing. Other than that, the plane is relatively safe to explore.

\textbf{Fire:} The Elemental Plane of Fire is naturally extremely hot. The ground is made from red-hot glowing coals and ash, and the seas are made of runny lava with swirls of molten metal through it. The lava is soft enough to swim in (assuming you can take the heat) but neither air nor water breathers can breathe it.

The atmosphere of the Elemental Plane of Fire is comprised entirely of flame. Although the flame will scorch the lungs of any who try to breathe it, those who are protected from the heat find that it is actually breathable by air breathers.

The sun and other large natural fire sources are, of course, simply white-hot fire.

At first glance, the Elemental Plane of Fire appears to be the least hospitable of the elemental planes. However, once the problem of heat is overcome by some kind of magical protection, it is not too bad. The atmosphere can be breathed, and the ground is solid.

\textbf{Water:}\label{sec:Elemental Plane of Water} On the Elemental Plane of Water, the ground is primarily made of ice and snow. Seas and rivers are made of clear fresh water.

The atmosphere in the Elemental Plane of Water is comprised of pure water vapor, and air breathing creatures who try to breathe it will drown. Water breathers can breathe it with no difficulty.

The sun and other large natural fire sources are made from glowing clouds of steam. These steam clouds are warm to the touch, but nowhere near as hot as real fire.

Providing travelers can breathe water, and wrap up well to survive the cold, the Elemental Plane of Water is a fairly safe place to explore; although the constant moisture can make travelers feel incredibly uncomfortable after a while.

\example{The Game Master has decided that the Celestial Sphere in which the campaign is set contains a sun (radiating body), two normal “earth-like” worlds (solid bodies), a water world (liquid body) and three gas giant planets (gaseous bodies).

The corresponding Elemental Plane of Air contains the same seven astronomical bodies, except that the sun is made of lightning; the earth-like worlds are made of solid clouds (with atmospheres and seas); the water world is made of vaporous smoke (also with an atmosphere); and the gas giants are basically big blobs of air.

On the Elemental Plane of Fire, the same bodies also exist. This time the sun is much like the prime plane’s sun; the earth-like worlds are large balls of cinder and coals surrounded with an atmosphere made of flames; the water world is made of lava and molten metals and has a similar flaming atmosphere; and the gas giants are simply large balls of fire.}

\subsection{Air and Breathing}
On the Elemental Plane of Air, breathing can be done normally, even in the smoky vapor that passes for liquid in the plane’s seas.

Breathing on the Elemental Plane of Earth requires both a \iref[spell:Create Air]{Create Air} spell or the equivalent and also blocking of the nose and mouth with cloth to prevent choking on the dust.

Trying to breathe in the Elemental Plane of Fire requires a \iref[spell:Resist Fire]{Resist Fire} spell or the equivalent in order to prevent taking 2d6 damage per round from the heat.

Providing that protection is in place, the fiery atmosphere of the plane can be breathed without problem.

Breathing in the Elemental Plane of Water requires either a Water Breathing or \iref[spell:Create Air]{Create Air} spell or the equivalent. However, either of those spells will allow breathing both in the water vapor that makes up the atmosphere and also the seas and rivers of the plane.

\subsection{Movement and Travel}
All four of the elemental planes touch the prime plane at all points, so a \iref[spell:Travel]{Travel} spell can take you from the prime to any one of the four or vice versa.

Because of the direct mapping between the planes, traveling to one of them and walking a mile north before returning will return you to the prime plane one mile north of where you left.

Traveling to the elemental planes while underground is emphatically not recommended, since whatever tunnels have been dug on the prime plane are highly unlikely to have also been dug on the elemental planes, and the unwary traveler is likely to appear in solid rock (or the equivalent) and be instantly killed.

Once on the elemental planes, movement and travel is usually no different to traveling on the prime plane. The only exception being that land movement is slowed to half normal rates on the Elemental Plane of Air due to the soft nature of the ground.

\section{The Outer Planes}\index[general]{The Outer Planes}\label{sec:The Outer Planes}
Outer planes, sometimes referred to as demiplanes, are small planes that sit outside the normal prime-ethereal-elemental plane structure.

Each outer plane is unique, although they can be no bigger than the size of a large planet plus atmosphere, and most are much smaller, down to the size of a large house or inn.

Unlike the Elemental Planes which mirror the structure of the prime plane, each outer plane has been deliberately created by one or more \iref[chap:Immortals]{Immortals} for a reason, and therefore there is no fixed structure to them.

Having said that, most outer planes have the same sorts of matter and rules of gravity and so on as the prime plane. Even \iref[chap:Immortals]{Immortals} find comfort in familiarity, after all.

The outer planes do not map exactly to the prime plane. Instead, there is a single location (with a radius of about 100 feet or so) on each outer plane which is its Anchor Point. This anchor point maps to a similar sized region on another plane—which can be any other plane; prime, inner or outer.

Within that region, \iref[spell:Travel]{Travel} spells will work between the outer plane and the plane that it is attached to, and the usual behavior of mapped planes is exhibited (i.e. traveling from the base plane to the attached outer plane, walking 50 feet to the left, and then returning to the base plane will bring you to 50 feet left of where you started).

Outside of that region, however, \iref[spell:Travel]{Travel} spells will not work, and the only way to move to other planes is via a Gate or the equivalent.

Since outer planes can be anchored onto other outer planes, it is possible for a whole “tree” of outer planes to end up being attached to the prime plane in a given location.

\example{After becoming Immortals, five characters work together to create a single plane and anchors it to the Prime Plane. Each of them then also creates a home plane for themselves and each anchors that plane to their shared plane. Finally, one of the Immortals wishes to experiment with creating new types of undead so she creates another plane in which she can perform her experiments and anchors it to her home plane.

	If someone wanted to travel from the prime plane to the plane where the experiment is being conducted using a \iref[spell:Travel]{Travel} spell, they would have to first locate the anchor point of the shared plane and then cross over onto that plane. From there, they would have to explore that shared plane until they find the anchor point of the Immortal’s home plane, and cross over once again. Finally, they would have to explore the Immortal’s home plane until they found the anchor point of the plane where the experiment is being conducted and cross over a third time.

If it were one of the other Immortals trying to get to the plane from their own home, they wouldn’t be able to travel directly there either. They too would have to go to the anchor point of their home plane and cross over to the shared plane, then find and cross the other two anchor points from there.}

Much as some \iref[chap:Immortals]{Immortals} may like to hide away, all outer planes must have an anchor point connecting them to an existing plane, and “circular” anchors are not possible (i.e. it is not possible to anchor plane A to a location on plane B, then anchor plane B to a location on plane C, then anchor plane C to a location on plane A). All outer planes must be anchored—directly or indirectly—to the prime plane or one of the Inner Planes.

Needless to say, most anchor points are well guarded on one side or the other, and traffic at those points is either forbidden or monitored carefully.

However, there are many outer planes that have been abandoned or simply forgotten. Such forgotten planes may well be unguarded, but the things they contain may be dangerous in their own right.

Since traveling from adjacent plane to adjacent plane can be time-consuming and difficult, it is common for an \iref[chap:Immortals]{Immortal} to create a permanent Gate between planes that they travel between often. \iref[spell:Gate]{Gate} spells can link any two planes for one-way or two-way traveling. The planes don’t have to be adjacent to one another; they can even be anchored to two different Celestial Spheres. However, \iref[spell:Gate]{Gate} spells still go from only a single fixed point on one plane to a single fixed point on the other plane. This can be either an advantage or a disadvantage, since it is even easier to defend a gate than it is to defend an anchor point.

\section{Luminiferous Aether}\index[general]{Luminiferous Aether}\label{sec:Luminiferous Aether}
The Celestial Sphere is not the entire universe. Beyond the sphere’s edge lies an expanse of a glowing swirling substance called Luminiferous Aether.

The Luminiferous Aether is chaotic mix of raw elemental matter and ether. It has no actual density or substance to it, and permeates everything outside the Celestial Sphere. However, it can only exist outside of a Celestial Sphere. If a ship with a Sail of Skysailing goes through the sphere and into the aether, it will take its air and gravity envelopes with it; but the air—while breathable—will be mixed with Luminiferous Aether. If the ship then re-enters the Celestial Sphere, it will bring its air envelope with it but leave the Luminiferous Aether outside the sphere.

Sages and philosophers—after discussion with \iref[chap:Immortals]{Immortals} capable of creating entire Celestial Spheres—have concluded that when a Celestial Sphere is created the inside of the Celestial Sphere becomes a region of order where the Luminiferous Aether cannot exist in its raw chaotic state. Instead, within the confines of the sphere, the Luminiferous Aether is split into its constituent parts—the four elements plus ether—and these parts are sorted into their own planes leaving the inside of the Celestial Sphere full of Void.

Whether this is true or not, Luminiferous Aether does have properties similar to each of the four elements, and it is true that Luminiferous Aether can never be brought into a Celestial Sphere. It simply ceases to exist.

The Luminiferous Aether may be infinite in size. At least, no-one has ever reached any kind of edge to it. It is dotted with innumerable Celestial Spheres, each of which has its own set of astronomical bodies and its own set of Inner Planes and Outer Planes.

Since the Inner Planes are attached within each Celestial Sphere, they do not extend out into the Luminiferous Aether and cannot be reached from there.

Similarly, no outer plane can be anchored in the Luminiferous Aether and no \iref[spell:Gate]{Gate} spell will reach it.

\iref[spell:Commune]{Commune} and \iref[spell:Contact Outer Plane]{Contact Outer Plane} spells cannot be used in the Luminiferous Aether, and those spells will not contact creatures who are there when cast from within Celestial Spheres.

\subsection{Air and Breathing}
Breathing within the Luminiferous Aether is not possible except for within the air envelope of an object; much as in the Void. Similarly, the rules for foul air apply.

However, any living creature that runs out of air in the Luminiferous Aether does not suffocate. Instead, the Luminiferous Aether itself supports them in some strange way. The creature loses consciousness and turns to a stone like substance. The creature remains that way until they are in an area with breathable air, at which point they return to normal.

This effect only happens to living creatures. Dead bodies, and undead creatures (as well as non-breathing things like golems) are unaffected.

The Luminiferous Aether is also incredibly flammable. Any spark will immediately burst into a 1-foot radius fireball doing 1d6 damage. If this catches other items alight, then take the normal damage done by such an object when burning and do three times the damage in an area three times the size of the object. For magical fires like a \iref[spell:Fireball]{Fireball} spell, use a similar guideline—although such spells will always explode around the caster as they are cast, rather than going off once they reach the desired target point.

Strangely, the Luminiferous Aether has the same inhibiting effect on Red Powder as large quantities of Red Powder has on itself. While in the Luminiferous Aether, Red Powder simply will not burn or explode.

\subsection{Movement and Travel}
Getting to the Luminiferous Aether is relatively straightforward. It can be reached from any Celestial Sphere by simply flying to the edge of the sphere and then opening it in some manner.

A \iref[spell:Travel]{Travel} spell will open a temporary hole in the crystal of the sphere wall that will last long enough to fly a ship through; or for creatures traveling without a ship, a \iref[spell:Dimension Door]{Dimension Door} or \iref[spell:Magic Door]{Magic Door} spell will allow passage through the crystal of the sphere wall.

Once in the Luminiferous Aether, travel is handled in a similar manner to in the Void. Creatures cannot fly using winged flight, but all forms of magical flight work as normal.

However, the problem of distances is even more pronounced in the Luminiferous Aether than in the Void. In fact, because of the immense distances involved, the lack of landmarks, and the fact that the Celestial Spheres drift around and do not stay static, distance is a largely meaningless concept for anything other than short range flight.

For long range travel, simply striking out in a random direction is completely pointless. It may take years or even centuries of travel to find another Celestial Sphere.

However, the Celestial Spheres are naturally tied together by “rivers” of flowing Luminiferous Aether that wind through three-dimensional paths. These rivers are huge things, around a thousand miles across, and always both start and end at particular spheres. They may flow in a single direction or flow in both directions with the two flows twisting around each other. By traveling along a river, a ship flying at voidspeed can both know that it is traveling towards another sphere and also take advantage of the river’s flow to get there quicker.

As far as anyone knows, there are no “orphan” spheres with no rivers flowing to them, and there are no “dead end” spheres with rivers flowing towards them but not away from them.

The rivers also have a secondary advantage—the flow of the river will attract debris that would otherwise simply drift randomly. This is particularly the case for debris and objects that are lost from a ship that is traveling the river, and means that on well traveled rivers a lost object has a much greater chance of being found than if it were simply drifting through the vast sea of Luminiferous Aether.

Unfortunately, the rivers twist and bend over a period of time; so while it is possible to map out which spheres are connected to which other spheres (and whether such connections are single or bi-directional), it is not possible to map any kind of meaningful length or distance for a river, since this will change over time (and by how well the navigator manages to keep to the center of the river where the flow is strongest). The time taken to travel along a river is therefore wildly unpredictable. It will take 10d10 days to travel the length of a river at voidspeed in the direction of its flow, although that may include one or more stops when another gravity mass is encountered such as a particularly large piece of flotsam or another ship.

Traveling against the flow is difficult but not impossible, providing one sticks to the outer edge of the river where the flow is least strong. Traveling against the flow in this manner makes a journey from sphere to sphere take five times the normal duration.

The Game Master is encouraged to create a “sphere map” for the region of space around the world on which their campaign starts, showing what spheres exist, and how they are connected.

No-one knows how many spheres there are in total. There even be an infinite number of them. Some \iref[chap:Immortals]{Immortals} have spent centuries exploring and have never managed to run out of new things to find and new places to see.

\end{multicols*}

