\chapter[red]{Ability Scores}
\index[general]{Ability Scores}\label{chap:Ability Scores}
\chapterimage[Ability Scores (Bordered)]
\thispagestyle{plain}

\begin{multicols*}{2}
The innate abilities of player characters are described by six values, called ability scores.

These represent the core abilities of the character and rarely change. These values are \iref[sec:Strength]{Strength}, \iref[sec:Dexterity]{Dexterity}, \iref[sec:Constitution]{Constitution}, \iref[sec:Intelligence]{Intelligence}, \iref[sec:Wisdom]{Wisdom}, and \iref[sec:Charisma]{Charisma}. In normal humans (and demi-humans), these ability scores will normally range from a minimum of 3 to a maximum of 18, with the average being 11.

These ability scores show the strengths and weaknesses of the character, and are used as target numbers for various tasks.

\section{The Ability Scores}
\subsection{Strength}\index[general]{Strength}\label{sec:Strength}
Strength (abbreviated to “Str”) needs little explanation. It represents the raw muscle power of a character. Strength checks are made when trying to perform tasks that rely on raw bodily strength rather than skill, for example when trying to break down doors.

Strength bonuses and penalties apply to a character’s melee attacks, and to the damage that a character does with melee or hurled weapons.

\subsection{Dexterity}\index[general]{Dexterity}\label{sec:Dexterity}
Dexterity (abbreviated to “Dex”) represents the coordination and agility of a character, as well as the speed of their reflexes. Characters with a high dexterity will be agile and graceful, whereas those with a lower dexterity may be clumsy and awkward. Dexterity checks are used when a character must do something involving balance or fine manipulation.

Dexterity penalties or bonuses are applied to a character’s attacks with thrown or missile weapons, and also to their armor class. In the case of armor class, bonuses are subtracted from the character's armor class and penalties are added. Dexterity may also provide a special bonus or penalty to initiative rolls.

\subsection{Constitution}\index[general]{Constitution}\label{sec:Constitution}
Constitution (abbreviated to “Con”) represents the toughness and general healthiness of a character. Characters with a high constitution are likely to be fit and healthy, whereas those with a low constitution are more likely to get ill and get winded easily.

Constitution checks are rarely made, although might be in some circumstances where stamina and endurance are important. 

Constitution bonuses or penalties are applied to the hit point rolls that a character gains each level.

\subsection{Intelligence}\index[general]{Intelligence}\label{sec:Intelligence}
Intelligence (abbreviated to “Int”) represents the memory and reasoning power of a character. Characters with a high intelligence will be able to perform difficult calculations and make deductive leaps, whereas those with lower intelligence will only do such things more slowly if at all.

Intelligence checks are used in a variety of situations where characters need to reason things out or remember things, particularly with academic or formally taught knowledge.

\subsection{Wisdom}\index[general]{Wisdom}\label{sec:Wisdom}
Wisdom (abbreviated to “Wis”) represents a combination of intuition, common sense, and spirituality. To a lesser extent, wisdom also represents the perceptiveness of a character and their ability to notice subtle clues and things out of place. Characters with high wisdom are likely to possess these traits, and be level headed, whereas those with lower wisdom may be rash or act without thinking.

Wisdom checks are used in situations where characters must notice something, and wisdom bonuses or penalties apply to characters’ saving throws against spells.

\subsection{Charisma}\index[general]{Charisma}\label{sec:Charisma}
Charisma (abbreviated to “Chr”) represents the likeability and force of personality of a character. Characters with a high charisma are born leaders and orators, whereas those with lower charisma may be boring or find it hard to communicate. On a physical level, charisma is unrelated to how attractive a character looks; although charismatic individuals often have better bearing and confidence which enhances their attractiveness.

Charisma checks are often used in social situations. Charisma also provides limits on a character’s leadership potential and provides a special bonus or penalty to the reactions of monsters that the character meets.

\section{Ability Score Modifiers}\index[general]{Ability Score Modifiers}
Each score also has one or more bonuses or penalties associated with it that are used to modify other die rolls and checks. \fullref{tab:Ability Score Bonuses and Penalties} shows the modifiers for different ability score values (it includes values much higher than 18, since \iref[chap:Immortals]{Immortal} characters may have much higher ability scores than normal humans).

\begin {table}[H]
  \caption{Ability Score Bonuses and Penalties}\label{tab:Ability Score Bonuses and Penalties}
	\begin{tabularx}{\columnwidth}{>{\bfseries}YYYYYM{.44in}}
		\thead{} &  \thead{} & \multicolumn{3}{c}{\thead{Charisma Only}} & \thead{Dexterity Only} \\
		\thead{Ability Score Value} & \thead{General Modifier} & \thead{Max Hirelings} & \thead{Hireling Morale} & \thead{Reaction Modifier} & \thead{Initiative Modifier}\\
		1 & -4 & 0 & 3 & -2 & -2\\
		2-3 & -3 & 1 & 4 & -2 & -2\\
		4-5 & -2 & 2 & 5 & -1 & -1\\
		6-8 & -1 & 3 & 6 & -1 & -1\\
		9-12 & +0 & 4 & 7 & +0 & +0\\
		13-15 & +1 & 5 & 8 & +1 & +1\\
		16-17 & +2 & 6 & 9 & +1 & +1\\
		18-19 & +3 & 7 & 10 & +2 & +2\\
		20-21 & +4 & 8 & 11 & +2 & +2\\
		22-23 & +5 & 9 & 12 & +3 & +3\\
		24-27 & +6 & 10 & 13 & +3 & +3\\
		28-32 & +7 & 11 & 14 & +4 & +4\\
		33-38 & +8 & 12 & 15 & +4 & +4\\
		39-45 & +9 & 13 & 16 & +5 & +5\\
		46-53 & +10 & 14 & 17 & +5 & +5\\
		54-62 & +11 & 15 & 18 & +6 & +6\\
		63-70 & +12 & 16 & 19 & +6 & +6\\
		71-77 & +13 & 17 & 20 & +7 & +7\\
		78-83 & +14 & 18 & 21 & +7 & +7\\
		84-88 & +15 & 19 & 22 & +8 & +8\\
		89-93 & +16 & 20 & 23 & +8 & +8\\
		94-96 & +17 & 21 & 24 & +9 & +9\\
		97-98 & +18 & 22 & 25 & +9 & +9\\
		99 & +19 & 23 & 26 & +10 & +10\\
		100 & +20 & 24 & 27 & +10 & +10\
  \end {tabularx}
\end {table}

\section{Prime Requisite}\index[general]{Prime Requisite}\label{sec:Prime Requisite}
Each class has an ability score that is most associated with the class’s primary function. This ability score is called a Prime Requisite.

During character creation, prime requisites may be raised by sacrificing 2 points in \iref[sec:Strength]{Strength}, \iref[sec:Intelligence]{Intelligence}, or \iref[sec:Wisdom]{Wisdom}. This can be done more than once but cannot be used to lower an ability score below the minimum required for the class.

Having a high or low Prime Requisite may alter the amount of experience gained by a character. \fullref{tab:Experience Adjustment} shows which Prime Requisite values adjustments occur.

\begin {table}[H]
  \caption{Experience Adjustment}\label{tab:Experience Adjustment}
  \begin{tabularx}{\columnwidth}{>{\bfseries}YY}
		\thead{Prime Requisite Value} & \thead{Adjustment}\\
		3-5 & -20\%\\
		6-8 & -10\%\\
		9-12 & 0\\
		13-15 & +5\%\\
		16-18 & +10\%\
  \end {tabularx}
\end {table}

\section{Ability Checks}\index[general]{Ability Checks}\label{sec:Ability Checks}
In general, adventurers are assumed to be competent individuals who can do things like riding horses, starting camp fires, and swimming in calm water.

If you particularly want to play a character whose competency is deficient in some area, for example if you decide that you specifically want to play a character who can’t swim, then you can do that. However, these rules assume that your character can do all these things in calm situations unless you decide otherwise.

However, sometimes there are situations where your character might fail. Perhaps they are trying to stay on a horse that is bolting in fright. Or perhaps they are trying to light a fire in torrential rain. Or perhaps they are trying to swim in turbulent water without getting washed downstream.

In these cases, the Game Master will call for an ability check in order for your character to succeed. To make an ability check, roll 1d20 and compare the score to the relevant ability score on your character sheet. If the roll is equal to or less than your character’s ability score then your character has succeeded. If the roll is higher than your character’s ability score then your character has failed.

The exact consequences of success and failure will depend on the exact situation your character faces, although it should be very rare (but not unheard-of) for a failed ability check to result in death, unless the character is attempting something spectacularly risky.

\example{Black Leaf is faced with a problem. She has been granted an audience with Queen Eloise and wishes to use the opportunity to ask for some royal sponsorship to aid her expedition to find the fabled Eye of Harmaz. Unfortunately she is panicking because she cannot remember the proper etiquette and is afraid that she will cause offense.

Marcie, her player, asks the Game Master if she can make an ability check against Black Leaf’s Intelligence in order for Black Leaf to “remember” the correct etiquette.

The Game Master agrees and Marcie rolls 1d20, getting a 14. This is higher than Black Leaf’s Intelligence score of 8, so the roll fails and Black Leaf fails to remember the correct etiquette for addressing the queen.

Some time later, Black Leaf can be found getting drunk in an inn, her expedition without royal patronage due in part to the queen’s offense at her frightful manners.}

\subsection{Modifiers to Ability Checks}\index[general]{Ability Check Modifiers}
In some circumstances the Game Master may decide that an ability check is particularly easy or hard. Maybe the ledge that the character is balancing on is slippery, or maybe the person that the character is trying to scrutinize in order to see if they are lying is wearing a mask, or maybe the piece of information they are trying to remember is reasonably common knowledge.

In these cases, the Game Master may assign a modifier to the character’s effective ability score when rolling the check. Unless there are exceptional circumstances that would be unknown to the character, the Game Master should always tell the player what modifiers are going to apply before the player rolls.

\example{Elfstar has come across an underground temple where cultists are performing some kind of ritual in front of a statue of a man who she assumes to be an Immortal.

Elfstar’s player, Debbie, wishes to roll an Intelligence check in order to recognize the Immortal.

The Game Master knows that although this Immortal is not one who has regular dealings with Elfstar’s home country (if he was, then the Game Master would simply tell Debbie who he is and not require a roll at all) he is commonly worshiped in the local area and so it is likely that Elfstar will have heard of him. He gives her a +4 bonus to her effective Intelligence for the roll, modifying it from a 13 to a 17.

Debbie rolls a 17, which is equal to her effective Intelligence for the roll—so the Game Master tells her that Elfstar recognizes the Immortal and tells her who he is.}

\begin {table}[H]
  \caption{Ability Checks by Ability}
  \begin{tabularx}{\columnwidth}{>{\bfseries}cYY}
		\thead{Ability} & \thead{Situations where this ability might be checked} & \thead{Skills useful in these situations}\\
		Strength & Where raw physical strength is required; such as lifting, pushing, or pulling. & Intimidation, Jumping, Swimming\\
		Dexterity & Where either agility or fine manipulative skills are required. & Balance, Craft, Escape Artist, Performance, Riding\\
		Constitution & Where stamina or raw endurance is required. & -\\
		Intelligence & Where calculation, memory or reasoning ability is required. & Arcane Lore, Engineering, Geography, History, Laws, Magical Engineering, Nature Lore, Religious Lore, Snares\\
		Wisdom & Where intuition or “common sense” are required. Also, where acuity of the senses is required. & Cooking, First Aid, Lip Reading, Navigating, Tracking, Sense Motive\\
		Charisma & Where personality and smooth talking are required. & Bluff, Diplomacy, Disguise, Etiquette, Gambling, Intimidation, Performance\
  \end {tabularx}
\end {table}
\end{multicols*}
