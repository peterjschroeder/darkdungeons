\chapter[green]{Immortals}
\label{chap:Immortals}
\chapterimage[Immortals (Bordered)]
\thispagestyle{plain}

\begin{multicols*}{2}
The ultimate goal of many characters is to reach the lofty heights of immortality.

Immortality doesn’t just mean not dying— although immortals are incredibly resilient—it actually means transcending flesh and transforming into a purely spiritual being of great power.

Since immortals are so different from mortal characters, and so much more powerful, an immortal level campaign will be very different in tone from a mortal level one. Most immortals don’t simply go out and kill monsters; and they certainly don’t hoard and spend treasure like mortal adventurers do. Instead, immortal level campaigns tend to center around political rivalries, machinations, and plotting.

The Game Master and players should take this difference into account when deciding whether or not to continue a campaign into the immortal levels. Many players may simply prefer to have their characters retire and die peacefully as mortals—or maybe use the gaining of immortality as the campaign finale rather than continue to play their characters once immortality is reached. 

Whether you decide to include the immortal levels in your campaign or not, it should be the result of a conscious decision; not the result of a “lucky” (or unlucky) die roll. Suddenly finding yourself in an immortal level campaign that you weren’t prepared for can be bewildering to both the players and the Game Master and is likely to kill the campaign if not prepared for. Similarly, being all geared up for an immortal level campaign and then discovering that one or more of the PCs doesn’t make the transition because their players rolled badly is equally unsatisfying.

\section{What is an Immortal?}
On the one hand, immortals are powerful spiritual beings that can create entire planes and species and move planets around.

On the other hand, immortals are simply people.

For all their great power, immortals still have the desires, goals and personalities that they had when they were mortal. Dark Dungeons assumes that all immortals are in fact former mortals, although since it is normally only possible to become an immortal by being sponsored by an existing immortal, this raises the question of where the first immortal(s) came from.

It is up to the Game Master to decide what the answer to that question is in their campaign. Maybe the first immortals were created by true gods (if they exist in the setting). Maybe the first immortals simply spontaneously appeared. Maybe the first mortals were able to become immortals even without sponsors. Or maybe it was something completely different.

Given that immortals are former mortals who have been given great power, what they do with that power (and what they do with their endless time— since immortals no longer age) is as varied as mortality itself. Some explore the universe. Some look after the mortals and protect them. Others play with mortals to amuse themselves, or play decadent political games with one another. Others are easily corrupted by the power and enjoy spoiling the plans of their peers and making life hard for mortals. 

The personalities of immortals are as varied as those of mortals; and even though they have great power, they do not necessarily have the wisdom that comes with great age. Some may well be as dumb as a bag of rocks, despite their power.

\section{The Three Forms}\index[general]{The Three Forms}
There are three forms that an immortal may take: Embodied Form, Spirit Form, and Mortal Form. Changing between forms normally takes a round, during which the immortal is treated (for the purposes of being attacked or other potentially damaging situations) as being in the least vulnerable of the forms.

\subsection{Embodied Form}\index[general]{Embodied Form}\label{sec:Embodied Form}
The most common form taken by an immortal is the embodied form. An immortal must always actually have an embodied form, even if they never use it.

This form is physically the most powerful and allows the immortal to use its powers more capably than other forms.

While in embodied form, immortals do not need to breathe, eat or drink—although they can do all three for pleasure if they choose. Similarly, immortals in embodied form are incapable of siring children, but can have sex for pleasure.

If an embodied form of an immortal is killed while the immortal is not on their home plane, the immortal immediately disappears and reappears on their home plane.

The immortal must immediately spend 1,000,000 XP to recreate the embodied form that just died, even if they have other embodied forms available to them. This must be done even if doing so forces them to lose a level. If this level loss would reduce them below \nth{1} level, then they lose their immortality and become merely a dead mortal who can be raised or reincarnated as normal. This is the only way that an immortal can lose their immortality.

Once the immortal has recreated the embodied form that died, they cannot leave their home plane or take on a different form (except to project a Spirit Form through an \iref[spell:Immortal Eye]{Immortal Eye spell}) for one day per hit point they have.

Additionally, their connection with all their clerics is temporarily disrupted, and their clerics may not cast spells during this period.

If an immortal’s embodied form is killed while on their home plane, the immortal is simply dead, and all their clerics permanently lose their power.

\subsection{Spirit Form}\index[general]{Spirit Form}
The spirit form is insubstantial and can travel freely through any mundane substance or any substance created by mortal level magic. This also means that the spirit form cannot interact with the mundane world in any way. For example, it cannot pick objects up or attack creatures.

The spirit form is always translucent and it glows gently (with a strength anywhere from a candle to a campfire) and cannot be made to appear invisible or solid, but the immortal may change the shape of their spirit form at will.

The most common shapes for immortals using spirit form to take are:

\begin{itemize}
	\item{A ball of light}
	\item{A transparent version of their Embodied Form}
	\item{A glowing version of the holy symbol used by their religion}
\end{itemize}
Immortals in spirit form do not need to eat, drink or breathe; in fact they are incapable of doing these things as they have no physical presence. This lack of physical presence also means that the immortal does not have a gravity or air envelope when in the Void or in the Luminiferous Aether.

\subsection{Mortal Form}\index[general]{Mortal Form}
Most immortals maintain one or more mortal forms. These mortal forms are, as the name suggests, mortal. They are completely indistinguishable from normal mortals, so an immortal can go incognito in a mortal form and manipulate things on the \ilink{sec:Prime Plane}{Prime Plane} without being noticed. No magical detection—not even the ESP spell or other forms of telepathy—can detect that the mortal form is anything other than a mortal.

When a mortal form dies, the immortal is immediately sent back to their home plane where they reappear in Embodied Form. The immortal cannot leave their home plane or take on a different form (except to project a Spirit Form through an \iref[spell:Immortal Eye]{Immortal Eye spell} spell) for 48 hours. The particular mortal form that was killed no longer exists, although there is nothing to stop the immortal from creating a new identical form if they wish.

When leaving mortal form, whether by changing forms or by death, any equipment that was created as part of the mortal form vanishes. Mundane items carried by the mortal form do not disappear in this manner.

\section{Becoming an Immortal}\index[general]{Becoming an Immortal}
Becoming an immortal is deceptively straightforward. All a character needs to do is to find an immortal who is willing to sponsor them and to create them their first Embodied Form. Their sponsor creates the form for them and Zap! they’re now an immortal.

Of course, it isn’t really that easy.

Firstly, only the strongest of life-forces can support an Embodied Form. A character needs to have at least 3,000,000 experience points to do this.

If the character doesn’t have at least 3,000,000 experience points, then they simply can’t be made into an immortal. Their life force is just not robust enough.

Secondly, the Embodied Form takes energy—and life force—to produce, and the sponsoring immortal must pay this. It costs the sponsoring immortal 1,000,000 experience points to create the Embodied Form for the prospective new immortal. Of course, no immortal is going to spend such a large amount of their own experience points on a whim.

So although becoming an immortal is a very straightforward process, getting strong enough to be able to go through the process and finding an immortal willing to significantly weaken themselves in order to take you through the process are not so straightforward.

The reasons why an immortal may be willing to sacrifice some of their own life force to create another like themselves can be varied. Some may do it for companionship or even love.

Others may help their own descendants become immortal out of a sense of familial duty. Others have more prosaic reasons. They do it to gain immortal allies, or as a significant reward for mortals who have served their interests well.

In the case of adventuring parties, this last reason is probably the most common. Although there is nothing to physically prevent immortals from acting in a blatant manner on the \ilink{sec:Prime Plane}{Prime Plane} (e.g. appearing in Embodied Form and blasting the armies of their worshipers’ enemies), in most campaign settings there will be large groups of immortals who “police” the \ilink{sec:Prime Plane}{Prime Plane} (or at least a particular Celestial Sphere) to prevent this. Experience shows time and again that unrestricted shows of immortal power on the \ilink{sec:Prime Plane}{Prime Plane} all too quickly lead to tit-for-tat wars and wholesale destruction of entire planets.

For this reason, most immortals restrict their work on the \ilink{sec:Prime Plane}{Prime Plane} to a series of churches, Mortal Forms and agents. Immortals therefore often show a large interest in high level adventuring parties, since they make useful agents— willing to risk great danger if the prize of potential immortality is dangled in front of them, and able to do things on the \ilink{sec:Prime Plane}{Prime Plane} that the immortal cannot do themselves because it would be too blatant. 

Of course, while some immortals may be very open and business-like about a “work for me and I’ll make you an immortal too” deal, others couch it in terms of sending the mortals on “quests” or “tests” in order to determine their “worthiness” to join the ranks of the immortals. Whether these immortals actually think of what they are doing in those terms or whether they are merely being euphemistic about the true nature of the deal may vary from individual to individual, of course. 

\example{Having reached 3,0000,000 experience, Elfstar is now powerful enough to become an immortal.

Diana, the immortal who Elfstar serves, visits her in a dream. She tells Elfstar that she has been a loyal servant and that now she is ready to be rewarded with the real power of being an immortal. However, because Elfstar is such a prominent member of her church, she can’t afford to lose her talents straight away. Diana tells Elfstar that in order to be given her reward, she must first train up a successor to carry on her good work.}

\section{Worshipers}\index[general]{Worshipers}
There is another wrinkle in becoming an immortal—and it is one that existing immortals don’t like to talk about. Immortals cannot exist without the worship of free-willed mortals. No-one knows exactly why this is, but an immortal that goes for over a year without worshipers dies. This is why even in campaigns that have pantheons of gods, immortals still act as their intercessors. They need the worship of their god’s followers.

The actual number of worshipers doesn’t matter; even having a single one is good enough (although most immortals naturally try to have as many worshipers as possible for safety’s sake). It also doesn’t matter if the worship is done out of love or fear, as long as it is done. This is a one-way dependency, in that although immortals need worshipers to survive, the worshipers get nothing out of it—at least by default.

Smart immortals know that looking after their worshipers and helping them with the occasional omen or answered prayer is a great way to keep them interested. Likewise, investing clerics who can go around healing and helping (or terrorizing if that’s what you prefer) the populace can gain and keep large numbers of worshipers.

An immortal without worshipers is fully aware of that state at all times, so there is no danger of an immortal—not even a new one—accidentally losing their last worshiper and not noticing until a year is up and it is too late.

\section{Home Plane}\index[general]{Home Plane}
The plane on which an immortal is first created is forever afterwards considered to be their home plane.

An immortal’s home plane is their seat of power. When on their home plane an immortal is treated as if six levels higher than their actual level, to a maximum of \nth{36} level for purposes of level-dependent abilities (e.g. hit points and power reserve). This effective level increase does not change the immortal’s experience total.

However, when on their home plane an immortal can only take on Spirit Form or Embodied Form, not Mortal Form.

Because of the importance of an immortal’s home plane, a sponsoring immortal will never bestow immortality onto someone on the \ilink{sec:Prime Plane}{Prime Plane}, since this would prevent them from ever taking Mortal Form there.

In some cases where there is an established pantheon of immortals who share a single home plane, new immortals may also be created on that plane. In most cases, however, the sponsor will create a tiny (house sized) outer plane anchored on their own home plane for the new immortal and give them their immortality there. Creating such a tiny plane with the Shape Reality spell costs only 200,000 XP.

That way, the new immortal can, once they are more experienced, expand and/or alter their home plane or move it to a new location of their choosing.

\section{Appearance}
When creating a new Embodied Form for the new immortal, the sponsor must specify exactly what that form will look like, and what powers it will have.

Most sponsors will ask the new immortal what they want their form to look like and powers they want, and even if the choice is not given in character to the new immortal, the choice should still be given out of character to the character’s player; for the same reason that players get to choose the class of a new mortal character they create even though the character themselves may have been apprenticed out and not had a choice in their career (and certainly not in their race).

Most new immortals already have a strong self-image, and wish to look like idealized versions of their mortal bodies. Since immortals don’t age, and any apparent age has no effect on their abilities, some immortals prefer to look young and virile as they did (or at least as they imagine they did) in their youth, while others prefer to look older and more worldly wise. Many simply wish to continue to appear as they did at the point when they became an immortal.

Some new immortals choose to make a complete break from their old mortal lives, and choose to look different— occasionally very different—from how they looked while mortal. Often this will involve taking on a new name to go with the new form. Unless powers say otherwise, an immortal’s form must be between three and seven feet in size.

\example{Elfstar has trained up her replacement and is in the process of investing her with her new role when Diana, not wanting to miss the chance to impress her followers, appears in her Embodied Form as the investiture rite is finishing. 

Normally an immortal simply showing up on the prime plane in Embodied Form would alarm the other immortals who are watching the prime plane for direct interference, but Diana has informed them in advance that she is going to appear to her worshippers in this way so while they keep watch, they don’t interfere.

Diana blesses Elfstar’s replacement and then wanders through the assembled crowd of worshippers dispensing healing and advice.

Finally, she takes Elfstar by the hand and returns with her to her home plane leaving no doubt in the minds of her worshippers that Elfstar has been invited to join her pantheon.

Once on the home plane that Diana shares with the rest of her pantheon, Elfstar stays as a guest in Diana’s palace for three days while Diana explains all about immortality to her. 

At the end of that time, Elfstar is ready and has decided that in order to attract worshippers of her own—and not to compete for them too much with the rest of Diana’s pantheon—she is going to appear as an emissary of youth and innocence (which won’t surprise people who knew her during her life, since she was always chaste).

She decides that her Embodied Form should look like she did when she was still a young teenager, and decides to give it the powers of Call Other, Detection Suite, Improved Saving Throws (vs. mental attacks), and Turn Undead.}

\section{The Immortal Class}\index[classes]{Immortal}
%\includegraphics[width=\columnwidth]{Classes/Immortal}
Once characters become immortals, they no longer have the character class (or even race) that they had in mortal life. Immortals, regardless of appearance, are a single class to themselves with thirty six levels just like any other class.

When a character becomes an immortal, they retain the experience total that they had in life. This means that a character becoming an immortal may begin their immortality at a level higher than \nth{1}.

Since the immortal has a completely new body which is fundamentally different to that of a mortal, the player of a new immortal character should not be surprised if some aspects of their character appear to have got worse since they became an immortal. For example, a \nth{1} level immortal may have fewer hit points than they had as a \nth{32} level fighter. This appearance can be deceptive, however, since immortals are significantly better than mortals in a number of ways and this makes direct comparisons of a single ability or score somewhat misleading.

\subsection{Skills}
First level immortals start with a number of skill points equal to 12 plus their \iref[sec:Intelligence]{Intelligence} bonus, and this increases to a maximum of 20 plus their \iref[sec:Intelligence]{Intelligence} bonus at \nth{36} level. Skills for immortals work in exactly the same way as skills for mortals. However, the higher ability scores of immortals can sometimes make skill and ability checks unnecessary unless there are very large penalties to the effective ability score.

While most physical skills can only be used in Embodied Form, some mental or social skills can also be used in Spirit Form.

\subsection{Weapon Feats}
Immortals have basic proficiency with all weapons and attacks, and have no restrictions on the weapons that they are allowed to use. Immortals do not normally gain extra weapon feats, but may have higher proficiency levels in some weapons because they have chosen the Weapon Expertise power as part of their Embodied Form.

\subsection{Level and Experience}
Compared to mortals, immortals need huge numbers of experience points to increase in level.

However, immortals don’t get experience for the same things that mortal level characters do. Immortals never gain experience for gaining treasure, and should only rarely—if ever, at the Game Master’s discretion—gain experience for killing mortal level monsters. Doing such things are simply not significant to an immortal.

The only experience immortals should get for killing monsters is if the monsters pose a significant threat to the immortals. 

Instead, the vast majority of experience gained by immortals will be for achieving plot goals. The Game Master is advised to give such rewards much more frequently in an immortal level campaign than in a mortal level campaign, and to make them large enough that the characters will advance in level at a rate the group finds reasonable.

\subsection{Saving Throws}
Since immortals are immune to mortal magic and posses Anti-Magic, they have only four types of saving throws which are spell attacks, psychical attacks, mental attacks, and power attacks.

\subsection{Abilities}
Some of these abilities require the immortal to spend experience points to use them. They are literally powering these abilities by using up their own life force. If spending experience points in this manner would reduce an immortal’s level, the immortal cannot spend the experience and cannot use the ability.

\textbf{Anti-Magic:} Immortals in Embodied Form or Spirit Form have Anti-Magic against mortal level magic cast by other immortals. Even if a spell cast by an immortal level caster gets through the immortal’s Anti-Magic, the immortal may still make a saving throw versus that spell where applicable. If the spell is mind-affecting, the immortal may make a saving throw vs. mental attacks, otherwise the immortal makes a saving throw vs. spell attacks.

\textbf{Aura of Power:} The Embodied Form of an immortal always radiates a glowing aura with a strength anywhere from that of a candle to that of a bonfire at the immortal’s whim.

Once per round (as an action), an immortal in Embodied Form can cause their aura to flare up. This will affect a number of mortal creatures equal to the immortal’s \iref[sec:Intelligence]{Intelligence} and \iref[sec:Wisdom]{Wisdom} bonuses added together; providing all the targets are within 60 feet of the immortal. Mortal creatures with no mind (such as unintelligent undead) are affected by these aura attacks.

The immortal must decide whether the desired effect of the aura is terror, awe, or beauty.

\textbf{Terror:} Each target will flee in terror for 30 minutes unless they can make a saving throw vs. spells. If the target is cornered, they will cower and fight only to defend themselves.

\textbf{Awe:} Each target will stand paralyzed for 30 minutes unless they can make a saving throw vs. spells.

\textbf{Beauty:} Each target will consider the immortal to be their best friend, and treat them accordingly. It does not make them fanatically loyal and will not make them attack their other friends.

All targets get a saving throw vs. spells to avoid the effect.

If a target fails its saving throw, then it gets another one periodically to throw off the charm effect. The frequency of the saving throw is based on the creature’s \iref[sec:Intelligence]{Intelligence} as indicated on \fullref{tab:Charm}.

If the immortal behaves in an overtly hostile manner to the charmed target, such as attacking it or ordering others to attack it, then the charm is broken.

\textbf{Control Dreams:} Immortals in Spirit Form can control the dreams of all dreaming mortal creature sleeping within 180 feet of them, and can do this to mortals on the \ilink{sec:Prime Plane}{Prime Plane} while the immortal is on the \iref[sec:Ethereal Plane]{Ethereal Plane}.

\textbf{Fast Healing:} Immortals heal quicker than mortals do. An immortal regains 1d4 hit points per day, or 1d8 hit points per day if resting.

\textbf{Immunity to Aging:} Immortals in Embodied Form are immune to aging (including magical aging).

\textbf{Immunity to Damage:} Immortals in Spirit Form are completely immune to any form of damage or attack.

\textbf{Immunity to Disease:} Immortals in Embodied Form are immune to all diseases.

\textbf{Immunity to Environmental Effects:} Immortals in Embodied Form are immune to mundane environmental effects such as fire, cold, lightning and so on.

\textbf{Immunity to Magic:} Immortals in Embodied Form or Spirit Form are completely immune to all mortal level magic cast by mortals, including such magical effects as dragon breath from mortal level dragons and \iref[sec:Energy Drain]{Energy Drain} from mortal level undead.

This immunity stretches to the magical creations of existing mortal level spells. For example an immortal can walk straight through a mortal level Force Field spell. However, it does not stretch to the mundane non-magical creations of such spells.

An immortal cannot see or walk through a Wall of Stone cast by a mortal.

Immortals in Spirit Form are completely immune to any form of magic, with the exception of the immortal spells Probe and Power Attack. The immortal gets normal Anti-Magic checks and saving throws against these spells while in Spirit Form.

\textbf{Immunity to Poison:} Immortals in Embodied Form are immune to mortal level poison. They are not immune to immortal level poisons. However, even such powerful venoms will be delayed for a number of rounds equal to the immortal’s \iref[sec:Constitution]{Constitution} bonus before taking effect; hopefully giving the immortal chance to cast a Neutralize Poison spell or similar before it is too late.

\textbf{Infravision:} Immortals in Embodied Form have \iref[sec:Infravision]{Infravision} (see \fullref{sec:Infravision}).

\textbf{Multilingual:} Immortals in Embodied Form or Spirit Form can speak and understand any language.

\textbf{Multiple Attacks:} An immortal in Embodied Form is able to make two attacks per round. At \nth{13} level, this rises to three attacks and at \nth{25} level it rises to four attacks. See \fullref{sec:Actions} for details of multiple attacks.

\textbf{Natural AC:} Immortals in Embodied Form have a natural armor class that rises when they gain levels. This armor class is not modified by armor and shields that are worn unless they are magical; in which case only the magical bonuses apply, not the base armor class normally granted by the armor.

\textbf{Power Reserve:} Immortals have a power reserve, which is a pool of points that can be spent on minor or temporary powers and abilities.

An immortal’s power reserve refreshes after a night’s rest, and any power points left unused are wasted.

If an immortal spends their entire power reserve, and has no more power points remaining, they are left in an exhausted state. If not already in Embodied Form, they immediately switch to their first Embodied Form, and can not leave that form until they have power points once more.

Additionally, all movement speeds are halved, as is unarmed damage; and the immortal has a -4 penalty on all saving throws.

This condition lasts until the immortal has had a chance to sleep and regain their power points back up to their normal power reserve level.

\textbf{Resistance to Mortal Damage:} If a mortal psychically attacks an immortal in Embodied Form, they will only cause damage if they hit with a +5 weapon or better. Even then, the immortal takes only minimum damage from each attack. If a fighter targets an immortal with their Smash ability, the fighter does not add their entire \iref[sec:Strength]{Strength} to the damage done—they only add their \iref[sec:Strength]{Strength} bonus instead.

\textbf{Spells:} By spending 18 power points, an immortal in Embodied Form may cast spells as a \nth{36} level cleric, druid, or wizard for a day.

Once these spells are cast they are forgotten and may not be relearned unless the immortal spends another 18 power points to regain them all. If the immortal spends additional power points, the spells are not forgotten when cast. The cost to do so is 7 power points for cleric and druid spells and 32 power points for wizard spells.

When casting these spells, the hit dice of an immortal are used to determine their effective level.

Immortals in Embodied Form can also cast powerful Immortal Level Spells.

All immortals can cast all of these spells, and there is no need to prepare them in advance.

There is no limit to how often an immortal can cast any of these spells other than their cost in experience and power points.

Immortals in Spirit Form may not use any Immortal Level Spells with the exception of the Power Attack, Probe, and Probe Shield spells. The immortal may use such mortal movement or travel spell at will, as often as they like; and if the spell has variable effects based on the level of the caster then the immortal is treated as a caster with a level equal to twice their hit dice.

\textbf{Strike to Kill Damage:} When attacking while unarmed, an immortal in Embodied Form can choose to do strike to kill when using the Unarmed Strike weapon feat instead of striking to stun (see \fullref{chap:Weapon Feats}). If they do so, they do more damage (and \iref[sec:Strength]{Strength} bonuses apply as normal), but lose the chance to stun or knock out their opponent. The damage done by an immortal of a particular level is listed on \fullref{tab:Immortal Progression}.

\textbf{Telepathy:} Immortals in Embodied Form or Spirit Form can communicate with any creature by transmitting and receiving thoughts. The immortal and the creature will understand the thoughts despite language differences.

\textbf{Increased Weapon Damage:} Immortals in Embodied Form of \nth{13} level and higher do additional damage when striking with weapons. This damage is based on the type of damage normally done by the weapon at the immortal’s level of expertise with that weapon, but with an additional die added to it. This increased damage does apply to unarmed strikes to stun, but does not apply to unarmed strikes to kill.

At \nth{25} level, two dice are added to the damage.

\subsection{Abilities: Powers}
Immortals gain four powers of the player’s choice upon creation to make them more unique. These power choices are permanent; no substitutions can be made after game play begins.

\subsubsection{Call Other}
The immortal can spend 10 power points in order to make a mental call for help back to their home plane.

If any other immortals share the same home plane and are on that plane at the time of the call, there is a 15\% chance that one of them (chosen randomly) will hear the call. This chance increases for every six levels the calling immortal has attained.

The immortal hearing the call will know the identity of the calling immortal, but not the circumstances in which the call is being made. They may choose to either ignore the call or to immediately spend 50 power points to open and step through a temporary Gate to the calling immortal’s location.

\subsubsection{Control Undead}
The immortal may speak with all intelligent undead, and may control undead as if they were a 33+ hit dice Undead Liege (see \fullref{sec:Undead Lieges}).

\subsubsection{Detection Suite}
The immortal gains all the special detection powers of the \iref[class:Dwarf]{Dwarf} and \iref[class:Elf]{Elf} classes.

\subsubsection{Dragon Breath}
The immortal can spend 50 power points to use the breath weapon of any of the normal types of dragon or dragon queen, doing damage equal to their current hit points. The immortal can only use the breath weapon of each type of dragon once per day.

If the immortal also has the Dragon Form power, these breath attacks may be used in addition to the breath attacks granted by that power.

\subsubsection{Dragon Form}
This power costs two power choices.

The immortal’s Embodied Form is that of a huge dragon. The immortal has a movement rate of 60 feet on foot or 140 feet flying.

The immortal gets nine attacks per round regardless of experience level. These are two bites for 6d8 damage each; and two claws, two wing strikes, two kicks and a tail swing, for 2d8 damage each. \iref[sec:Strength]{Strength} bonuses apply to each of these.

Additionally, the immortal must choose either a single color or a mix of two colors for their scales. They can spend 50 power points to use the breath weapon of a dragon or dragon queen of either of their colors, doing damage equal to their current hit points. The immortal can only use the breath weapon twice per day, but each time may be from the same or a different type of dragon.

If the immortal also has the Dragon Breath power, these breath attacks may be used in addition to the breath attacks granted by that power.

\subsubsection{Enhanced Reflexes}
This power may be taken more than once.

The immortal gets a +2 bonus on their Surprise and Initiative rolls.

\subsubsection{Extra Attacks}
This power may be taken more than once.

The immortal gets one extra attack per round, over and above the normal number of attacks granted by their level or other powers.

\subsubsection{Fighter Abilities}
The immortal gains the Smash and Parry fighter abilities.

\subsubsection{Groan}
Once per ten minutes, the immortal can spend 20 power points to make a horrible noise (although the power is called “Groan” the noise does not actually have to be a groan—it could be a different type of noise).

All creatures (including other immortals) within 180 feet must make a saving throw. In the case of mortal creatures, this is a saving throw vs. spells with a -2 penalty. In the case of undead creatures, this is a saving throw vs. spells with no penalty. In the case of immortals, this is a saving throw vs. mental attacks with a +4 bonus.

Any creature that fails the saving throw is paralyzed for ten minutes.

Any creature that makes the saving throw can only move at half their normal speed for ten minutes.

Multiple groans from different immortals have no additional effect on a creature already affected by this power.

\subsubsection{Height Decrease}
The immortal’s Embodied Form can grow and shrink anywhere from normal human-sized to as small as three inches tall.

It takes 10 minutes for the immortal to change size, although they can remain at any given size indefinitely.

Changing size does not affect the immortal’s other abilities or powers.

An immortal may have both the height decrease and Height Increase powers.

\subsubsection{Height Increase}
The immortal’s Embodied Form can grow and shrink anywhere from normal human-sized to as large as twenty-two feet tall.

It takes 10 minutes for the immortal to change size, although they can remain at any given size indefinitely. Changing size does not affect the immortal’s other abilities or powers.

An immortal may have both the Height Decrease and height increase powers.

\subsubsection{Howl}
The immortal may make a terrifying sound (although this power is called “Howl”, the sound does not actually have to be a howl—it could be a different type of sound).

All creatures (including other immortals) within 180 feet must make a saving throw. In the case of mortal creatures, this is a saving throw vs. spells with a -2 penalty. In the case of undead creatures, this is a saving throw vs. spells with no penalty.

In the case of immortals, this is a saving throw vs. mental attacks with a +4 bonus.

Any creature that fails the saving throw must flee in terror for 3d6 rounds.

\subsubsection{Improved Saving Throws}
This power may be taken more than once.

The immortal is particularly good at resisting a certain type of effect. When this power is taken, the player must choose one of the types of immortal saving throws.

Whenever the immortal must make a saving throw of that type to avoid taking damage, success means that the immortal only takes a quarter of the normal damage from the attack, and failure means that the immortal takes a half of the normal damage from the attack.

If the attack is an all-or-nothing effect rather than an effect that does damage, then success means that the immortal is completely unaffected by the attack and failure means that the immortal takes the full effect.

If this power is taken more than once, it must apply to a different saving throw each time.

\subsubsection{Increased Damage}
This power may be taken more than once.

The immortal increases the damage done by each physical attack by one die of the type done by the attack. This power does not increase the damage done by spells cast by the immortal.

\subsubsection{Increased Movement Rate}
This power may be taken more than once.

The immortal moves at double normal speed in all modes of travel except when flying in at voidspeed in Spirit Form.

If this power is taken more than once, the multiplier increases by one for each additional time the power is taken; so an immortal who has taken this power three times moves at four times their normal movement speeds.

\subsubsection{Leech}
The immortal may suck the life force out of creatures they touch, including other immortals. This power must be consciously used—the immortal won’t accidentally kill people when shaking their hands—and requires a successful attack roll against an unwilling target.

When used on a mortal creature, the touch will drain three levels of experience from the victim. There is no saving throw against this drain, and the victim will not even notice that the drain has happened unless they can make a saving throw vs. death ray at a -2 penalty. The immortal using the attack gains 3d4 hit points from the drained life force.

When used on an immortal, the victim must make a saving throw vs. power attacks. If the victim fails the saving throw then they lose 100,000 experience points.

This loss can cause the victim to lose a level, but cannot reduce them below 3,000,000 experience (i.e. it cannot reduce them below \nth{1} level). If the victim makes the saving throw then they lose 10 power points. The immortal using the attack gains 10 power points from the drained life force.

If the attacking immortal gains more hit points or power points than their normal maximum, the excess disappear after ten minutes.

\subsubsection{Monk Abilities}
This power costs three power choices.

The immortal has the number of attacks, damage, and special abilities of a \nth{36} level monk.

\subsubsection{Poison}
This power can be taken twice.

The immortal has a poisonous bite or a poisonous stinger. If this power is taken twice, then the immortal has both.

If the immortal makes a successful attack with either a bite or a sting, the victim must make a saving throw.

Mortal victims must make a saving throw vs. poison with a -4 penalty. If they fail the saving throw then they die instantly. If they make the saving throw then they take 6d6 damage and can do nothing but writhe in agony for a full day, being unable to even think clearly.

Immortal victims must make a saving throw vs. physical attacks. If they fail the saving throw then they take 6d6 damage and are in such pain that they cannot speak, fight, cast spells or use powers for a full day. Turning to Spirit Form will ease the pain, but turning back to an Embodied Form will make it return and turning to a Mortal Form will cause that form to instantly die. If they make the saving throw, they are unaffected.

If a mortal is slain by the poison (either from failing their saving throw or from the 6d6 damage), their blood remains poisonous enough that it can be used as twelve doses of normal save-or-die poison; although the poisonous blood will not poison the blood of its victims in turn.

\subsubsection{Rogue Abilities}
The immortal gains the special abilities of a \nth{36} level rogue, with the exception of the Sneak Attack ability.

\subsubsection{Snap}
The immortal can stretch out a body part (hair, tongue, arms, tentacles or some other part chosen when the power is chosen) to a distance of 20 feet and make an attack with it.

If the attack hits its target, the target is grabbed and pulled to the immortal who can then make a normal melee attack against the victim. If the snap attack was made by surprise, the resulting melee attack does double damage.

Once the melee attack has been made, the victim is no longer grappled by the snapping body part.

\subsubsection{Spit Poison}
The immortal may spit poison into the eyes of any target within 30 feet. No attack roll is needed, but the target gets a saving throw.

Mortal victims must make a saving throw vs. poison at a -2 penalty. If they fail then they die instantly.

If they succeed they take 3d6 damage and are \iref[sec:Blinded]{Blinded} until cured by a Neutralize Poison spell cast by an immortal.

Immortal victims must make a saving throw vs. physical attacks. If they fail they take 3d6 damage and are \iref[sec:Blinded]{Blinded} for 2d10 rounds or until they receive a Neutralize Poison spell. If they succeed then the attack has no effect.

\subsubsection{Summon Weapons}
This power may be taken more than once.

The immortal must designate one or two weapons as their chosen weapons when they take this power. Those weapons must be hidden in a secure place on the immortal’s home plane.

At any time, the immortal can summon one or both weapons to their hand instantly (this happens during the Statement of Intent phase in combat and does not affect initiative or actions).

If either of the weapons is dropped by the immortal, either deliberately or accidentally, then they immediately return to their hiding place.

If either of the weapons is ever stolen from its hiding place, it may not be summoned until it is found and returned to that place.
\subsubsection{Swoop}

The immortal can make a swoop attack while flying. This attack is treated as a Charge action, even though the immortal is not mounted.

This power can only be used once every three rounds.

\subsubsection{Turn Undead}
The immortal is able to turn undead as if a \nth{36} level cleric.

\subsubsection{Weapon Expertise}
This power may be chosen more than once.

The immortal has the grand master level of expertise with three types of weapon chosen at the time the power is chosen.

\textbf{Ability Requirements:} None

\textbf{Prime Requisite:} None

\textbf{Hit Dice:} Non-Variable Hit Points

\textbf{Movement (Embodied Form):} 50 ft., 120 ft. (Fly), 50 ft. (Swim)

\textbf{Movement (Spirit Form):} Voidspeed, 240 ft. (Attentive)

\textbf{Weapons:} Any

\textbf{Armor:} Any

\textbf{Special Abilities:} Anti-Magic, Aura of Power, Control Dreams, Immunity to Aging, Immunity to Damage, Immunity to Disease, Immunity to Environmental Effects, Immunity to Magic, Immunity to Poison, Infravision, Multilingual, Multiple Attacks, Power Reserve, Powers, Resistance to Mortal Damage, Spells, Strike to Kill Damage, Telepathy, Increased Weapon Damage

\end{multicols*}
\begin {table}[H]
  \caption{Immortal Progression}\label{tab:Immortal Progression}
	\begin{tabularx}{\columnwidth}{>{\bfseries}ccM{.15in}M{.25in}M{.35in}M{.38in}M{.35in}M{.38in}M{.35in}M{.35in}Y}
		\thead{} & \thead{} & \thead{} & \thead{} & \thead{} & \thead{} & \multicolumn{4}{c}{\thead{Saving Throws}} & \thead{}\\
		\thead{Level} & \thead{Experience} & \thead{Hit Dice} & \thead{Hit Points} & \thead{Attack Bonus} & \thead{Natural AC} & \thead{Spell Attacks} & \thead{Physical Attacks} & \thead{Mental Attacks} & \thead{Power Attacks} & \thead{Special}\\
	1 & 3,000,000 & 15 & 75 & +12 & 9 & 20 & 15 & 18 & 17 & Anti-Magic, Aura of Power, Control Dreams, Immunity to Aging, Immunity to Damage, Immunity to Disease, Immunity to Environmental Effects, Immunity to Magic, Immunity to Poison, Infravision, Multilingual, Multiple Attacks (2), 
Power Reserve, Powers, Resistance to Mortal Damage, 
Spells, Strike to Kill Damage, Telepathy\\
	2 & 3,250,000 & 16 & 80 & +13 & 9 & 20 & 15 & 18 & 17 & -\\
	3 & 3,500,000 & 17 & 85 & +13 & 9 & 20 & 15 & 18 & 17 & -\\
	4 & 3,750,000 & 18 & 90 & +14 & 9 & 20 & 15 & 18 & 17 & -\\
	5 & 4,000,000 & 19 & 95 & +14 & 9 & 20 & 15 & 18 & 17 & +1 Skill Point\\
	6 & 4,500,000 & 20 & 100 & +15 & 9 & 20 & 15 & 18 & 17 & -\\
	7 & 5,000,000 & 21 & 110 & +15 & 8 & 20 & 14 & 17 & 16 & -\\
	8 & 6,000,000 & 22 & 120 & +16 & 8 & 20 & 14 & 17 & 16 & -\\
	9 & 7,000,000 & 23 & 130 & +16 & 7 & 19 & 13 & 16 & 15 & +1 Skill Point\\
	10 & 8,000,000 & 24 & 140 & +17 & 7 & 19 & 13 & 16 & 15 & -\\
	11 & 9,000,000 & 25 & 150 & +17 & 6 & 18 & 12 & 15 & 14 & -\\
	12 & 10,000,000 & 26 & 160 & +18 & 6 & 18 & 12 & 15 & 14 & -\\
	13 & 12,000,000 & 27 & 180 & +18 & 5 & 17 & 11 & 14 & 13 & Increased Weapon Damage, Multiple Attacks (3), +1 Skill Point\\
	14 & 14,000,000 & 28 & 200 & +19 & 5 & 17 & 11 & 14 & 13 & -\\
	15 & 16,000,000 & 29 & 220 & +19 & 4 & 16 & 10 & 13 & 12 & -\\
	16 & 18,000,000 & 30 & 240 & +20 & 4 & 16 & 10 & 13 & 12 & -\\
	17 & 20,000,000 & 31 & 260 & +20 & 3 & 15 & 9 & 12 & 11 & +1 Skill Point\\
	18 & 22,000,000 & 32 & 280 & +21 & 3 & 15 & 9 & 12 & 11 & -\\
	19 & 25,000,000 & 33 & 300 & +21 & 2 & 14 & 8 & 11 & 10 & -\\
	20 & 30,000,000 & 34 & 330 & +22 & 2 & 14 & 8 & 11 & 10 & -\\
	21 & 35,000,000 & 35 & 360 & +22 & 1 & 13 & 7 & 10 & 9 & +1 Skill Point\\
	22 & 40,000,000 & 36 & 390 & +23 & 1 & 13 & 7 & 10 & 9 & -\\
	23 & 45,000,000 & 37 & 420 & +24 & 0 & 12 & 6 & 9 & 8 & -\\
	24 & 50,000,000 & 38 & 450 & +25 & 0 & 12 & 6 & 9 & 8 & -\\
	25 & 55,000,000 & 39 & 500 & +26 & -1 & 11 & 5 & 8 & 7 & Increased Weapon Damage (2), Multiple Attacks (4), +1 Skill Point\\
	26 & 60,000,000 & 40 & 550 & +27 & -1 & 10 & 5 & 8 & 7 & -\\
	27 & 70,000,000 & 41 & 600 & +28 & -2 & 9 & 4 & 7 & 6 & -\\
	28 & 80,000,000 & 42 & 650 & +29 & -3 & 8 & 4 & 7 & 6 & -\\
	29 & 90,000,000 & 43 & 700 & +30 & -4 & 7 & 3 & 6 & 5 & +1 Skill Point\\
	30 & 100,000,000 & 44 & 750 & +31 & -5 & 6 & 3 & 6 & 5 & -\\
	31 & 110,000,000 & 45 & 800 & +32 & -6 & 5 & 2 & 5 & 4 & -\\
	32 & 120,000,000 & 46 & 900 & +33 & -7 & 5 & 2 & 5 & 4 & -\\
	33 & 130,000,000 & 47 & 1,000 & +34 & -8 & 5 & 2 & 5 & 4 & +1 Skill Point\\
	34 & 140,000,000 & 48 & 1,250 & +35 & -9 & 5 & 2 & 5 & 4 & -\\
	35 & 150,000,000 & 49 & 1,500 & +36 & -10 & 5 & 2 & 5 & 4 & -\\
	36 & 160,000,000 & 50 & 2,000 & +37 & -11 & 5 & 2 & 5 & 4 & -
  \end {tabularx}
\end {table}
\newpage
\begin{multicols*}{2}

\begin {table}[H]
  \caption{Immortal Special Abilities Progression}
  \begin{tabularx}{\columnwidth}{>{\bfseries}YYYY}
	\thead{Level} & \thead{Anti-Magic} & \thead{Power Reserve} & \thead{Strike to Kill Damage}\\
	1 & 50\% & 300 & 2d6\\
	2 & 50\% & 325 & 2d6\\
	3 & 50\% & 350 & 2d6\\
	4 & 50\% & 375 & 2d6\\
	5 & 50\% & 400 & 2d6\\
	6 & 50\% & 450 & 2d6\\
	7 & 50\% & 500 & 2d6\\
	8 & 50\% & 600 & 2d6\\
	9 & 50\% & 700 & 2d6\\
	10 & 50\% & 800 & 2d6\\
	11 & 50\% & 900 & 2d6\\
	12 & 50\% & 1,000 & 2d6\\
	13 & 60\% & 1,200 & 3d6\\
	14 & 60\% & 1,400 & 3d6\\
	15 & 60\% & 1,600 & 3d6\\
	16 & 60\% & 1,800 & 3d6\\
	17 & 60\% & 2,000 & 3d6\\
	18 & 60\% & 2,200 & 3d6\\
	19 & 70\% & 2,500 & 3d6\\
	20 & 70\% & 3,000 & 3d6\\
	21 & 70\% & 3,500 & 3d6\\
	22 & 70\% & 4,000 & 3d6\\
	23 & 70\% & 4,500 & 3d6\\
	24 & 70\% & 5,000 & 3d6\\
	25 & 80\% & 5,500 & 4d6\\
	26 & 80\% & 6,000 & 4d6\\
	27 & 80\% & 7,000 & 4d6\\
	28 & 80\% & 8,000 & 4d6\\
	29 & 80\% & 9,000 & 4d6\\
	30 & 80\% & 10,000 & 4d6\\
	31 & 90\% & 11,000 & 4d6\\
	32 & 90\% & 12,000 & 4d6\\
	33 & 90\% & 13,000 & 4d6\\
	34 & 90\% & 14,000 & 4d6\\
	35 & 90\% & 15,000 & 4d6\\
	36 & 90\% & 16,000 & 4d6
  \end {tabularx}
\end {table}

\section{Immortal Level Spells}\label{sec:Immortal Level Spells}
These spells may only be cast by immortals while in Embodied Form. These spells are not stored in spell books and it is not possible to write these spells onto scrolls or make magic items that duplicate their effects.

Immortal level spells can be either lesser or greater. Lesser spells generally have temporary effects and cost the immortal power points to cast; whereas greater spells generally have permanent effects and cost the immortal experience points to cast. None of these spells can be dispelled by either mortal or immortal casters.

\begin {table}[H]
  \caption{Immortal Level Spells}
  \begin{tabularx}{\columnwidth}{>{\bfseries}YY}
	\thead{Lesser Spells} & \thead{Greater Spells}\\
	Conceal Magical Nature & Bestow*\\
	Create Mundane Object‡ & Create Artifact\\
	Detect Immortal Magic & Create Embodied Form\\
	Hear Prayers & Create Mortal Form\\
	Immortal Eye & Create Mundane Object‡\\
	Increase Spell Duration & Create Species\\
	Power Attack & Create Time Gate\\
	Prepare Mortal Magic & Grant Power\\
	Probe & Improve Ability\\
	Probe Shield & Invest Cleric*\\
	Reduce Saving Throw & Shape Reality\\
	& Time Travel\\
	& Transform
	\end {tabularx}
	*Reversible spell
	‡Can be cast as either a Lesser Spell for temporary effect or a Greater Spell for permanent effect
\end {table}

\subsection{Bestow}\index[spells]{Bestow}
\statblock{\textit{Immortal (Greater)}

\textbf{Cost:} 50,000 XP per ability bestowed

\textbf{Range:} 60 ft.

\textbf{Duration:} Permanent}

The bestow pell grants one or more permanent abilities to a mortal creature. An ability can be either a +1 bonus to an ability score or the equivalent of a mortal spell between \nth{1} and \nth{7} level having been cast on the mortal and made permanent by a Permanence spell.

However, unlike an actual Permanence spell, this bestows the power on the mortal as an innate ability which cannot be dispelled.

\textbf{Reverse:} Diminish removes one or more abilities from a mortal creature. An ability can be one granted by this spell or an innate ability such as \iref[sec:Infravision]{Infravision}, or it can result in a -1 penalty to one of the mortal’s ability scores.

The mortal gets a saving throw vs. spells with a -2 penalty in order to avoid the effects.

\subsection{Conceal Magical Nature}\index[spells]{Conceal Magical Nature}
\statblock{\textit{Immortal (Lesser)}
\textbf{Cost:} 10 pp

\textbf{Range:} 60 ft.

\textbf{Duration:} One year, or until item used}

This spell will cause a single magical object (including an artifact.) to fail to show up on Detect Magic, Detect Evil, Know Alignment, and Truesight spells.

This concealment of the item’s magical nature will last for one year or until the item is used; at which point this spell is canceled. and the item’s magical nature will become apparent once more.

This spell does not work on living creatures, and multiple castings are not cumulative in duration.

\subsection{Create Artifact}\index[spells]{Create Artifact}
\statblock{\textit{Immortal (Greater)}

\textbf{Cost:} Varies

\textbf{Range:} Touch

\textbf{Duration:} Permanent}

This spell creates an artifact (see \fullref{sec:Artifacts})—an extremely powerful magical item.

\subsection{Create Embodied Form}\index[spells]{Create Embodied Form}
\statblock{\textit{Immortal (Greater)}

\textbf{Cost:} 1,000,000 XP

\textbf{Range:} 10 ft.

\textbf{Duration:} Permanent}

This spell creates a new Embodied Form for either the caster or for a willing target within 10 feet.

The new Embodied Form can have any combination of four Embodied Form powers, and will look like the caster wishes.

When cast on a mortal with fewer than 3,000,000 experience points, this spell will fail. If the mortal has at least 3,000,000 experience points then casting this spell on them will make them become an immortal with the same number of experience points that they had as a mortal; and whose home plane is the plane on which this spell was cast on them.

If cast on an immortal (usually the caster themselves), this spell will give them an additional Embodied Form which may look different and have different powers to their existing form.

An immortal can have as many Embodied Forms as they like, and may change freely between them taking a round to do so.

\subsection{Create Mortal Form}\index[spells]{Create Mortal Form}
\statblock{\textit{Immortal (Greater)}

\textbf{Cost:} 50,000 XP

\textbf{Range:} Caster

\textbf{Duration:} Permanent}

This spell creates a new Mortal Form for the caster.

The new Mortal Form can be of any race, class or monster species; but its abilities are limited to those that a normal member of that race, class or monster species would have.

Once created, the Mortal Form lasts indefinitely (although it may age as a normal member of its race or class if the caster chooses so at the time of casting) until slain. A Mortal Form cannot be raised or reincarnated. The caster may shift away from the Mortal Form and then shift back to it at a later time without needing to cast this spell again, and the Mortal Form will have aged appropriately if the caster chose for it to do so.

The caster can, of course, simply create a new Mortal Form that is identical to a previous one in order to give the appearance of restored youth to an aged one or give the appearance that a dead one has been raised.

The caster may cast Create Mundane Object in conjunction with this spell in order to create clothing and equipment for their new Mortal Form.

Any equipment created in this manner will disappear when the caster shifts to a different form and reappear when the caster shifts back to the Mortal Form.

An immortal can have as many Mortal Forms as they like, and may change freely between them taking a round to do so.

\subsection{Create Mundane Object}\index[spells]{Create Mundane Object}
\statblock{\textit{Immortal (Greater)}

\textbf{Cost:} 1 XP per 1 gp value of the object

\textbf{Range:} 10 ft.

\textbf{Duration:} Permanent}

This spell creates a mundane object out of nothingness. It can create any non-magical object or any magic item; but can not create artifacts.

This spell can create complex items such as buildings complete with fixtures and fittings, providing the immortal pays for the total value of the object.

If the immortal chooses, they can make temporary items that disappear after either 24 hours or when the immortal switches out of Embodied Form whichever comes sooner. Temporary items do not cost experience points to create, but cost 1 power point per 10,000 gp of value created (rounded up).

\subsection{Create Species}\index[spells]{Create Species}
\statblock{\textit{Immortal (Greater)}

\textbf{Cost:} Varies

\textbf{Range:} 10 ft.

\textbf{Duration:} Permanent}

Each casting of this spell creates a single creature, which may be intelligent.

The creature created does not have to be from an existing species or race; the immortal can simply invent a new species by casting this spell, and if they cast the spell repeatedly to create the same type of creature they can create a breeding population of this new species.

It is up to the Game Master (and the actions of the immortal in protecting, teaching and guiding their new species) to determine whether the new species will thrive or die out. This is how many of the humanoid species in the world started.

The immortal has no direct control over creatures created by this spell, but they will usually be grateful for existence and worship the immortal.

This spell costs the immortal 100,000 experience points to cast, plus an additional 100,000 experience points for each special ability of the creature created. At the Game Master’s discretion, having a high number of hit dice may count as one or more special abilities in its own right.

Intelligent creatures created by this spell cannot take on human classes, but can potentially become sorcerers or shamans.

The player and Game Master may optionally create a custom racial class for the new species, similar to the racial classes that exist for dwarves, gnomes, elves and halflings.

\subsection{Create Time Gate}\index[spells]{Create Time Gate}
\statblock{\textit{Immortal (Greater)}

\textbf{Cost:} 50 pp

\textbf{Range:} 10 ft.

\textbf{Duration:} Concentration}

When you cast this spell, you create a portal between your time and location and another specified time and location on the same plane. The portal is circular, with a diameter anywhere between 5 feet and 20 feet.

The portals at either end of the link are one-sided, and anything passing into one emerges from the other. The portals both travel through time at the normal rate of one second per second, so they always remain the same relative distance apart in time. 

Creating the portals branches the time line firstly at the point of creation and then secondly at the far end, as if you had traveled from here to there, as does ceasing concentration so that the portals disappear.

Additionally, if anything passes through the portals in either direction the time line branches at their point of departure and then almost immediately afterwards at their point of arrival in the normal manner for time travel journeys.

If multiple travelers wish to travel together without the time line branching multiple times, they must hold hands or otherwise remain in physical contact while using readied actions to pass through the time gate simultaneously.

\subsection{Detect Immortal Magic}\index[spells]{Detect Immortal Magic}
\statblock{\textit{Immortal (Lesser)}

\textbf{Cost:} 10 pp

\textbf{Range:} One astronomical body on the Prime Plane

\textbf{Duration:} 1 Day}

This spell will scan one astronomical body (on the \ilink{sec:Prime Plane}{Prime Plane} only) for immortal activity. A particularly large or complex astronomical body may require more than one detect immortal magic spell to cover it, at the Game Master’s discretion.

Each round that an immortal is on the planet in Embodied Form, there is a 5\% cumulative chance that this spell will detect their presence. If the embodied immortal uses spells then this becomes a 10\% cumulative chance per round. There is also a 1\% cumulative chance per round that this spell will detect the active use of an artifact.

The spell will not reveal the identity of the immortal, or even whether it was triggered by an immortal or an artifact, but it will reveal the location that the immortal power was detected at.

In most campaign settings, the major planets where the campaign takes place will normally have some kind of rules set up by the most powerful immortals that prohibit direct immortal activity except for particular prescribed activities (such as investing clerics or sending omens to worshipers).

This is for the safety of those living on the planet, because it is relatively simple for a dispute between immortals to destroy huge areas of civilization. Of course, not all immortals will obey such rules, so there is often a council of high level immortals on “guard duty” using this spell to monitor events and prepared to step in and stop any unauthorized direct meddling.

This spell will not detect immortals in Spirit Form, nor will it detect immortals in Mortal Form.

\subsection{Grant Power}\index[spells]{Grant Power}
\statblock{\textit{Immortal (Greater)}

\textbf{Cost:} Varies

\textbf{Range:} Touch

\textbf{Duration:} Permanent}

This spell allows the immortal so simply give away experience points.

The immortal can spend any number of experience points when casting this spell (providing they can afford to spend them without losing a level).

The target of the spell will gain the same number of experience points that the immortal has spent.

The experience can be given to a mortal, or it can be given to another immortal who has fewer experience points than the caster. However, it cannot give enough experience points to bring the target’s experience total higher than the caster’s.

This spell may only be cast once per experience level. Once an immortal casts grant power, they can not cast it again until they have increased in level.

\subsection{Hear Prayers}\index[spells]{Hear Prayers}
\statblock{\textit{Immortal (Lesser)}

\textbf{Cost:} 5 pp

\textbf{Range:} Everywhere

\textbf{Duration:} 30 minutes}

This spell allows the immortal to hear all prayers that were made to them over the previous day.

It does not grant the immortal any special ability to answer those prayers, but most immortals are aware that the occasional intervention is a great way to keep worshipers loyal.

This spell is also an ideal way for an immortal to keep up to date on the actions of their agents and minions—assuming those minions report those actions in prayer.

\subsection{Immortal Eye}\index[spells]{Immortal Eye}\label{spell:Immortal Eye}
\statblock{\textit{Immortal (Lesser)}

\textbf{Cost:} 5 pp

\textbf{Range:} Anywhere

\textbf{Duration:} 1 hour}

This spell allows the immortal to create an invisible magical eye in any location that they have previously visited, whether on the same plane or a different plane. The eye is detectable by a Detect Magic spell cast in its location, but is will not show up on a Detect Immortal Magic spell directed at the location.

The caster can move the eye at a speed of 240 feet per round by concentrating, and when they stop concentrating the eye will remain in place. At the end of the spell’s duration, the caster can spend an additional 5 power points in order to extend it for another hour.

If the caster concentrates for ten minutes, they can cause their Embodied Form to fall into a trance, and their Spirit Form to appear in the eye’s location. This is the only time at which an Immortal can have more than one form simultaneously.

The Spirit Form can stay at the eye’s location until the duration of the spell runs out, and cannot shift to a different form or move from that point (although it can be moved by the caster concentrating in order to move the eye itself), but it can communicate with people or enter into people’s dreams as normal.

\subsection{Improve Ability}\index[spells]{Improve Ability}
\statblock{\textit{Immortal (Greater)}

\textbf{Cost:} See \fullref{tab:Improve Ability}

\textbf{Range:} Caster

\textbf{Duration:} Permanent}

This spell permanently improves one of the ability scores of the caster. The score is improved in all Embodied Forms (and in the case of mental ability scores, also in Spirit Form).

The cost to improve an ability score and the maximum value to which an ability score can be raised depend on the caster’s level, and can be found on \fullref{tab:Improve Ability}.

\begin {table}[H]
  \caption{Improve Ability}\label{tab:Improve Ability}
  \begin{tabularx}{\columnwidth}{>{\bfseries}YYYYY}
	\thead{Level} & \thead{Cost} & \thead{Max Ability Score}\\
	1-12 & 100,000 & 25\\
	13-18 & 200,000 & 50\\
	19-24 & 400,000 & 75\\
	25-30 & 800,000 & 100\\
	31-36 & 1,600,000 & 100
  \end {tabularx}
\end {table}

\subsection{Increase Spell Duration}\index[spells]{Increase Spell Duration}
\statblock{\textit{Immortal (Lesser)}

\textbf{Cost:} 1 pp per increase

\textbf{Range:} -

\textbf{Duration:} -}

This spell is cast at the same time as the caster casts a mortal level spell.

For each power point spent on this spell, the mortal level spell will have its duration extended by an amount equal to its normal duration.

If the caster spends 1 pp on this spell, for example, the mortal spell it is cast with will have its duration doubled. If the caster spends 2 pp, the mortal spell it is cast with will have its duration tripled. If the caster spends 3 pp, the mortal spell it is cast with will have its duration quadrupled. And so forth.

Both this spell and the Reduce Saving Throw spell can be cast at the same time on the same mortal level spell.

This spell cannot be cast with other immortal level spells.

\subsection{Invest Cleric}\index[spells]{Invest Cleric}
\statblock{\textit{Immortal (Greater)}

\textbf{Cost:} 50,000 XP

\textbf{Range:} Touch

\textbf{Duration:} Permanent}

This spell will either turn a willing human mortal into a cleric of the caster or a willing non-human mortal into a shaman of the caster. If the target already has a class then this class is lost and replaced with the cleric class.

The mortal neither gains nor loses experience, so a normal commoner will become a first level cleric with no experience points but an experience adventurer may become a higher level cleric with the same amount of experience points that they had in their previous class.

\textbf{Reverse:} Excommunicate Cleric removes the clerical, druid or shaman class from a mortal worshiper of the caster.

The target of the spell immediately loses their cleric class and becomes fighter with the same amount of experience points that they had in the cleric class.

There is no saving throw against this excommunication, but it can only be used on clerics or shamans who are worshipers of the caster (or druids who are former worshipers of the caster). It cannot be used to neutralize the clerics or shamans of rivals.

\subsection{Power Attack}\index[spells]{Power Attack}\label{spell:Power Attack}
\statblock{\textit{Immortal (Lesser)}

\textbf{Cost:} 2 pp per 1d6 of attack

\textbf{Range:} 720 ft.

\textbf{Duration:} Instantaneous}

This spell can only be cast on other immortals or Exalted beings.

The caster can spend up to two power points per level when casting the spell. The spell automatically hits its target, although the target’s Anti-Magic does apply and may stop the spell from working.

The target loses 1d6 power points per 2 power points used to cast the spell. If the target makes a saving throw vs. power attacks then they only take half damage.

This spell has no effect on creatures who do not have a power reserve.

This spell can be cast both by and against immortals in Spirit Form, and when cast by an immortal in Spirit Form it can be detected by Detect Immortal Magic.

\subsection{Prepare Mortal Magic}\index[spells]{Prepare Mortal Magic}
\statblock{\textit{Immortal (Lesser)}

\textbf{Cost:} Varies

\textbf{Range:} Caster

\textbf{Duration:} 1 Day}

This spell allows the caster to cast mortal level spells for the rest of day as if they had prepared them. This spell can be cast at any time, but only once per day.

When casting this spell, the caster decides how many mortal level spells they wish to cast, and must spend power points accordingly.

For 1 power point per two levels desired, the caster can prepare a number and type of spells equivalent to a cleric, druid or wizard of that level. For example an immortal could spend 10 power points (7+3) to prepare the same number and type of spells that a \nth{14} level wizard and a \nth{6} level cleric can prepare each day.

Alternately, for 25 power points the caster can cast any number of druid/cleric spells for the rest of the day without needing to prepare them in advance, or for 50 power points the caster can cast any number of wizard spells for the rest of the day without needing to prepare them in advance.

These options can be mixed, so the caster could spend 30 power points (25+5) to cast any number of druid/cleric spells plus a number of wizard spells equivalent to a \nth{10} level wizard. Most immortals simply spend 75 power points at the beginning of each day to be able to cast any mortal spell any number of times during the day.

In any of the above cases, the mortal level spells cast by the immortal are cast as if by a spellcaster of equal level to the immortal’s hit dice; and any saving throws that mortals make against them are made at a -2 penalty unless the individual spell specifies otherwise.

When casting mortal level spells, an immortal caster does not need to speak or gesture, and cannot have these spells disrupted by being hit before their initiative.

\subsection{Probe}\index[spells]{Probe}
\statblock{\textit{Immortal (Lesser)}

\textbf{Cost:} 5 pp

\textbf{Range:} 720 ft.

\textbf{Duration:} Instant}

This spell can be cast on any creature, including another immortal, and gives the caster information about that creature. There is no saving throw against the spell, and Anti-Magic does not apply.

The spell informs the caster of the level, hit dice, power points, hit points, and Anti-Magic of the creature (which, of course, will also inform the caster indirectly whether the creature is mortal or immortal).

The spell will also inform the caster of the name of the target, although this additional information can be blocked by the target’s Anti-Magic or by the target making a saving throw (vs. spells at a -2 penalty in the case of mortals, or vs. spell attacks in the case of immortals).

When cast at an immortal who is in Mortal Form, this spell only reveals the details (including name) of that Mortal Form, not the details and name of the immortal.

This spell can be cast both by and against immortals in Spirit Form, and when cast by an immortal in Spirit Form it can be detected by Detect Immortal Magic.

\subsection{Probe Shield}\index[spells]{Probe Shield}
\statblock{\textit{Immortal (Lesser)}

\textbf{Cost:} 5 pp

\textbf{Range:} Caster

\textbf{Duration:} 10 Minutes}

This spell makes the caster immune to the Probe spell. Anyone casting the Probe spell at the caster during the duration of this spell will only detect that the probe shield is active and will get no other information—although they may infer from the presence of the probe shield that the caster is an immortal.

This spell can be cast by immortals in Spirit Form, and when cast by an immortal in Spirit Form it can be detected by Detect Immortal Magic.

\subsection{Reduce Saving Throw}\index[spells]{Reduce Saving Throw}
\statblock{\textit{Immortal (Lesser)}

\textbf{Cost:} 1 pp per -1 penalty per target

\textbf{Range:} -

\textbf{Duration:} -}

This spell is cast at the same time as the caster casts a mortal level spell.

For each power point spent on this spell, one mortal target of the mortal level spell will get a -1 penalty on its saving throw against the spell.

If the caster spends 5 pp on this spell, for example, a single mortal could be given a -5 penalty. If the caster spends 15 pp, a single mortal could be given a -15 penalty or one mortal could be given a -10 penalty and a second one given a -5 penalty, or fifteen mortals could be given a -1 penalty each. And so forth.

Both this spell and the Increase Spell Duration spell can be cast at the same time on the same mortal level spell.

This spell cannot be cast with other immortal level spells, and cannot penalize the saving throws of immortal level creatures.

\subsection{Shape Reality}\index[spells]{Shape Reality}
\statblock{\textit{Immortal (Greater)}

\textbf{Cost:} Varies

\textbf{Range:} Special

\textbf{Duration:} Permanent}

This spell allows the immortal to reshape reality. The spell can be used to create or move astronomical bodies, planes or even crystal spheres. A group of immortals can cast this spell together, sharing the experience cost of the casting equally between them.

\textbf{Create a Body:} This spell can be used to create an astronomical body of any size. The body must be created on the \ilink{sec:Prime Plane}{Prime Plane} within a Crystal Sphere.

The body can be of any of the four basic types (gaseous body, solid body, radiating body, liquid body), and will take on an orbit of the caster’s choosing either around the center of the Crystal Sphere or around an existing body in the sphere.

It is possible to create a stationary body in the exact center of a sphere with this spell.

When a body is created using this spell, four equivalent bodies are also created in the four \iref[sec:The Elemental Planes]{Elemental Planes} and an ethereal copy of the body is created in the \iref[sec:Ethereal Plane]{Ethereal Plane}. See \fullref{chap:Other Worlds} for details on how the \iref[sec:The Elemental Planes]{Elemental Planes} and the \iref[sec:Ethereal Plane]{Ethereal Plane} work.

The orbit of the body created must not take it more than half way from the center of the sphere to its edge.

When created, a body will have no plant or animal life, although such life can be taken there. It will, however, be created with an air and gravity envelope of its own.

When an immortal uses this spell to create a body, they will become aware of the danger of the body colliding with other bodies, and therefore no immortal will accidentally cause such a collision by using this spell. However, a malicious immortal could use this spell to deliberately set up a collision if they desired.

The bodies created by this spell will normally be spherical, but the caster may create them in other shapes such as rings or flat discs if they desire.

The cost to create a body depends on the size of the body as indicated on \fullref{tab:Create a Body}.

\begin {table}[H]
  \caption{Create a Body}\label{tab:Create a Body}
  \begin{tabularx}{\columnwidth}{>{\bfseries}YY}
	\thead{Body} & \thead{Cost}\\
	Asteroid (50 mi radius) & 50,000 XP\\
	Small Moon (500 mi radius) & 100,000 XP\\
	Large Moon (1,000 mi radius) & 200,000 XP\\
	Small Planet (2,000 mi radius) & 400,000 XP\\
	Medium Planet (4,000 mi radius) & 600,000 XP\\
	Large Planet (10,000 mi radius) & 800,000 XP\\
	Small Sun (100,000 mi radius) & 1,200,000 XP\\
	Large Sun (500,000 mi radius) & 1,400,000 XP\\
	Unusual Shape & +200,000 XP
  \end {tabularx}
\end {table}

\textbf{Move a Body:} The spell can also be used to move an existing astronomical body. Moving an existing body has the same rules as creating one—the body must be moved to an orbit around the center of the sphere or around an existing body, and the Immortal cannot accidentally create a situation where two bodies will collide.

The Elemental and Ethereal equivalents of the body are moved with it.

This spell can not move a body out of the Crystal Sphere that contains it. Nor can it move a body out of the \ilink{sec:Prime Plane}{Prime Plane}.

The cost to move a body depends on its size as indicated on \fullref{tab:Move a Body}.

\begin {table}[H]
  \caption{Move a Body}\label{tab:Move a Body}
  \begin{tabularx}{\columnwidth}{>{\bfseries}YY}
  \thead{Body} & \thead{Cost}\\
	Asteroid (50mi radius) & 10,000 XP\\
	Small Moon (500mi radius) & 20,000 XP\\
	Large Moon (1,000mi radius) & 40,000 XP\\
	Small Planet (2,000mi radius) & 80,000 XP\\
	Medium Planet (4,000mi radius) & 120,000 XP\\
	Large Planet (10,000mi radius) & 160,000 XP\\
	Small Sun (100,000mi radius) & 240,000 XP\\
	Large Sun (500,000mi radius) & 280,000 XP
  \end {tabularx}
\end {table}

\textbf{Create a Celestial Sphere:} An Immortal with enough experience to spend can use this spell to create an empty Celestial Sphere. The Immortal must be in the Luminiferous Aether to use this function of the spell.

There is no chance of the new sphere colliding with an existing sphere, since the spheres will repel each other.

The newly created sphere can be from 3-9 billion miles in radius, and will be full of Void and contain no astronomical bodies, although it will have stars embedded in it. However, creating the sphere will also create a matching set of Inner Planes (Elemental Planes of Air, Earth, Fire and Water; and an \iref[sec:Ethereal Plane]{Ethereal Plane}).

If the Immortal is not in an existing river of aether when they cast this spell, a bi-directional river will spontaneously form to connect the new sphere to the nearest other sphere, and there is a 25\% chance that a bi-directional river will also form to connect the new sphere to another “nearby” sphere at the Game Master’s discretion.

If the Immortal is in an existing river of aether connecting two existing spheres when they cast this spell, the result is determined randomly as indicated on \fullref{tab:Create a Celestial Sphere}.

\begin {table}[H]
  \caption{Create a Celestial Sphere}\label{tab:Create a Celestial Sphere}
  \begin{tabularx}{\columnwidth}{>{\bfseries}cY}
	\thead{1d8} & \thead{Result}\\
	1-4 & Bi-directional rivers form to both existing spheres. The old river between the existing spheres still exists.\\
	5-6 & Single direction rivers form to one of the existing spheres and from the other (chosen randomly). The old river between the existing spheres still exists.\\
	7 & A bi-directional river forms to one of the existing spheres (chosen randomly). The old river between the existing spheres still exists.\\
	8 & The existing river is split into two rivers, each of which keeps the direction of flow that it had before the sphere was created. It is no longer possible to travel directly between the two existing spheres without going via the new sphere.
  \end {tabularx}
\end {table}

Creating a Celestial Sphere costs 6,400,000 experience points.

\textbf{Create an Outer Plane:} An Immortal who is inside a Celestial Sphere (or inside any of the Inner Planes or Outer Planes attached to that sphere) can use this spell to create a new Outer Plane.

The new plane will be anchored at the Immortal’s current location. If this location is on an orbiting astronomical body, the anchor point of the plane will stay is position relative to the rotation and movement of the body. See \fullref{sec:The Outer Planes} for more details on Outer Planes and their anchor points.

The plane will always be roughly spherical, and the geographic edge of the plane is similar to the crystal of a Celestial Sphere, except that it is not possible for anything to move through the crystal in any way.

The cost to create an outer plane depends on the size of the plane as indicated on \fullref{tab:Create an Outer Plane}.

\begin {table}[H]
  \caption{Create an Outer Plane}\label{tab:Create an Outer Plane}
  \begin{tabularx}{\columnwidth}{>{\bfseries}YY}
	\thead{Size} & \thead{Cost}\\
	Dwelling & 200,000 XP\\
	Town & 300,000 XP\\
	Island/Asteroid & 400,000 XP\\
	Continent/Moon & 800,000 XP\\
	Planet & 1,600,000 XP
  \end {tabularx}
\end {table}

The caster can choose the (initial) contents of the plane from any of the possibilities that can be chosen when altering an outer plane.

However, regardless of the options chosen, Outer Planes never contain Void. They are always full of atmosphere.
A newly created plane will not contain any plant or animal life.

\textbf{Move an Outer Plane:} If an Immortal is at the anchor point of an outer plane, they can cast this spell to move that anchor point. This is only possible if the plane is not the home plane of an unwilling Immortal.

When the Immortal casts this spell, the anchor point of the plane is folded up into a glowing sphere about the size of a grapefruit in the Immortal’s hand.

The Immortal then moves to the location that they wish to move the anchor point to, and lets go of it. The anchor point of the outer plane is then fixed in the new location.

If the Immortal switches from their Embodied Form to a different form (other than directly switching to a different Embodied Form), the anchor point is immediately dropped in the current location.

The anchor point can be carried through a Gate, and can be transferred to another plane via the Travel spell or a similar ability. However, it can not be carried directly (or indirectly) into the Luminiferous Aether. Attempting to do so will cause the spell to fail and the anchor point of the plane to revert to its previous location.

Similarly, it is not possible to use this spell create a “loop” of Outer Planes by taking the anchor point into the plane that is being moved or into any plane that is anchored (directly or indirectly) onto that plane. Attempting this will also cause the spell to fail and the anchor point of the plane to return to its previous location.

While the anchor point of the plane is being carried, it cannot be crossed using Travel spells or similar abilities. Gates to and from the plane being moved remain fully functional, and Travel spells can still be used to pass between the plane being moved and planes that are anchored to it.

The cost to move the anchor point of an outer plane depends on the size of the plane as indicated on \fullref{tab:Move an Outer Plane}.

\begin {table}[H]
  \caption{Move an Outer Plane}\label{tab:Move an Outer Plane}
  \begin{tabularx}{\columnwidth}{>{\bfseries}YY}
	\thead{Size} & \thead{Cost}\\
	Dwelling & 50,000 XP\\
	Town & 75,000 XP\\
	Island/Asteroid & 100,000 XP\\
	Continent/Moon & 200,000 XP\\
	Planet & 400,000 XP
  \end {tabularx}
\end {table}

\textbf{Alter an Outer Plane:} An Immortal can also use this spell to alter the characteristics of an existing Outer Plane.
The plane must be either the Immortal’s home plane or a plane that the Immortal created; and if the plane is also the home plane of another Immortal, that Immortal must be willing for the alteration to take place.

Each casting of the spell can alter one of the following facets of the plane:

\begin{itemize}
	\item{The basic conditions and matter can be altered to match either the \ilink{sec:Prime Plane}{Prime Plane} or one of the Elemental Planes.}
	\item{The magical laws of the plane can be changed to one of the following selections: All magic works; All mortal-level spells are considered to be X levels higher than normal; Only magic cast by Immortals works; Only Immortal level spells work; No magic works.}
	\item{The appearance of the plane and the native matter within it can be altered (e.g. pink sky, or everything appears to be made of wood).}
	\item{The type of “body” on the plane can be changed to any one of the following types: air body, earth body, fire body, water body.}
	\item{The matter in the plane can be changed to any one of the following types: A single round astronomical body in the center of the plane; A set of tiny flat astronomical bodies floating in the plane; A hemispherical astronomical body filling half of the plane.}
	\item{The gravity on the plane can be changed to any one of the following types: Pulling in a constant “down” direction; Pulling towards the center of the plane; Pulling towards each body the plane contains as if those bodies were in the Void.}
\end{itemize}
Note that changing the properties of an inhabited plane can be very dangerous to those inhabitants. Also, no matter how restricted the magic is on a plane, Immortals in Spirit Form from can always use their equivalent of the Travel spell to exit the plane at its anchor point, and the Shape Reality spell can still be used to change the restriction on magic.

The cost to alter a single facet of a plane depends on the size of the plane as indicated on \fullref{tab:Alter an Outer Plane}.

\begin {table}[H]
  \caption{Alter an Outer Plane}\label{tab:Alter an Outer Plane}
  \begin{tabularx}{\columnwidth}{>{\bfseries}YY}
	\thead{Size} & \thead{Cost}\\
	Dwelling & 10,000 XP\\
	Town & 10,000 XP\\
	Island/Asteroid & 10,000 XP\\
	Continent/Moon & 20,000 XP\\
	Planet & 40,000 XP
  \end {tabularx}
\end {table}

\subsection{Time Travel}
\statblock{\textit{Immortal (Greater)}

\textbf{Cost:} 20 pp

\textbf{Range:} Caster

\textbf{Duration:} Concentration}

You are taken out of your current time line, and can travel up and down it at will.

You disappear from your current time (this causes time line branching as normal) and are then able to view the time line you are now on. Everything around you looks stretched like ribbons or cords, with every object and creature extended back into its past and forward into its future.

You are able to move around as normal, including the use of movement spells, but you are unable to interact with anything or affect anything. Time continues to pass while you are in the time stream, and while not actively moving forward or back in time you drift forwards at the normal rate (i.e. you drift forward at one second per second of actual time). By concentrating, you are able to actively move forward or backward in time at a speed of up to one year per minute.

When you stop concentrating, you reappear at your current position in the time line that you have been moving through. Again, this causes time line branching as normal.

Because each time you use this spell it causes the time line to branch, if someone else uses this spell it is not possible for you to cast it after they do in order to follow them, since your casting will create a new time line and you will therefore not be travelling up and down the same time line as the person you are following.

If multiple casters wish to use this spell to travel together along the same time line, they must hold hands or otherwise touch each other and then use readied actions to cast the spell simultaneously. Similarly, if they wish to arrive in the same time line together they must be in physical contact with each other while they simultaneously stop concentrating.

\subsection{Transform}\index[spells]{Transform}
\statblock{\textit{Immortal (Greater)}

\textbf{Cost:} 50,000 XP

\textbf{Range:} 60 ft.

\textbf{Duration:} Permanent}

This spell changes the form of a mortal creature in a similar manner to a Polymorph Other spell.

However, the change is non-magical and therefore cannot be detected by a Detect Magic spell or undone by a Dispel Magic spell.

Once the change has happened, the target is a normal non-magical creature of the appropriate type. However, a Probe spell cast at the target will reveal their “true” pre-transform identity as well as their post-transform identity.

The caster decides whether the target keeps their old mind; has their mind changed along with their body but keeps their memories; or has their mind changed along with their body and has their memories replaced with fake ones suitable for the new body.

If the memories of the target are suppressed, the caster may optionally set a command word that will restore them.

The target may make a saving throw vs. spells with a -2 penalty to avoid the effect, although a willing target does not need to make this saving throw.

This spell can be used to change the race and/or class of a mortal, and it can also be used to lower the amount of experience that they have. It cannot raise the experience that the target has unless the spell is being used to restore them to their prior form.

If this spell is cast on someone who has already been transformed by a previous application of this spell, the caster can choose to revert the target back to their original form without necessarily knowing what that form is.

\end{multicols*}

